% -*- mode: tex -*-
%%%%%%%%%%%%%%%%%%%%%%%%%%%%%%%%%%%%%%%%%%%%%%%%%%%%%%%%%%%%%%%%%%%%%%%%%%%%%%%%
%                                STANDARD                                      %
%%%%%%%%%%%%%%%%%%%%%%%%%%%%%%%%%%%%%%%%%%%%%%%%%%%%%%%%%%%%%%%%%%%%%%%%%%%%%%%%
\definefontfeature[default][default]
                  [protrusion=quality,
                    expansion=quality,
                    script=latn]
\setupalign[hz,hanging]
\setuptolerance[tolerant]
\setbreakpoints[compound]
\setupindenting[yes,1em]
\setupfootnotes[way=bychapter,align={hz,hanging}]
\setupbodyfont[modern] % this is a stinky workaround to load lmodern
\setupbodyfont[libertine,11pt]

\setuppagenumbering[alternative=singlesided,location={footer,middle}]
\setupcaptions[width=fit,align={hz,hanging},number=no]

\startmode[a4imposed,a4imposedbc,letterimposed,letterimposedbc,a5imposed,%
  a5imposedbc,halfletterimposed,halfletterimposedbc]
  \setuppagenumbering[alternative=doublesided]
\stopmode

\setupbodyfontenvironment[default][em=italic]


\setupheads[%
  sectionnumber=no,number=no,
  align=flushleft,
  align={flushleft,nothyphenated,verytolerant,stretch},
  indentnext=yes,
  tolerance=verytolerant]

\definehead[awikipart][chapter]

\setuphead[awikipart]
          [%
            number=no,
            footer=empty,
            style=\bfd,
            before={\blank[force,2*big]},
            align={middle,nothyphenated,verytolerant,stretch},
            after={\page[yes]}
          ]

% h3
\setuphead[chapter]
          [style=\bfc]

\setuphead[title]
          [style=\bfc]


% h4
\setuphead[section]
          [style=\bfb]

% h5
\setuphead[subsection]
          [style=\bfa]

% h6
\setuphead[subsubsection]
          [style=bold]


\setuplist[awikipart]
          [alternative=b,
            interaction=all,
            width=0mm,
            distance=0mm,
            before={\blank[medium]},
            after={\blank[small]},
            style=\bfa,
            criterium=all]
\setuplist[chapter]
          [alternative=c,
            interaction=all,
            width=1mm,
            before={\blank[small]},
            style=bold,
            criterium=all]
\setuplist[section]
          [alternative=c,
            interaction=all,
            width=1mm,
            style=\tf,
            criterium=all]
\setuplist[subsection]
          [alternative=c,
            interaction=all,
            width=8mm,
            distance=0mm,
            style=\tf,
            criterium=all]
\setuplist[subsubsection]
          [alternative=c,
            interaction=all,
            width=15mm,
            style=\tf,
            criterium=all]


% center

\definestartstop
  [awikicenter]
  [before={\blank[line]\startalignment[middle]},
   after={\stopalignment\blank[line]}]

% right

\definestartstop
  [awikiright]
  [before={\blank[line]\startalignment[flushright]},
   after={\stopalignment\blank[line]}]


% blockquote

\definestartstop
  [blockquote]
  [before={\blank[big]
    \setupnarrower[middle=1em]
    \startnarrower
    \setupindenting[no]
    \setupwhitespace[medium]},
  after={\stopnarrower
    \blank[big]}]

% verse

\definestartstop
  [awikiverse]
  [before={\blank[big]
      \setupnarrower[middle=2em]
      \startnarrower
      \startlines},
    after={\stoplines
      \stopnarrower
      \blank[big]}]

\definestartstop
  [awikibiblio]
  [before={%
      \blank[big]
      \setupnarrower[left=1em]
      \startnarrower[left]
        \setupindenting[yes,-1em,first]},
    after={\stopnarrower
      \blank[big]}]
                
% same as above, but with no spacing around
\definestartstop
  [awikiplay]
  [before={%
      \setupnarrower[left=1em]
      \startnarrower[left]
        \setupindenting[yes,-1em,first]},
    after={\stopnarrower}]



% interaction
% we start the interaction only if it's not an imposed format.
\startnotmode[a4imposed,a4imposedbc,letterimposed,letterimposedbc,a5imposed,%
  a5imposedbc,halfletterimposed,halfletterimposedbc]
  \setupinteraction[state=start,color=black,contrastcolor=black,style=bold]
  \placebookmarks[awikipart,chapter,section,subsection,subsubsection][force=yes]
  \setupinteractionscreen[option=bookmark]
\stopnotmode



\setupexternalfigures[%
  maxwidth=\textwidth,
  maxheight=\textheight,
  factor=fit]

\setupitemgroup[itemize][each][packed][indenting=no]

\definemakeup[titlepage][pagestate=start,doublesided=no]

%%%%%%%%%%%%%%%%%%%%%%%%%%%%%%%%%%%%%%%%%%%%%%%%%%%%%%%%%%%%%%%%%%%%%%%%%%%%%%%%
%                                IMPOSER                                       %
%%%%%%%%%%%%%%%%%%%%%%%%%%%%%%%%%%%%%%%%%%%%%%%%%%%%%%%%%%%%%%%%%%%%%%%%%%%%%%%%

\startusercode

function optimize_signature(pages,min,max)
   local minsignature = min or 40
   local maxsignature = max or 80
   local originalpages = pages

   -- here we want to be sure that the max and min are actual *4
   if (minsignature%4) ~= 0 then
      global.texio.write_nl('term and log', "The minsig you provided is not a multiple of 4, rounding up")
      minsignature = minsignature + (4 - (minsignature % 4))
   end
   if (maxsignature%4) ~= 0 then
      global.texio.write_nl('term and log', "The maxsig you provided is not a multiple of 4, rounding up")
      maxsignature = maxsignature + (4 - (maxsignature % 4))
   end
   global.assert((minsignature % 4) == 0, "I suppose something is wrong, not a n*4")
   global.assert((maxsignature % 4) == 0, "I suppose something is wrong, not a n*4")

   --set needed pages to and and signature to 0
   local neededpages, signature = 0,0

   -- this means that we have to work with n*4, if not, add them to
   -- needed pages 
   local modulo = pages % 4
   if modulo==0 then
      signature=pages
   else
      neededpages = 4 - modulo
   end

   -- add the needed pages to pages
   pages = pages + neededpages
   
   if ((minsignature == 0) or (maxsignature == 0)) then 
      signature = pages -- the whole text
   else
      -- give a try with the signature
      signature = find_signature(pages, maxsignature)
      
      -- if the pages, are more than the max signature, find the right one
      if pages>maxsignature then
	 while signature<minsignature do
	    pages = pages + 4
	    neededpages = 4 + neededpages
	    signature = find_signature(pages, maxsignature)
	    --         global.texio.write_nl('term and log', "Trying signature of " .. signature)
	 end
      end
      global.texio.write_nl('term and log', "Parameters:: maxsignature=" .. maxsignature ..
		   " minsignature=" .. minsignature)

   end
   global.texio.write_nl('term and log', "ImposerMessage:: Original pages: " .. originalpages .. "; " .. 
	 "Signature is " .. signature .. ", " ..
	 neededpages .. " pages are needed, " .. 
	 pages ..  " of output")
   -- let's do it
   tex.print("\\dorecurse{" .. neededpages .. "}{\\page[empty]}")

end

function find_signature(number, maxsignature)
   global.assert(number>3, "I can't find the signature for" .. number .. "pages")
   global.assert((number % 4) == 0, "I suppose something is wrong, not a n*4")
   local i = maxsignature
   while i>0 do
      -- global.texio.write_nl('term and log', "Trying " .. i  .. "for max of " .. maxsignature)
      if (number % i) == 0 then
	 return i
      end
      i = i - 4
   end
end

\stopusercode

\define[1]\fillthesignature{
  \usercode{optimize_signature(#1, 40, 80)}}


\define\alibraryflushpages{
  \page[yes] % reset the page
  \fillthesignature{\the\realpageno}
}


% various papers 
\definepapersize[halfletter][width=5.5in,height=8.5in]
\definepapersize[halfafour][width=148.5mm,height=210mm]
\definepapersize[quarterletter][width=4.25in,height=5.5in]
\definepapersize[halfafive][width=105mm,height=148mm]
\definepapersize[generic][width=210mm,height=279.4mm]

%% this is the default ``paper'' which should work with both letter and a4

\setuppapersize[generic][generic]
\setuplayout[%
  backspace=42mm,
  topspace=31mm,% 176 / 15
  height=195mm,%130mm,
  footer=9mm, %
  header=0pt, % no header
  width=126mm] % 10.5 x 11

\startmode[libertine]
  \usetypescript[libertine]
  \setupbodyfont[libertine,11pt]
\stopmode

\startmode[pagella]
  \setupbodyfont[pagella,11pt]
\stopmode

\startmode[antykwa]
  \setupbodyfont[antykwa-poltawskiego,11pt]
\stopmode

\startmode[iwona]
  \setupbodyfont[iwona-medium,11pt]
\stopmode

\startmode[helvetica]
  \setupbodyfont[heros,11pt]
\stopmode

\startmode[century]
  \setupbodyfont[schola,11pt]
\stopmode

\startmode[modern]
  \setupbodyfont[modern,11pt]
\stopmode

\startmode[charis]
  \setupbodyfont[charis,11pt]
\stopmode        

\startmode[mini]
  \setuppapersize[S33][S33] % 176 × 176 mm
  \setuplayout[%
    backspace=20pt,
    topspace=15pt,% 176 / 15
    height=280pt,%130mm,
    footer=20pt, %
    header=0pt, % no header
    width=260pt] % 10.5 x 11
\stopmode

% for the plain A4 and letter, we use the classic LaTeX dimensions
% from the article class
\startmode[a4]
  \setuppapersize[A4][A4]
  \setuplayout[%
    backspace=42mm,
    topspace=45mm,
    height=218mm,
    footer=10mm,
    header=0pt, % no header
    width=126mm]
\stopmode

\startmode[letter]
  \setuppapersize[letter][letter]
  \setuplayout[%
    backspace=44mm,
    topspace=46mm,
    height=199mm,
    footer=10mm,
    header=0pt, % no header
    width=126mm]
\stopmode


% A4 imposed (A5), with no bc

\startmode[a4imposed]
% DIV=15 148 × 210: these are meant not to have binding correction,
  % but just to play safe, let's say 1mm => 147x210
  \setuppapersize[halfafour][halfafour]
  \setuplayout[%
    backspace=10.8mm, % 146/15 = 9.8 + 1
    topspace=14mm, % 210/15 =  14
    height=182mm, % 14 x 12 + 14 of the footer
    footer=14mm, %
    header=0pt, % no header
    width=117.6mm] % 9.8 x 12
\stopmode

% A4 imposed (A5), with bc
\startmode[a4imposedbc]
  \setuppapersize[halfafour][halfafour]
  \setuplayout[% 14 mm was a bit too near to the spine, using the glue binding
    backspace=17.3mm,  % 140/15 + 8 =
    topspace=14mm, % 210/15 =  14
    height=182mm, % 14 x 12 + 14 of the footer
    footer=14mm, %
    header=0pt, % no header
    width=112mm] % 9.333 x 12
\stopmode


\startmode[letterimposedbc] % 139.7mm x 215.9 mm
  \setuppapersize[halfletter][halfletter]
  % DIV=15 8mm binding corr, => 132 x 216
  \setuplayout[%
    backspace=16.8mm, % 8.8 + 8
    topspace=14.4mm, % 216/15 =  14.4
    height=187.2mm, % 15.4 x 11 + 15 of the footer
    footer=14.4mm, %
    header=0pt, % no header
    width=105.6mm] % 8.8 x 12
\stopmode

\startmode[letterimposed] % 139.7mm x 215.9 mm
  \setuppapersize[halfletter][halfletter]
  % DIV=15, 1mm binding correction. => 138.7x215.9
  \setuplayout[%
    backspace=10.3mm, % 9.24 + 1
    topspace=14.4mm, % 216/15 =  14.4
    height=187.2mm, % 15.4 x 11 + 15 of the footer
    footer=14.4mm, %
    header=0pt, % no header
    width=111mm] % 9.24 x 12
\stopmode

%%% new formats for mini books
%%% \definepapersize[halfafive][width=105mm,height=148mm]

\startmode[a5imposed]
% DIV=12 105x148 : these are meant not to have binding correction,
  % but just to play safe, let's say 1mm => 104x148
  \setuppapersize[halfafive][halfafive]
  \setuplayout[%
    backspace=9.6mm,
    topspace=12.3mm,
    height=123.5mm, % 14 x 12 + 14 of the footer
    footer=12.3mm, %
    header=0pt, % no header
    width=78.8mm] % 9.8 x 12
\stopmode

% A5 imposed (A6), with bc
\startmode[a5imposedbc]
% DIV=12 105x148 : with binding correction,
  % let's say 8mm => 96x148
  \setuppapersize[halfafive][halfafive]
  \setuplayout[%
    backspace=16mm,
    topspace=12.3mm,
    height=123.5mm, % 14 x 12 + 14 of the footer
    footer=12.3mm, %
    header=0pt, % no header
    width=72mm] % 9.8 x 12
\stopmode

%%% \definepapersize[quarterletter][width=4.25in,height=5.5in]

% DIV=12 width=4.25in (108mm),height=5.5in (140mm) 
\startmode[halfletterimposed] % 107x140
  \setuppapersize[quarterletter][quarterletter]
  \setuplayout[%
    backspace=10mm,
    topspace=11.6mm,
    height=116mm,
    footer=11.6mm,
    header=0pt, % no header
    width=80mm] % 9.24 x 12
\stopmode

\startmode[halfletterimposedbc]
  \setuppapersize[quarterletter][quarterletter]
  \setuplayout[%
    backspace=15.4mm,
    topspace=11.6mm,
    height=116mm,
    footer=11.6mm,
    header=0pt, % no header
    width=76mm] % 9.24 x 12
\stopmode

\startmode[quickimpose]
  \setuppapersize[A5][A4,landscape]
  \setuparranging[2UP]
  \setuppagenumbering[alternative=doublesided]
  \setuplayout[% 14 mm was a bit too near to the spine, using the glue binding
    backspace=17.3mm,  % 140/15 + 8 =
    topspace=14mm, % 210/15 =  14
    height=182mm, % 14 x 12 + 14 of the footer
    footer=14mm, %
    header=0pt, % no header
    width=112mm] % 9.333 x 12
\stopmode

\startmode[tenpt]
  \setupbodyfont[10pt]
\stopmode

\startmode[twelvept]
  \setupbodyfont[12pt]
\stopmode

%%%%%%%%%%%%%%%%%%%%%%%%%%%%%%%%%%%%%%%%%%%%%%%%%%%%%%%%%%%%%%%%%%%%%%%%%%%%%%%%
%                            DOCUMENT BEGINS                                   %
%%%%%%%%%%%%%%%%%%%%%%%%%%%%%%%%%%%%%%%%%%%%%%%%%%%%%%%%%%%%%%%%%%%%%%%%%%%%%%%%


\mainlanguage[en]


\starttext

\starttitlepagemakeup
  \startalignment[middle,nothanging,nothyphenated,stretch]


  \switchtobodyfont[18pt] % author
  {\bf \em

Workers’ Solidarity Federation  \par}
  \blank[2*big]
  \switchtobodyfont[24pt] % title
  {\bf

Why Class Struggle and Revolution From Below? and Why Do We Oppose Capitalism and the State?

\par}
  \blank[big]
  \switchtobodyfont[20pt] % subtitle
  {\bf 

\par}
  \vfill
  \stopalignment
  \startalignment[middle,bottom,nothyphenated,stretch,nothanging]
  \switchtobodyfont[global]

  \stopalignment
\stoptitlepagemakeup



\title{Contents}

\placelist[awikipart,chapter,section,subsection]



\page[yes,right]

\chapter{Why Class Struggle and Revolution From Below?
}

Capitalism and the State are based on the exploitation of the many by the few — the ruling class. These structures are also the main cause of special oppressions like racism. Capitalism, the State and all forms of oppression must be fought and ultimately replaced by a democratic stateless socialist (anarcho-syndicalist) system. This task can only be successfully accomplished by the working class, the poor and the working peasantry.


\section{What are Classes?
}

Society is divided into different classes. By “class”, we mean a group of people with a common relationship to the structures of political and economic power.


The biggest class in South Africa is the working class. It consists of manual and white collar workers and their families, farm workers, service sector workers, labour tenants, the unemployed poor, rank-and-file soldiers, and the marginalised youth. All these people lack political and economic power. They do not own the means of life — factories, land, offices etc. They are therefore dependent on wages earned by working for the bosses to survive (either working themselves or being supported by a family member).


The other key class in society is the ruling class. It is made up of the bosses (large capitalists), top military officials like generals, and top government officials (like directors and professional politicians). This class owns all the land and factories. It makes profits by exploiting and oppressing the workers and poor. This is called capitalism. It also has political power in that it controls the State (army, police, government departments, parliament). The State acts to defend this class.


There is also a third class — the middle class. This is not everyone who gets a good wage! It means those who stand between the two key classes: small capitalists who only employ a few workers; middle level management; and professionals (like lawyers). This class is caught in the middle of the class struggle and is divided over which key class it supports. But its privileged position tends to make it conservative.


Finally, in some countries there is also a class called the working peasantry.  This refers to small family farmers who do not employ wage labour. This class is the natural ally of the working class and has the same enemies — the ruling class.


\section{Why Revolution?
}

Both capitalism and the State are tools of the ruling class. They are anti-democratic structures that out the needs of the few ahead of the needs of the many. They are the key cause of racism — racism was created to divide workers, justify genocide, slavery, colonialism and apartheid-capitalism. Racism can only be finally defeated with destruction of capitalism and the State. Capitalism and the state cannot be reformed — they must be abolished.


\section{Class Struggle?
}

We believe that the only people who can achieve these tasks are the working class and the working peasants. Why?


Class struggle is central because it is the struggle of the majority — the working and poor people.


The ruling class and the working class have totally opposite interests. The rich get richer at the expense of the poor; every gain the working class is at the expense of the ruling class.  This leads to a struggle between the two key classes which we call the “class struggle”.  Only the working class and peasantry can create a free society because only these classes are not based on exploitation.


The majority of people are working class and therefore have a direct interest in opposing capitalism and the State. That is why most people are actually engaged in class struggle, through actions like strikes, squatting unofficially, land invasions, rent boycotts, getting back at the boss etc.


\section{Factories
}

The class struggle is central because it is as workers that we have the “social power” to fight back — we produce all wealth and goods under capitalism, which means that we have the ability to hit the bosses where it hurts — in the pocket — by workplace actions.


The class struggle is central because our experience under capitalism forces us to organise as a class. Workers are often concentrated in large factories and face similar conditions — this makes it easy and vital to organise mass struggle.  fighting structures like trade unions, Large working class neighbourhoods and schools have a similar effect.


\section{Community Action
}

Class struggle is not just about factory resistance, although this is essential — it is about every action by working and poor people to resist the bosses and rulers.


For example, in the community, workers pay rent to local government. This local government is thus just like any other landlord — wanting the most rent for the least services.  To resist it is an act of class struggle. For the poor to occupy land held by the rich is a class struggle.


This does not mean that we are in favour of the “whole community” (everyone from workers and the unemployed to local business-men and “leaders”) being mobilised. We are opposed to any attempt to bring bosses and elites into our struggles. Poor people must do it for themselves — to build an alliance with the rich and powerful means to sacrifice the struggle against the rich and powerful to confusion and treachery.


As for the middle class, we recognise that it will split before and during revolution. They can join the struggle — not as leaders and intellectuals giving orders, but as comrades who come to repay a debt to the masses, and honestly rejecting the privileges of their class.


\section{Rich Scum
}

It is therefore in the interests of the working class to resist the current system. By contrast, the ruling class fights to defend capitalism and the State, and is thus guilty of all products of this system. The power and wealth of the few depends on the poverty, powerlessness and oppression of the many. So it is clear that these bosses will never act to progressively change society — they will agree to reforms if forced to do so; revolutionary change will be opposed tooth and nail. It is therefore rubbish to talk about the “progressive” bourgeoisie or to put our hopes in any famous member of the ruling class — they are all rich scum.


\section{Fighting Oppression
}

lass struggle is central because it is the key to defeating special oppressions such as racism and sexism.  By special oppression, we mean forms of oppression other than class exploitation.


To defeat our enemy we must understand it. Oppression does not fall from the sky — it is rooted in capitalism and the State — it is a tool to divide the working masses and super-exploit less socially powerful sections of the masses (like women and Blacks).


This means that oppression can therefore only be defeated by a struggle against capitalism and the State. Only a stateless socialist society can end the causes of racism and provide redistribution to the victims of racism by putting human need before profit.


Only the working class can win this struggle against the ruling class because only it has the numbers, power and interest to destroy capitalism and the State.


Rich Blacks may not like racism, but they do like capitalism and the State and will therefore fight to defend these structures. Strange as it may seem, they therefore defend the causes of racism.


Perhaps it is because your class affects your experience of oppression — the Black elite can go to fancy schools and live in nice houses, the Black working class cannot. The majority of people affected by special oppression, and those who in fact are affected the worst, are working/ poor people, with class grievances against the system.


\section{One Struggle
}

The fight against racism and other oppression is thus not something separate to the class struggle — it is central to the struggle of the working class. As mentioned, the struggle against racism requires a class struggle against capitalism and the State.


And successful class struggle requires maximum unity amongst the working class.  So it requires overcoming the divisions created by the bosses. This includes racial, sexual and national divisions. Special oppression like racism is against the interests of all workers as it divides and weakens the workers struggle, resulting in worse conditions for all workers. Class unity is in the interests of all doubly oppressed groups as it provides the unity and solidarity required to win gains and defeat the root cause of racism — capitalism and the State.


Successful class struggle requires the mobilisation of as many people and strata of the working class as possible. This requires a programme that mobilises people around all issues that affect all sectors of the working class: racism, wages, rents etc. In other words, the class struggle can only succeed through mobilising and uniting the working class on the basis of opposition to all forms of oppression and exploitation. In other words, the struggle against capitalism and the State requires a struggle against racism and all oppression, and the struggle against racism and all oppression requires a class struggle against capitalism and the State.


\chapter{Why Do We Oppose Capitalism and the State?
}

We oppose capitalism and the State because they are tools used by the ruling class of bosses, military leaders and top government officials to exploit and rule the rest of us — the working class, the poor and the working peasantry.  These structures will never create freedom, they must go.


\section{What is Capitalism?
}

We live in a capitalist society. Capitalism is an economic system in which bosses compete with each other to make profits.


Capitalism is usually based around the wage system. The bosses and top government officials own and control the means of life: the land, factories and so on. The rest of us — the working class — don’t. Therefore we have to work for the bosses for a wage and/or pay them rents, taxes etc. This is the situation in South Africa.


Sometimes capitalism operates slightly differently. It takes over, or even re-established pre-capitalist forms of labour control. For example, establishing slave plantations in the USA before the 1860s, or turning African farmers into peasants growing cash crops for the market. All of these labour systems are dominated by the broader capitalist system — profit \& money. They are linked to it through trade or labour supply.


\section{We Hate Capitalism
}

We are fundamentally opposed to the capitalist system:



\startitemize[1]\relax
\item[] capitalism is based on exploitation. The wages we are paid are always less than the value of the goods we make. The surplus produced goes to the bosses and rulers as profit. We do the work, they take the credit. This results in poverty as the rich get rich by stealing from the poor.




 \item[] capitalism is authoritarian and undemocratic. At the workplace, the bosses make all the decisions in their own interests; the worker majority has no real say.  Poverty, long hours etc. all act to make it difficult for the workers and the poor to be active in politics.




 \item[] capitalism puts profit before people. Capitalists have goods made so they can make a profit so goods only go to people with money. This means that although there is definitely enough food to feed everyone, many starve because the they do not have money.




 \item[] capitalism is inefficient and wasteful. The bosses do not plan production to meet needs. Instead, they have goods made and hope that they can sell them. If unsold, they are thrown away.




 \item[] capitalism promotes ruling class values like greed, aggression and a thirst for power, instead of values like mutual aid and solidarity.




 \item[] capitalism is a primary cause of oppression, racism and environmental destruction. Racism was created to divide workers, justify genocide, slavery, colonialism and the super-exploitation of Black workers under apartheid-capitalism.




 
\stopitemize
\section{What is the State?
}

By the State we mean the army, police, government departments and parliament. It is a centralised, heretical (top-down) bureaucratic structure of rule over a country.


\section{Abolish the State?
}

The State is a direct result of the fact that we live in an unequal society. In South Africa, 5\% of the population own 85\% of all wealth, 120,000 capitalist farmers control all the land in the historically “White” areas, and 5 companies own 80\% of all shares on the Stock exchange.


The State exists to defend the ruling class minority at the top of this system, if not by persuasion then by force. This is true of all States, whether or not they call themselves capitalist, socialist or democratic. Wherever you have a State, you have the rule of a privileged few over the exploited many. The State protects capitalism and the all the veils that accompany it. When workers strike, they are attacked by police — who ever heard of a boss being arrested for paying low wages?


The State is built to allow this minority to rule over the working and poor majority. It has a top-down structure that concentrates power in the hands of the few at the top. There is no way that ordinary people can participate in the running of this apparatus. Its structure makes it inevitable that all States lead to the few ruling the many.


This means that real change cannot come through elections. Instead, we can only win reforms through mass struggle against capitalism and the State. We must defend all political freedoms, but at the same time boycott elections and rely on people power and mass organisation to win. This is the way to win gains like better pay, and also the way to defeat both capitalism, the State and all oppression — and establish a free stateless socialist society (anarcho-syndicalism).









\page[yes]

%%%% backcover

\startmode[a4imposed,a4imposedbc,letterimposed,letterimposedbc,a5imposed,%
  a5imposedbc,halfletterimposed,halfletterimposedbc,quickimpose]
\alibraryflushpages
\stopmode

\page[blank]

\startalignment[middle]
{\tfa The Anarchist Library
\blank[small]
Anti-Copyright}
\blank[small]
\currentdate
\stopalignment

\blank[big]
\framed[frame=off,location=middle,width=\textwidth]
       {\externalfigure[logo][width=0.25\textwidth]}



\vfill
\setupindenting[no]
\setsmallbodyfont

\startalignment[middle,nothyphenated,nothanging,stretch]

\blank[line]
% \framed[frame=off,location=middle,width=\textwidth]
%       {\externalfigure[logo][width=0.25\textwidth]}


Workers’ Solidarity Federation



Why Class Struggle and Revolution From Below? and Why Do We Oppose Capitalism and the State?







\stopalignment
\blank[line]

\startalignment[hyphenated,middle]




Retrieved on January 1, 2005 from \goto{www.cat.org.au}[url(http://www.cat.org.au/aprop/whyclass.txt)]


\stopalignment

\stoptext


