% -*- mode: tex -*-
%%%%%%%%%%%%%%%%%%%%%%%%%%%%%%%%%%%%%%%%%%%%%%%%%%%%%%%%%%%%%%%%%%%%%%%%%%%%%%%%
%                                STANDARD                                      %
%%%%%%%%%%%%%%%%%%%%%%%%%%%%%%%%%%%%%%%%%%%%%%%%%%%%%%%%%%%%%%%%%%%%%%%%%%%%%%%%
\definefontfeature[default][default]
                  [protrusion=quality,
                    expansion=quality,
                    script=latn]
\setupalign[hz,hanging]
\setuptolerance[tolerant]
\setbreakpoints[compound]
\setupindenting[yes,1em]
\setupfootnotes[way=bychapter,align={hz,hanging}]
\setupbodyfont[modern] % this is a stinky workaround to load lmodern
\setupbodyfont[libertine,11pt]

\setuppagenumbering[alternative=singlesided,location={footer,middle}]
\setupcaptions[width=fit,align={hz,hanging},number=no]

\startmode[a4imposed,a4imposedbc,letterimposed,letterimposedbc,a5imposed,%
  a5imposedbc,halfletterimposed,halfletterimposedbc]
  \setuppagenumbering[alternative=doublesided]
\stopmode

\setupbodyfontenvironment[default][em=italic]


\setupheads[%
  sectionnumber=no,number=no,
  align=flushleft,
  align={flushleft,nothyphenated,verytolerant,stretch},
  indentnext=yes,
  tolerance=verytolerant]

\definehead[awikipart][chapter]

\setuphead[awikipart]
          [%
            number=no,
            footer=empty,
            style=\bfd,
            before={\blank[force,2*big]},
            align={middle,nothyphenated,verytolerant,stretch},
            after={\page[yes]}
          ]

% h3
\setuphead[chapter]
          [style=\bfc]

\setuphead[title]
          [style=\bfc]


% h4
\setuphead[section]
          [style=\bfb]

% h5
\setuphead[subsection]
          [style=\bfa]

% h6
\setuphead[subsubsection]
          [style=bold]


\setuplist[awikipart]
          [alternative=b,
            interaction=all,
            width=0mm,
            distance=0mm,
            before={\blank[medium]},
            after={\blank[small]},
            style=\bfa,
            criterium=all]
\setuplist[chapter]
          [alternative=c,
            interaction=all,
            width=1mm,
            before={\blank[small]},
            style=bold,
            criterium=all]
\setuplist[section]
          [alternative=c,
            interaction=all,
            width=1mm,
            style=\tf,
            criterium=all]
\setuplist[subsection]
          [alternative=c,
            interaction=all,
            width=8mm,
            distance=0mm,
            style=\tf,
            criterium=all]
\setuplist[subsubsection]
          [alternative=c,
            interaction=all,
            width=15mm,
            style=\tf,
            criterium=all]


% center

\definestartstop
  [awikicenter]
  [before={\blank[line]\startalignment[middle]},
   after={\stopalignment\blank[line]}]

% right

\definestartstop
  [awikiright]
  [before={\blank[line]\startalignment[flushright]},
   after={\stopalignment\blank[line]}]


% blockquote

\definestartstop
  [blockquote]
  [before={\blank[big]
    \setupnarrower[middle=1em]
    \startnarrower
    \setupindenting[no]
    \setupwhitespace[medium]},
  after={\stopnarrower
    \blank[big]}]

% verse

\definestartstop
  [awikiverse]
  [before={\blank[big]
      \setupnarrower[middle=2em]
      \startnarrower
      \startlines},
    after={\stoplines
      \stopnarrower
      \blank[big]}]

\definestartstop
  [awikibiblio]
  [before={%
      \blank[big]
      \setupnarrower[left=1em]
      \startnarrower[left]
        \setupindenting[yes,-1em,first]},
    after={\stopnarrower
      \blank[big]}]
                
% same as above, but with no spacing around
\definestartstop
  [awikiplay]
  [before={%
      \setupnarrower[left=1em]
      \startnarrower[left]
        \setupindenting[yes,-1em,first]},
    after={\stopnarrower}]



% interaction
% we start the interaction only if it's not an imposed format.
\startnotmode[a4imposed,a4imposedbc,letterimposed,letterimposedbc,a5imposed,%
  a5imposedbc,halfletterimposed,halfletterimposedbc]
  \setupinteraction[state=start,color=black,contrastcolor=black,style=bold]
  \placebookmarks[awikipart,chapter,section,subsection,subsubsection][force=yes]
  \setupinteractionscreen[option=bookmark]
\stopnotmode



\setupexternalfigures[%
  maxwidth=\textwidth,
  maxheight=\textheight,
  factor=fit]

\setupitemgroup[itemize][each][packed][indenting=no]

\definemakeup[titlepage][pagestate=start,doublesided=no]

%%%%%%%%%%%%%%%%%%%%%%%%%%%%%%%%%%%%%%%%%%%%%%%%%%%%%%%%%%%%%%%%%%%%%%%%%%%%%%%%
%                                IMPOSER                                       %
%%%%%%%%%%%%%%%%%%%%%%%%%%%%%%%%%%%%%%%%%%%%%%%%%%%%%%%%%%%%%%%%%%%%%%%%%%%%%%%%

\startusercode

function optimize_signature(pages,min,max)
   local minsignature = min or 40
   local maxsignature = max or 80
   local originalpages = pages

   -- here we want to be sure that the max and min are actual *4
   if (minsignature%4) ~= 0 then
      global.texio.write_nl('term and log', "The minsig you provided is not a multiple of 4, rounding up")
      minsignature = minsignature + (4 - (minsignature % 4))
   end
   if (maxsignature%4) ~= 0 then
      global.texio.write_nl('term and log', "The maxsig you provided is not a multiple of 4, rounding up")
      maxsignature = maxsignature + (4 - (maxsignature % 4))
   end
   global.assert((minsignature % 4) == 0, "I suppose something is wrong, not a n*4")
   global.assert((maxsignature % 4) == 0, "I suppose something is wrong, not a n*4")

   --set needed pages to and and signature to 0
   local neededpages, signature = 0,0

   -- this means that we have to work with n*4, if not, add them to
   -- needed pages 
   local modulo = pages % 4
   if modulo==0 then
      signature=pages
   else
      neededpages = 4 - modulo
   end

   -- add the needed pages to pages
   pages = pages + neededpages
   
   if ((minsignature == 0) or (maxsignature == 0)) then 
      signature = pages -- the whole text
   else
      -- give a try with the signature
      signature = find_signature(pages, maxsignature)
      
      -- if the pages, are more than the max signature, find the right one
      if pages>maxsignature then
	 while signature<minsignature do
	    pages = pages + 4
	    neededpages = 4 + neededpages
	    signature = find_signature(pages, maxsignature)
	    --         global.texio.write_nl('term and log', "Trying signature of " .. signature)
	 end
      end
      global.texio.write_nl('term and log', "Parameters:: maxsignature=" .. maxsignature ..
		   " minsignature=" .. minsignature)

   end
   global.texio.write_nl('term and log', "ImposerMessage:: Original pages: " .. originalpages .. "; " .. 
	 "Signature is " .. signature .. ", " ..
	 neededpages .. " pages are needed, " .. 
	 pages ..  " of output")
   -- let's do it
   tex.print("\\dorecurse{" .. neededpages .. "}{\\page[empty]}")

end

function find_signature(number, maxsignature)
   global.assert(number>3, "I can't find the signature for" .. number .. "pages")
   global.assert((number % 4) == 0, "I suppose something is wrong, not a n*4")
   local i = maxsignature
   while i>0 do
      -- global.texio.write_nl('term and log', "Trying " .. i  .. "for max of " .. maxsignature)
      if (number % i) == 0 then
	 return i
      end
      i = i - 4
   end
end

\stopusercode

\define[1]\fillthesignature{
  \usercode{optimize_signature(#1, 40, 80)}}


\define\alibraryflushpages{
  \page[yes] % reset the page
  \fillthesignature{\the\realpageno}
}


% various papers 
\definepapersize[halfletter][width=5.5in,height=8.5in]
\definepapersize[halfafour][width=148.5mm,height=210mm]
\definepapersize[quarterletter][width=4.25in,height=5.5in]
\definepapersize[halfafive][width=105mm,height=148mm]
\definepapersize[generic][width=210mm,height=279.4mm]

%% this is the default ``paper'' which should work with both letter and a4

\setuppapersize[generic][generic]
\setuplayout[%
  backspace=42mm,
  topspace=31mm,% 176 / 15
  height=195mm,%130mm,
  footer=9mm, %
  header=0pt, % no header
  width=126mm] % 10.5 x 11

\startmode[libertine]
  \usetypescript[libertine]
  \setupbodyfont[libertine,11pt]
\stopmode

\startmode[pagella]
  \setupbodyfont[pagella,11pt]
\stopmode

\startmode[antykwa]
  \setupbodyfont[antykwa-poltawskiego,11pt]
\stopmode

\startmode[iwona]
  \setupbodyfont[iwona-medium,11pt]
\stopmode

\startmode[helvetica]
  \setupbodyfont[heros,11pt]
\stopmode

\startmode[century]
  \setupbodyfont[schola,11pt]
\stopmode

\startmode[modern]
  \setupbodyfont[modern,11pt]
\stopmode

\startmode[charis]
  \setupbodyfont[charis,11pt]
\stopmode        

\startmode[mini]
  \setuppapersize[S33][S33] % 176 × 176 mm
  \setuplayout[%
    backspace=20pt,
    topspace=15pt,% 176 / 15
    height=280pt,%130mm,
    footer=20pt, %
    header=0pt, % no header
    width=260pt] % 10.5 x 11
\stopmode

% for the plain A4 and letter, we use the classic LaTeX dimensions
% from the article class
\startmode[a4]
  \setuppapersize[A4][A4]
  \setuplayout[%
    backspace=42mm,
    topspace=45mm,
    height=218mm,
    footer=10mm,
    header=0pt, % no header
    width=126mm]
\stopmode

\startmode[letter]
  \setuppapersize[letter][letter]
  \setuplayout[%
    backspace=44mm,
    topspace=46mm,
    height=199mm,
    footer=10mm,
    header=0pt, % no header
    width=126mm]
\stopmode


% A4 imposed (A5), with no bc

\startmode[a4imposed]
% DIV=15 148 × 210: these are meant not to have binding correction,
  % but just to play safe, let's say 1mm => 147x210
  \setuppapersize[halfafour][halfafour]
  \setuplayout[%
    backspace=10.8mm, % 146/15 = 9.8 + 1
    topspace=14mm, % 210/15 =  14
    height=182mm, % 14 x 12 + 14 of the footer
    footer=14mm, %
    header=0pt, % no header
    width=117.6mm] % 9.8 x 12
\stopmode

% A4 imposed (A5), with bc
\startmode[a4imposedbc]
  \setuppapersize[halfafour][halfafour]
  \setuplayout[% 14 mm was a bit too near to the spine, using the glue binding
    backspace=17.3mm,  % 140/15 + 8 =
    topspace=14mm, % 210/15 =  14
    height=182mm, % 14 x 12 + 14 of the footer
    footer=14mm, %
    header=0pt, % no header
    width=112mm] % 9.333 x 12
\stopmode


\startmode[letterimposedbc] % 139.7mm x 215.9 mm
  \setuppapersize[halfletter][halfletter]
  % DIV=15 8mm binding corr, => 132 x 216
  \setuplayout[%
    backspace=16.8mm, % 8.8 + 8
    topspace=14.4mm, % 216/15 =  14.4
    height=187.2mm, % 15.4 x 11 + 15 of the footer
    footer=14.4mm, %
    header=0pt, % no header
    width=105.6mm] % 8.8 x 12
\stopmode

\startmode[letterimposed] % 139.7mm x 215.9 mm
  \setuppapersize[halfletter][halfletter]
  % DIV=15, 1mm binding correction. => 138.7x215.9
  \setuplayout[%
    backspace=10.3mm, % 9.24 + 1
    topspace=14.4mm, % 216/15 =  14.4
    height=187.2mm, % 15.4 x 11 + 15 of the footer
    footer=14.4mm, %
    header=0pt, % no header
    width=111mm] % 9.24 x 12
\stopmode

%%% new formats for mini books
%%% \definepapersize[halfafive][width=105mm,height=148mm]

\startmode[a5imposed]
% DIV=12 105x148 : these are meant not to have binding correction,
  % but just to play safe, let's say 1mm => 104x148
  \setuppapersize[halfafive][halfafive]
  \setuplayout[%
    backspace=9.6mm,
    topspace=12.3mm,
    height=123.5mm, % 14 x 12 + 14 of the footer
    footer=12.3mm, %
    header=0pt, % no header
    width=78.8mm] % 9.8 x 12
\stopmode

% A5 imposed (A6), with bc
\startmode[a5imposedbc]
% DIV=12 105x148 : with binding correction,
  % let's say 8mm => 96x148
  \setuppapersize[halfafive][halfafive]
  \setuplayout[%
    backspace=16mm,
    topspace=12.3mm,
    height=123.5mm, % 14 x 12 + 14 of the footer
    footer=12.3mm, %
    header=0pt, % no header
    width=72mm] % 9.8 x 12
\stopmode

%%% \definepapersize[quarterletter][width=4.25in,height=5.5in]

% DIV=12 width=4.25in (108mm),height=5.5in (140mm) 
\startmode[halfletterimposed] % 107x140
  \setuppapersize[quarterletter][quarterletter]
  \setuplayout[%
    backspace=10mm,
    topspace=11.6mm,
    height=116mm,
    footer=11.6mm,
    header=0pt, % no header
    width=80mm] % 9.24 x 12
\stopmode

\startmode[halfletterimposedbc]
  \setuppapersize[quarterletter][quarterletter]
  \setuplayout[%
    backspace=15.4mm,
    topspace=11.6mm,
    height=116mm,
    footer=11.6mm,
    header=0pt, % no header
    width=76mm] % 9.24 x 12
\stopmode

\startmode[quickimpose]
  \setuppapersize[A5][A4,landscape]
  \setuparranging[2UP]
  \setuppagenumbering[alternative=doublesided]
  \setuplayout[% 14 mm was a bit too near to the spine, using the glue binding
    backspace=17.3mm,  % 140/15 + 8 =
    topspace=14mm, % 210/15 =  14
    height=182mm, % 14 x 12 + 14 of the footer
    footer=14mm, %
    header=0pt, % no header
    width=112mm] % 9.333 x 12
\stopmode

\startmode[tenpt]
  \setupbodyfont[10pt]
\stopmode

\startmode[twelvept]
  \setupbodyfont[12pt]
\stopmode

%%%%%%%%%%%%%%%%%%%%%%%%%%%%%%%%%%%%%%%%%%%%%%%%%%%%%%%%%%%%%%%%%%%%%%%%%%%%%%%%
%                            DOCUMENT BEGINS                                   %
%%%%%%%%%%%%%%%%%%%%%%%%%%%%%%%%%%%%%%%%%%%%%%%%%%%%%%%%%%%%%%%%%%%%%%%%%%%%%%%%


\mainlanguage[en]


\starttext

\starttitlepagemakeup
  \startalignment[middle,nothanging,nothyphenated,stretch]


  \switchtobodyfont[18pt] % author
  {\bf \em

Margaret Killjoy  \par}
  \blank[2*big]
  \switchtobodyfont[24pt] % title
  {\bf

Anarchism Versus Civilization

\par}
  \blank[big]
  \switchtobodyfont[20pt] % subtitle
  {\bf 

\par}
  \vfill
  \stopalignment
  \startalignment[middle,bottom,nothyphenated,stretch,nothanging]
  \switchtobodyfont[global]

2010

  \stopalignment
\stoptitlepagemakeup



\page[yes,right]

In his 2003 polemic {\em Anarchism versus Primitivism}, Brian Oliver Sheppard makes the case that primitivism is inherently in contradiction with anarchism.


Much can be inferred from his tone, which is openly mocking. He makes references to how “[u]nfortunately for anarchists, plunging into the primitivist miasma has become necessary,” openly condescending to engage the primitivists at all. But his arguments are mired in absurdities: he mocks primitivists as hypocrites for engaging in technological practices while ignoring the fact that nearly every anarchist of any stripe in capitalist and statist society is not living as she or he preaches.


The core of his argument is that primitivism is authoritarian and therefore irreconcilable with anarchism. But the anarchism he promotes is rather clearly a simplistic and “classical” one, a red anarchism that argues for worker control of a stateless society. He argues that primitivists are stuck in an illusory past that cannot be supported by evidence, yet never acknowledges his complicity in the same behavior; here is a man arguing that anarchism has always been about worker control and communistic ideas, completely ignoring the heterogeneous past and present of anarchism. The individualists, the anarchists-without-adjectives, the mutualists\unknown{} these people simply never existed, if one is to infer from Brian’s\footnote{It is, of course, the norm to refer to a writer by their last name rather than their first name. This applies much more often to men than women; compare Kropotkin and Bakunin with Voltairine DeCleyre and Emma Goldman.} piece.


Well-reasoned critiques of primitivism exist, but they are rarely distributed. Instead, self-defeating and remarkably sectarian missives are the norm. But this basic idea, that anarcho-primitivism is no more anarchist than the largely dismissed ideas of “anarcho”-capitalists and “anarcho”-nationalists, is a curious one.


For the sake of argument, I make the opposite case: anarchism is and always has been anti-civilization, and that civilization and anarchism are completely irreconcilable. Anyone who claims to be for civilization and anarchism both is deluding themselves.\footnote{Or simply use different semantic set and tend to define things differently than I, or this article, do.}


An anthropologist named Elman Service\footnote{Elman Service, by the way, for some red-anarcho cred, was an American volunteer in the Abraham Lincoln Brigade of the Spanish Civil War, fighting against Franco and the fascists.} suggested a widely-used system of classification for human cultures that contains four rough categories. Firstly, there are gatherer-hunter bands, which are generally egalitarian; secondly there are tribal societies that are larger, slightly more formal, and have bits of social ranking; third are chiefdoms, which continue down the path of social stratification; and finally there are civilizations, which are anthropologically understood by their complex social hierarchies and organized, institutional governments.


The rejection of complex social hierarchies and government means, therefore, the rejection of civilization. If an anarchist society were to develop, it would be by definition a non-civilized society.


Sure, an argument can be made that “classical” anarchists\footnote{The word “classical” is getting the quotes treatment in this article because I personally disapprove of this oversimplification of “what anarchists have always wanted” that is presented to us by Brian Sheppard as much as I disapprove of the oversimplification of what “primitive people were like” that indeed many primitivists are guilty of.} are in opposition to the concept of the State rather than the idea of government per say, but the overwhelming majority of contemporary anarchist thought and dialogue speaks to the rejection of government as something that is inherently tied to the stateform.


So an anarchist society would necessitate either a return to the gatherer-hunter bands or it would — and I consider this option much more likely and much preferable, personally — mean developing something entirely new. I would personally like to call it the post-civilization, but I don’t believe we {\em need} to call it that. We simply need to understand it as anarchism.


Elman understood his four-part typology to be illustrative of a linear loss of autonomy. In a band, an individual had liberty. In a civilization, an individual ceded or lost this liberty. Now, Elman was an integration theorist; he believed that citizens in early civilizations gave up their autonomy willingly — in essence, that they signed the social contract, ceding their liberty so as to allow for a more complex society. The opposing theory is conflict theory: that states have, from the beginning, sought to consolidate power into the hands of the few for the benefit of those few.


But no one is arguing that the development from band to civilization hasn’t resulted in hierarchy and a lack of autonomy. This has, historically, been quite simple and linear: the further a society “advances” along these lines towards civilization, the more that liberty has waned.


Anarchism argues for a classless, egalitarian society devoid of coercive authority and therefore argues — and always has — against some of the primary, distinguishing traits that define civilization. To argue in favor of civilization is as absurd as to argue in favor of the state.


Very few modern anarchists would argue against anarcha-feminism. Anarcha-feminism is not understood as a separate thing, alien to anarchism as a whole, but rather as an essential component to the struggle against domination. It is generally understood that there are those who identify more strongly with anarcha-feminism than others. There are those who use it as their personal lens with which to address the world, who lay down important theory and practical organizing to address and overcome patriarchy.


And this, I would argue, is the role of the anti-civilized, the anarcho-primitivists. Anti-civilization thought has greatly deepened our understandings of oppression, with its critique of the division of labor and of linear concepts of progress.


It is as much of a mistake to reject all anarcho-primitivists as genocidal hypocrites as it is to reject all communist anarchists as technophiles who want the enslavement of nature in service of the almighty Worker\footnote{Of course, it would be easier for me to not make this mistake myself if I didn’t personally know more than few people who fit these rude stereotypes\unknown{}}.


Patriarchy, government, capitalism, nationalism, racism, civilization\unknown{} none of it has a place in the society we envision. And more importantly, none of it has a place in our struggles, here and now.


 









\page[yes]

%%%% backcover

\startmode[a4imposed,a4imposedbc,letterimposed,letterimposedbc,a5imposed,%
  a5imposedbc,halfletterimposed,halfletterimposedbc,quickimpose]
\alibraryflushpages
\stopmode

\page[blank]

\startalignment[middle]
{\tfa The Anarchist Library
\blank[small]
Anti-Copyright}
\blank[small]
\currentdate
\stopalignment

\blank[big]
\framed[frame=off,location=middle,width=\textwidth]
       {\externalfigure[logo][width=0.25\textwidth]}



\vfill
\setupindenting[no]
\setsmallbodyfont

\startalignment[middle,nothyphenated,nothanging,stretch]

\blank[line]
% \framed[frame=off,location=middle,width=\textwidth]
%       {\externalfigure[logo][width=0.25\textwidth]}


Margaret Killjoy



Anarchism Versus Civilization






2010


\stopalignment
\blank[line]

\startalignment[hyphenated,middle]




Retrieved on 24 August 2010 from \goto{www.postcivilized.net}[url(http://www.postcivilized.net/2010/08/anarchism-versus-civilization/)]


\stopalignment

\stoptext


