% -*- mode: tex -*-
%%%%%%%%%%%%%%%%%%%%%%%%%%%%%%%%%%%%%%%%%%%%%%%%%%%%%%%%%%%%%%%%%%%%%%%%%%%%%%%%
%                                STANDARD                                      %
%%%%%%%%%%%%%%%%%%%%%%%%%%%%%%%%%%%%%%%%%%%%%%%%%%%%%%%%%%%%%%%%%%%%%%%%%%%%%%%%
\enabletrackers[fonts.missing]
\definefontfeature[default][default]
                  [protrusion=quality,
                    expansion=quality,
                    script=latn]
\setupalign[hz,hanging]
\setuptolerance[tolerant]
\setbreakpoints[compound]
\setupindenting[yes,1em]
\setupfootnotes[way=bychapter,align={hz,hanging}]
\setupbodyfont[modern] % this is a stinky workaround to load lmodern
\setupbodyfont[libertine,11pt]

\setuppagenumbering[alternative=singlesided,location={footer,middle}]
\setupcaptions[width=fit,align={hz,hanging},number=no]

\startmode[a4imposed,a4imposedbc,letterimposed,letterimposedbc,a5imposed,%
  a5imposedbc,halfletterimposed,halfletterimposedbc]
  \setuppagenumbering[alternative=doublesided]
\stopmode

\setupbodyfontenvironment[default][em=italic]


\setupheads[%
  sectionnumber=no,number=no,
  align=flushleft,
  align={flushleft,nothyphenated,verytolerant,stretch},
  indentnext=yes,
  tolerance=verytolerant]

\definehead[awikipart][chapter]

\setuphead[awikipart]
          [%
            number=no,
            footer=empty,
            style=\bfd,
            before={\blank[force,2*big]},
            align={middle,nothyphenated,verytolerant,stretch},
            after={\page[yes]}
          ]

% h3
\setuphead[chapter]
          [style=\bfc]

\setuphead[title]
          [style=\bfc]


% h4
\setuphead[section]
          [style=\bfb]

% h5
\setuphead[subsection]
          [style=\bfa]

% h6
\setuphead[subsubsection]
          [style=bold]


\setuplist[awikipart]
          [alternative=b,
            interaction=all,
            width=0mm,
            distance=0mm,
            before={\blank[medium]},
            after={\blank[small]},
            style=\bfa,
            criterium=all]
\setuplist[chapter]
          [alternative=c,
            interaction=all,
            width=1mm,
            before={\blank[small]},
            style=bold,
            criterium=all]
\setuplist[section]
          [alternative=c,
            interaction=all,
            width=1mm,
            style=\tf,
            criterium=all]
\setuplist[subsection]
          [alternative=c,
            interaction=all,
            width=8mm,
            distance=0mm,
            style=\tf,
            criterium=all]
\setuplist[subsubsection]
          [alternative=c,
            interaction=all,
            width=15mm,
            style=\tf,
            criterium=all]


% center

\definestartstop
  [awikicenter]
  [before={\blank[line]\startalignment[middle]},
   after={\stopalignment\blank[line]}]

% right

\definestartstop
  [awikiright]
  [before={\blank[line]\startalignment[flushright]},
   after={\stopalignment\blank[line]}]


% blockquote

\definestartstop
  [blockquote]
  [before={\blank[big]
    \setupnarrower[middle=1em]
    \startnarrower
    \setupindenting[no]
    \setupwhitespace[medium]},
  after={\stopnarrower
    \blank[big]}]

% verse

\definestartstop
  [awikiverse]
  [before={\blank[big]
      \setupnarrower[middle=2em]
      \startnarrower
      \startlines},
    after={\stoplines
      \stopnarrower
      \blank[big]}]

\definestartstop
  [awikibiblio]
  [before={%
      \blank[big]
      \setupnarrower[left=1em]
      \startnarrower[left]
        \setupindenting[yes,-1em,first]},
    after={\stopnarrower
      \blank[big]}]
                
% same as above, but with no spacing around
\definestartstop
  [awikiplay]
  [before={%
      \setupnarrower[left=1em]
      \startnarrower[left]
        \setupindenting[yes,-1em,first]},
    after={\stopnarrower}]



% interaction
% we start the interaction only if it's not an imposed format.
\startnotmode[a4imposed,a4imposedbc,letterimposed,letterimposedbc,a5imposed,%
  a5imposedbc,halfletterimposed,halfletterimposedbc]
  \setupinteraction[state=start,color=black,contrastcolor=black,style=bold]
  \placebookmarks[awikipart,chapter,section,subsection,subsubsection][force=yes]
  \setupinteractionscreen[option=bookmark]
\stopnotmode



\setupexternalfigures[%
  maxwidth=\textwidth,
  maxheight=\textheight,
  factor=fit]

\setupitemgroup[itemize][each][packed][indenting=no]

\definemakeup[titlepage][pagestate=start,doublesided=no]

%%%%%%%%%%%%%%%%%%%%%%%%%%%%%%%%%%%%%%%%%%%%%%%%%%%%%%%%%%%%%%%%%%%%%%%%%%%%%%%%
%                                IMPOSER                                       %
%%%%%%%%%%%%%%%%%%%%%%%%%%%%%%%%%%%%%%%%%%%%%%%%%%%%%%%%%%%%%%%%%%%%%%%%%%%%%%%%

\startusercode

function optimize_signature(pages,min,max)
   local minsignature = min or 40
   local maxsignature = max or 80
   local originalpages = pages

   -- here we want to be sure that the max and min are actual *4
   if (minsignature%4) ~= 0 then
      global.texio.write_nl('term and log', "The minsig you provided is not a multiple of 4, rounding up")
      minsignature = minsignature + (4 - (minsignature % 4))
   end
   if (maxsignature%4) ~= 0 then
      global.texio.write_nl('term and log', "The maxsig you provided is not a multiple of 4, rounding up")
      maxsignature = maxsignature + (4 - (maxsignature % 4))
   end
   global.assert((minsignature % 4) == 0, "I suppose something is wrong, not a n*4")
   global.assert((maxsignature % 4) == 0, "I suppose something is wrong, not a n*4")

   --set needed pages to and and signature to 0
   local neededpages, signature = 0,0

   -- this means that we have to work with n*4, if not, add them to
   -- needed pages 
   local modulo = pages % 4
   if modulo==0 then
      signature=pages
   else
      neededpages = 4 - modulo
   end

   -- add the needed pages to pages
   pages = pages + neededpages
   
   if ((minsignature == 0) or (maxsignature == 0)) then 
      signature = pages -- the whole text
   else
      -- give a try with the signature
      signature = find_signature(pages, maxsignature)
      
      -- if the pages, are more than the max signature, find the right one
      if pages>maxsignature then
	 while signature<minsignature do
	    pages = pages + 4
	    neededpages = 4 + neededpages
	    signature = find_signature(pages, maxsignature)
	    --         global.texio.write_nl('term and log', "Trying signature of " .. signature)
	 end
      end
      global.texio.write_nl('term and log', "Parameters:: maxsignature=" .. maxsignature ..
		   " minsignature=" .. minsignature)

   end
   global.texio.write_nl('term and log', "ImposerMessage:: Original pages: " .. originalpages .. "; " .. 
	 "Signature is " .. signature .. ", " ..
	 neededpages .. " pages are needed, " .. 
	 pages ..  " of output")
   -- let's do it
   tex.print("\\dorecurse{" .. neededpages .. "}{\\page[empty]}")

end

function find_signature(number, maxsignature)
   global.assert(number>3, "I can't find the signature for" .. number .. "pages")
   global.assert((number % 4) == 0, "I suppose something is wrong, not a n*4")
   local i = maxsignature
   while i>0 do
      -- global.texio.write_nl('term and log', "Trying " .. i  .. "for max of " .. maxsignature)
      if (number % i) == 0 then
	 return i
      end
      i = i - 4
   end
end

\stopusercode

\define[1]\fillthesignature{
  \usercode{optimize_signature(#1, 40, 80)}}


\define\alibraryflushpages{
  \page[yes] % reset the page
  \fillthesignature{\the\realpageno}
}


% various papers 
\definepapersize[halfletter][width=5.5in,height=8.5in]
\definepapersize[halfafour][width=148.5mm,height=210mm]
\definepapersize[quarterletter][width=4.25in,height=5.5in]
\definepapersize[halfafive][width=105mm,height=148mm]
\definepapersize[generic][width=210mm,height=279.4mm]

%% this is the default ``paper'' which should work with both letter and a4

\setuppapersize[generic][generic]
\setuplayout[%
  backspace=42mm,
  topspace=31mm,% 176 / 15
  height=195mm,%130mm,
  footer=9mm, %
  header=0pt, % no header
  width=126mm] % 10.5 x 11

\startmode[libertine]
  \usetypescript[libertine]
  \setupbodyfont[libertine,11pt]
\stopmode

\startmode[pagella]
  \setupbodyfont[pagella,11pt]
\stopmode

\startmode[antykwa]
  \setupbodyfont[antykwa-poltawskiego,11pt]
\stopmode

\startmode[iwona]
  \setupbodyfont[iwona-medium,11pt]
\stopmode

\startmode[helvetica]
  \setupbodyfont[heros,11pt]
\stopmode

\startmode[century]
  \setupbodyfont[schola,11pt]
\stopmode

\startmode[modern]
  \setupbodyfont[modern,11pt]
\stopmode

\startmode[charis]
  \setupbodyfont[charis,11pt]
\stopmode        

\startmode[mini]
  \setuppapersize[S33][S33] % 176 × 176 mm
  \setuplayout[%
    backspace=20pt,
    topspace=15pt,% 176 / 15
    height=280pt,%130mm,
    footer=20pt, %
    header=0pt, % no header
    width=260pt] % 10.5 x 11
\stopmode

% for the plain A4 and letter, we use the classic LaTeX dimensions
% from the article class
\startmode[a4]
  \setuppapersize[A4][A4]
  \setuplayout[%
    backspace=42mm,
    topspace=45mm,
    height=218mm,
    footer=10mm,
    header=0pt, % no header
    width=126mm]
\stopmode

\startmode[letter]
  \setuppapersize[letter][letter]
  \setuplayout[%
    backspace=44mm,
    topspace=46mm,
    height=199mm,
    footer=10mm,
    header=0pt, % no header
    width=126mm]
\stopmode


% A4 imposed (A5), with no bc

\startmode[a4imposed]
% DIV=15 148 × 210: these are meant not to have binding correction,
  % but just to play safe, let's say 1mm => 147x210
  \setuppapersize[halfafour][halfafour]
  \setuplayout[%
    backspace=10.8mm, % 146/15 = 9.8 + 1
    topspace=14mm, % 210/15 =  14
    height=182mm, % 14 x 12 + 14 of the footer
    footer=14mm, %
    header=0pt, % no header
    width=117.6mm] % 9.8 x 12
\stopmode

% A4 imposed (A5), with bc
\startmode[a4imposedbc]
  \setuppapersize[halfafour][halfafour]
  \setuplayout[% 14 mm was a bit too near to the spine, using the glue binding
    backspace=17.3mm,  % 140/15 + 8 =
    topspace=14mm, % 210/15 =  14
    height=182mm, % 14 x 12 + 14 of the footer
    footer=14mm, %
    header=0pt, % no header
    width=112mm] % 9.333 x 12
\stopmode


\startmode[letterimposedbc] % 139.7mm x 215.9 mm
  \setuppapersize[halfletter][halfletter]
  % DIV=15 8mm binding corr, => 132 x 216
  \setuplayout[%
    backspace=16.8mm, % 8.8 + 8
    topspace=14.4mm, % 216/15 =  14.4
    height=187.2mm, % 15.4 x 11 + 15 of the footer
    footer=14.4mm, %
    header=0pt, % no header
    width=105.6mm] % 8.8 x 12
\stopmode

\startmode[letterimposed] % 139.7mm x 215.9 mm
  \setuppapersize[halfletter][halfletter]
  % DIV=15, 1mm binding correction. => 138.7x215.9
  \setuplayout[%
    backspace=10.3mm, % 9.24 + 1
    topspace=14.4mm, % 216/15 =  14.4
    height=187.2mm, % 15.4 x 11 + 15 of the footer
    footer=14.4mm, %
    header=0pt, % no header
    width=111mm] % 9.24 x 12
\stopmode

%%% new formats for mini books
%%% \definepapersize[halfafive][width=105mm,height=148mm]

\startmode[a5imposed]
% DIV=12 105x148 : these are meant not to have binding correction,
  % but just to play safe, let's say 1mm => 104x148
  \setuppapersize[halfafive][halfafive]
  \setuplayout[%
    backspace=9.6mm,
    topspace=12.3mm,
    height=123.5mm, % 14 x 12 + 14 of the footer
    footer=12.3mm, %
    header=0pt, % no header
    width=78.8mm] % 9.8 x 12
\stopmode

% A5 imposed (A6), with bc
\startmode[a5imposedbc]
% DIV=12 105x148 : with binding correction,
  % let's say 8mm => 96x148
  \setuppapersize[halfafive][halfafive]
  \setuplayout[%
    backspace=16mm,
    topspace=12.3mm,
    height=123.5mm, % 14 x 12 + 14 of the footer
    footer=12.3mm, %
    header=0pt, % no header
    width=72mm] % 9.8 x 12
\stopmode

%%% \definepapersize[quarterletter][width=4.25in,height=5.5in]

% DIV=12 width=4.25in (108mm),height=5.5in (140mm) 
\startmode[halfletterimposed] % 107x140
  \setuppapersize[quarterletter][quarterletter]
  \setuplayout[%
    backspace=10mm,
    topspace=11.6mm,
    height=116mm,
    footer=11.6mm,
    header=0pt, % no header
    width=80mm] % 9.24 x 12
\stopmode

\startmode[halfletterimposedbc]
  \setuppapersize[quarterletter][quarterletter]
  \setuplayout[%
    backspace=15.4mm,
    topspace=11.6mm,
    height=116mm,
    footer=11.6mm,
    header=0pt, % no header
    width=76mm] % 9.24 x 12
\stopmode

\startmode[quickimpose]
  \setuppapersize[A5][A4,landscape]
  \setuparranging[2UP]
  \setuppagenumbering[alternative=doublesided]
  \setuplayout[% 14 mm was a bit too near to the spine, using the glue binding
    backspace=17.3mm,  % 140/15 + 8 =
    topspace=14mm, % 210/15 =  14
    height=182mm, % 14 x 12 + 14 of the footer
    footer=14mm, %
    header=0pt, % no header
    width=112mm] % 9.333 x 12
\stopmode

\startmode[tenpt]
  \setupbodyfont[10pt]
\stopmode

\startmode[twelvept]
  \setupbodyfont[12pt]
\stopmode

%%%%%%%%%%%%%%%%%%%%%%%%%%%%%%%%%%%%%%%%%%%%%%%%%%%%%%%%%%%%%%%%%%%%%%%%%%%%%%%%
%                            DOCUMENT BEGINS                                   %
%%%%%%%%%%%%%%%%%%%%%%%%%%%%%%%%%%%%%%%%%%%%%%%%%%%%%%%%%%%%%%%%%%%%%%%%%%%%%%%%


\mainlanguage[en]


\starttext

\starttitlepagemakeup
  \startalignment[middle,nothanging,nothyphenated,stretch]


  \switchtobodyfont[18pt] % author
  {\bf \em

Ihar Alinevich  \par}
  \blank[2*big]
  \switchtobodyfont[24pt] % title
  {\bf

On the Way to Magadan

\par}
  \blank[big]
  \switchtobodyfont[20pt] % subtitle
  {\bf 

prisoner’s diary

\par}
  \vfill
  \stopalignment
  \startalignment[middle,bottom,nothyphenated,stretch,nothanging]
  \switchtobodyfont[global]

2014

  \stopalignment
\stoptitlepagemakeup



\title{Contents}

\placelist[awikipart,chapter,section,subsection]



\page[yes,right]

\awikipart{Introduction. Ihar
}

This book is written by my son about what is happening in Belarus today, about the choice of a person in a situation between life and death, freedom and captivity, conscience and betrayal. Everything that happened to him, happened in real life and in the 21\high{st} century, in a country that considers itself a civilized European state, before and after the presidential elections of 2010. The idea of this book came up in spring of 2011 during the only meeting between us that was allowed in the KGB prison. We could communicate with great caution, but my husband and I were so happy to see him\unknown{} Everything that has happened and is happening with Ihar is very similar to the situation described in the book “Children of the Arbat” by Anatoly Rybakov. Although Sasha Pankratov was arrested in 1933, the story unfortunately recurs. I suggested that Ihar wrote about everything that had happened to him, so we don’t forget, so he leaves his memory for history. We all think that lawlessness, brutality and repression will never touch us and our loved ones. It is very important that this situation becomes public knowledge.


In spring 2011 I didn’t know the details of his abduction in Moscow and imprisonment in the KGB detention center. The trial had just passed, which showed the absurdity of charges against him and his comrades, Mikalai Dziadok and Aliaksandr Frantskevich. A whole series of such trials were held at that time. The Youth Front members Dashkevich and Lobov, candidates for presidency Sannikov, Uss, Nekyayev, and the members of their electoral offices, and other young Belarusian opposition activists were already convicted by that time. Prisons and detention centers were overcrowded. Almost simultaneously with our trial was the trial of Nikolai Statkevich, another candidate for presidency. The last sentences were shocking because of their severity – 6 years of prison for Statkevich, 8 years for Ihar. There already appeared some rumors about conditions in the KGB prison, beatings, tortures, psychological pressure. I tried to drive away from all that I heard, I didn’t believe that such lawlessness and victimisation could happened to the child that I gave birth to and raised to live a happy, creative and free life, not to serve as cannon fodder for the regime.


I’m often asked about how I feel when I read the diary of my son. I am writing these lines, and though it’s been three years, I shed tears again when I recollect it. Of course, I feel pain. This pain grids into my soul and heart. It is eternal and felt all the time. This pain is not only for my son, for our family – this pain is for the people of Belarus, for the Belarusian youth who are forced to serve completely unfair prison sentences in terrible inhumane conditions, to emigrate from their country, hide, be humiliated and beaten just for the wish of a better and free life for its people. History repeats itself. Trying to retain power, the regime destroys the future of the country. At the same time, I am proud that I have been able to raise such a worthy man as my son. In the face of grave danger, remaining in complete isolation without any support from the outside, he has managed to stay true to himself, to his personal values. And not only him. The event of December 2010 showed how many decent people live in Belarus. They can identify themselves with various segments of civic society, with different parties and movements, they can look in different directions, but there is one heart beating inside them all. This is the heart of brave, honest, sincere patriots who have preserved their mind, honor and conscience and have not decided to cooperate with the repressive regime. Ihar, as well as all political prisoners in Belarus, turned out to be more respected than those who have been part of the herd of bulls, trampling our country.


When Ihar was 16 years old, I gave him a poem by R. Kipling. I wanted him to learn the truth that is important for a real person before he entered into adulthood. I sent him the same poem before the trial to support him, to make him realise that the world does exist, that human values are always there, even if it seems that the world is gone and you are left face to face with a terrible dragon.


As a mother, I am responsible for what kind of life I have prepared for my children. As a mother, I am proud that I have raised a son, who has remained a true person.


{\em Valiantsina Alinevich}


\section{The Commandment
}

If you can keep your head when all about you
Are losing theirs and blaming it on you,
If you can trust yourself when all men doubt you,
But make allowance for their doubting too;
If you can wait and not be tired by waiting,
Or being lied about, don’t deal in lies,
Or being hated don’t give way to hating,
And yet don’t look too good, nor talk too wise:
If you can dream – and not make dreams your master;
If you can think – and not make thoughts your aim,
If you can meet with Triumph and Disaster
And treat those two impostors just the same;
If you can bear to hear the truth you’ve spoken
Twisted by knaves to make a trap for fools,
Or watch the things you gave your life to, broken,
And stoop and build ‘em up with worn-out tools:
If you can make one heap of all your winnings
And risk it on one turn of pitch-and-toss,
And lose, and start again at your beginnings
And never breathe a word about your loss;
If you can force your heart and nerve and sinew
To serve your turn long after they are gone,
And so hold on when there is nothing in you
Except the Will which says to them: ‘Hold on!’
If you can talk with crowds and keep your virtue,
Or walk with Kings – nor lose the common touch,
If neither foes nor loving friends can hurt you,
If all men count with you, but none too much;
If you can fill the unforgiving minute
With sixty seconds’ worth of distance run,
Yours is the Earth and everything that’s in it,
And – which is more – you’ll be a Man, my son!


{\em Rudyard Kipling}


\awikipart{Context: The Case of Belarusian Anarchists
}

‘The case of Belarusian anarchists’ was started after an attack on the Russian embassy in Minsk which was carried out on the 30\high{th} of August, 2010, when an unknown anarchist group named “Friends of Freedom” threw several Molotov cocktails at the embassy. According to their statement, it was an action of solidarity with Russian political prisoners, a protest against repressions in Russia after the public unrest due to the destruction of the Khimki forest.


The development and the results of this case expose the conditions of social and political activism in Belarus.


The population of Belarus is still very clumsy in terms of protecting their rights and freedoms which are slowly but steadily limited not only by the law, but also by the President’s decrees or public statements by the authorities.


At the same time, the consciousness of young people is often limited by the generally accepted axiom that nothing depends on you and everything is already decided – almost 18 years of Lukashenko’s\high{\footnote{Alexander Lukashenko – president of the Republic of Belarus.}} paternalism has raised a whole generation. At the same time there is no possibility for public political expression. Absolutely all peaceful political initiatives in Belarus are harshly suppressed: it is impossible to carry out a legal picket or demonstration, an open discussion or meeting – people are detained even at punk concerts. No wonder that, in the absence of any dialogue between the government and society, the most strongly minded social activists see confrontation as the only possible way to defend their views.


In this context the arguments for “guerrilla tactics” have started to appear more consistent: firstly, publicity is substituted for activists’ heads on the block; secondly, in these circumstances, legal and semi-legal activities hardly perform their task of attracting new people; thirdly, the impact of peaceful initiatives on the general public remains minimal. The repression exerted by the authorities and the passivity of the masses worked as a call to confrontation. Thus, the years of 2009 and 2010 were particularly rich in radical actions, the responsibility for which was claimed by Belarusian anarchist groups.


The starting point of the transition from picketing and distributing leaflets to more noticeable actions was an anti-war march against Russian-Belarusian military exercises conducted near the General Staff of the Armed Forces building in September 2009. A common march with banners and slogans was supplemented by a smoke grenade which was thrown into the territory of the General Staff.


After the anti-war march other actions occurred in which not only leaflets, banners and megaphones were used, but also fireworks and Molotov cocktails. These included the attack on the “Shangri-La” casino in Minsk with paint bulbs and hand flares as a protest against the mass transfer of Russian casinos to Belarus after the prohibition of casinos in Russia; the attack on a police station in Soligorsk\high{\footnote{A provincial town in Belarus.}} dedicated to a common day of action against police brutality in Belarus (a hand flare was thrown through a broken window); the attack on the Federation of Trade Unions which has long been the “defender” of workers’ rights, but only settled conflicts in favour of employers; the attack on BelarusBank headquarters in Minsk which was set on fire to protest against the existing financial system.


Despite the use of new techniques, actions still had a symbolic nature – the damage inflicted by them cannot be considered significant. Their main advantage was the media effect. However, press releases were only circulated by certain opposition-minded media. The state media was at best limited to mentioning the fact, and at worst, spread their speculations and judgments that painted anarchists as unmotivated aggressors and mindless hooligans – the characteristic reaction of the state media to any organized opposition in Belarus.


The attack on the Russian embassy with Molotov cocktails on August 30, 2010, which resulted in the damage of an official car, was a motive for the repressions that hit the Belarusian libertarian movement.


By the autumn of 2010 more than 150 people had been interrogated, 19 people were arrested as suspects and 5 of them were subsequently convicted of involvement in the direct actions mentioned. They are – Ihar Alinevich (8 years of prison), Mikalai Dziadok (4.5 years of prison), Aliaxandar Frantskevich (3 years of prison), Maxim Vetkin (4 years of custodial restraint) and Jauhen Silivonchyk (1.5 years of custodial restraint).


During their detention, another radical action happened: “Friends of Freedom” attacked the gates of a pre-trial detention facility where the detainees were held with Molotov cocktails. In a press release they stated that the authorities had seized the wrong people and took responsibility for all previous radical actions.


At the same time an international solidarity campaign spread in Europe and Russia, with mostly peaceful protests, and even in Mexico, where an attack on banks took place. It’s noteworthy that radical methods of struggle have also continued in Belarus. In October 2010 there was an attack with Molotov cocktails on the KGB headquarters in Bobruisk\high{\footnote{Another town in Belarus.}}. After this action, Jauhen Vaskovich, Artsiom Prakapenka and Pavel Syramolatau were arrested and sentenced to 7 years in prison each.


Ihar Alinevich, Mikalai Dziadok, Jauhen Vaskovich and Artsiom Prakapenka are still behind bars\high{\footnote{Pavel Syramolatau was released in July 2012 after having signed a petition for pardon. Aliaksandr Frantskevich was set free in September 2013 after having done all his time.}}. While remaining in custody, the ‘Decembrists’\high{\footnote{The Decembrists or Russian Dekabrist are the Russian revolutionaries who led an unsuccessful uprising on Dec. 14 (Dec. 26, New Style), 1825, and through their martyrdom provided a source of inspiration to succeeding generations of Russian dissidents. The people arrested in Belarus after the elections in December 2010 were also called ‘Decembrists’.}} have a good chance of being released soon (the regime will be forced to release some political prisoners as part of the normalization of relations with Europe). The prospect of those convicted in the ‘case of anarchists’ is not as obvious.


{\em A. Zhinevich, sociologist, activist.}


\awikipart{On the way to Magadan
}

\chapter{1
}

November 28 2010, Moscow, a cafe in the “U Gorbushki” shopping mall. 2.45 pm. I can barely keep my eyes open after a sleepless night. Crowds of people, vanity, anxious faces. Every other passer-by seems to be an officer, like these three in black jackets, for example, with grimly serious faces. Dima\high{\footnote{Dzmitry Dubouski – Belarusian anarchist who left the country with Ihar. Now is wanted by the police.}} is sitting in front of me. We are laughing at our own paranoia. Last night Buratino\high{\footnote{Buratino (Pinocchio) – a nickname of Anton Laptenok, a traitor who made an appointment with the wanted anarchists so that the special service could catch them.}} suggested meeting in a safe place.


There was a persistent desire to reject the invitation, because we know he is suspicious. But we must meet him. Dima is nervous. According to the plan, he has to watch the meeting through binoculars, but he was against it from the beginning. Certainly, Dima is right. Both the place and the plan should have been considered better, but three months underground dulls your sense of danger; moreover, we do not want to believe that this one has also betrayed us. I must manage to spot agents during the meeting, and run away spraying tear gas in their faces if needed. It is too late to change anything. It’s time to go.


When we come out of the mall, four shadows rush from all sides, clutching my arms. I was not surprised, not a muscle trembled. Dima jumped aside and ran away. A random passer-by trips him up, but fortunately fails to stop him. Country of masters, country of slaves.


One of the “men in black” reassures me, “This way we help your special forces.” Hmm, it’s your special forces, not ours. Clank of bracelets, I am pushed into the car. Pocket search: a mobile phone, a wallet, an mp3-player. An hour and a half before the meeting I turned the phone on, from which I had called Buratino. I thought that in a crowded place they wouldn’t manage to find it or won’t even try to spot it. Foolish mistake\unknown{} Hat covering my eyes, one car, then another, “men in black” – servants of FSB\high{\footnote{Russian Federal Security Service – a successor of the KGB.}} – don’t exchange a word with one another, they are texting each other. A couple of stops to pee, I see a field, a forest, and it all seems to be a dream\unknown{}


\unknown{}Belarusian border. They push my head down to the floor, it means the operation is illegal. They hand me over to the local agents in a van. The Russians say:


“Don’t ask us to do this kind of shit any more!”


“Certainly, we owe you a favour, lads,” answer the agents.


The car pulls away. They start with threats:


“You know what you need to tell us, don’t you? Or shall we stop somewhere for explanations?”


“Yes, I get it, I get it,” I answer.


Yeah, of course. Don’t interrogate your memory, don’t regret, count seconds, calm down. I must mobilize, focus on the only truth: “Don’t trust, don’t be afraid, don’t beg\unknown{}”\high{\footnote{A well-known jail proverb, popularized by Shalamov and Solzhenitsyn.}}


“It’s already 20.30, drive in.”


The gate clanks and the car drives in. The cap is still over my eyes. I am completely disorientated. They push me into the room, throw me on a chair, face on a table, the edge of someone’s palm falls on my neck. The longest night in my life is beginning\unknown{}


“\unknown{}Ihar, let’s talk man to man”, sounds the voice in front of me.


“People usually don’t talk in such conditions,” my own voice surprises me.


They probably didn’t expect resistance and they linger for a while. It gives me confidence. Then they start:


“We know everything, speak, confess!”


“I don’t know anything, I wasn’t there.”


“Everyone has snitched on you, why would you refuse?”


Only one question is tormenting me: Is Dima gone or was he taken as well? But how can I find out?


“What about Dima? Did he give evidence?”


“Which Dima? Do you mean Dubouski?”


I see! They didn’t take him! So, everything is not that bad.


“What about the news on the Internet? No one was kidnapped? Bad work. We knew about Buratino in advance. We were ready”, I say.


The door opens, someone says:


“Indeed, there are already publications on the website.”


The awkward pause hangs. It seems that it is offensive for them to acknowledge that they couldn’t get both of us. All the investigators left the room, there were three or four of them. It’s a load off my mind: Dima is free, he kept his head, and Buratino is completely revealed. Now I have to go through the inquiry. When I happened to be at a human rights workshop with Markelov\high{\footnote{Stanislav Markelov (20 May 1974 – 19 January 2009) – a Russian human rights lawyer. He participated in a number of publicized cases, including those of left-wing political activists and antifascists, as well as of victims of police violence, journalists, etc. Markelov was murdered by neonazis on 19 January 2009 in Moscow.}} (rest in peace), I learned firmly: no admissions! Generally, the whole case is based on the evidence given during the first days of interrogation.


The investigators return.


“You’re naive. Do you think you have friends? Everyone snitched on you, and you suspected the wrong person!”


But I don’t listen to this bullshit any more. The first rule is “Don’t trust!” Everything they say is a lie, a half-truth. And even if it is the truth, they use it only for further manipulations. The method is simple: they begin with one episode but as soon as they face any psychological resistance, they pass to another one. The army\unknown{} The casino\unknown{} The billboards\unknown{} Labour unions\unknown{} The bank\unknown{} The embassy\unknown{} The detention centre\unknown{} The bank\unknown{} The casino\unknown{} It is endless.


They starve me out. I fall asleep many times and wake up; when they feel my fatigue, they strengthen pressure at once. Everything is used: threats, flattery, blackmail, assurance of the senselessness of our fight, suspicion towards companions, emphasis on egoism, etc. I don’t know how much time has passed. It ceases to exist. It isn’t clear, what is real and what is a dream\unknown{}


“We will throw you into a boneheads’ cell! We have a special cell full of nazi skinheads!.. You are a handsome man, they will love you in prison\unknown{} You haven’t been beaten properly yet\unknown{} Why do you need it? You could have lived like everyone else. It is still possible!.. Do you do karate? It is hierarchical, you contradict your principles!.. You are afraid to admit it, you are a coward!.. You will finish up in jail. This issue is resolved. But for five or ten years, it’s up to you\unknown{} I would sentence you to 12, no, even 20 years\unknown{} (“And I would shoot you without a doubt, bitch,” I thought). I’ll call your grandmother. We’ll let her know everything about you\unknown{} Nobody will hire a lawyer for you\unknown{} We need to know only one thing: who paid you?..”


I can only say “I don’t know”, “No, I wasn’t there,” and lose consciousness again. The second rule says “Don’t be afraid”. As a rule, they fake it. Even if they don’t, it’s still the only way to learn if you can withstand it or not. The one who is scared, looses everything. If you show you’re frightened, you are on a hook and they’ll pull everything out from you.


They take the hat from my eyes for a while. There is only one man sitting at the table:


“You’re a good guy. An engineer with a healthy life-style, doing sports\unknown{} You shouldn’t ruin yourself like this. I understand that many things the opposition are saying are right, but the implementation causes suffering. Why don’t you just let it all go?”


During the inquiry I sometimes have the feeling that I have already read bits and pieces somewhere. This thought makes me sober and confirms that it is all a performance. After all there is a certain feeling of isolation, subconsciously there is a wish to believe their arguments and therefore stop everything. Psychological defense reaction. Nowhere to escape from it.


They pull the hat over my eyes again. Somebody new comes in. He doesn’t speak a lot, but with select phrases and specific intonations he starts to rub in what a chicken shit I am\unknown{}waiting again for something. I am really thirsty and and tempted to ask them for a cigarette. But I know I can’t do that. The third rule is “Don’t beg”. Every request should be put in a form of demand Any request makes the psychological ambiance softer and it can be enough to let them dominate.


They take off my hat, bring some food. The detectives sit lackluster, exhausted. We are waiting for something, for a long time. I see a light through a little window. It means it is already daytime. Suddenly it’s time to get up and again through corridors, stairs, short passage threw the inside yard, passing multiple rooms with signs saying “Interrogation in Progress”. They take me to the investigator. There is also a lawyer. Everything is done cordially. They hand in the arrest warrant, accusing me of taking part in the action near Okrestina detention center. They start the inquiry. It is 16.00. I have been there for a day already, 24 hours in their claws. The inquiry lasted 19 hours. They finally take off the handcuffs\unknown{} This wonderful feeling of freedom to move my hands\unknown{} I take the responsibility for a smoke grenade near the Army HQ. Anyway I am on the video – they will identify me, and it is better to lift the charges from Mikola. They want to make him the organiser, but it will not work out. Anyway, I don’t consider myself guilty.


A search. They take away my things, my boots, give me some old slippers. I can’t stand it anymore, I’m falling asleep right on the bench. They’re waking me up, taking me to a big round hall with massive walls. A narrow ladder to the second floor. I have a feeling I’m in some kind of symbiosis of anti-nuclear bunker and the Coliseum. A horizontal grating separates the first floor from the second. In the middle there is a security console with a phone. The guard takes me in a roundabout way past the doors, one by one. I carry a mattress, a pillow and sheets. We stop, the door no 3 opens and I enter the cell. Nobody’s there. Two iron beds with iron rods, two stools built in to the wall as well as a table. In the corner there’s a plastic bucket with a cover. On the bedside table I see a tray with potatoes, herring and juice. A small window with a double lattice in the form of muzzle connects me with the outside world.. which, for me, is a brick wall. The door closes with a clang. I fall down on the mattress and immediately fall asleep.


\chapter{2
}

I am woken up by the senior warrant officer who comes into the cell to ask for a report.


“There’s one person in the cell, there are no letters or written requests. The walk is 1 hour. Alinevich is on cell duty” – that’s how it sounds every day.


The time is crawling. I have nothing to do. It is wildly cold and draughty, but I can’t wrap myself in a blanket\high{\footnote{The prison regime in Belarus allows lying on beds or using bedclothes only at night.}}. For those who arrive here without warm clothes it’s a real torture. The lack of shoes hits you the most. The feet are cold in any socks, even knitted ones. The only thing that helps is to wrap your feet in a jumper. But these are small things. The most important thing is the permanent silence and the absence of time. Sometimes you can hear steps, the squeak of handcuffs, the clanks of the “feeders” (a small vertical door for delivering food), knocks on the door, whistle and the whisper of controllers (they never talk!).


In a couple of days you’re able to catch and recognize the slightest sounds. The feeder is opened several times a day: for breakfast, lunch, dinner and medicine. The door opens 4 times: mornings and evenings to go to the toilet, plus in the morning for the round of the man on duty, and once more for a walk (if there is one). And that’s how it goes for months, for some people, years with a 24-hour luminescent light.


Complete obscurity, where I am and what comes next. They take my watch away. Days blend into one\unknown{} You wake up and go to sleep, without knowing how long you were asleep, or what the time is.


What are the thoughts of a prisoner in the early days? It is a swarm of creativity, of the imagination catalyzed by subconscious animal fear. Only regular physical exercise brings you back to reality. Isolation\unknown{} What does it feel like? The life of a human is woven from a thousand social threads: communication, commitment, plans, relationships, work, even the salad in the fridge – everything has a thread in our minds. And in a moment you begin to slide off from this solid floor. Not immediately, but gradually. Suddenly you remember about some matters, from more to less urgent, the mind starts to shudder, to rush, you need to do something. You’re trying to take hold of the threads, not to miss them, to link them somehow in a new way, but instead you lose them one after another and fall into the abyss of emptiness. This is not the worst thing: here you see at least what you are losing\unknown{}


\unknown{}In this tremendous vacuum the first package and the first letter from relatives is like a ray of light, breaking through the darkness and burning with warmth. I remember how I pulled out of the package warm socks and a blanket. I wrapped myself in it and then fell asleep with a sense of home and parental care\unknown{}


“\unknown{}I have something to talk to you about,” says a gray-haired, but sturdy colonel of the 4\high{th}\high{\footnote{The 4\high{th} department of State Security Committee of Republic of Belarus is the ‘ideological’ department, just as 4\high{th} department of KGB of USSR was dedicated to ‘fighting the anti-soviet elements’.}} KGB\high{\footnote{KGB – Belarusian security service; still called KGB as in the time of the Soviet Union.}} department. The view of the city at night, the central avenue of Minsk suddenly opens from the window of the furthest office. I do not believe that someone wouldn’t be impressed by the view, after the cell. So close and yet so far, as long as years\unknown{} Tea, cookies, cakes, other courtesies, like in the movies.


“Do you know why you’re here?” sounded the insidious question, like the Inquisition used centuries ago.


“Firstly, I would like to know where I am?” I reply.


“This is not a jail, thank God, but the KGB detention center. There is a difference. ‘Amerikanka’, as the folks call it. In the 30’s more than 30,000 people were shot dead here. It is sad, but I assure you, neither I nor my colleagues would ever think of doing something like that again”, continues the colonel.


Three interviews until the night: about the anarchist movement, methods, personal choice, the meaning of life, etc. I just decide to talk exclusively within the information available on the Internet. When they ask a question, I imagine an open source, where such information is available and only then answer. No details.


The colonel is interested in such things as “financing”, “leaders”, “international connections”. The potential of the movement, in case it is used by external forces to destabilize the situation in the country. Clearly, they have one-sided thinking. No one believes that people can do something themselves out of ideological motives. On the third day it ends with the question:


“Can anarchists and those in power go together towards a better future? Would you like to create your own organisation?”


A fragment from ‘The Informant’s Diary’\high{\footnote{‘The Informant’s Diary’ – an anonymous book of a former KGB informant.}} clicks in my memory. There the recruitment happens through the same proposal!


“When my sentence is over, I’m going to work on the issues of alternative energy”, I rap out slowly, word by word. My answer saddens the colonel greatly\unknown{}


On the way to the cell I remember Mayakovsky\high{\footnote{Vladimir Mayakovsky (July 19 [O.S. July 7] 1893 – April 14, 1930) was a Russian and Soviet poet and playwright. He is among the foremost representatives of early-20\high{th} century Russian Futurism.}} and his great words: “\unknown{}I would rather serve pineapple liquor to whores at the bar\high{\footnote{A quotation from Mayakovsky’s famous verse “For You”.}}.”


\unknown{}The first walk under the falling wet snow in holey fabric slippers. A walk is within three-meter high stern walls, a patio of three to six steps in size and grating overhead with electric barbed wire. The first time discourages you from going out again for a long time, but only until you understand that the sky, even grated, is better than the dirty-white ceiling with the same light 24 hours a day. Cold rain drops are running down my face, just like in the forest which we often went through with Dima on the way to the train station, when we had been hiding in Moscow.


\chapter{3
}

When the first arrests started in early September, no one thought it would turn out to be so serious. I immediately got in touch with Dima, and together we were waiting to see what would happen next, hoping that everything would go well and everybody would be released. But within the next three days we learned that UBOP\high{\footnote{Department for Organized Crime Control in Belarus.}} wanted to raid five apartments. Every day the number of detainees increased, and information appeared that the comrades were not only interrogated about the embassy, ​​but about many other episodes, even absolutely fantastical ones. But we still didn’t believe that someone would go to jail with a real sentence. Over the years we’ve learned that nobody cares about us: neither cops, nor journalists, nor politicians. However, the warning signs started to appear recently. Plain-clothed police started actively visiting punk concerts. They also tried to join us as sympathizers, and in spring of 2010 raided Bespartshkola (public lectures on anarchism), but somehow no one paid much attention to it.


More intriguing events developed on the Internet a couple of days before the detention. Belarusian Indymedia, a free news platform for anarchists and close-to-anarchist initiatives, applied censorship; having removed the message about the action at the embassy. Moreover, their collective declared the action provocative. It must be pointed out that anarchist radical actions have been carried out regularly since 2008, both in Belarus and in Russia. Events in Greece certainly became the main catalyst. For the first time over the years it was announced that the revolt was masterminded not by some abstract antiglobalists, but by actual anarchists. For the death of the young man\high{\footnote{A 15-year-old, Alexandros Grigoropoulos was killed by two policemen in Athens on 6 December 2008. The murder of a young student by police that resulted in large protests and demonstrations.}}, such a response; so comprehensive and uncompromising! But during the last three years Indymedia saw a provocation only in the last action. Using general confusion and the downtime of a Revolutionary Action web-page, Indymedia managed to impose its own evaluation onto the majority of the movement in the country as well as abroad. Another part of the movement, a smaller one, didn’t follow that way, but the forces were unequal at that time. Infuriated, we looked at the sheer apostasy and madness. It was sickening to realize that the majority of supporters of freedom and reason behaved like a herd, following the assurances of two or three people. It was obvious that some people tried to save their own skin, and the pityful demagogy\high{\footnote{After Belarusian Indymedia had deleted the statement, some people started arguing in comments that it was against Indymedia’s rules to delete content of that kind. Ihar disapproves of the attempts by the Indymedia collective to defend themselves by trying to find gaps in the rules.}} about Indymedia’s rules couldn’t hide it. Alas, in that situation we had to quickly solve another issue: to share the fate of the detained or to escape repressions.


It is not so easy – to gather and throw everything away. Important and interesting projects wait at work, the weekend house is at the peak of repair, plans to attend a rave party on the weekend with friends. Tens of social web threads hold you and set the movement. And then at one point you have to abandon everything. Reasonings lead to a deep introspection, when you have to find out your true values, conviction in ideas, life purposes, readiness for sacrifice. A peculiar check of what is more important: the will, even poor and hungry, or comfort and a ‘cross-your-fingers’ mentality.


Last days of feverish packing and attempts to finish at least some affairs. A trip to grandma and grandpa, some help with the garden in the weekend house. They are already old and, most likely, I won’t see them anymore. Then, a visit to parents, to wire the garage for electricity as I promised long ago. Mother talks about the plans for the next week, and I have a lump in my throat. We spend a night in a friend’s country house. I don’t explain what’s going on. He doesn’t ask. It’s good when friends understand that if you need something, there are weighty reasons for it\unknown{}


\unknown{}The road to the border. I have a heavy heart. You break away from everything native and close. The fate of comrades is up in the air. But I feel better than Dima. He has to leave his beloved. The drama which took place in the movement is killing us. Articles are published and opinions are expressed saying that “we and radicals are not on the same paths.” It is easy to say when the answer can attract the attention of the snitches and, thus, the cops. When, for the sake of safety comrades peck comrades, the unified movement ceases to exist. Solidarity is the minimum base on which interaction of various opinions and political trends is possible. A differentiation occurred, same as in Germany, Poland, France, Greece, Spain. Well, so be it. The spirit of adventures prevails, and we are full of optimism again. We will fight further, for the sake of ourselves and our comrades. Let the whole world turn against us. We won’t step back and won’t give up.


\unknown{}Moscow. Crash pads, cross-overs, searches for free Wi-Fi, acquaintances, insomnia, sometimes daily change of apartments. We know that we are already wanted, and they are really looking for us. A search for a safe place, hard work in cold and rain, dirty-tricks by employers, sometimes hunger. But it was that autumn that I saw an unprecedented solidarity in action. Accommodation, food, money, communication, leisure. We would never have survived unaided. Those days, lines of Kropotkin about mutual aid were read somehow in a new way. Brotherly support and feelings were poured before us with bright colours in all their beauty and greatness.


The information on the case was arriving in crumbs. The clouds were gathering. Sanya and Mikola were accused of a number of episodes and locked in a pre-trial detention. KGB officers schemed, wrote provocative articles on the Internet, sent false letters, put pressure on relatives. They used especially cruel and brutal methods against Dima. His soul was full of drama, but the will is stronger. It was all in vain, too clumsy (with some exceptions). Then detectives decided to send a double agent. We got too many disturbing warnings about Buratino, but there was no direct proof. There was no wish to take a risk, besides, we had just settled in a safe place and found normal jobs. But it was necessary to expose a Judas by all means. It was simply impossible to leave such a person in the movement.


Before going to the meeting we sent a letter to reliable people in case something went wrong\unknown{}


\chapter{4
}

\unknown{}A few days later investigative activities began: three confrontations with people who have given evidence. I have some lingering hope that these people won’t dare to repeat them directly to my face. Arsen is absolutely despairing, Vetkin hides his eyes and speaks like a wuss. Denya is very worried, but looks into my eyes. Anyway it is worth it for me to refrain from an assessment before court.


Certainly, confrontations leave a hard feeling. It turns out that everything has its own price. Meanwhile, one thing is clear: I am in trouble and in trouble for long time.


***


“Take your stuff and go out!” sounded the order from the guard. Two weeks in solitary confinement are over, now it’s time to move to another one. I go, say hello. In front of me there are people, ordinary people with human faces. I imagined criminals looked different. A fellow dark from tattoos in an vest approaches me and asks: “Have you chanted for MTZ\high{\footnote{MTZ-RIPO (now ‘Partisan’) – a Minsk football team, largely supported by antifascists.}}?” They say that the world is small. But who could think that in the KGB prison, where there are only 18 cells for 60 people, I’d meet the person with whom I chanted for MTZ-RIPO some years ago! Indeed, the world is like a village! The mood has improved. We start smoking. Max, my acquaintance, 22 years old, punk rock, antifa, football, amphetamines, 9 years of prison for drug trade (Art. 328 p. 3, from 8 to 13 years). Kirill, a presentable guy, 29 years old, worked for the State Control Committee, accused according to Art. 209 (“Fraud”). According to the version of investigation he borrowed money from amorous girls without giving it back. The scale was striking: a total of 1.5 mln rubles\high{\footnote{Equals \$500.}} on four counts! I would never have believed that anyone could be sentenced for it, especially by the KGB, if he had not read out excerpts from the case. Vladimir, an elderly man, 55 years old. Previously a City Official from the municipality of Mogilyov\high{\footnote{A city in Belarus.}}. A few years ago he stole a couple of trucks loaded with sand and a little bit more and transferred them to his country house. And then he crossed the wrong guy, that’s when the sand emerged. He faced up to 10 years without the right to amnesty.


Days and weeks passed by\unknown{} It is much better to be in jail in the company of appropriate people than in a solitary confinement. Considering everyday life, the shortages of different essentials is easily solved. Garlic, onions, soap, toothpaste, matches, a boiler, a kettle, a pen, a pencil, a sheet of paper, an envelope, a basin, threads and various hygiene items\unknown{} You won’t remember everything. But what is more important is the understanding of the future prospects of staying in jail. How things are solved with the administration, what processes occur in the investigative and judicial system, the expected timing of pre-trial investigation, the Criminal Code articles that can be applied. All in all, a complete view on the current situation. But most important is a sense of community. Prison solidarity is developed very quickly, although in Americanka the conning culture is not established. All natural desire to communicate is practiced, such as mutual assistance, the feeling of involvement, games, jokes, and, certainly, laughter. The trouble pulls people together, and it is noticeable how a person who was individualistic and closed outside the jail, becomes more social and open. Cooking, cleaning, washing, even simple movement in the cell or in a formation demand constant awareness of the others. In a way, the primary fear of the unknown and the severity of the prison dwelling goes away. After all, the main enemy is your own imagination. Soon the circumstances will make you realise the truth of this assertion. All of us, the Americanka prisoners of that time, got convinced of it.


But meanwhile we were playing dominoes, arranging checkers tournaments playing Hacky Sack during the walks, watching TV in the evening, told tall tales and life stories.


Letters of support and solidarity arrived from family and friends, colleagues and strangers. I had a meeting with a lawyer. He brought a particle of something absolutely alien to this stone vacuum world, something very friendly to me. It inspired me and strengthened the thought that I was not alone. Self-reliance absolutely dominated me and choked the destructive and despairing voices about my broken life, career, life style and other things. There is nothing to hide, at first everyone thinks about this. The question is whether these thoughts will end during the first few days of custody or will continue to torment the soul.


\chapter{5
}

We learnt the results of the presidential elections on the night of December 19, 2010, when a fifth person was thrown into our cell – Oleg Korban. It turned out that tens of thousands of people had gone to the streets and there had been some riots at the Government House. It was hard to believe\unknown{}


The next day we saw dozens of wooden beds in the corridor. Most of them had a very fresh appearance, that suggested a preplanned operation. Which suggested another conclusion\unknown{}


One thing became clear: there were mass arrests. In the news they were talking about six hundred detainees. How many it was in reality, we will never know. The same day, Oleg was taken away. Anatoly Lebedko was brought up instead, chairman of the OGP\high{\footnote{OGP (Russian ‘Obyedinennaya Grazhdanskaya Partiya’) – The United Civic Party.}}. But we renamed the party OPG\high{\footnote{OPG (Russian ‘Organizovannaya Prestupnaya Gruppirovka’) – an organised criminal group.}}. It sounds more familiar in prison dungeons. An experienced politician, he had visited many places and participated in many affairs, including Lukashenko’s coming to power. Irony of fate. By the way, that didn’t prevent him from beating us in dominoes and other games. He became familiar with them, probably, during his former administrative arrests. Lebedko started a hunger strike and bravely held out until the New Year.


There was a mood at that time that the government had decided to intimidate the opposition a bit and keep them locked up for about ten days. The biggest dirty trick that our imagination could draw, was the fact that people would be kept inside for the holidays, would have to stay in prison for a few days longer. But even that thought seemed almost unbelievable. Everyone was so used to the Belarusian left-handed dictatorship; which couldn’t make serious actions and bonded together only on the slave mentality of the people.


We didn’t pay attention to the appearance of security in balaclavas. It seemed somehow logical, that once the prison got overcrowded, they called for reinforcements. We didn’t know what the commission of SWAT-units to the prison meant. There were no experienced prisoners among us\unknown{} That time we still snapped in anger at the order “face to the floor”, and their rudeness and crudeness was explained by the fact that these thugs were from some riot police unit. Even when they made us ‘stretch\high{\footnote{“To put on stretch” means to make somebody put their hands on the wall and spread their legs widely. The cops can force you to stay in that position for a long time as a form of torture.}} during a search, which was usually done in the gym, we took it as a rotten show-off, a crude attempt to intimidate, a slapstick comedy, which was about to end. After all, the whole society was watching the unfolding events, the entire West was closely watching Belarus.


All illusions vanished when they took away the TV, when they almost gave Vladimir a heart attack (to all complaints they just responded: “If you die – we will take you out”), when we were forced to walk around in circles in the yard, when at 10pm on December 31\high{st} Lebedko left with his things and in half an hour he returned\unknown{} The detention center had changed, so had the country. The government made a clear step towards an outright dictatorship, showing the confidence in its power, firmness, impunity.


That New Year’s celebration was the most incredible in my life. Even in a fantastic dream I couldn’t imagine that I would meet the year of 2011 in KGB dungeons in such freakish company. With Coca-Cola and chocolate cake on the table, I mean a nightstand, to the accompaniment of old songs and the vague expectation of a grand raid.


Being the oldest of us, Vladimir delivered a kind of toast: “How wonderful that we all are here today!”. Lebedko was short: “Long live Belarus!”


\chapter{6
}

The year of 2011 started gloomy. On the 1\high{st} of January Max and I drew a smile with snowballs and a slogan “Vivat anarchia” during a walk. As soon as we got back into the view of the cameras a guard’s head appeared in the doorway and asked: “Well, who is the artist here?”. I took it upon myself and went to clean it alone. That was a mistake. Somehow I didn’t pay any attention to my escort of two masks, who followed me to the yard. Suddenly they ordered me to take off my sweater (knitted by grandma) and clean the snow art with it. With certainty I refused and immediately got a baton to the head. In the first few seconds I was shocked, I couldn’t believe that they seriously expected a normal person to undress in the frost and scrub that dirty rough wall with their own clothes. But they really wanted it! Order – refuse – hit, Order – refuse – hit\unknown{} They hit me in the head, ears, neck, groin, under knees, poked into my teeth and eyes. My blood boiled, fists clenched. Seeing that turn, the masks moved a couple of steps back and stood with clubs raised. They shouted that I unclench my fists, but I did not hear them any more. The situation was settled by a warden who had suddenly grown behind their backs. They didn’t dare to continue. Everything inside me burned\unknown{} On the way back they stopped me again at a staircase. Those same masks or other, I couldn’t make out. They demanded that I bow my head. I refused. I felt a powerful hit to the neck from behind. I refused again. And again I got a hit. Refused one more time. The totally exasperated mask yelled:


“Are you ideological?!”


“Yes, I’m ideological.”


“I can’t understand, you’re a code-bound criminal\high{\footnote{A top-rank criminal according to the Russian prisoners’ code.}} or what?!”


“No.”


“So, what the f\unknown{} is your idea?!”


“I’m for freedom!”


And the guard still yelled: “Shit! Get the fuck out of here!!!”


\unknown{}What a drama queen.


Next morning the torment continued. They pull me aside on the way back from the toilet. This time the masks gathered together, four or five. They block my way and order me to bow my head. I refused. A couple of hits, no reaction. They make me do a stretch at the wall. They wonder if I will continue to refuse. The answer is positive. A sharp hit to the legs, knocked down, I fall on my knees and elbows. Batons, legs\unknown{} They try to lift me, but my mind is blown, there is a red mist in my eyes. It is not me any more. I wrestle out of the holds, spin on the floor like a top. They immobilize me, hand-cuffs, click bracelets on the wrists. They drag me to the gym, make me stretch out violently; my head comes to the wall. The masks stretch my legs with their boots, the soles tear the skin of my shin. I can make out hits to the stomach here and there, but don’t feel pain any more. There is a lion’s share of adrenaline in my blood. They place a working taser close my face. It’s scary, but I just grit my teeth stronger. Negotiations. We agree that I will only look down, to the order “head down”. At least something. Surreptitiously they clean my wounds with hydrogen peroxide.


One day later I register with the medical center to report the injuries. There is a hematoma on my forehead, abrasions cover my knees and elbows. There is a scar on my shin. Lips, an ear – it’s more than enough. However, instead of visiting a doctor the whole cell goes to the prison governor. In a spacious, well-equipped office there sat a man of small stature, but with an imperious and self-confident face.


“Are you a terrorist?” colonel Orlov asked rigidly.


“No.”


“Why did you beat two guards? I have a report here. One of them had to take a sick day. The other one has an injured hand.”


Yah! I told everything like it was, but the governor only upheld of the actions of his subordinates.


“It’s like in the army here,” continued Orlov. “Discipline requires punishment even for the innocent. I need discipline and do not want any trouble. As you can see, the country has faced great hardships.”


On the way back to the cell I realized that everything was controlled here and those events had not been an accident. Nor was the appearance of Orlov, who replaced a former governor right after the elections. The situation became absolutely gloomy.


\chapter{7
}

Days, already bleak, started turning into torture. It all started at 6am with the roar of masks in the corridor, when people were flushed out to the toilet\high{\footnote{In this particular prison there are no toilets in the cells, prisoners have to wait for a special time when they are taken to the toilet in the corridor all together. They have a bucket in the cell for emergencies.}}. Sharp baton hits on walls, banisters, the floor, constant yelling “Head down!”, “Hurry!”, “Move!” with a voice of an SS-man shouting “Schneller!” to Jews in Auschwitz gas chambers. Clang of the door, and it moves to the next cell.


All together it created a imposing and rigid cacophony which suppressed the will and fed the fear. After the morning round of a guard, everything followed the trodden path. From 8.30am when the first shift of prisoners went for a walk, then every 1–2 hours up to 12.30pm when the last shift came back. Six jog to the outside, six back to the cells, exactly by the number of exercise yards. If there were less times we concluded that some cells had refused to go out. Over time we began to notice that guards yelled a lot at some prisoners, less at others, and at some of us they didn’t shout at all. A differentiated approach.


From 1pm to 3pm – lunch. A couple of hours of respite. After 3pm began the second round: shakedown. If earlier searches of the cells had been carried out once a month, now it turned into a weekly procedure. Usually we were expelled to the gym where we had to undress and squat a few times. After the examination of clothes we were made to do the stretch at the wall, often with palms curved back, like LPs (life prisoners). Once Max and I stood like that for half an hour while the search was going on. I remember, the first time we were staying like that it was ‘fun’ for five minutes, but anyway it was a torture, after which it was very hard to move your legs. After 30 minutes you don’t want anything. You just hold on trying not to faint, with a pool of your own sweat under your feet and a wild tremor in your hands.


At 4.30pm followed the second walk-out to the toilets. All the same as the morning scheme. Shakedowns again till 6pm. Dinner. At 8pm a new shift of guards started duty, who also tried to catch up for “operations”. They usually bothered us with a so-called “pat-down search”. We had to collect everything, curling the mattress with sheets, pack the food, etc. Then we went down to the gym again with all the trunks, forced to carry everything at once, we were only allowed to bring things out separately in the first days. The guards tossed the bags, searched our belongings, letters, packages; then we had to squat again. And all the time they hurried us: “Move!”, “Faster!” If they didn’t like the speed there was another round. We didn’t pack things tidily, we just pushed them inside the bags. We were short of time. There were many other prisoners, waiting for their turn. Then began the most difficult part – the way back. At first we walked, later we ran. Eventually, it ended with multiple races. On signal we had to run up the narrow steep stairs, loaded with trunks and mattresses, eternally dropping sheets. Almost at the finish of that race the masks would stop us and force us to go back down. And again upstairs\unknown{} Even the most physically strong man couldn’t stand that! You would crawled to your plank bed like a steaming horse and didn’t even unpack; you didn’t care for anything any more.


Having tortured us “physically and spiritually”, the punishment began on the brain. From 6 to 10 pm local “jail TV” (common TV was turned off in December) started broadcasting programs, 90\% of which really sucked. Mysticism, pseudo-history, Chechen fighters, terrorists, politicians, drug addicts, the Jewish plot, the blood-sucking dollar, to put it short, front-page stories, aimed at intimidating the population. All fine and dandy, but its repeated every day. Dozens of times the same thing. They battered our brains with anxiety and danger. It was most likely aimed at the development of neuroses, first of all neurasthenia. That TV-zombieing was the worst of all. Sometimes it brought us to panic and self-destruction. Besides, programs of explicit far-right content were broadcasted. They showed movies, “Russia with the Knife in the Back”, etc. It was completely idiotic, especially when prisoners were told that Putin was a Jew and Russia is a Zionist country. Wardens, accompanied by the masks with batons, came in from time to time to check if we were watching. Over time we started cheating – had the sound as a background and later switched it off all together.


A walk in the yard for 1 or 2 hours was an outlet, despite the gloomy dull grey walls and its small size of 3 to 6 steps (there were smaller yards too). The guards in the watch tower turned on a radio (but it was disconnected over time) or CDs, sometimes with some quite decent electronic music. But the masks managed to spoil our lives even there: they forced us to walk around, firstly with our hands behind our backs. If you refused they took you back to the cell. As a result, some of the prisoners absolutely refused to go for a walk. Guards started to force them to go. It’s not easy to walk around for 2 hours, when in 15 minutes the floor turned into an ice rink. Naturally, no one thought of gritting the ice. Only a few weeks later, when the snow started thawing and people fell down in the greasy slush everyday, they put down some grit.


Our letters disappeared abruptly. In December I’d managed to receive a whole pile, but since January I had got almost nothing. Only some letters from certain people reached me: from parents, relatives, a few friends and single letters from comrades where it was difficult to see read anything political in them. The time between letters went from several days to one month, depending on behaviour and conversations in the cells. On average, it took two weeks for those that were sent from Minsk to arrive.


Letters are a tricky thing! They cheer you up, especially when they describe various triviality from normal everyday life. But the guards could insidiously give you a false impression about the real attitude of people towards you and the real situation in the world by regulating the stream of letters and filtering the post. For example, several equally important people could write to you, but they’ll let through letters of only one of them, and then they could even cut that person off. You start to think that you are forgotten, and no one needs you. Moreover, they would bundle together several letters containing negative information. It’s hard. Certainly, the mind repeats thousands of times that it is not true, that you shouldn’t take it seriously, but a worm of doubt perforates the subconscious. It is impossible to escape it. And it is what they expect. Under the conditions of such an information vacuum, a selective supply of information influences you, whether you want it to or not. In this situation it makes sense to repeat as a prayer: “The time will come and I will learn everything.” That is what I did every day. Despite the technology of filtration, they’d miss somethings. A couple of letters, not by their contents but just because I had received them, orientated me and opened up a whole layer of lies from the investigators, so I knew some aspects of my case and could count on them. The information was worth more than rubies. News from friends which I managed to receive right at the beginning gave me additional support. Friends\unknown{} How many years of a common way, cheerful parties, frivolous adventures, sincere understanding. It seemed to be eternal. Who could think that instead of storming the wooden walls of a castle in armor and helmets\high{\footnote{Ihar used to be involved in medieval role-plays before prison.}} you would storm the confines of the “red house” with letters from the outside. Each message with words of support that I received was priceless. Those words excited my memory, didn’t allow me to forget who I was, who I am, and didn’t let the punishers mold a blind obedient puppet out of me.


\chapter{8
}

At the beginning of January there were some rearrangements in the cell, which re-defined the future composition of the experimental enclosure No 4.


Volodya and Lebedko were moved away and replaced with Sanya Molchanov and Alexander Feduta. Both were political prisoners.


Molchanov was picked up in early January in Borisov – he was identified on surveillance cameras. His appearance was that of a painfully thin, idealistic student. And maybe it was because of this, he was suppressed especially hard. He was regularly subjected to forced marches. Quite often they grabbed his neck and head, shouted and humiliated. Security officers actively ‘gave him the works’, forced a public confession out of him and videotaped it. I remember once the door opened and one of the masks entered the cell, but without a mask (!) and with a baton raised. We thought that now a masks-show\high{\footnote{A word-play. Masks-show was a popular comic show back in the 90s. Masks-show in prison means beating of everyone in the cell.}} would begin, but he pulled only Sanya out and brought him to the corridor. An with a look as he if was being taken to an execution. Kirill’s question “What are we going to do, guys?” hung in the air.


Over time we got to know Sanya. A participant in the democratic movement since his younger years, he had been spotted at different activities by his twenties. In time he realised that he was being framed by the KGB, and by some miracle he managed to get away from their surveillance.


In his free time from activism Sanya was fond of stalkerism\high{\footnote{Stalker is a computer game and a movie where the hero wanders in abandoned nuclear power plants and similar environments. A stalkerist is a person who tries to do the same in real life, visiting the Chernobyl zone or other secured areas.}}, reading, and the Internet. Such a positive way of life made Molchanov some kind of a TV-star of the square on December 19. He was captured grabbing a national flag from the KGB headquarters and waving a white-red-white flag\high{\footnote{Historical national flag of Belarus before the communist times, the symbol of the opposition, was also the official flag of Belarus 1991–1995 from the proclamation of independence until a referendum staged by Lukashenko.}} on top of a snow-removal machine\unknown{} After any act of oppression Sasha didn’t despair and swore at security officers like a sailor, even though he knew that the cell were under surveillance and bugged.


Alexander Feduta was a 46 year-old man in glasses. A large man, he was a large political figure too, during the elections as well as generally. He was a political strategist and chief of the electoral office for candidate Neklyaev. He gave the impression of a great humanist of the 18\high{th} century and a mysterious power broker. Alexander was almost spending his nights in interrogations. 8–12 hours with the inquisitors belonged to him every day. Despite his role as a political shark, we quickly got on well. Alexander used to be a school teacher, the leader of Komsomol\high{\footnote{Communist Youth Union in the USSR.}}, and a journalist who had met with Gorbachov. Then he became one of the fat cats in Lukashenko’s crew, and now he was his prisoner. Life is life. Apart from politics, Alexander appeared to be a professional writer and a remarkable story-teller. We took advantage of that characteristic. I remember how the whole cell used to listen with bated breath to his retelling of “The Count of Monte-Cristo”, as well as a lot of travel stories. It was interesting to watch the faces of cellmates when the Count carried out the next act of revenge. For sure, everyone twisted that situation in their own imagination, trying on the role of the Count. There were a number of characters, exactly four of them, in my murderous fantasies, too.


The harder we were pressed, the louder the laughter was in our cell and the more actively we played board games. There was the most popular prison game, which Feduta, being a genetic intellectual, immediately named “tarakashka” (cockroachy). Certainly, we took advantage of this situation to make a mockery of it.


Exclusive solidarity in our cell helped us to resist the on-going nightmare. Strong laughter and black humour served as a mode of psychological protection, as the mind couldn’t put up with what was happening otherwise.


In terms of food supply we constructed a communism in the cell. As a rule, for lunch and dinner we sat down all together at the table (a bedside-table covered with a newspaper, to be more exact). Usually Kirill, and later me, made a salad. Vegetables, onions, greens, garlic, bread, lard and sausages were cut. I nevertheless returned to being a vegetarian and even talked Max into it. Tea, coffee, cookies, sweets, fruits usually were shared. Everyone simply knew when to stop with a scarce product and there were no problems.


But nevertheless, with increasing mockery from the masks, there were days when almost no one spoke because of fear and despair. We resisted: we whispered more often in a dead zone hidden from the camera. When we were forced to run (with hands behind our backs and bowed heads), we put Feduta to the front. Thus, nobody lagged behind and the last wasn’t hurried on with batons. The masks tried to destroy our solidarity. As if intentionally the shakedowns and pat searches were carried out while we were playing a board game, when in a gust of excitement we would have forgotten about everything. Or the guards would only searched some of us, and not the others, to cause envy and suspicion. We were not tricked by it, but it left behind some food for thought. That was what they expected, as we later realised. Deception was wide spread. Thus, after another interrogation, Feduta asked right off the bat: “Guys, what have I done? Why did you write an application for my transfer to another cell?” We were taken aback by such a bold lie. However, there was nothing to be surprised at: our cell was passed off as a press-cell\high{\footnote{A press-cell is a term used for a cell with particularly brutal and aggressive prisoners, who cooperate with the prison administration. The cops usually use the idea of press-cells to frighten other prisoners because if sent to one you may be beaten, tortured, raped or psychologically oppressed by the other inmates. They are also used to extract confessions.}} with drug addicts and terrorists; other prisoners were scared of it. One shouldn’t think that such deception only took place because of people’s stupidity and trustfulness, after all, the master-classes of manipulation were performed by the new governor, colonel Orlov himself.


\chapter{9
}

Being summoned to the governor of Americanka was a separate issue. In all that madhouse only his office was a lagoon of calmness and tranquility. Usually he met you in a very friendly fashion, he was able to speak entertainingly and listen. He said that he had participated in the operations in the mountains, had done some time in an Asian prison, didn’t drink, didn’t smoke. Somehow you could not believe that it was he who tormented us like a cat playing with mice, who conducted all this tragicomedy. He didn’t make clumsy mistakes like simple investigators. He could unsurreptitiously ask your opinion of a cellmate. He could force you to think something that wasn’t your opinion, only by making you reject a more unacceptable position. I remember how he suggested looking at the photos showing that Lebedko didn’t observe the hunger strike. This was a bad shot because he had played the same trick on someone else earlier. The computer “suddenly” got stuck when he tried to open the photos. The same thing happened to me so I wasn’t surprised. After all, professionals make mistakes too. Besides, I personally saw Lebedko weakened, he was shaking and his face turned pale for some days. Orlov talked about the situation in prison as “the answer to a challenge which our country had faced.” He compared the opposition with French revolutionaries who pushed the circumstances to terror, and “you never know whose terror will be worse.”


All the time I had to be very alert: watch every word, follow the train of thought (his and mine), watch the dialogue in general. It was challenging. There was always a danger of expressing an opinion or some detail which he would be able to weave for plausibility in conversation with other inmates, thereby sowing doubt and discord. It was impossible not to talk to him. Many times I entered with an intention to bring the conversation towards monosyllabic answers, but over and over again Orlov managed to draw me out. The same as that colonel from the 4\high{th} squad, they knew their craft and did it well. There was a beautiful tea laid in the office, gingerbread, cognac – everything like in the movies. No tea with the punishers! I said this once and repeated it every time the governor offered it to me. For some reason, people think that it is not an issue. Like, if a cop behaves gently and decently, you can drink tea with him. Thus you find valid all abuses and deceptions which prisoners are subject to. Jailers are divided into two kinds: bad and very bad. This is a fundamental truth. Any courtesy from their side is an element of a web, aimed at creating a subconscious trust. They practice ‘adaptation’ by admitting a system of values (agreement with a number of your thoughts), as well as ‘mirroring’ (copying postures). Alas, my knowledge in this area is very modest, and the biggest part of that psychological siege remained invisible to me.


I counteracted. As a protection, looking at the governor with a half-smile, I repeated in my mind over and over again: “He presses, smothers and humiliates me, and brings pain to those closest to me. He is an enemy. Everything he says is a lie.”


Hatred to an enemy\unknown{} A friend of mine once said that while seducing a girl it is necessary to undress her with your eyes. I adapted that approach to the present situation in the following way: I imagined how I’d grab the governor’s throat with my hand, reaching him over the table, and suffocating him; he rattles, froths at the mouth, his eyes bug out, and he is in panic unsuccessfully trying to unclench the steel grasp. It was the perfect help!


Orlov openly discussed methods of torture, the moral aspect of the cases. He claimed that his goal was to make us doubt. He didn’t reply to my objection that we had not been convicted yet. However, it was already clear that for them a person is guilty not when the court passes the sentence, but when the person falls under suspicion. After all, they cannot be wrong! It is the KGB. Mere mortals don’t compare to them!


From those conversations I learned that they practiced an individual approach to each cell and to each person. In total there were 18 cells, with about 60 prisoners. They didn’t need large facilities and lots of staff. Everything was well-planned: when, where and for how many times. Where to have a shakedown, who to individually torture or humiliate, where to leave lights on for the night, where to forbid smoking. They even planned who to hurt during runs to the toilet or on a walk. While I was in the office, Orlov called the guards and gave instructions not to touch Molchanov for two days. And, of course, the cells were shuffled to complicate life or make it easier.


“The world is a pack of wolves. And a stronger pack always tries to pick on their weaker neighbour. I identify myself with my pack, and its well-being is the well-being of my family, friends and fellows,” Orlov speculated on his general outlook.


I had something to say about those patriotic arguments.


“It’s classic fascism. Read Mussolini, you’ll like it”, I commented. “Yes, the picture of civilization is similar to what you describe, I won’t argue. And it’s clear with the wolves. But what should simple antelopes, who drag these wolves on their backs, do in your world?”


It looked like antelopes could only sit and sweat their guts out, carry out food rationing systems\high{\footnote{Ihar here makes an allusion to a Soviet policy and campaign involving the confiscation of grain and other agricultural products from peasantry for a nominal fixed price according to specified quotas. It allowed the Soviet government to solve a significant problem: supplying the Red Army and urban population and providing raw material for different industries.}}, provide the skin for the production of military belts and backpacks. In the meantime, antelopes, depending on their behaviour, were brought to the governor either gently with no hand-cuffs, or bent-double. That is when the shackles clap behind your back and they twist your arms up so high that it is possible to kiss your own boots if desired. And in this position you are driven through corridors and stairs, like a life prisoner.


Once they forgot instructions for me. They called the governor who specified exactly how to behave.


“I’ll make a Guantanamo here for you all,” threatened Orlov. That was how Americanka became Guantanamka.


\chapter{10
}

At the end of January they took away Feduta. The cell had become totally bored. I spent time reading an encyclopedia of psychology which had been bought for me after a few weeks of my regular formal requests. In the evenings I learned to draw using a self-teaching manual sent by my parents. Alternatively I did push-ups on the floor or between the plank beds, like on gym bars.


Sometimes I heard the chilling roar of masks in the corridor – as usual they were mocking someone. That led to an outright panic. Only sports allowed me to regain self-possession.


We quietly discussed hard-core scenes from the movie “Saw VIII”, of course, with guards playing the main roles.


Hatred brimmed over and imagination bubbled in bright bloody colours. Laughter was hysterical more and more often. Specific prison humour, created by critical life conditions and a closed space, made our humour look like inhumane treatment, replacing normal human jokes.


“There was of course no way of knowing whether you were being watched at any given moment. How often, or on what system, the Thought Police plugged in on any individual wire was guesswork. It was even conceivable that they watched everybody all the time. But at any rate they could plug in your wire whenever they wanted to. You had to live – did live, from habit that became instinct – in the assumption that every sound you made was overheard, and, except in darkness, every movement scrutinized.”


These words from George Orwell’s novel “1984” best reflect the psychological atmosphere which prevailed in the soul of every prisoner. Only we had the lights on all the time. There was a daylight lamp of 100 watts, and the night time light, so bright that we could read. It happened that for several days we slept with both lights on. A guard would helplessly reply: “It’s the order.” With every new week it was getting worse. Systematically, step by step. We half-consciously waited for an outcome, because it was impossible to continue living on like this. We were feeling our the limits.


Then the “second wave” appeared. One thing is to feel loss, an impotence to prevent something. But while this “something” exists, you realize yourself, your position, who you are. And then it starts to seem that there is nothing indeed, and there won’t be anything any more. Barren monotony is killing so much that you no longer imagine another way. A stone bag in the emptiness with a simple set of external irritants, always the same. A feeling of timelessness, with no beginning and no end. You are lying on the plank bed, you can not stand up, because it’s unclear what is more important – a meteorite in space or a cup of tea while you have a boiler\high{\footnote{In general, any electric appliances are forbidden in prison and prisoners cannot cook in the cell. But the administration may let you have a domestic immersion heater, a thing that you can put in a cup with water and boil it .}}. Schizophrenic disintegration of consciousness. And then you crazily make a salad, cutting everything you have, do push-ups in between, force yourself to play chess. Doing household duties every day is like a ritual, it becomes your armour against the madness. But another guest visited us quite often – the fear. Then only a peculiar self-therapy could help.


Sometimes I curled up in the bed, covered myself so as not to see or hear anything, and secretly looked at a photo sent by a stranger from St. Petersburg. It depicted a solid black cloud in which you could barely discern the outlines of a stone figure, clenched in a fist. It seemed, that the cloud had absolutely covered the stone, but it still stood indestructible, as a beacon in the fog. How many people went through prisons, labor camps, persecutions and tortures. Many times I read about them, and I knew those people had had hard times. How many of them perished in extreme poverty and obscurity. But anyway, they still went on, they didn’t reconcile. This fight, the opposition of freedom and slavery is a common thread passing through the history of mankind.


Epochs, civilizations, names changed, but the essence remained the same: the antagonism of aspirations of a simple man and aspirations of the masters (clans, faith, money, position). Man vs Power at all times. I am only a small particle in this environment of feelings, thoughts and actions. As a drop in the ocean: without me there won’t be less water, but complete, it consists of such droplets, and each of them makes a contribution to the general rhythm of the oceanic beating. Let the punishers do what they want to me, I have won anyway\unknown{}


\chapter{11
}

In early February the masks became utterly brutal. It was Friday, three people from our cell fell ill with flu. Shaking from fever and shivering all night long. Next morning we visited a doctor and he prescribed some pills. In the evening they ordered me and Molchanov to go out with all our stuff. Taught by a bitter experience, we left out books and other heavy things. As usual, we were brought to the gym, but for some reason at a slow pace and without yelling. It was suspiciously silent. It didn’t mean anything good. They turned our bags and packages inside out, dumping the contents into one pile. We were waiting for something, naked and barefoot on the concrete floor. And it began.


“COLLECT THE STUFF!”, “MOVE!”, “WHAT IS UNCLEAR?!!”, “HURRY UP!”, “I SAID IT!!”, “RUN!!!!”


We run through the stairs and corridors. They clap us into a madcell (a special cell in the cellar, upholstered with rubber and faux leather for the suppression of violent prisoners). One more shakedown. We are made to do the splits and left alone, but they approach the peephole from time to time to check on us. Collecting the stuff again, I hardly manage to put on my clothes, grab the mattress and run along the corridor to the stairs leading to the center of this neocoliseum. At the finish we see a mask, who shouts: “Too slowly, go back!” The other one catches up. I hear how on the other staircase they chase Sanya Molchanov. Bastards! I come back to the madcell. It’s wildly hot, sweat flows like water, I have a mental block. It is a limit. If I cross it, what happens next? That’s it, I don’t care any more.


The punisher shouts:


“Take two! QUICK MARCH!!!”


“No.”


“I said, take the things and run, faster!!!”


“I refuse.”


“Take the stuff and run!”


“Do what you want. I won’t run any more.”


He looks at me for some time, then goes to the another mask and discusses something in a whisper. Then the other one approaches me and indifferently, in a most polite and peaceful tone says: “Take your things and go to your cell.” I was taken aback, couldn’t believe my ears. His tone contradicted the situation. It turns out, that they are able to speak properly. This is the end\unknown{}


\unknown{}Without laying the mattress out, I fall right on iron rods. I am powerless. My head is shaking. Sanya barges in right after me. He is barely alive, as pale as death. He feels bad and sick. Deathly silence hangs in the cell. Kirill and Max haven’t been treated like that yet. Everyone was scared. Scared?! Horror penetrated so deeply into every atom of body and mind, that everyone was sitting silent, afraid to drop a word\unknown{}


I feverishly think what to do. The situation approached the line after which it is already impossible to respect yourself. One more step and anything becomes possible. There comes a clear understanding that such things cannot be tolerated. Point-blank. You must refuse in the beginning, do not collect your things. If it’s not the next chase, but a transfer to another cell, a duty warden will come. But he is never present. He is intentionally absent, he doesn’t want to be a witness. This is cunning, bastards.


The following morning a guard asked: “Well, who else is sick?”


A few days later the most hard oppression in all that black winter happened. The yelling was so loud, that the sound reached us distinctly from the gym through two doors and a central hall. We could neither read nor write, play, even simply lie in bed. Someone paced up and down, someone did push-ups. A few hours of expectation, all the time we heard doors of the neighbouring cells open, but for some reason the punishers skipped us. We heard the yelling from outside: “Lie down! STAND UP! LIE DOWN! STAND UP!..” Terrible. Nobody looks each other in the eye and only lips move: “Bitches, beasts, bastards\unknown{}” Dinner. It means the danger is over for us this time. But for how long? Miracles don’t happen twice, next time we will be hooked for sure.


On Saturday early in the morning I was taken to the governor. Orlov, dressed in civil clothes, met me hospitably. He asked immediately: “What is bothering you?” Then he immediately said something about my condition and that he had been instructed to dispel the “misunderstanding”.


“After all, this is nothing more than a theatrical performance,” he said in a confidential and benevolent tone. “We closely watch the condition of everyone, and I can assure you, nothing threatens anyone.” He added: “The main enemy is your own fear. You don’t have to take hasty actions and any problem can be solved right here.”


I was shocked. How did he know that? Was everything I’d heard about the psychological control of prisoners just another rumour? I asked him point-blank:


“How did you know?”


“By your eyes”, said the governor of the KGB pre-trial detention center.


\chapter{12
}

Mid-February was marked by the beginning of ‘climate warming’. Something obviously happened outside the prison walls, and the level of pressure decreased considerably. Chases with the so-called pat-search disappeared. Shakedowns became softer, rarer, with walking to the toilet like the old times.


On February 23\high{rd} Kirill fell down the stairs. Out of habit set by the masks he ran down with his hands behind his back, without holding the banisters. For the first time I saw a hematoma the size of half his back. Kirill had a painful shock. He was shaking so much that he could not speak properly, not even smoke. As a result, he was taken to the hospital. Judging by the crumbs of conversations among the guards, there had been similar cases before. Anyway, the following day the masks shouted at us because\unknown{} we ran without holding the banisters!!! Every day we saw the masks less and less, and finally by the beginning of March they had gone altogether.


It was a load off our minds. It became easier to breathe. We didn’t go around in circles during the walks, but repeatedly heard our neighbours forced to do so. The differentiated approach remained, but became more discerning.


It was interesting to observe how the rank and file was changing in the conditions of increasing authority. From polite and good-natured, some of them became complete bastards. For example, there was a pair of guards remembered by everyone: Vasya and Frog. They had pulled me to the masks in the corridor just because I hadn’t stood up when the door opened. They made me do the splits and surrounded me. Frog hit my leg so hard that I could hardly keep my balance. It was evident that those two wanted to prove to the masks that they were ‘tough guys’ too. Pathetic fools, with in a month they were toeing the line. There were also those who didn’t turn into cattle and remained humans. But anyway you must understand that a brown-nose is a brown-nose. Those decent ones had to follow orders as well. Even if they didn’t carry out the torture, they brought us to those who did. The problem is not in the people, it is in the system that allows the creation of lawlessness.


Sanya and Kirill were convicted. Molchanov picked up three years, and Kirill either got a custodial sentence at a penal labour settlement or a conditional release: he didn’t return to the cell. This sobered us up a bit: it appeared, that there was a way out of this submarine. What was remarkable was that neither Sanya, nor Kirill had an opportunity to meet with a lawyer.


Approximately at that time we learned that Mikhalevich – one of the “Decembrists” – claimed to have experienced torture in Americanka and fled to the Czech Republic. In amazement we read (we began to receive Belgazeta\high{\footnote{Belgazeta – a Belarusian independent newspaper, which tries to ridicule the government as well as the opposition.}}!) that his behaviour provoked some criticism. Probably, the critic didn’t understand that the life of a political refugee is no bowl of cherries. A foreign country, strange people and the only hope to return home being the change of the regime. Every day we were rotting in that hell and no one knew what could have happend next, had it not been for his self-sacrificing act. Many prisoners of the red house at that time were grateful for it.


The prosecutor’s investigation of prisoners’ rights was organised, but it was a farce.


However, at that time I cared for other news which had filtered through in spite of near complete isolation. All those months I had been tortured by thoughts about my friends. Had Dima managed to find a shelter, had Sasha and Kolya managed not to give up (I sent them a verbal message with Molchanov) and what actions would Denya undertake? If all the other people who had given evidence were mediocre to me, the situation with Denya was radically different. Long ago we met at a punk-concert and got into trouble with fascists: we took revenge for Toro Bravo’s concert failure, right at the October square in Minsk, with the force of 150 people. During those years (the end of the 90s – beginning of the 2000s) a non-commercial DIY music scene was created and it required protection because right-wing ultras didn’t tolerate people who openly said “Fascism is shit” and “Fascism must die!” There was an invisible war in the streets, something constantly happening: concerts, parties, meetings, skirmishes, violent confrontations, political actions. It was the world we lived in, fought for, the world which enriched and formed our personalities, tempered our characters, refusal to compromise and our will to win. So we lived through many troubles together and actually won during those years. Though after two or three years I left the subculture, and Denya had focused on football hooliganism, we didn’t lose sight of each other and always helped each other in difficult situations. Especially last year when I experienced hard personal losses and the world around me seemed to be very grey, I became cold-hearted. It was the communication with Denya that helped me to overcome that black line. There are not many people in life who you can call real friends. But in prison you understand that your number of real friends is much smaller. By that time it was clear that his evidence was not important, Vetkin’s words were quite enough. The issue was not in those words or the sentence but in what was stronger – the fear or the friendship. No matter how long the sentence was it would pass, but a true friend would stay forever. I had serious reservations, but I didn’t stop believing. And when I learnt that Denya published a video on the Internet revealing his false statement and then went abroad, I felt euphoria for three days. Those beasts couldn’t always suck our blood! They could oppress and scare us as much as they liked, but there were things which defeated them. The day will come when these human values will break the back of this despicable authority.


There were some allusions to solidarity in my mother’s letters: the guys called, came to visit her. It made me stronger.


The investigator, who hadn’t made his appearance since last year, suddenly burst in with the lawyer and expert evidence. The lawyer said that he hadn’t been allowed to visit me “for absence of technical possibility” like many others. I wondered who had been allowed at all?


During that meeting I learnt that Dima hadn’t been arrested and I showed my delight, disregarding the presence of the investigator.


There was nothing at all against me in the expert examination! No phone conversations, no correspondence, not even a trace in a computer’s memory, no proofs after searches in flats and in the car, no coinciding mobile calls. All in all, my vitality improved. I only had to wait for the access to the case and then the trial and a prison camp. But that “only” lasted forever.


\chapter{13
}

In early March two more people appeared in the cell substituting the departed. Sergey Martselev turned out to be a political strategist of Nikolay Statkevich, the candidate for the presidency. At first though I thought he was a Polish criminal. Max and I were taken aback by his facial expression when he entered the cell. But he proved to be one of us through and through. In another cell he had got a nickname ‘Student’ because he had three degrees.


Aleksandr Kiselev was a big Russian businessman and concurrently a local sight of Amerikanka nicknamed “Oligarch”. He was involved in investments that increased a company’s capital following resale. According to the market and bureaucratic capitalism Aleksandr must have been a good guy. According to the KGB, he was a criminal. Kiselev was a highly organized man, he applied a constructive approach to everything. At the same time he was strong, good-natured and positive. Governor Orlov hated him. So did some guards (for example, Vasya didn’t give Oligarch bread when he distributed thin broth!). He was constantly thrown from one cell to another (he was in 14 out of 18 in total) and transferred to Volodarka\high{\footnote{Another pre-trial detention centre in Minsk.}} to apply additional pressure. But Aleksandr was firm and used to say: “Let them make me rot here, but I will never give up because there should be limits, and the KGB don’t have them.” He did physical exercise daily. Every day he learnt German, every day he had to stick to his point. In short he was an iron man.


The whole week we discussed world capitalism and the financial crisis, prospects of the Belarusian economy, workers’ control and self-government, the criminal origin of Russian politics and, of course, the lawlessness of the Belarusian regime. We also got to know the details of the ‘eternal winter’ in other cells. It involved the dangerous cell \#13. Orlov once mentioned it as a cell where they kept ‘people not content with their lives’. It turned out that there were just two approaches to the inhabitants of that cell those days: a shakedown every other day and a shakedown every day. They emptied their bags out on the floor. The masks could enter and break tea containers with their batons. They could beat you just because a prisoner made a complaint against them during a trial; they pulled another prisoner by the collar and, intending to throw him into the yard, smashed him against the wall\unknown{} During a pat-down search they made you do the splits naked, while a guard asked dirty questions.


By the way, there had been a prosecutor’s investigation the day before Martselev and Kiselev appeared. Deputy prosecutor Shved himself with a group of white collars was so kind as to visit us. Governor Orlov entered the cell for a few seconds, gave a strict glance around and then left. The prosecutors entered. Shved asked a few times if everything was OK and that was that. Outside, in front of the door the guards lined up; they looked like they had come to a final showdown. Naturally, we kept silent. Nobody believed the prosecutors. We had already seen in December what their investigation was about. Anatoly Lebedko had been called to the doctor (for the second time that day) during the same half hour when the prosecutor visited the prison as part of his monthly investigation. Of course, Lebedko would say many things. Was there any sense, though? The prosecutor never even asked where the missing prisoner was. It was the same that time. There is a good illustration for the situation: a monkey which closes its mouth, ears and eyes.


10 days later Kiselev was transferred. A young boy, Denis, joined us. A talented car mechanic, he left at home his mother and a fiance. For the first time I happened to see the behaviour of a fresh prisoner from outside. It was painful to watch him realize day by day, step by step, that with a high probability, everything – the business, the mother, the fiance – remained behind.


There were no letters during the whole of March. No news from relatives. The only thread which connected me with the world was broken. It seemed that someone didn’t like my speeches at all. Though it is quite difficult to understand their logic, which is why I didn’t even try to. Reading, drawing and conversations helped us to waste the time. Max and I quite often recollected the good old days from antifa-action, football-hooligans’ maneuvers and punk concerts. Sometimes we discussed how we would create a cafe-club with a cyberpunk style. With Serega we could argue about history, for example patriotism during the World Wars. Or, what should a decent man do in the case of NATO occupation? Martselev told me about aspects of political technologies and advertising very cleverly and eloquently. Feduta spoke in a general way about the functioning of electoral headquarters. It is a serious thing. A machine! I learnt that during the electoral campaign a candidate is a totally dependent figure – he should follow all the directives of his staff. What he wears, what he says, with whom he meets – it is a headed by strategic staff. So the real geniuses of the elections became not the candidates but the headquarters’ chiefs. Listening to those stories I understood two things: how much we are inferior in the public sphere and how a ‘democracy’ is an illusion in reality. The people are just an electoral mass and a spare trump card. Its mood is manipulated and used for mercantile aims just as Bolsheviks had glorified the proletariat. In the name of the nation they deliver pompous speeches, but in reality we have only a bourgeois democracy and its political apparatus. In order to keep the nation away from control, the first bourgeois republics introduced a property qualification so only wealthy citizens managed to filter through it. Only when the bourgeoisie were convinced that the right to elect was no threat to the domination of bourgeois parties, the property qualification was abolished. The people, for a politician, are like the sea for a seaman: just a means for mobility, the source of income and a big source of folklore. But the people are also an eternal indication of disaster, sweeping anything and everything in their path.


And yet, in spite of the absence of outside financing and the lack of knowledge of social patterns and technologies, anarchism has two key advantages – unlimited enthusiasm and pure truth. Technical capabilities are just things one can gain.


\chapter{14
}

At the end of March, finally, I was introduced to the case. They incriminated us using article 218 “Damage of property”. Sanya’s and Kolya’s sentences were from 3 years right up to 10, my sentence – from 7 to 12 years. As it happens, such unreal sentences can appear out of the blue. I thought that such a big sentence could only be given for sabotage, terrorism, murders and so on. Firstly I was looking for the guys’ evidence in order not to be tormented by the question: “Did they manage to stand up to the questioning?” Everything was all right. They managed. Also I wondered what other witnesses had said and on what evidence the case was built in general. Fingerprints, an old technique of detectives that is a relic of the past. Odor (perspiration), skin cells, saliva and even air in a closed room are real identifiers of a person. They found gloves with Vetkin’s sweat and traces of fluid which could have been the same as a single mass of substances found on the broken fragments of a bottle. They also traced his telephone-call from the nearest bus stop. Factually, those things didn’t prove anything, but he had made another decision. Vetkin had borne it for just five days and then capitulated. At first he named me and an unknown person as accomplices then added Dima Dubovski and then changed Dima to Denis. That’s how you can exchange comrades for punishers’ mercy and an illusive hope for freedom. A snitch is a snitch, he will burn in Hell.


The interrogator lied about Sasha\high{\footnote{Meaning Aliaksandr Frantskevich.}}, that he had lost heart, and about the sentence: that it won’t be severe and it’s not him who decides it, and similar crap. But the fake excuses won’t be believed – he is an accomplice of the judicial-investigative lawlessness. He tried to throw dust in my eyes about responsibility, but he couldn’t get one thing: I had put my bet on the trial and a long sentence already. I only wanted to see my comrades and leave that madhouse for the colony as soon as possible.


I received a newspaper with an article about Feduta’s release with travel restrictions. But we couldn’t celebrate fully. There was one problem – it appeared that he had been transferred from our cell to another, a transit one, where he had been alone\unknown{} for 55 days! I doubted anybody had paid attention to that fact, but we understood everything without hints. The worst of all torments is to be alone and hear how they torture others every day. On the 3\high{rd} day you want to hit the ceiling, on the 5\high{th} you come undone. I spent two weeks in a solitary cell (the maximum term of punishment in a disciplinary cell) and I was happy to be transferred, to be with people. But almost 2 months\unknown{} a nightmare. Especially when you hear other people being tormented. We discussed for a long time how our writer had spent those days, and nobody joked or smiled. We reached the following conclusion: every prisoner of Amerikanka had rough times, but the most terrifying lot was given to Feduta. It’s not about how not to fall to pieces. It’s about how to survive.


***


April. The case was at the Prosecutor’s. Martselev was waiting for his trial, Max was wondering if a second case would be started, Denis wasn’t waiting for anything, it was just the beginning for him. On the night of the 11\high{th} we heard a weird noise from the avenue, as if there was a fire. The duty guard ordered us to switch on the state channel (although officially the antenna was out of order). Thus we got to know about the terrorist attack in the metro. At first we only thought about our relatives. The next day they brought the list of the casualties. It was awful to look for familiar names on such a sheet. Later I learned that colonel Orlov burst into cells with political prisoners (to Sannikov\high{\footnote{Another candidate for the presidency.}}, in particular) and shouted, blaming them for what had happened. Frankly speaking, later he excused himself, but the apologies were not accepted.


But what astonished me most was the statement by Zaytsev, the KGB chairman, who said that one of the motives for the terrorist attack could be anarchists’ revenge! What had he taken to think of such nonsense? I wondered how Dima felt at that moment\unknown{}


Naturally, they would bother me. They came in the evening\unknown{} There were two people in the governor’s office – Orlov himself and the other one, who I had seen in other offices before. They suggested that we watch a video from the metro surveillance cameras. The other one sat to my right, as if to watch the video, but in fact he was watching my eyes. It was just the same as during the interrogation about the arson of the Bobruisk KGB headquarters in the 4\high{th} police department, when there were investigators from Mogilev. We watched the video, they wondered about my observations, but I had nothing to say; basically they didn’t invite me for that. Finally, they gave me a still of the best frame, although the quality was detestable, and I couldn’t find photos similar to it in the newspapers.


On the day of mourning they switched on classical music. The prison was deathly silent. Only the idiot guard Vasya managed to cry “Move!” when we were going for a walk. Serega and Max filed an application to donate blood for the injured, but they got an official refusal.


\chapter{15
}

Martselev was waiting for a trial at the end of April. They changed his case from organisation of mass disorders (from 5 to 15 years in prison) to a less serious charge (3 to 6 years). He was in two minds whether to admit it or not. He was thrown into a dilemma: an old accusation or a new one, but with the confession of guilt. We tried to persuade him to agree, because it was evident that having a narrow escape is already a victory in this system which had nothing to do with justice. Sergey was very nervous and was getting ready to go to a colony, we were teasing him and arguing that he would receive a suspended sentence. As it appeared later, Pavel Severinets\high{\footnote{Pavel Severinets served as a head of the Electoral Office of former presidential candidate Vitali Rymashevsky. He was arrested after the presidential elections in December 2010.}} a totally real custodial restraint in a form of a penal labour settlement\high{\footnote{A less severe regime, wherein inmates can have domestic appliances, see relatives more often, and go home for the weekend.}}, so Sergey’s fears hadn’t been groundless.


We quite often argued with Martselev. He was a social-democrat and an Orthodox patriot. So the spectrum of our arguments was pretty wide. Most of all I was agitated by his attitude to the First World War, as he supported Martov’s\high{\footnote{Julius Martov – the leader of the Mensheviks in early twentieth century Russia.}} position, i.e. a defensive position. And I considered mutual annihilation of workers by other workers for the sake of the profit of national military-industrial and military circles of all warring parties to be meaningless. In the first days of my restraint I happened to read “The Thibaults, Part 2”. It’s a book about the beginning of that massacre, about the socialists of France, Germany, Austria, Switzerland, who preached internationalism just before that, the class war and the all-out strike. But the more the situation heated up, the faster they fell into a patriotic mode and in the end all together started to kill those who had been their comrades just the day before.


Here they are, the socialists! Martselev stated that fighting the war met the interests of the working class, since usually an occupation by another army lowers their living standards. But hasn’t the war itself caused even greater devastation? But that’s not what it is all about. Can the possibility to choose from ten different types of sausages really be a moral yardstick? Is it moral to choose rulers for yourself and do it at the cost of your class brothers’ blood? Patriotism was announced to be one of the most important moral values of people. National interests were above all things. Interest is greed. But since when does greed (in a territorial or ethnic sense) equal ethical concepts, justice and kindness? “What is profitable is what is fair” – such a view on patriotism agreed with that of the KGB colonel from the 4\high{th} police department, and Orlov was 100\% for this approach. This is the way to define what is moral by greed nowadays. Such a model has nothing in common with natural ethics, because respect, honor, equality, altruism, mutual aid, rights and freedoms cannot possibly have the colour of flags or borders. But these are exactly the qualities and values that are used everywhere and in all times to define the humanness of a human. Patriotism plays on the love a person has of their land, it tries to identify itself with a natural affection for native terrain. But there is such a holy notion as love of the mother, and no one even thinks to build any ideology around such a natural and intimate feeling. Then why may one create ideological theories on the basis of the love of a land? Moreover, the same theories demand rejection of ethics, and rise over universal values.


It is said that we need to foster pride of “our own”. But what if some fact of a country’s history makes you feel ashamed and not proud at all? Belarusian history is a history of different periods: Slavic-Baltic tribes, The Grand Duchy of Lithuania, Tsarist Russia, the Red Empire. On what grounds are some periods denied, and some glorified? I want to be proud of neither Bolshevism, nor Tsarism, nor the Duchy. For some reason it is forgotten that in Lithuanian times people were treated like slaves. And how is it possible to be proud or ashamed of something that you have nothing to do with? We can be filled with admiration for certain pages of our history and such pages can be found in any period. The same as pages of sorrow. At school we need to instill not the false patriotism that depicts the past one-dimensionally, but interest in our history. It will strengthen self-consciousness to give examples of our ancestors’ behaviour (no matter what they were like), enabling a better understanding of the present.


A community based on freedom and justice is stronger than a society based on filtered history and exaggerated collectivism.


I am a Belarusian because due to my parentage I am assigned to this unique historical-cultural community. This is neither good, nor bad; it is not a reason either to be proud or to be ashamed. It is what it is and that is enough. With concern to values, humanity has worked out a strong ethical base by the means of all its history, philosophy and science. This base is humanism.


Arguing about socialism, I pointed out the fact that it was under the rule of “raspberry-reds” when Europe hit a wall. It is under the rule of social-democratic governments that the wrecking of the so-called “social state” is happening. The difference between right-wing liberals and SDs become purely cosmetic. It looks like there are no distinctions between them anymore except for their ideas on the income tax rate. The right-wingers want to stop at 25\% (it’s an average number in the USA), the leftists want a larger number. In fact, both the market-based liberal system of the USA and the market-based social system of the EU are in a most dramatic crisis. Both have taken loads of loans and cannot pay them back. The governments and the real economy both comply with financial capital and are inferior to it. The state capitalism of the USSR claimed that it was striving for communism. And it crashed down, it wasn’t even close to the communism described in books. Market capitalism claims that great depressions remain in the past. Why did America and Europe end up in such a shitty position in the first place? Because sustainable development is a myth just as state communism is. Predators remain predators no matter what you call them.


Martselev liked to say that today, class is no longer valid and it is the social structure of society that supposedly defines stratas. Allegedly, even at some congress of the Social International in the 1960s the class approach was abolished. This decision can not have any authority since socialists in all their forms (Bolsheviks, Social Democrats) have shown the utter bankruptcy of their own theories. Capitalism has not failed, the Bolshevik government led to totalitarianism and parliamentary reform socialists turned in to the next party of bourgeoisie.


I remember how in 2006 I went to work in the West and I had a personal experience of what the much boasted ‘social partnership’ meant. I got a job in the Carnival Cruise Lines company which is engaged in travel on cruise ships, mainly in the Caribbean. Like many others, I expected that hard work would mean earning more money, because the USA has always been associated with the formula ‘more work – more money’. But the reality was quite different\unknown{} Work on the liner involved a 10-hour working day with literally no weekends. The majority of the workers such as cooks, storekeepers, cleaners and painters had a salary of about 450–780 dollars per month. They didn’t pay overtime because it wasn’t official: the managers corrected the working schedule. Part-time additional work was officially strictly prohibited. No fee, no bonuses. All in all, we worked for money that you could earn working less hard, even in Belarus. Before that, I had to work as a pavement tiler and as a house-painter during house construction (the concrete casting of foundations). So I had something to compare the work with. The workload on the ship was quite extraordinary. Blood blisters on our feet, arm sprains and back problems were the most common companions of life there. It got to the point that women had menstrual disorders. Even experienced guys who had to work as strawberry pickers in the EU, on poultry farms in England and in US restaurants, cursed the job. The salary of waiters and housekeepers was not much better than 900–1100 dollars, with almost no possibility to get tips. The moral atmosphere was even worse than the physical work environment. Everywhere there were supervisors threatening the workers, security with batons, you couldn’t smile or talk to a passenger. Alcohol and drugs were the only way to relieve stress. The situation was aggravated by absolute impotence: managers could openly blame you for their failures and you couldn’t do anything. A compulsory sharing of responsibility left no chances to prove your point. The workers were divided into several hierarchical castes, each with its own rules and rights. Even the dining table for each caste was different. A common worker couldn’t wear shorts, have long hair, talk with passengers. Another caste, a service-providing one, was allowed to talk to tourists. White-collars were allowed to have any appearance, get acquainted with customers, use their services. The officers could do everything, including battering workers. Everything was done to prevent contact among the castes. White collars were forbidden to come to a dinning room for the ‘blacks’, walk around and hang out in the bar. I remember one violin-player who came from Minsk. A normal girl. She didn’t care for the hierarchy. Once we got together and went to the beach. Someone snitched on her, she was called to the office to have a brainwash: “You can not hang out with these people.” I met a punk from Canada, he was a sound engineer. We played table football; an hour later someone reported us. It was insane, but that was how it worked. Divide and rule. Sometimes workers came on the ship as a couple, but one of them was below the other and typically the couple would break up. What the system couldn’t ruin was smashed by envy, hypocrisy, snitching, contempt, malevolence. In that atmosphere, even a one night stand with people of different castes was seen as a protest against ‘morality’. After such working conditions and the beastly attitude, all illusions about capitalism were gone. The level of democracy that exists in Western countries is there not because of the market system, but because of the desire and readiness of many people to defend their interests and take to the streets.


In this regard, I strongly admired Americans and Europeans. Dignity and class consciousness are much more developed there. However, the ruling class and the working class are still in place. Them – the 1\%, we – the 99\%. Belarus, with its mixed economy and the dictatorship of bureaucracy, will eventually absorb modern corporate management techniques (they already exist in the IT sector and some areas of trade). But it will bring nothing good. No liberation, but a more sophisticated exploitation and a dependence on bank credit which is akin to addiction.


I was spending time reading “The Bull’s Hour” by Yefremov. A slow, thoughtful read. It was similar to the brothers Strugatsky’s novels. A united society triumphed on the Earth. The greatest ideals of freedom were realised. The State ceased to exist. People lived comfortably and interestingly and were governed by horizontal structures that served as a means of coordination for the common good. Multilateral development, a harmonious and moral person was the center of the system. Some day such social organisation will become a common thing. But today we have to live in a world full of grief and misfortune, where the sprouts of reason and joy fight their way through despite the dark conditions.


Martselev got a suspended sentence. Prosecutors sent back Max’s case, and he was immediately transferred to Volodarka. Denis was also taken away. I haven’t heard about him since.


I was moved to the neighbouring cell \#5. May began with the expectation of the trial\unknown{}


\chapter{16
}

The new cell was different from the other quadruples, because it had two windows instead of one. Unprecedented luxury! But more notable was the company. I met Vladimir again, the one that I had done time with before. I was amazed by the changes that had occurred in him: within 4 months he turned from a vigorous elderly man into a noticeably grey, wobbly old man. Let’s assume he will be acquitted. But how can the state compensate this?


Two other inmates turned out to be law enforcement officers who had fallen under the hammer of the system which they had served. The first one, Sergey Yelin, was a deputy prosecutor from the Grodno region. He was accused of accepting a bribe. The details of the case are interesting: a man who bribed him tried to solve his problems through Sergey, but the latter refused. Then this guy went to the KGB agents and offered to make the impracticable investigator suffer. During the first attempt he tried to slip him 5000 dollars, but to no avail. The next day he repeated the maneuver, but then after another refusal, he offered him just 500 dollars for ‘legal service and his time’. Yelin agreed and was immediately arrested. The first attempt was not in the case, as if it had never existed. An audio-recording of the conversation during the money transfer was provided by the provocateur himself, although the arrest was carried out by the KGB. Not surprisingly, there was only one audio file on a CD. The recording was missing on the voice recorder (allegedly, it had been erased due to a lack of space). According to Yelin, many phrases that he had said were missing in that recording. KGB agents weren’t sanctioned for wiretapping and couldn’t provide it in court. So, apparently, they came up with the voice recorder story. And in fact, they were wiretapping and recorded onto the CD only what they wanted and filed it. By an odd coincidence it was Yelin who convicted a director of the ‘Lida Flour’ factory, who had been in friendly contact with the KGB previously.


The second law enforcement officer, Zakhar Djilavdari, managed to work in the Economic Crimes Department, Financial Investigations Department and Organized Crime Department. He participated in the investigation of the explosion\high{\footnote{The 2008 Minsk bombing took place at a concert celebrating Belarus’ independence on 4 July 2008, in Minsk, Belarus and wounded 54 people.}} in Minsk in 2008. He was the one working with the last Belarusian mafia boss Birya. But the most interesting moment of his biography was the arrest of the vice-chairman of the KGB. If I’m not mistaken, the case was connected with customs crimes; he was also part of the investigating team of Baykova\high{\footnote{Svetlana Baykova, former chief investigator at the Office of the Prosecutor General, was arrested by KGB agents on February 25, 2010. She was accused of office abuse and sentenced to 2 years of house arrest.}}, the disgraced state prosecutor. Zakhar himself was facing charges of an attempt to take a bribe, according to the testimonies of a witness. But in the court that witness confused the sum of the bribe and didn’t manage to name which currency it was (dollar, euro, ruble). The prosecution didn’t lose their grip and changed the charges to “Nonfeasance in office” (up to 3 years). It could seem trivial, but a person (although a pig) had already spent 9 months in this concrete well\unknown{} One interesting detail: within the last three weeks Zakhar accidentally met in the corridor\unknown{} Baykova! The thing is that there are no accidental meetings in the KGB prison. The guards convoy people, whistling, so the other approaching group can stop in time. In extreme cases they order us to turn our faces to the wall; that happened to me several times. And once a person appeared in our cell. They mixed up the numbers of the cells. We could see how scared the guards were and they dragged him out the very next moment. I suppose it is considered a huge mistake.


This was how they made Zakhar realise that he had ended up smashed by the system he was serving. The thing is that in 2009 there was a real war between different security agencies, and it was a holy cause to imprison someone from the enemy bloc. After allowing his arrest to be prolonged for 2 months without a prosecutor decision (as well as wiretapping) the war took a grand scale. Now the Ministry of Internal Affairs (several departments to be precise), the KGB and the Department of Financial Investigations\high{\footnote{Over the time of Lukshenko’s rule the KGB became more powerful. Before the function of the KGB was limited to special investigations, concerning the state security. Later the policy changed and the KGB started to control and watch their colleagues from other special agencies. Their resistance to such behaviour led to a never-ending confrontation between the agencies.}} (financial police of the Committee of State Control) got the opportunity to start cases and arrest people, which wouldn’t have passed through the prosecutor’s office before. Even more – before there was a separation between different structures in terms of which crimes they investigated, but now this differentiation was gone. And naturally they started choking each other. The statistics of 2009–2010 indicate a raise in corruption crimes by two times. Baykova’s case indicated the death of the prosecutor’s office as a serious structure. The Department of Financial Investigations also became a victim of the KGB: first they appointed their own director (a former KGB agent), and then arrested all more or less independent staff (for example, the case of Adamovich\high{\footnote{Dmitry Adamovich – a head investigator of the Committee of State Control, arrested in August 2010 by the KGB, allegedly for covering up crimes of big business. Sentenced to 3 years of prison with confiscation of property.}} who was under investigation at the same time as us). By the way, here I can also mention Aleksandr Kiselev. It was ironic that the same investigator from the Department of Financial Investigations who had started a case against him also ended up in Amerikanka\unknown{} in the same cell with Kiselev. The world is a village indeed! He was the one who told Aleksandr the whole truth. How once, two KGB agents came to him to start a case where there was no actual crime. The investigator said that it wouldn’t work in the prosecutor’s office, but the agents reassured him that they would take care of it. That’s how Kiselev ended up here. Courts are partly controlled, including some district courts in Minsk and the municipal court. The Supreme Court might not be controlled yet, but this is just a matter of time. But to cope with the Ministry of Internal Affairs is way harder. First of all, the numbers: 100,000 against 5,000 members. Secondly, the Ministry of Internal Affairs has its own criminal police, the Organised Crime Department. And although the KGB has a technical and personnel advantage, looking at the prisoners of Americanka, you could say that they are doing the same thing. But two goats are too much for one garden, so the criminal police officers have the same experience and are ready to send any agent on vacation to Volodarka. Yelin was sad as he said that professionalism had decreased because of that struggle. Nowadays they are not even capable of making a proper case anymore, they rely more on prosecutors’ and lawyers’ loyalty.


This situation among the law enforcement agencies is encouraged by the ruling class and the method itself is not that new – there is a historical figure, a Russian Georgian, who liked to shuffle staff cards. This method is called “rat king breeding”. Officers fight and tear each other apart and as a result the dominant position is taken by the most unprincipled but effective (“the rat king”). The power itself is still impregnable – if there are doubts about the “rat king’s” loyalty, the ruler uses damaging evidence against him that has already been prepared.


Zakhar told me how they once found a brothel for VIP clients, all top-rank officials. At the point when they wanted to raid it they got instructions from their bosses: “Get the fuck out of there right now!” There was also a case when the new director of the Department of Financial Investigations didn’t know all the rules. Somebody managed to submit a list of companies for a routine check. Everything was fine but “Triple”\high{\footnote{The TRIPLE Group of Companies is a wholesale company for industrial and food products. Top managers of the company are believed to be closely connected to Lukashenko. In March 2012 most of the group’s companies were placed under the economical sanctions of the EU for supporting the regime.}} was on the list. When the time for the check came, the director started panicking and told Zakhar: “Do you know who that belongs to?”


We also talked about the opposition. There were rumours about details of the political cases. I didn’t have any chance to figure out what was authentic, that’s why I will skip these moments. After all, those stories with Rymashevski and Romanchuk\high{\footnote{Vitali Rymashevski and Yaroslav Romanchuk are opposition politicians, candidates for presidency, who escaped long KGB arrests and were blamed for cooperating with the authorities}} and the reaction of the opposition media distorted the reputation of the opposition. But one story was witnessed by Yelin and Djilavdari. Let everybody know it. It was the 31\high{st} of December 2010 in cell \#1, where Dmitriev\high{\footnote{Andrei Dmitriev – one of the oppositional politicians.}}, the chief of Neklyaev electoral office was kept. In the evening Dmitriev was taken to an interrogation, where he managed to trade for his release. Nobody knows what the price of it was. After that they brought him back to the cell so he could pack his items, but Dmitriev said to his cellmates that he was to spend the whole night with the masks. The guys packed his two bags with food and said good-bye, worrying about him. But Dmitriev went straight back home\unknown{} In conversations he easily talked about how he built a flat and got loads of money from the presidential campaign, although he didn’t care about any politics from the democratic camp. Most probably Dmitriev took the bags to the first garbage bin, but maybe they also became part of his New Year table, but who knows, a greedy politician will take advantage of everything.


\chapter{17
}

All day long we talked about the investigative unit of the judicial system and the characteristics of the various operatives and investigators. We also argued about politics. Anyway, we agreed on the fact that without the active participation of the people, with no real public control of governance structures, the republic would devolve to dictatorship. This might seem strange, but cops realise it perfectly well because they see the situation from the inside everyday and have no illusions regarding the true nature of power.


The forensic investigation system is simple and nightmarish. Key point: the real power lies with the operatives. They deal with criminal profiling and accumulating material for the prosecution – there are no limitations. Everything is based on personal experience, logic and intuition, they make use of any evidence. Their legal awareness is not important. The most important thing for an operative is to recreate the flow of events so that it fits into a conventional crime cliche. His boss decides whether to start a case based on his subjective impressions. The defining criterion of success for an operative is the number of cases he brings to court during his career. It is favourable for him to interpret the situation as a crime and prove the implication of a suspect in the crime. The grounds for suspicion is the subjective opinion of an operative. Any person can be detained for three days simply because he wants it so.


Then comes the moment of truth. One can be accused because of any lead which is more or less direct. A simple cigarette-end with traces of DNA (saliva, sweat), a cell phone signal nearby, an empty bottle with fingerprints, or any statement of a spiteful person can be sufficient to arrest and imprison someone. Before, an arrest was sanctioned by the prosecutor, and for a positive decision in an uncertain situation they used their personal networks, influential persons, etc. Now, for a 2-month arrest, the signature of the head of a law enforcement agency is enough. Once a person is arrested he slides down the ramp straight to the sentence. From that point, all the other stages of the system are mere formality. When the case is filed it is passed to an investigator. Although this stage is called “preliminary investigation”, in fact the investigator only inspects and draws on the materials gathered by an operative, according to legal regulations. An indicator of a successful career for an investigator is the ratio of cases brought to indictment to the total number of cases under investigation, essentially, the number of people convicted. Another important indicator is the gravity of offense according to the Criminal Code (a misdemeanour, a serious offense, an especially serious crime). Therefore, each investigator gets benefits if a person a) goes to jail; and b) for the gravest offenses.


The worst thing is that these indicators of success are not hidden or informal, they are officially confirmed: based on these numbers, investigators receive financial bonuses, appointments and grade titles. There is no greater disaster for an investigator than a not-guilty verdict. That can lead to the most severe corrective actions including dismissal. If, during a preliminary investigation, an investigator realises that he can’t bring a case against a person he can offer to reclassify the case on a less severe article, whereby the person can get a probation or be set free with a sentence equal to the time he had already done before the trial.


For the accused this is still a victory: it is almost impossible to get out of this meat grinder unharmed. Such mechanisms of the system generate the following statistics: the efficiency of convictions in Belarus is more than 99.7\%! In Europe it is 80\%. Even in 1937, with its special troikas\high{\footnote{NKVD troika or Special troika in Soviet Union history were institutional commissions of three people who issued sentences to people without trial.}} and trials in absentia, 10\% of cases were acquittals.


Once an investigator has made his case, he sends it to the prosecutor’s office. Only during this period do the accused and the lawyer finally get access to the case. Before that an investigator can only reveal some of the cards. A lawyer is in fact a powerless person, they can do very little. The prosecutor’s office is there to check if the case is not complete bullshit. Judging by what can appear during trials, they work half-heartedly.


And finally, the last stage – the trial. The trial is bluffing, a performance. A judge never sees any inconsistencies and transgressions. For them the process has turned into a routine. Even when there is an obvious logical contradiction, the judge will pretend that there is nothing unusual. He doesn’t want to go against the flow because everything is ready, chewed and served on a plate. It’s easier for a judge to convict the accused with some questionable circumstances in the case and, thus, to shift responsibility for the destiny of the person to a higher court (municipal or Supreme), where subsequent appeals and complaints are filed.


Thus, the judgment is actually delivered by an investigator. Then it all comes down the pipeline, more precisely the system locks down: it starts with something small, but the mechanism of the system transforms it into something bigger, and from bigger to massive, detailed and thorough.


The question is how to find initial clues that will later become “evidence”. An investigator, having found no evidence (because ‘Sherlocks’ quit their job or are imprisoned), will scratch his bird-brain head and resort to an old and reliable method – testimonies.


Witnesses are threatened that they will be turned in to suspects and later the accused. A classic scheme: if you give testimony, you are named a witness in a case, otherwise, you are an accomplice. If the circumstances of the case don’t involve complicity, they point to the concealment, make threats about problems at work/school for you and your relatives.


Another scheme is more complex, it presupposes the use of the testimony of another accused in exchange for the promise of a less severe punishment. It works flawlessly with article 328 (drugs). It is assumed that addicts make up a single social environment, so each of them, one way or another, buys drugs for themselves or resells them to others. It becomes a market of networks.


For example, they detain three addicts with a few grams of weed or amphetamines. This qualifies as article 328, part 1 (possession of drugs, up to 3 years, a misdemeanour). Two of them can get probation, be freed after the trial or get a few years of colony, but only if they are lucky, because the investigator may not keep his word. However, they can become scapegoats themselves and get 8 years (if they admit the guilt) or 9 years (if they deny it). It’s like a merry-go-round.


The most valuable material for an investigator is, of course, the testimony of the accused. Ideally, an admission of guilt. If so, an investigator will slowly complete the case with full confidence that nothing will go wrong. To get testimonies they use a diverse set of tools. Usually they try to frighten you, saying that if you give testimony things will be better for you. Or they lie, saying that they understand the situation, they believe in your innocence, they will do everything to release you, but you have to give testimony “so that they are able to help you, man.” After all, you are a “good guy”. Or they threaten you with the horrors of prison, in particular that you will be put in to a press-cell, a cell with people sick with tuberculosis, or with the homeless. They can say they will guarantee problems for your relatives, including real attempts to put pressure on them. The main emphasis is on moral coercion. They alternate methods; exhaust you with long interrogations, pressure you by shouting, swearing, insulting you, restraint, thirst, bright light.


If they can’t make people testify, without giving it a second thought, they throw you in a cell and starve you out. The prison atmosphere in the early days is perceived as hard; the uncertainty and severity provokes panic and fear. Sometimes you find the courage to survive the inquiry. But the cell, with all its uncertainty, horror and dreadful isolation from everything, breaks down the defense and you lose courage. When they call you next time, they pretend they want to help you or they use violence, from threats to pretending to begin to torture you or minor hits (a slap in the face or to the side of the head, a punch in the chest, in the leg, etc.). Serious violence and torture, as well as the implementation of threats of being killed by the cellmates is not usually practiced. Such methods are an obvious crime. Nobody wants to lose their job, or what is more, their freedom. Although, there are exceptions, of which there are many examples. People who suffer, but don’t give up and come out as winners in the end.


Even a negative testimony has something for them. Indeed, in the case of denial of guilt, they have a base to start from. 80\% of cases are filed based on the testimony given during the preliminary investigation. The system is used to such a course of events. But it is also its major vulnerability: an investigator is afraid of the refusal to testify most of all. In this case they will have to do everything by themselves, search for a strong evidential basis, saying nothing of the fact that there is no guarantee that their falsifications will go unnoticed. For example, if an investigator has prepared a false witness and based the case on his testimony. Suddenly the accused gives such a testimony in the trial that it becomes obvious that the witness is false. The investigator can get in a scrape. So, a refusal to testify is the most efficient way, the most advantageous position.


Either way, once a person has been arrested, his chances for acquittal are lousy. To get off lightly is a victory. But, even in the best case, it will be a Pyrrhic victory.


Even three months in prison is a long time, akin to punishment. Those who have hope of aquittal and keep struggling can stay there for a year, two or even more. During this time many things in life will have changed, lost, broken, crashed and forgotten. It’s a severe punishment anyway, a punishment for having dared to go against the flow, to question the power of authority.


To fully understand how the investigation machine works, it is necessary to mention the core component, which drives the system, determines specific numbers and answers the question: “Why so many cases?” The name of this component is the ‘Plan’. It does not exist officially, but it actually defines all the quantitative work of the investigation. The essence of the ‘Plan’ is defined by the following formula: the number of cases of significant severity that were filed last year should remain at least the same in the current year. It happens similarly with the crime detection rate. In other words, an operative and an investigator are guided not only by career motives: they turn themselves inside out to get the ‘Plan’ done. If you can’t cope with it – clear a space for the others, maybe less professional, but more dodgy and immoral.


Presumption of guilt – that’s the basic credo of the punitive system (it can’t be called any other way) of the Belarusian regime. This is what morality and justice in the Belarusian state looks like!


\chapter{18
}

Very soon, the first court session started. I was waiting for it as though for deliverance. When the case was taken to court they allowed me a brief meeting with my father. Glass, phone receivers\unknown{} It’s hard to be a meter away from such a close person and not be able to touch them. Father makes it clear that he was aware that I would be sent to prison. I was glad he understood it and was mentally ready for this. After all, all those assurances in letters, like, “you will be acquitted for sure”, brought me only a bitter grin. The truth is always better. I wanted him to understand that my spirit was strong and I would accept the sentence serenely.


The last night before the trial\unknown{} I wanted to have a change of scene, to see my parents, relatives, friends, acquaintances, comrades, sympathizers. Finally, I could at least have a little chat with my comrades. So many years of common hopes, expectations, trial and error, frustrations, achievements, meetings, disputes. We all started from nothing, with unclear motives for freedom, for the truth, for justice, for brotherhood. The frameworks of youth movements were too narrow for us, because our instinctive craving for freedom didn’t accept partial solutions. Human personality doesn’t need to have limits. The first articles on anarchism on floppy disks, the first book by Kropotkin. There isn’t anything more precious because the dream was unfolded there. Since then, neither the omnipotence of the authorities nor the servility of the people, nor the indifference of the laypeople could stop us. From conversations on the bench with a beer to the first zine. The first small group in the neighbourhood, then at the university. Participation in demonstrations, acquaintances with similar enthusiasts. DIY-subculture, meetings, social events, stickers, pamphlets, periodicals\unknown{} Street wars with Nazis, concerts, trips\unknown{} With the first drop-outs vanishes the first romanticism. Those who are left stick together even closer\unknown{} Crises in personal lives mow the ranks like sniper shots\unknown{} A cloudless childhood was over: employment, housing issues, payments made us look in a different way at the words “social justice”, “economic exploitation”. That was what we breathed in every day\unknown{} Fewer words – more responsibility, that understanding was put at the forefront. Growth was felt everywhere, quantitative and qualitative. It started in 2008. Anarchism. Completely destroyed by the Gulag\high{\footnote{The government agency that administered the main Soviet forced labor camp systems during the Stalin era, from the 1930s through the 1950s.}} it re-emerged during Perestroika\high{\footnote{Literally, restructuring; the policy of economic and governmental reform instituted by Mikhail Gorbachev in the Soviet Union during the mid-1980s.}}. It took 20 painful years, until it replaced several generations of activists that were groping for methods and forms of organisation. We are now a mature social movement and are willing to fight for the total implementation of humanistic ideals.


The authorities long for our repentance, they want us to betray each other and attempt to expose us as disturbed people who regret their ‘ruined’ lives. The authorities want to make this trial a showcase and discourage other people, so they can enjoy their power. But this will not happen! We will not change the essence of our lives for mercy and pity. We will not give our comrades a reason to doubt our life choice! We love freedom too much to beg for it. Our relatives will see determination and persistence on our faces. The pride and respect of our families and friends, this is all we need. We will go to jail, but stay the same and keep our personality\unknown{}


\chapter{19
}

\unknown{}It’s the morning. The guards, handcuffs, a police van, a blind section in total darkness. The car races along the green lane with flashing lights. To maintain my position, I have to set my head against the wall. They drive right up to the back. I see a double line of cops stretching to the door. They lead us to the basement. They divide us among the cells, this time concrete ones, half a meter by a meter in size. A pat search: they carefully check my clothes. At this moment we meet eyes. Sanya, Kolya\unknown{} I want to say so much right now, embrace, shake hands. But while greeting, we meticulously evaluate each other. It’s clear that everyone wants to get rid of their tiny inner doubts: “Have they lost spirit or not?” But judging by the firmness of our voices, the way we handle the police, it is obvious that no one has caved in. We communicate more confidently, despite the constant remarks of the guards, and these first words make our hearts warm. Vetkin is trying to communicate, but no one talks to him. There’s only curiosity in his eyes. He could have been anything, but became nobody. This is sad. We are waiting in the cells. The walls are covered with writing: nicknames, articles, sentences, wishes. Mostly it’s 205’s and 328’s, theft and drugs. I add mine, draw symbols and slogans. Let them know that you can do time not only for self-interest. Time passes very slowly.


\unknown{}Finally, it’s our turn. They line us up and take us to the court room. There are a lot of people in the room, cameras are flashing. All this puts me into a stupor. At the entrance there are metal detectors, a lot of police and plain-clothed cops, it’s such nonsense. In the cage they remove the handcuffs. We are trying to communicate, but they are vigilantly watching us and suppressing the communication. It’s said that the court is surrounded by riot police. In brief, the circus around the circus. The lawyers come one after the other, there are a lot of familiar and unfamiliar faces. After many months in prison you become estranged from society and get lost in such an abundance of people. Parents, relatives, friends, comrades. This support strengthens us. After all, your eyes convince you that you are not alone and you can count on all these caring people. The isolation rips at the seams.


The judge and two yes-men pretend not to notice the absurdity of some evidence and testimonies, the pressure of the operatives, etc. Zombies. Most witnesses deny their previous testimonies. The prosecutor presses them, but with no success. It’s just many tedious hours of absolutely useless words spoken by unimportant people, and I look out of the window. I never thought that I would be so happy to see green trees and clear blue sky. Not from behind the bars\unknown{}


The prosecutor, who has got the nickname “buddy beaver”, claims that we only acknowledge the laws of physics and chemistry. It’s true, as well as all natural laws of life: the laws of biology, history, and the most important moral law, approved by the whole essence of human nature and social development.


The last word. I wasn’t prepared, I thought it would be tomorrow. I decided to talk about Dima Dubovsky, our slandered and persecuted comrade. Vetkin and Konofalsky, scum, claimed that he was responsible for some things, but they lied so much that it came up in trial. Out of the four of us, he faced the hardest test, even if he managed to keep something that is called “Freedom” in this wretched society. They applied to him the vilest and the most disgusting methods of operative work. But Dima survived it and will overcome all the difficulties. Such people – forever. And the years in the dungeons are no barrier for our brotherly comradeship. Sanya and Kolya said very decent words. There is no guilt in our conscience, meaning any deprivation is just a reward.


The sentence. Well, Makhno did time, so, so do we. Eight years in one breath! Last glance at the people close to me. With the exception of my parents, I won’t see them in the near future. I say goodbye to my lawyer. His appearance in the KGB detention center was like a breath of fresh air; in this desperate situation, he was able to help me. I shake hands and hug with Kolya. I am honored to share the fate of such people.


Vetkin was granted mercy: a 4 years of custodial restraint. He, Zakhar, Arsen and Buratino will live miserable lives. If they have children one day, what will they learn from their fathers?


\unknown{}Again to the police van, next stop – “The KGB”. Going out of the court, I shout, “Comrades, see you!”


\chapter{20
}

A meeting with my parents. This time they let my mother attend too. Our dear mothers\unknown{} They are the ones who are really unhappy. Our fathers are also suffering but by their nature they understand that hardship will be good for their child. But a mother does not listen to any arguments if her son is imprisoned. There are always two people imprisoned. A mother cannot live a single day without worrying for her child. Queuing up to send a parcel, waiting for letters, catching any news about the prison or the penal colony where we serve the sentence. This is their sentence, day in, day out, year in, year out. So prisoners’ mothers are the real heroines, the real martyrs. I know they are very worried about me. But I am glad to see them cheerful and proud. We discuss the trial. I get to hear of various people’s opinions, their greetings and wishes. This defeat is in fact our victory. The regime is digging its own grave with trials like this. They did not keep the lessons from the Stalin-era purges in mind.


Last days at the Americanka. I feel how this place is losing its power. Rays of sun look very beautiful on the rugged wall, but something worrying remains in them. These six months were not given in vain. A reflection of this building, this red building will forever stay on my soul. I will never be able to forget this dimension, where the world outside falls apart, where even hope dies, where neither time nor space exist. And in this bleak permanence, life turns into a knot of pure fear and pure will. For the last time I am looking at these massive and severe walls, corridors, stairs, handrails, watchtower, rolls of barbed wire, metal doors. Hundreds of details, and all of them are forming a monolith which serves only one purpose: to trample the individual underfoot. But it was in this hell, it was thanks to this nightmare that I was able to look into myself and to understand many things. It’s perfect material for dystopian films, for industrial ambient music. It’s a pity I don’t know anything about art, otherwise the substance of this place would seep through the style of my writing. Alas!


Three times within a single week I ran into a pretty girl with white plaits. She was one of the service staff. Why are the wardens so lax? I don’t really care though. Morally, I am not here anymore. I am waiting for prison transport to Volodarka any day now. In the cell, everyone has withdrawn into themselves. Everyone is going to trial. Vladimir has been refused a pardon. Zakhar said that “Uncle Vova\high{\footnote{Meaning Vladimir.}}” is very much expecting that Article 70 (less than the minimum recommended sentence) will be applied, but such clemency has to be earned somehow. I remember when Molchanov got his case file, there was a paper found in it, about investigative activities on our cell, unfruitful though. We figured out when that paper was filed, and remembered that Vladimir was transferred two days later. His questions about who the anarchists’ leader is and who gives me orders also tied in. As did his attempts to find out how to make a Molotov cocktail and his incitement for actions in the style of the Combat Group\high{\footnote{According to the novel, the Combat Group was a revolutionary group with an outstanding leader, which was engaged in murders of government officials and bombings in Tsarist Russia in 1891.}} (in Boris Akunin’s novel “The State Counselor”). For my part, I played the fool and pulled this knobhead’s leg\unknown{}


On one of the last days the cell door opened and in came\unknown{} colonel Orlov, large as life and twice as ugly! Naturally, he was on to me. A conversation took place. The chief was interested in my mood, in my attitude to the expected sentence. He even voiced some sympathy. I do not believe in sentimental agents, so I was waiting to uncover the reason for the conversation. Despite that, I was still caught off guard. Orlov bluntly blurted out: “Come on, join us to be a hacker? You know how the Chinese are rolling! We’ll supply you with a personal laptop.” Truth be told, I was dumb-founded and disorientated. Then Orlov offered another helping: “Well, if you don’t want to be a hacker, you can work here as part of the service staff. There are good conditions here, many advantages.” All patterns torn, brain puzzled, total shock\unknown{} Have I ever given them any reason to offer something like that to me? How many people have died fighting this organisation? And how many of the best people in the country have they eliminated?! And how are they brutalizing the population now? And they expect me to trade my conscience for their mere pittance. Comfort, advantages\unknown{} I used to have it all, and I do not regret the loss of it. I replied:


“I would rather do my time in a penal colony.”


“8 years is not short.”


“I do not care about the sentence, I will develop myself.”


“Everybody says that. The first three years are bearable, and then\unknown{}”


“I will have a chance to see all of that for myself. Our humane state has provided me with such an opportunity.”


Honestly, I cannot understand these KGB colonels. They can speak with strong conviction even when they are lying. But lying is their professional duty, so for me it remained unclear which words were utilitarian and calculated, and which reflected their actual opinions. It all sounded the same. Orlov repeatedly told us that his aim was to cast doubt. Well, that much he achieved, no doubt about it. I came to the conclusion that the KGB colonel is a master of sensitive tasks, no more, no less. As for Orlov, it seems to me that he pitied us. But this feeling should not be confused with normal people’s pity. It was something different. He was a bit like John Kramer, that semi-maniac from the film “Saw”. But not exactly. The Jigsaw Killer had ethical considerations. He wanted the humanistic transformation of a person under extreme conditions. There’s no humanism involved here at all. Orlov is more similar to O’Brien from George Orwell’s novel “1984”: convinced, systemic, merciless.


\unknown{}Prison transport to Volodarka. It is finished! I bid goodbye to my cellmates, take my stuff, a shakedown, formal procedures. I’m led to a van. I look back and inspect this place which is soaked in suffering, grief, desperation. Americanka\unknown{} One day it will be a museum.


\chapter{21
}

Volodarka – large, gloomy vaults and long corridors. But the pushy manner of the coppers and prisoners indicates that the stern silence is just a facade. This is an anthill pierced with thousands of threads, and it is bursting with life. I get a mattress, then a cold shower, wait in the room where a search takes place and finally I go up to the cell. My impressions are quite the opposite of what I felt when the door to an Americanka cell first opened in front of me\unknown{} It feels like entering a migrant workers’ barrack. People look at you from the top and bottom levels of the bunk beds, from the table and even up from the floor. Fifteen men, sweating from heat and stuffiness, sit around in their underwear with impenetrable clouds of tobacco smoke around. Now I am in a real prison!


June 4\high{th} – the day of happiness. There are 10 beds and 16 people: half are banged up for economic crimes, three junkies, a car thief, a fraudster, a member of parliament, an alimony dodger, a killer, a gangster, a political prisoner (Kazakov) – in short, it is Noah’s Arc. Movement does not cease for 24 hours a day and there is some sort of freedom hanging in the air, not just the smoke and sweat.


I was offered tea and given a newspaper with a report about our trial to read. Cellmates compared me to the photo – do I look like it or not? The new rhythm and the atmosphere of freedom had a strange effect on me: I was in a stupor for three days. After half a year in Americanka you become so estranged from large society, retreat so deeply into yourself! The other men noticed this, expressed sympathy and interest in the conditions at Americanka, in the pressure techniques used there. I tried to tell them how it was but I felt that I could not put some things into words. How does one convey the feeling of waiting for abuse to start, how it grows with each passing day? Or the feeling of a constant watch? Here in the cell there were blindspots for the eye, there was a fenced-off (!) toilet, there was at least a little bit of privacy. One has to lose even that to understand what it means for an individual to be deprived of autonomous space.


I was put on the night shift: from 8am until 8pm the bed was mine, then in the next 12 hours I had to swap with another person. And so it went – I slept in the daytime and at night I socialized, played backgammon and chess, sorted out my affairs. Prisoners’ wit enabled me to get in touch with Sasha – now that was a real gift! We made the best of that opportunity. We found ourselves to be completely in sync in our analysis of the events. It’s so cool when there is unity and understanding despite isolation and depression. Outside everything was not quite the way we would have liked it to be: there were plenty of losses and disappointments. Well, you can’t help it; as the legionnaires said, march or die.


A brief visit from my mother. Finally we got to speak without looking over our shoulders at the KGB agents. I found out about events of the last six months in some detail. It was as though a new world opened. The vacuum created an illusion of silence but in fact the movement outside was most active. The floodgates crashed. In came a flood of letters with words of support and solidarity from all sorts of people, people I knew before and people I did not know, and sometimes very unexpected people too. Such situations somehow fill you with life energy, make you much stronger.


..In 10 days I learnt some basics of a convict’s life. Different people, different paths, different lifestyles, but mean-old fate brought us all together, in trouble. The reality of the “law-enforcement” system became clear from speaking to people about their cases: the situation in prisons and penal colonies; the tactics of the defendants and the investigators, developed over thousands upon thousands of incidents and passed from prisoner to prisoner like ancient wisdom. The punitive system itself is an organic part of Belarusian society in general. The information I got didn’t come from any anarchist reasoning, but from conversations of solidarity and trust, which appear in heavy and extreme conditions of imprisonment. These are the opinions of politicians, business owners, scientists, officials, representatives of security services and the criminal world.


\chapter{22
}

\unknown{}Prisoner transport. I don’t even know where. They took me in the morning and until the evening I stayed in a special transit cell with dozens of other unfortunate guys. All of them are youth from all over the country, their faces perplexed and alarmed. First there is a prison shakedown, then a shakedown by the escort convoy with rubber gloves and metal detectors. In the transit cell I catch up with a political prisoner – Kirkevich, who also used to be in Americanka; he still speaks only Belarusian, and in Volodarka he stаyed with Kolya. We take out a kettle, a cup, and drink tea. After a new shuffle I appear in a new cell. There is no water here. Finally we are taken out. We line up along the wall under the yard arch. They call your last name, and you grab your stuff and get into a prison transit van. With all the stuff we fill the vans’ sections, like sardines in a tin. They take us to the station. The transfer point is closed off, we get on the train one by one through a corridor of guards. Here it is. The famous Stolypin car\high{\footnote{Stolypin car – a type of railroad carriage used to implement a massive resettlement of peasants to Siberia during the Stolypin reform. It consisted of two compartments: one for passengers and another for livestock and agricultural tools. After the Bolshevik Revolution, Cheka and NKVD found these carriages convenient for transporting large numbers of incarcerated convicts and exiles: the passenger part was used for prison guards, whereas the cattle part was used for prisoners.}}. Three-tier sleeping car, no windows, bars separate it from the corridor. Each tier has a solid ceiling, to get to the next one you need to crawl through a manhole. Chinese express train.


That’s it; ahead of me is a new stage. What have I learnt over this time? Real wealth is when people stay with you despite all the hardships. You can be sure only in those who share all your hardships with you, nearby and at a distance. The rest is fragile.


The past exists only in your memories, the future – in your imagination, but what is truly important is the present, a specific moment of time. The past will fade, it will be distorted and lied about, the expected future may not come true at all, but looking back and going forward fills this Here and Now with meaning.


Today there is no freedom. While the state exists we cannot be free. But we can sense freedom; feel its breath by fighting for it. The struggle brings to life all the feelings and thoughts that are suppressed by the state discipline. In the fight for freedom we not only anticipate the long awaited day of justice and triumph but we also save our own personality from the boredom of existence and degradation. Any act of liberation makes sense only Here and Now.


Years of darkness and hardships lie ahead of me, but it doesn’t sadden me. The worse the better, as what doesn’t kill us makes us stronger. To turn the entire ordeal for my own good is the right decision. To learn from within this world, born of the century-old cycle of millions of human destinies in the dungeons of prisons and colonies.


\unknown{} The train hurries to the north. Everybody is sleeping; only two weary cons discuss life in a colony and a guard wearing an armoured jacket walks slowly along the corridor of the Stolypin-car. A popular song “Going to Magadan”\high{\footnote{A song about a prisoner who is being transported to Magadan. Magadan is the farthest eastern point of the Russian Federation, it is renowned for a labour camp, organised in the time of the USSR.}} runs in my head and I feel relieved.


{\em Prisoner}


{\em Ihar Alinevich}


{\em Summer 2011}









\page[yes]

%%%% backcover

\startmode[a4imposed,a4imposedbc,letterimposed,letterimposedbc,a5imposed,%
  a5imposedbc,halfletterimposed,halfletterimposedbc,quickimpose]
\alibraryflushpages
\stopmode

\page[blank]

\startalignment[middle]
{\tfa The Anarchist Library
\blank[small]
Anti-Copyright}
\blank[small]
\currentdate
\stopalignment

\blank[big]
\framed[frame=off,location=middle,width=\textwidth]
       {\externalfigure[logo][width=0.25\textwidth]}



\vfill
\setupindenting[no]
\setsmallbodyfont

\startalignment[middle,nothyphenated,nothanging,stretch]

\blank[line]
% \framed[frame=off,location=middle,width=\textwidth]
%       {\externalfigure[logo][width=0.25\textwidth]}


Ihar Alinevich



On the Way to Magadan



prisoner’s diary




2014


\stopalignment
\blank[line]

\startalignment[hyphenated,middle]


You can find more about Ihar and other Belarusian imprisoned anarchists and antifascists at www.abc-belarus.org



Retrieved on July 29, 2014 http://abc-belarus.org/files/2014/01/\%D0\%BC\%D0\%B0\%D0\%BA\%D0\%B5\%D1\%82-eng.pdf


\stopalignment

\stoptext


