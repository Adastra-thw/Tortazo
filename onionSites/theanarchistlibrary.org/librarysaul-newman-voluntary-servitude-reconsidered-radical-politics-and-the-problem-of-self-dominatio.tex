% -*- mode: tex -*-
%%%%%%%%%%%%%%%%%%%%%%%%%%%%%%%%%%%%%%%%%%%%%%%%%%%%%%%%%%%%%%%%%%%%%%%%%%%%%%%%
%                                STANDARD                                      %
%%%%%%%%%%%%%%%%%%%%%%%%%%%%%%%%%%%%%%%%%%%%%%%%%%%%%%%%%%%%%%%%%%%%%%%%%%%%%%%%
\definefontfeature[default][default]
                  [protrusion=quality,
                    expansion=quality,
                    script=latn]
\setupalign[hz,hanging]
\setuptolerance[tolerant]
\setbreakpoints[compound]
\setupindenting[yes,1em]
\setupfootnotes[way=bychapter,align={hz,hanging}]
\setupbodyfont[modern] % this is a stinky workaround to load lmodern
\setupbodyfont[libertine,11pt]

\setuppagenumbering[alternative=singlesided,location={footer,middle}]
\setupcaptions[width=fit,align={hz,hanging},number=no]

\startmode[a4imposed,a4imposedbc,letterimposed,letterimposedbc,a5imposed,%
  a5imposedbc,halfletterimposed,halfletterimposedbc]
  \setuppagenumbering[alternative=doublesided]
\stopmode

\setupbodyfontenvironment[default][em=italic]


\setupheads[%
  sectionnumber=no,number=no,
  align=flushleft,
  align={flushleft,nothyphenated,verytolerant,stretch},
  indentnext=yes,
  tolerance=verytolerant]

\definehead[awikipart][chapter]

\setuphead[awikipart]
          [%
            number=no,
            footer=empty,
            style=\bfd,
            before={\blank[force,2*big]},
            align={middle,nothyphenated,verytolerant,stretch},
            after={\page[yes]}
          ]

% h3
\setuphead[chapter]
          [style=\bfc]

\setuphead[title]
          [style=\bfc]


% h4
\setuphead[section]
          [style=\bfb]

% h5
\setuphead[subsection]
          [style=\bfa]

% h6
\setuphead[subsubsection]
          [style=bold]


\setuplist[awikipart]
          [alternative=b,
            interaction=all,
            width=0mm,
            distance=0mm,
            before={\blank[medium]},
            after={\blank[small]},
            style=\bfa,
            criterium=all]
\setuplist[chapter]
          [alternative=c,
            interaction=all,
            width=1mm,
            before={\blank[small]},
            style=bold,
            criterium=all]
\setuplist[section]
          [alternative=c,
            interaction=all,
            width=1mm,
            style=\tf,
            criterium=all]
\setuplist[subsection]
          [alternative=c,
            interaction=all,
            width=8mm,
            distance=0mm,
            style=\tf,
            criterium=all]
\setuplist[subsubsection]
          [alternative=c,
            interaction=all,
            width=15mm,
            style=\tf,
            criterium=all]


% center

\definestartstop
  [awikicenter]
  [before={\blank[line]\startalignment[middle]},
   after={\stopalignment\blank[line]}]

% right

\definestartstop
  [awikiright]
  [before={\blank[line]\startalignment[flushright]},
   after={\stopalignment\blank[line]}]


% blockquote

\definestartstop
  [blockquote]
  [before={\blank[big]
    \setupnarrower[middle=1em]
    \startnarrower
    \setupindenting[no]
    \setupwhitespace[medium]},
  after={\stopnarrower
    \blank[big]}]

% verse

\definestartstop
  [awikiverse]
  [before={\blank[big]
      \setupnarrower[middle=2em]
      \startnarrower
      \startlines},
    after={\stoplines
      \stopnarrower
      \blank[big]}]

\definestartstop
  [awikibiblio]
  [before={%
      \blank[big]
      \setupnarrower[left=1em]
      \startnarrower[left]
        \setupindenting[yes,-1em,first]},
    after={\stopnarrower
      \blank[big]}]
                
% same as above, but with no spacing around
\definestartstop
  [awikiplay]
  [before={%
      \setupnarrower[left=1em]
      \startnarrower[left]
        \setupindenting[yes,-1em,first]},
    after={\stopnarrower}]



% interaction
% we start the interaction only if it's not an imposed format.
\startnotmode[a4imposed,a4imposedbc,letterimposed,letterimposedbc,a5imposed,%
  a5imposedbc,halfletterimposed,halfletterimposedbc]
  \setupinteraction[state=start,color=black,contrastcolor=black,style=bold]
  \placebookmarks[awikipart,chapter,section,subsection,subsubsection][force=yes]
  \setupinteractionscreen[option=bookmark]
\stopnotmode



\setupexternalfigures[%
  maxwidth=\textwidth,
  maxheight=\textheight,
  factor=fit]

\setupitemgroup[itemize][each][packed][indenting=no]

\definemakeup[titlepage][pagestate=start,doublesided=no]

%%%%%%%%%%%%%%%%%%%%%%%%%%%%%%%%%%%%%%%%%%%%%%%%%%%%%%%%%%%%%%%%%%%%%%%%%%%%%%%%
%                                IMPOSER                                       %
%%%%%%%%%%%%%%%%%%%%%%%%%%%%%%%%%%%%%%%%%%%%%%%%%%%%%%%%%%%%%%%%%%%%%%%%%%%%%%%%

\startusercode

function optimize_signature(pages,min,max)
   local minsignature = min or 40
   local maxsignature = max or 80
   local originalpages = pages

   -- here we want to be sure that the max and min are actual *4
   if (minsignature%4) ~= 0 then
      global.texio.write_nl('term and log', "The minsig you provided is not a multiple of 4, rounding up")
      minsignature = minsignature + (4 - (minsignature % 4))
   end
   if (maxsignature%4) ~= 0 then
      global.texio.write_nl('term and log', "The maxsig you provided is not a multiple of 4, rounding up")
      maxsignature = maxsignature + (4 - (maxsignature % 4))
   end
   global.assert((minsignature % 4) == 0, "I suppose something is wrong, not a n*4")
   global.assert((maxsignature % 4) == 0, "I suppose something is wrong, not a n*4")

   --set needed pages to and and signature to 0
   local neededpages, signature = 0,0

   -- this means that we have to work with n*4, if not, add them to
   -- needed pages 
   local modulo = pages % 4
   if modulo==0 then
      signature=pages
   else
      neededpages = 4 - modulo
   end

   -- add the needed pages to pages
   pages = pages + neededpages
   
   if ((minsignature == 0) or (maxsignature == 0)) then 
      signature = pages -- the whole text
   else
      -- give a try with the signature
      signature = find_signature(pages, maxsignature)
      
      -- if the pages, are more than the max signature, find the right one
      if pages>maxsignature then
	 while signature<minsignature do
	    pages = pages + 4
	    neededpages = 4 + neededpages
	    signature = find_signature(pages, maxsignature)
	    --         global.texio.write_nl('term and log', "Trying signature of " .. signature)
	 end
      end
      global.texio.write_nl('term and log', "Parameters:: maxsignature=" .. maxsignature ..
		   " minsignature=" .. minsignature)

   end
   global.texio.write_nl('term and log', "ImposerMessage:: Original pages: " .. originalpages .. "; " .. 
	 "Signature is " .. signature .. ", " ..
	 neededpages .. " pages are needed, " .. 
	 pages ..  " of output")
   -- let's do it
   tex.print("\\dorecurse{" .. neededpages .. "}{\\page[empty]}")

end

function find_signature(number, maxsignature)
   global.assert(number>3, "I can't find the signature for" .. number .. "pages")
   global.assert((number % 4) == 0, "I suppose something is wrong, not a n*4")
   local i = maxsignature
   while i>0 do
      -- global.texio.write_nl('term and log', "Trying " .. i  .. "for max of " .. maxsignature)
      if (number % i) == 0 then
	 return i
      end
      i = i - 4
   end
end

\stopusercode

\define[1]\fillthesignature{
  \usercode{optimize_signature(#1, 40, 80)}}


\define\alibraryflushpages{
  \page[yes] % reset the page
  \fillthesignature{\the\realpageno}
}


% various papers 
\definepapersize[halfletter][width=5.5in,height=8.5in]
\definepapersize[halfafour][width=148.5mm,height=210mm]
\definepapersize[quarterletter][width=4.25in,height=5.5in]
\definepapersize[halfafive][width=105mm,height=148mm]
\definepapersize[generic][width=210mm,height=279.4mm]

%% this is the default ``paper'' which should work with both letter and a4

\setuppapersize[generic][generic]
\setuplayout[%
  backspace=42mm,
  topspace=31mm,% 176 / 15
  height=195mm,%130mm,
  footer=9mm, %
  header=0pt, % no header
  width=126mm] % 10.5 x 11

\startmode[libertine]
  \usetypescript[libertine]
  \setupbodyfont[libertine,11pt]
\stopmode

\startmode[pagella]
  \setupbodyfont[pagella,11pt]
\stopmode

\startmode[antykwa]
  \setupbodyfont[antykwa-poltawskiego,11pt]
\stopmode

\startmode[iwona]
  \setupbodyfont[iwona-medium,11pt]
\stopmode

\startmode[helvetica]
  \setupbodyfont[heros,11pt]
\stopmode

\startmode[century]
  \setupbodyfont[schola,11pt]
\stopmode

\startmode[modern]
  \setupbodyfont[modern,11pt]
\stopmode

\startmode[charis]
  \setupbodyfont[charis,11pt]
\stopmode        

\startmode[mini]
  \setuppapersize[S33][S33] % 176 × 176 mm
  \setuplayout[%
    backspace=20pt,
    topspace=15pt,% 176 / 15
    height=280pt,%130mm,
    footer=20pt, %
    header=0pt, % no header
    width=260pt] % 10.5 x 11
\stopmode

% for the plain A4 and letter, we use the classic LaTeX dimensions
% from the article class
\startmode[a4]
  \setuppapersize[A4][A4]
  \setuplayout[%
    backspace=42mm,
    topspace=45mm,
    height=218mm,
    footer=10mm,
    header=0pt, % no header
    width=126mm]
\stopmode

\startmode[letter]
  \setuppapersize[letter][letter]
  \setuplayout[%
    backspace=44mm,
    topspace=46mm,
    height=199mm,
    footer=10mm,
    header=0pt, % no header
    width=126mm]
\stopmode


% A4 imposed (A5), with no bc

\startmode[a4imposed]
% DIV=15 148 × 210: these are meant not to have binding correction,
  % but just to play safe, let's say 1mm => 147x210
  \setuppapersize[halfafour][halfafour]
  \setuplayout[%
    backspace=10.8mm, % 146/15 = 9.8 + 1
    topspace=14mm, % 210/15 =  14
    height=182mm, % 14 x 12 + 14 of the footer
    footer=14mm, %
    header=0pt, % no header
    width=117.6mm] % 9.8 x 12
\stopmode

% A4 imposed (A5), with bc
\startmode[a4imposedbc]
  \setuppapersize[halfafour][halfafour]
  \setuplayout[% 14 mm was a bit too near to the spine, using the glue binding
    backspace=17.3mm,  % 140/15 + 8 =
    topspace=14mm, % 210/15 =  14
    height=182mm, % 14 x 12 + 14 of the footer
    footer=14mm, %
    header=0pt, % no header
    width=112mm] % 9.333 x 12
\stopmode


\startmode[letterimposedbc] % 139.7mm x 215.9 mm
  \setuppapersize[halfletter][halfletter]
  % DIV=15 8mm binding corr, => 132 x 216
  \setuplayout[%
    backspace=16.8mm, % 8.8 + 8
    topspace=14.4mm, % 216/15 =  14.4
    height=187.2mm, % 15.4 x 11 + 15 of the footer
    footer=14.4mm, %
    header=0pt, % no header
    width=105.6mm] % 8.8 x 12
\stopmode

\startmode[letterimposed] % 139.7mm x 215.9 mm
  \setuppapersize[halfletter][halfletter]
  % DIV=15, 1mm binding correction. => 138.7x215.9
  \setuplayout[%
    backspace=10.3mm, % 9.24 + 1
    topspace=14.4mm, % 216/15 =  14.4
    height=187.2mm, % 15.4 x 11 + 15 of the footer
    footer=14.4mm, %
    header=0pt, % no header
    width=111mm] % 9.24 x 12
\stopmode

%%% new formats for mini books
%%% \definepapersize[halfafive][width=105mm,height=148mm]

\startmode[a5imposed]
% DIV=12 105x148 : these are meant not to have binding correction,
  % but just to play safe, let's say 1mm => 104x148
  \setuppapersize[halfafive][halfafive]
  \setuplayout[%
    backspace=9.6mm,
    topspace=12.3mm,
    height=123.5mm, % 14 x 12 + 14 of the footer
    footer=12.3mm, %
    header=0pt, % no header
    width=78.8mm] % 9.8 x 12
\stopmode

% A5 imposed (A6), with bc
\startmode[a5imposedbc]
% DIV=12 105x148 : with binding correction,
  % let's say 8mm => 96x148
  \setuppapersize[halfafive][halfafive]
  \setuplayout[%
    backspace=16mm,
    topspace=12.3mm,
    height=123.5mm, % 14 x 12 + 14 of the footer
    footer=12.3mm, %
    header=0pt, % no header
    width=72mm] % 9.8 x 12
\stopmode

%%% \definepapersize[quarterletter][width=4.25in,height=5.5in]

% DIV=12 width=4.25in (108mm),height=5.5in (140mm) 
\startmode[halfletterimposed] % 107x140
  \setuppapersize[quarterletter][quarterletter]
  \setuplayout[%
    backspace=10mm,
    topspace=11.6mm,
    height=116mm,
    footer=11.6mm,
    header=0pt, % no header
    width=80mm] % 9.24 x 12
\stopmode

\startmode[halfletterimposedbc]
  \setuppapersize[quarterletter][quarterletter]
  \setuplayout[%
    backspace=15.4mm,
    topspace=11.6mm,
    height=116mm,
    footer=11.6mm,
    header=0pt, % no header
    width=76mm] % 9.24 x 12
\stopmode

\startmode[quickimpose]
  \setuppapersize[A5][A4,landscape]
  \setuparranging[2UP]
  \setuppagenumbering[alternative=doublesided]
  \setuplayout[% 14 mm was a bit too near to the spine, using the glue binding
    backspace=17.3mm,  % 140/15 + 8 =
    topspace=14mm, % 210/15 =  14
    height=182mm, % 14 x 12 + 14 of the footer
    footer=14mm, %
    header=0pt, % no header
    width=112mm] % 9.333 x 12
\stopmode

\startmode[tenpt]
  \setupbodyfont[10pt]
\stopmode

\startmode[twelvept]
  \setupbodyfont[12pt]
\stopmode

%%%%%%%%%%%%%%%%%%%%%%%%%%%%%%%%%%%%%%%%%%%%%%%%%%%%%%%%%%%%%%%%%%%%%%%%%%%%%%%%
%                            DOCUMENT BEGINS                                   %
%%%%%%%%%%%%%%%%%%%%%%%%%%%%%%%%%%%%%%%%%%%%%%%%%%%%%%%%%%%%%%%%%%%%%%%%%%%%%%%%


\mainlanguage[en]


\starttext

\starttitlepagemakeup
  \startalignment[middle,nothanging,nothyphenated,stretch]


  \switchtobodyfont[18pt] % author
  {\bf \em

Saul Newman  \par}
  \blank[2*big]
  \switchtobodyfont[24pt] % title
  {\bf

Voluntary Servitude Reconsidered: Radical Politics and the Problem of Self-Domination

\par}
  \blank[big]
  \switchtobodyfont[20pt] % subtitle
  {\bf 

\par}
  \vfill
  \stopalignment
  \startalignment[middle,bottom,nothyphenated,stretch,nothanging]
  \switchtobodyfont[global]

2010

  \stopalignment
\stoptitlepagemakeup



\title{Contents}

\placelist[awikipart,chapter,section,subsection]



\page[yes,right]

\section{Abstract
}

In this paper I investigate the problem of voluntary servitude — first elaborated by Etienne de la Boëtie — and explore its implications for radical political theory today. The desire for one’s own domination has proved a major hindrance to projects of human liberation such as revolutionary Marxism and anarchism, necessitating new understandings of subjectivity and revolutionary desire. Central here, as I show, are micro-political and ethical projects of interrogating one’s own subjective attachment to power and authority — projects elaborated, in different ways, by thinkers as diverse as Max Stirner, Gustav Landauer and Michel Foucault. I argue that the question of voluntary servitude must be taken more seriously by political theory, and that an engagement with this problem brings to the surface a counter-sovereign tradition in politics in which the central concern is not the legitimacy of political power, but rather the possibilities for new practices of freedom.


\section{Introduction
}

In this paper I will explore the genealogy of a certain counter-sovereign political discourse, one that starts with the question ‘why do we obey?’ This question, initially posed by the philosopher Etienne de la Boëtie in his investigations on tyranny and our voluntary servitude to it, starts from the opposite position to the problematic of sovereignty staked out by Bodin and Hobbes. Moreover, it remains a central and unresolved problem in radical political thought which works necessarily within the ethical horizon of emancipation from political power. I suggest that encountering the problem of voluntary servitude necessitates an exploration of new forms of subjectivity, ethics and political practices through which our subjective bonds to power are interrogated; and I explore these possibilities through the revolutionary tradition of anarchism, as well as through an engagement with psychoanalytic theory. My contention here is that we cannot counter the problem of voluntary servitude without a critique of idealization and identification, and here I turn to thinkers like Max Stirner, Gustav Landauer and Michel Foucault, all of whom, in different ways, develop a micropolitics and ethics of freedom which aims at undoing the bonds between the subject and power.


\section{The Powerlessness of Power
}

The question posed by Etienne De La Boëtie in the middle of the sixteenth century in {\em Discours de la servitude volontaire, ou Le Contr’Un,} remains with us today and can still be considered a fundamental political question:



\startblockquote
My sole aim on this occasion is to discover how it can happen that a vast number of individuals, of towns, cities and nations can allow one man to tyrannize them, a man who has no power except the power they themselves give him, who could do them no harm were they not willing to suffer harm, and who could never wrong them were they not more ready to endure it than to stand in his way (Etienne De La Boëtie, 1988).



\stopblockquote
La Boëtie explores the subjective bond which ties us to the power that dominates us, which enthrals and seduces us, blinds us and mesmerizes us. The essential lesson here is that the power cannot rely on coercion, but in reality rests on our power. Our active acquiescence to power at the same time constitutes this power. For La Boëtie, then, in order to resist the tyrant, all we need do is turn our backs on him, withdraw our active support from him and perceive, through the illusory spell that power manages to cast over us — an illusion that we participate in — his weakness and vulnerability. Servitude, then, is a condition of our own making — it is entirely voluntary; and all it takes to untie us from this condition is the desire to no longer be subjugated, the will to be free.


This problem of voluntary servitude is the exact opposite of that raised by Hobbes a century later. Whereas for La Boëtie, it is unnatural for us to be subjected to absolute power, for Hobbes it is unnatural for us to live in any other condition; the anarchy of the state of nature, for Hobbes, is precisely an unnatural and unbearable situation. La Boëtie’s problematic of self-domination thus inverts a whole tradition of political theory based on legitimizing the sovereign — a tradition that is still very much with us today. La Boëtie starts from the opposite position, which is that of the primacy of liberty, self-determination and the natural bonds of family and companionship, as opposed to the unnatural, artificial bonds of political domination. Liberty is something which must be protected not so much against those who wish to impose their will on us, but against our own temptation to relinquish our liberty, to be dazzled by authority, to barter away our liberty in return for wealth, positions, favours, and so on. What must be explained, then, is the pathological bond to power which displaces the natural desire for liberty and the free bonds that exist between people.


Boëtie’s explanations for voluntary servitude are not entirely adequate or convincing, however: he attributes it to a kind of denaturing, whereby free men become effeminate and cowardly, thus allowing another to dominate them. Nevertheless, he raises, I think, one of the fundamental questions for politics — and especially for radical politics — namely, why do people at some level desire their own domination? This question inaugurates a counter-sovereign political theory, a libertarian line of investigation which is taken up by a number of thinkers. Wilhelm Reich, for instance, in his Freudo-Marxist analysis of the mass psychology of fascism, pointed to a desire for domination and authority which could not be adequately explained through the Marxist category of ideological false consciousness (Reich, 1980). Pierre Clastres, the anthropologist of liberty, saw the value of La Boëtie in showing us the possibility that domination is not inevitable; that voluntary servitude resulted from a misfortune of history (or pre-history), a certain fall from grace, a lapse from the condition of primitive freedom and statelessness into a society divided between dominators and the dominated. Here, man occupies the condition of the {\em unnameable} (neither man nor animal): so alienated is he from his natural freedom, that he freely chooses, {\em desires,} servitude — a desire which was entirely unknown in primitive societies (Clastres, 1994: 93–104). Following on from Clastres’ account, Gilles Deleuze and Felix Guattari explored the emergence of the state, and the way in which it relies not so much, or not entirely, on violent domination and capture, but rather on the self-domination of the subject at the level of his or her desire — a repression which is itself desired. The state acts to channel the subject’s desire through authoritarian and hierarchical structures of thought and modes of individualization.\footnote{Deleuze and Guattari point to the mysterious way that we are tied to State power, something which the term ‘voluntary servitude’ both illuminates and obscures: “The State is assuredly not the locus of liberty, nor the agent of forced servitude or capture. Should we then speak of ‘voluntary servitude’?” See Deleuze \& Guattari (2005: 460).}


Moreover, the Situationist Raoul Vanegeim showed, in an analysis that bears many a striking likeness to La Boëtie’s, that our obedience is bought and sustained by minor compensations, a little bit of power as a psychological pay-off for the humiliation of our own domination:



\startblockquote
Slaves are not willing slaves for long if they are not compensated for their submission by a shred of power: all subjection entails the right to a measure of power, and there is no such thing as power that does not embody a degree of submissions. That is why some agree so readily to be governed (Vanegeim, 1994: 132).



\stopblockquote
\section{Another Politics\unknown{}?
}

The problem of self-domination shows us that the connection between politics and subjectification must be more thoroughly investigated. To create new forms of politics — which is the fundamental theoretical task today — requires new forms of subjectivity, new modes of subjectivisation. Moreover, to counter voluntary servitude will involve new political strategies, indeed a different understanding of politics itself. Quite rightly, La Boëtie recognizes the potential for domination in any democracy: the democratic leader, elected by the people, becomes intoxicated with his own power and teeters increasingly towards tyranny. Indeed, we can see modern democracy itself as an instance of voluntary servitude on a mass scale. It is not so much that we participate in an illusion whereby we are deceived by elites into thinking we have a genuine say in decision-making. It is rather that democracy itself has encouraged a mass contentment with powerlessness and a general love of submission.


As an alternative, La Boëtie asserts the idea of a free republic. However, I would suggest that the inverse of voluntary servitude is not a free republic, but another form of politics entirely. Free republics have a domination of their own, not only in their laws, but in the rule of the rich and propertied classes over the poor. Rather, when we consider alternative forms of politics, when we think about ways of enacting and maximizing the possibilities of non-domination, I think we ought to consider the politics of anarchism — which is a {\em politics of anti-politics}, a politics which seeks the abolition of the structures of political power and authority enshrined in the state.


Anarchism, this most heretical of radical political philosophies, has led for a long time a marginalized existence. This is due in part to its heterodox nature, to the way it cannot be encompassed within a single system of ideas or body of thought, but rather refers to a diverse ensemble of ideas, philosophical approaches, revolutionary practices and historical movements and identities. However, what makes a reconsideration of anarchist thought essential here is that out of all the radical traditions, it is the one that is most sensitive to the dangers of political power, to the potential for authoritarianism and domination contained within any political arrangement or institution. In this sense, it is particularly wary of the bonds through which people are tied to power. That is why, unlike the Marxist-Leninists, anarchists insisted that the state must be abolished in the first stages of the revolution: if, on the other hand, state power was seized by a vanguard and used — under the ‘dictatorship of the proletariat’ — to revolutionize society, it will, rather than eventually ‘withering away’, expand in size and power, engendering new class contradictions and antagonisms. To imagine, in other words, that the state was a kind of neutral mechanism that could be used as a tool of liberation if the right class controlled it, was, according to the classical anarchists of the nineteenth century, engaged as they were in major debates with Marx, a pure fantasy that ignored the inextricable logic of state domination and the temptations and lures of political power. That was why the Russian anarchist Peter Kropotkin insisted that the state must be examined as a specific structure of power which could not be reduced to the interests of a particular class. It was — in its very essence — dominating: “And there are those who, like us, see in the State, not only its actual form and in all forms of domination that it might assume, but in its very essence, an obstacle to the social revolution” (Kropotkin, 1943). The power of the state, moreover, perpetuates itself through the subjective bond that it forms with those who attempt to control it, through the corrupting influence it has on them. In the words of another anarchist, Mikhail Bakunin, “We of course are all sincere socialists and revolutionists and still, were we to be endowed with power [\unknown{}] we would not be where we are now” (Bakunin, 1953: 249).


This uncompromising critique of political power, and the conviction that freedom cannot be conceived within the framework of the state, is what distinguishes anarchism from other political philosophies. It contrasts with liberalism, which is in reality a politics of security, where the state becomes necessary to protect individual liberty from the liberty of others: indeed, the current securitization of the state through the permanent state of exception reveals the true face of liberalism. It differs also in this respect from socialism, which sees the state as essential for making society more equal, and whose terminal decline can be witnessed in the sad fate of social democratic parties today with their authoritarian centralism, their law and order fetishes and their utter complicity with global neoliberalism. Furthermore, anarchism is to be distinguished from revolutionary Leninism, which now represents a completely defunct model of radical politics. What defines anarchism, then, is the refusal of state power, even of the revolutionary strategy of {\em seizing} state power. Instead, the focus of anarchism is on self-emancipation and autonomy, something which cannot be achieved through parliamentary democratic channels or through revolutionary vanguards, but rather through the development of alternative practices and relationships based on free association, equal liberty and voluntary cooperation.


It is because of its alterity or exteriority to other state-centred modes of politics that anarchism has been largely overshadowed within the radical political tradition. Yet, I would argue that currently we are in a kind of anarchist moment politically. What I mean is that with the eclipse of the socialist state project and revolutionary Leninism, and with the devolving of liberal democracy into a narrow politics of security, that radical politics today tends to situate itself increasingly outside the state. Contemporary radical activism seems to reflect certain anarchist orientations in its emphasis on decentralized networks and direct action, rather than party leadership and political representation. There is a kind of disengagement from state power, a desire to think and act beyond its structures, in the direction of greater autonomy. These tendencies are becoming more pronounced with the current economic crisis, something which is pointing to the very limits of capitalism itself, and certainly to the end of the neoliberal economic model. The answer to the failings of neoliberalism is not more state intervention. It is ludicrous to talk about the return of the regulatory state: the state in fact never went away under neoliberalism, and the whole ideology of economic ‘libertarianism’ concealed a much more intensive deployment of state power in the domain of security, and in the regulation, disciplining and surveillance of social life. It is clear, moreover, that the state will not help us in this current situation; there is no point looking to it for protection. Indeed, what is emerging is a kind of disengagement from the state; the coming insurrections will challenge the hegemony of the state, which we see increasingly governing through the logic of exception.


Moreover, the relevance of anarchism is also reflected at a theoretical level. Many of themes and preoccupations of contemporary continental thinkers for instance — the idea of non-state, non-party and post-class forms of politics, the coming of the multitudes and so on — seem to invoke an anarchist politics. Indeed, this is particularly evident in the search for a new political subject: the multitudes of Michael Hardt and Antonio Negri, the people for Ernesto Laclau, the excluded part-of-no-part for Jacques Rancière, the figure of the militant for Alain Badiou; all this reflects an attempt to think about new modes of subjectification which are perhaps broader and less constraining than the category of the proletariat as politically constituted through the Marxist-Leninist vanguard. A similar approach to political subjectivity was proposed by the anarchists in the nineteenth century, who claimed that the Marxist notion of the revolutionary class was exclusivist, and who sought to include the peasantry and {\em lumpenroletariat} as revolutionary identities.\footnote{See Bakunin’s notion of the revolutionary mass as opposed to the Marxist category of class (Bakunin, 1984: 47).} In my view, anarchism is the ‘missing link’ in contemporary continental political thought — a spectral presence which is never really acknowledged.\footnote{For a discussion of the relevance of classical anarchism and contemporary radical political philosophy, see my paper (2007) ‘Anarchism, Poststructuralism and the Future of Radical Politics Today’, {\em Substance} (113)36/2.}


\section{The Anarchist Subject
}

Anarchism is a politics and ethics in which power is continually interrogated in the name of human freedom, and in which human existence is posited in the absence of authority. However, this raises the question of whether there is an anarchist subject as such. Here I would like to reconsider anarchism through the problem of voluntary servitude. While the classical anarchists were not unaware of the desires for power that lay at the heart of the human subject — which is why they were so keen to abolish the structures of power which would incite these desires — the problem of self-domination, the desire for one’s own domination, remains insufficiently theorized in anarchism.\footnote{This acknowledgement of the desire for power at the heart of human subjectivity does not endorse the Hobbesian position affirming the need for a strong sovereign. On the contrary, it makes the goal of fragmenting and abolishing centralized structures of power and authority all the more necessary. Surely if, in other words, human nature is prone to the temptations of power and the desire for domination, the last thing we should do is trust a sovereign with absolute power over us. A similar point is made by Paolo Virno (see the essay ‘Multitude and Evil’), who argues that if we are to accept the ‘realist’ claim that we have as humans a capacity for ‘evil’, then, rather than this justifying centralized state authority, we should be even more cautious about the concentration of power and violence in the hands of the state (cf., Virno, 2008).} For the anarchists of the eighteenth and nineteenth centuries — such as William Godwin, Pierre-Joseph Proudhon, Mikhail Bakunin and Peter Kropotkin — conditioned as they were by the rationalist discourses of Enlightenment humanism, the human subject naturally desired freedom; thus the revolution against state power was part of the rational narrative of human emancipation. The external and artificial constraints of state power would be thrown off, so that man’s essential rational and moral properties could be expressed and society could be in harmony with itself. There is a kind of Manichean opposition that is presupposed in classical anarchist thought, between human society which is governed by natural laws, and political power and man-made law, embodied in the state, which is artificial, irrational and a constraint on the free development of social forces. There is, furthermore, an innate sociability in man — a natural tendency, as Kropotkin saw it, towards mutual aid and cooperation — which was distorted by the state, but which, if allowed to flourish, would produce a social harmony in which the state would become unnecessary (cf., Kropotkin, 2007).


While the idea of a society without a state, without sovereignty and law is desirable, and indeed the ultimate horizon of radical politics, and while there can be no doubt that political and legal authority is an oppressive encumbrance on social life and human existence generally, what tends to be obscured in this ontological separation between the subject and power is the problem of voluntary servitude — which points to the more troubling complicity between the subject and the power that dominates him. To take this into account, to explain the desire for self-domination and to develop strategies — ethical and political strategies — to counter it, would be to propose an anarchist theory of subjectivity, or at least a more developed one than can be found in classical anarchist thought. It would also imply a move beyond some of the essentialist and rationalist categories of classical anarchism, a move that elsewhere I have referred to as postanarchism (Newman, 2010). This is not to suggest that the classical anarchists were necessarily naïve about human nature or politics; rather that its humanism and rationalism resulted in a kind of blind-spot around the question of desire, whose dark, convoluted, self-destructive and irrational nature would be revealed later by psychoanalysis.


\section{Psychoanalysis and Passionate Attachments
}

So it is important to explore the subjective bond to power at the level of the psyche.\footnote{This is similar to what Jason Glynos refers to as the problem of self-transgression (see Glyno, 2008) The argument here is that conceptualization and practice of freedom is often complicated by various forms of self-transgression, where the subject engages in activities which limit his or her freedom — which prevent him or her from achieving one’s object of desire, or achieving a certain ideal that one might have of oneself — because of the unconscious enjoyment ({\em jouissance}) derived from this transgression. Thus the limitation to the subject’s freedom is not longer {\em external} (as in the paradigm of negative freedom) but {\em internal}. This might be another way of thinking about the problem of voluntary servitude through the lens of psychoanalysis.} A psychological dependency on power, which was explored by Freudo-Marxists such as Marcuse and Reich\footnote{See also Theodore Adorno’s study [et al] {\em The Authoritarian Personality} (1964).}, meant that the possibilities of emancipatory politics are at times compromised by hidden authoritarian desires; that there was always a risk of authoritarian and hierarchical practices and institutions emerging in post-revolutionary societies. The central place of the subject — in politics, philosophy — is not abandoned here but complicated. Radical political projects, for instance, have to contend with the ambiguities of human desire, with irrational social behaviour, with violent and aggressive drives, and even with unconscious desires for authority and domination.


This is not to suggest that psychoanalysis is necessarily politically or socially conservative. On the contrary, I would maintain that central to psychoanalysis is a libertarian ethos by which the subject seeks to gain a greater autonomy, and where the subject is encouraged, through the rules of ‘free association’, to speak the truth of the unconscious.\footnote{According to Mikkel Borch-Jacobsen, Freud’s psychoanalytic theory of groups implies “something like a “revolt or an uprising against the hypnotist’s unjustifiable power” (1988: 148).} To insist on the ‘dark side’ of the human psyche — its dependence on power, its identification with authoritarian figures, its aggressive impulses — can serve as a warning to any revolutionary project which seeks to transcend political authority. This was really the same question that was posed by Jacques Lacan in response to the radicalism of May ’68: “the revolutionary aspiration has only a single possible outcome — of ending up as the master’s discourse. This is what experience has proved. What you aspire to as revolutionaries is a master. You will get one” (Lacan, 2007: 207). What Lacan is hinting at with this rather ominous prognostication — one that could be superficially, although, in my view, {\em incorrectly}, interpreted as politically conservative — is the hidden link, even dependency, between the revolutionary subject and authority; and the way that movements of resistance and even revolution may actually sustain the symbolic efficiency of the state, reaffirming or reinventing the position of authority.


Psychoanalysis by no means discounts the possibility of human emancipation, sociability and voluntary cooperation: indeed, it points to conflicting tendencies in the subject between aggressive desires for power and domination, and the desire for freedom and harmonious co-existence. As Judith Butler contends, moreover, the psyche — as a dimension of the subject that is not reducible to discourse and power, and which exceeds it — is something that can explain not only our passionate attachments to power and (referring to Foucault) to the modes of subjectification and regulatory behaviours that power imposes on us, but also our resistance to them (Butler, 1997: 86).


\section{Ego Identification 
}

One of the insights of psychoanalysis, something that was revealed, for instance, in Freud’s study of the psychodynamics of groups, was the role of identification in constituting hierarchical and authoritarian relationships. In the relationship between the member of the group and the figure of the Leader, there is a process of identification, akin to love, in which the individual both idealizes and identifies with the Leader as an ‘ideal type’, to the point where this object of devotion comes to supplant the individual’s ego ideal (Freud, 1955). It is this idealization that constitutes the subjective bond not only between the individual and the Leader of the group, but also with other members of the group. Idealization thus becomes a way of understanding voluntary submission to the will of authoritarian leaders.


However, we also need to understand the place of idealization in politics in a broader sense, and it is here that, I would argue, the thinking of the Young Hegelian philosopher, Max Stirner, becomes important. Stirner’s critique of Ludwig Feuerbach’s humanism allows us to engage with this problem of self-domination. Stirner shows that the Feuerbachian project of replacing God with Man — of reversing the subject and predicate so that the human becomes the measure of the divine rather than the divine the human (Feuerbach, 1957) — has only reaffirmed religious authority and hierarchy rather than displacing it. Feuerbach’s ‘humanist insurrection’ has thus only succeeded in creating a new religion — Humanism — which Stirner connects with a certain self-enslavement. The individual ego is now split between itself and an idealized form of itself now enshrined in the idea of human essence — an ideal which is at the same time outside the individual, becoming an abstract moral and rational spectre by which he measures himself and to which he subordinates himself. As Stirner declares: “Man, your head is haunted [\unknown{}] You imagine great things, and depict to yourself a whole world of gods that has an existence for you, a spirit-realm to which you suppose yourself to be called, an ideal that beckons to you” (Stirner, 1995: 43).


For Stirner, the subordination of the self to these abstract ideals (‘fixed ideas’) has political implications. Humanism and rationalism become in his analysis the discursive thresholds through which the desire of the individual is bound to the state. This occurs through identification with state-defined roles of citizenship, for instance. Moreover, for Stirner, in a line of thought that closely parallels La Boëtie’s, the state itself is an ideological abstraction which only exists because we allow it to exist, because we abdicate our own power over ourselves to what he called the ‘ruling principle’. In other words, it is the {\em idea} of the state, of sovereignty, that dominates us. The state’s power is in reality based on our power, and it is only because the individual has not recognized this power, because he humbles himself before an external political authority, that the state continues to exist. As Stirner correctly surmised, the state cannot function only through repression and coercion; rather, the state relies on us {\em allowing} it to dominate us. Stirner wants to show that ideological apparatuses are not only concerned with economic or political questions — they are also rooted in psychological needs. The dominance of the state, Stirner suggests, depends on our willingness to let it dominate us:



\startblockquote
What do your laws amount to if no one obeys them? What your orders, if nobody lets himself be ordered? [\unknown{}] The state is not thinkable without lordship [{\em Herrschaft}] and servitude [{\em Knechtschaft}] (subjection); [\unknown{}] He who, to hold his own, must count on the absence of the will in others is a thing made by these others, as the master is a thing made by the servant. If submissiveness ceased it would be all over with lordship (Stirner, 1995: 174–5).



\stopblockquote
Stirner was ruthlessly and relentlessly criticized by Marx and Engels as ‘Saint Max’ in {\em The German Ideology:} they accused him of the worst kind of idealism, of ignoring the economic and class relations that form the material basis of the state, and thus allowing the state to be simply {\em wished} out of existence. However, what is missed in this critique is the value of Stirner’s analysis in highlighting the subjective bond of voluntary servitude that sustains state power. It is not that he is saying that the state does not exist in a material sense, but that its existence is sustained and supplemented through a psychic attachment and dependency on its power, as well as the acknowledgement and idealization of its authority. Any critique of the state which ignores this dimension of subjective idealization is bound to perpetuate its power. The state must first be overcome as an {\em idea} before it can be overcome in reality; or, more precisely, these are two sides of the same process.


The importance of Stirner’s analysis — which broadly fits into the anarchist tradition, although breaks with its humanist essentialism in important ways\footnote{See my reading of Stirner as a poststructuralist anarchist in {\em From Bakunin to Lacan} (2001).} — lies in exploring this voluntary self-subjection that forms the other side of politics, and which radical politics must find strategies to counter. For Stirner, the individual can only free him or herself from voluntary servitude if he abandons all essential identities and sees himself as a radically self-creating void:



\startblockquote
I on my part start from a presupposition in presupposing myself; but my presupposition does not struggle for its perfection like ‘Man struggling for his perfection’, but only serves me to enjoy it and consume it [\unknown{}] I do not presuppose myself, because I am every moment just positing or creating myself (Stirner, 1995: 150).



\stopblockquote
While Stirner’s approach is focused on the idea of the individual’s self-liberation — from essences, fixed identities — he does raise the possibility of collective politics with his notion of the ‘union of egoists’, although in my view this is insufficiently developed. The breaking of the bonds of voluntary servitude cannot be a purely individual enterprise. Indeed, as La Boëtie suggests, it always implies a collective politics, a collective rejection of tyrannical power by the people. I am not suggesting that Stirner provides us with a complete or viable theory of political and ethical action. However, the importance of Stirner’s thought lies in the invention of a {\em micropolitics}, an emphasis on the myriad ways we are bound to power at the level of our subjectivity, and the ways we can free ourselves from this. It is here that we should pay close attention to his distinction between the Revolution and the insurrection:



\startblockquote
Revolution and insurrection must not be looked upon as synonymous. The former consists in an overturning of conditions, of the established condition or {\em status}, the state or society, and is accordingly a {\em political} or {\em social} act; the latter has indeed for its unavoidable consequence a transformation of circumstances, yet does not start from it but from men’s discontent with themselves, is not an armed rising but a rising of individuals, a getting up without regard to the {\em arrangements} that spring from it. The Revolution aimed at new arrangements; insurrection leads us no longer to {\em let} ourselves be arranged, but to arrange ourselves, and sets no glittering hopes on ‘institutions’. It is not a fight against the established, since, if it prospers, the established collapses of itself; it is only a working forth of me out of the established (Italics are Stirner’s; Stirner, 1995: 279–80).



\stopblockquote
We can take from this that radical politics must not {\em only} be aimed at overturning established institutions like the state, but also at attacking the much more problematic relation through which the subject is enthralled to and dependent upon power. The insurrection is therefore not only against external oppression, but, more fundamentally, against the self’s internalized repression. It thus involves a transformation of the subject, a micro-politics and ethics which is aimed at increasing one’s autonomy from power.


Here we can also draw upon the spiritual anarchism of Gustav Landauer, who argued that there can be no political revolution — and no possibility of socialism — without at the same time a transformation in the subjectivity of people, a certain renewal of the spirit and the will to develop new relationships with others. Existing relationships between people only reproduce and reaffirm state authority — indeed the state itself is a certain relationship, a certain mode of behaviour and interaction, a certain imprint on our subjectivity and consciousness (and I would argue on our {\em unconscious}) and therefore it can only be transcended through a spiritual transformation of relationships. As Landauer says, “we destroy it by contracting other relationships, by behaving differently” (in Martin Buber, 1996: 47).


\section{A Micro-Politics of Liberty
}

Therefore, overcoming the problem of voluntary servitude, which has proved such a hindrance to radical political projects in the past, implies this sort of ethical questioning of the self, an interrogation of one’s subjective involvement and complicity with power. It relies on the invention of micropolitical strategies that are aimed at a disengagement from state power; a certain politics of {\em dis-identification} in which one breaks free from established social identities and roles and develops new practices, ways of life and forms of politics that are no longer conditioned by state sovereignty. This would mean thinking about what freedom means beyond the ideology of security (rather than simply seeing freedom as conditioned by or necessarily constrained by security). We also need to think what democracy means beyond the state, what politics means beyond the party, economic organization beyond capitalism, globalization beyond borders, and life beyond biopolitics.


Central here has to be, for instance, a critical interrogation of the desire for security. Security, in our contemporary society, has become a kind of metaphysics, a fundamentalism, where not only is it the impetus behind an unprecedented expansion and intensification of state power, but also becomes a kind of condition for life: life must be secure against threats — whether they are threats to our safety, financial security, etc — but this means that the very existential possibility of not only human freedom, but politics itself, is being negated. Can the law and liberal institutional frameworks protect us from security; can it counter the relentless drive towards the securitization of life? We must remember that, as Giorgio Agamben and others have shown, biopolitics, sovereign violence and securitization are only the other side of the law, and that it is simply a liberal illusion to imagine that law can limit power in this way. No, we must invent a new relationship to law and institutions, no longer as subjects who are obedient, nor as subjects who simply transgress (which is only the other side to obedience — in other words, transgression, as we understand from Lacan, continues to affirm the law\footnote{See Lacan’s discussion of the dialectic of law and transgression in ‘Kant avec Sade’ (1962).}). Rather, we must transcend this binary of obedience/transgression. Anarchism is more than a transgression, but a learning to live beyond the law and the state through the invention of new spaces and practices for freedom and autonomy which will be, by their nature, somewhat fragile and experimental.


To take such risks requires discipline, but this can be a kind of ethical discipline that we impose on ourselves. We need to be disciplined to become undisciplined. Obedience to authority seems to come easily, indeed ‘naturally’, to us, as La Boëtie observed, and so the revolt against authority requires the disciplined and patient elaboration of new practices of freedom. This was something that Foucault was perhaps getting at with his notion of {\em askesis}, ethical exercises that were part of the care of the self, and which were for him indistinguishable from the practice of freedom (cf., Foucault, 1988). The aim of such strategies, for Foucault, was to invent modes of living in which one is ‘governed less’ or governed not at all. Indeed, the practice of critique itself, according to Foucault, is aimed not only at questioning power’s claim to legitimacy and truth, but more importantly, at questioning the various ways in which we are bound to power and regimes of governmentality through certain deployments of truth — through power’s insistence that we conform to certain truths and norms. For Foucault, then: “Critique will be the art of voluntary {\em inservitude}, of reflective indocility” (emphasis added; Foucault, 1996: 386). Foucault therefore speaks of an interrogation of the limits of our subjectivity that requires a “patient labour giving form to our impatience for liberty” (Foucault, 2000: 319). Perhaps, then, we can counter the problem of voluntary servitude through a radical {\em discipline of indiscipline.}


\section{Conclusion: A Politics of Refusal
}

Voluntary inservitude — the refusal of power’s domination over ourselves — should not be confused with a refusal of politics. Rather it should be seen as the construction of an alternative form of politics, and as intensification of political action; we might call it a politics of withdrawal from power, a politics of non-domination. There is nothing apolitical about such a politics of refusal: the politics of refusal is not a refusal of politics as such, but rather a refusal of the established forms and practices of politics enshrined in the state, and the desire to create new forms of politics outside the state — the desire, in other words, for a politics of autonomy. Indeed, the notion of the ‘autonomy of the political’, invoked by Carl Schmitt to affirm the sovereignty of the state — the prerogative of the state to define the friend/enemy opposition (Schmitt, 1996) — should be seen, on my alternative reading, as suggesting a politics of autonomy. The proper moment of the political is outside the state and seeks to engender new non-authoritarian relationships and ways of life.


A number of contemporary continental thinkers such as Giorgio Agamben, and Michael Hardt and Antonio Negri, have proposed a similar notion of refusal or withdrawal as a way of thinking about radical politics today. Indeed, the recent interest in the figure of Bartleby (from Melville’s {\em Bartleby the Scrivener}) as a paradigm of resistance to power, points to a certain realization of the limits of existing models of radical and revolutionary politics, and an acknowledgement, moreover, of the need to overcome voluntary subjection to power. Bartleby’s impassive gesture of defiance towards authority — “I would prefer not to” — might be seen as an active withdrawal from participation in the practices and activities which reaffirm power, and without which power would collapse. In the words of Hardt and Negri, “These simple men [Bartleby and Michael K, a character from a J.M Coetzee novel] and their absolute refusals cannot but appeal to our hatred of authority. The refusal of work and authority, the refusal of voluntary servitude, is the beginning of liberatory politics” (Hardt \& Negri, 2000: 204).


In this paper I have placed the problem of voluntary servitude — diagnosed long ago by La Boëtie — at the centre of radical political thought. Voluntary servitude, whose contours have been sharpened by psychoanalytic theory, might be understood as a threshold through which the subject is bound to power at the level of his or her desire. At the same time, the idea of voluntary servitude also points to the very fragility and undecidability of domination, and the way that, through the invention of micro-political and ethical strategies of de-subjectification — an anarchic politics of voluntary inservitude — one may loosen and untie this bond and create alternative spaces of politics beyond the shadow of the sovereign.


\section{References.
}

Adorno, Theodore. (1964) {\em The Authoritarian Personality.} New York: Wiley.


Bakunin, Mikhail. (1984) {\em Marxism, Freedom and the State} (K. J. Kenafick, Trans.). London: Freedom Press.


— . (1953) {\em Political Philosophy: Scientific Anarchism} (G.P Maximoff, Ed.). London: The Free Press.


Borch-Jacobsen, Mikkel. (1988) {\em The Freudian Subject} (Catherine Porter, Trans.). Stanford: Stanford University Press.


Buber, Martin. (1996) {\em Paths in Utopia.} New York: Syracuse University Press.


Butler, Judith. (1997) {\em The Psychic Life of Power: Theories in Subjection.} Stanford, Ca.: Stanford University Press.


Clastres, Pierre. (1994) {\em Archaeology of Violence,} Chapter 7: ‘Freedom, Misfortune, the Unnameable’ (Jeanine Herman, Trans.). New York: Semiotext(e).


De La Boëtie, Etienne. (1988) {\em La Servitude Volontaire, or the Anti-Dictator} [{\em Slaves by Choice}]. Egham: Runnymede Books.


Deleuze, Gilles., \& Felix Guattari. (2004) {\em A Thousand Plateaus: Capitalism and Schizophrenia} (Brian Massumi, Trans.). University of Minnesota Press.


Feuerbach, Ludwig. (1957) {\em The Essence of Christianity} (George Eliot, Trans.). New York. London: Harper \& Row.


Foucault, Michel. (2000) ‘What is Enlightenment?’, in {\em Essential Works of Michel Foucault 1954–1984: Volume 1, Ethics} (Paul Rabinow, Ed., Robert Hurley, Trans.). London: Penguin Books.


— . (1996) ‘What is Critique?’, in {\em What is Enlightenment: Eighteenth Century Answers and Twentieth Century Questions} (James Schmidt, Ed.). Berkeley: University of California Press.


— . (1988) {\em T}{\em he History of Sexuality, Volume 3: The Care of the Self.} New York: Vintage.


Freud, Sigmund (1955) {\em Group Psychology and the Analysis of the Ego}. {\em The Standard Edition of the Complete Psychological Works of Sigmund Freud, Volume XVIII (1920–1922): Beyond the Pleasure Principle, Group Psychology and Other works.} Psychoanalytic Electronic Publishing.


Glynos, Jason. (2008) ‘Self-Transgressive Enjoyment as a Freedom Fetter’, {\em Political Studies,} Vol. 56/3: 679–704.


Hardt, Michael., \& Antonio Negri.. (2000) {\em Empire,} Cambridge MA,: Harvard University Press.


Kropotkin, Peter. (2007) {\em Mutual Aid, A Factor of Evolution}, Dodo Press.


— . (1943) {\em The State: Its Historic Role,} London: Freedom Press.


Lacan, Jacques (2007) ‘Analyticon’, in {\em T}{\em he Seminar of Jacques Lacan, Book XVII: The Other Side of Psychoanalysis} (Jacques-Alain Miller, Ed., Russell Grigg, Trans.). New York and London: W.W. Norton \& Co.


— . (1962) ‘Kant avec Sade’. {\em Critique} \#191, September: 291–313.


Newman, Saul. (2010) {\em The Politics of Postanarchism.} Edinburgh: Edinburgh University Press.


— . (2001) {\em From Bakunin to Lacan: Anti-authoritarianism and the Dislocation of Power.} MA: Lexington Books.


Schmitt, Carl. (1996) {\em The Concept of the Political} (George Schwab, Trans.). Chicago: University of Chicago Press.


Stirner, Max. (1995) {\em T}{\em he Ego and Its Own} (David Leopold, Ed.). Cambridge: Cambridge University Press.


Reich, Wilhelm. (1980) {\em The Mass Psychology of Fascism.} New York: Farrar, Straus and Giroux.


Vaneigem, Raoul. (1994) {\em The Revolution of Everyday Life} (Donald Nicholson-Smith, Trans.). London: Rebel Press.


Virno, Paolo. (2008) {\em Multitude: Between Innovation and Negation.} New York: Semiotext(e).


 









\page[yes]

%%%% backcover

\startmode[a4imposed,a4imposedbc,letterimposed,letterimposedbc,a5imposed,%
  a5imposedbc,halfletterimposed,halfletterimposedbc,quickimpose]
\alibraryflushpages
\stopmode

\page[blank]

\startalignment[middle]
{\tfa The Anarchist Library
\blank[small]
Anti-Copyright}
\blank[small]
\currentdate
\stopalignment

\blank[big]
\framed[frame=off,location=middle,width=\textwidth]
       {\externalfigure[logo][width=0.25\textwidth]}



\vfill
\setupindenting[no]
\setsmallbodyfont

\startalignment[middle,nothyphenated,nothanging,stretch]

\blank[line]
% \framed[frame=off,location=middle,width=\textwidth]
%       {\externalfigure[logo][width=0.25\textwidth]}


Saul Newman



Voluntary Servitude Reconsidered: Radical Politics and the Problem of Self-Domination






2010


\stopalignment
\blank[line]

\startalignment[hyphenated,middle]


From {\em Anarchist Developments in Cultural Studies}, Volume 2010.1




\stopalignment

\stoptext


