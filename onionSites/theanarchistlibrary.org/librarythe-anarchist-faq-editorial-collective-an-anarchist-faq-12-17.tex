% -*- mode: tex -*-
%%%%%%%%%%%%%%%%%%%%%%%%%%%%%%%%%%%%%%%%%%%%%%%%%%%%%%%%%%%%%%%%%%%%%%%%%%%%%%%%
%                                STANDARD                                      %
%%%%%%%%%%%%%%%%%%%%%%%%%%%%%%%%%%%%%%%%%%%%%%%%%%%%%%%%%%%%%%%%%%%%%%%%%%%%%%%%
\definefontfeature[default][default]
                  [protrusion=quality,
                    expansion=quality,
                    script=latn]
\setupalign[hz,hanging]
\setuptolerance[tolerant]
\setbreakpoints[compound]
\setupindenting[yes,1em]
\setupfootnotes[way=bychapter,align={hz,hanging}]
\setupbodyfont[modern] % this is a stinky workaround to load lmodern
\setupbodyfont[libertine,11pt]

\setuppagenumbering[alternative=singlesided,location={footer,middle}]
\setupcaptions[width=fit,align={hz,hanging},number=no]

\startmode[a4imposed,a4imposedbc,letterimposed,letterimposedbc,a5imposed,%
  a5imposedbc,halfletterimposed,halfletterimposedbc]
  \setuppagenumbering[alternative=doublesided]
\stopmode

\setupbodyfontenvironment[default][em=italic]


\setupheads[%
  sectionnumber=no,number=no,
  align=flushleft,
  align={flushleft,nothyphenated,verytolerant,stretch},
  indentnext=yes,
  tolerance=verytolerant]

\definehead[awikipart][chapter]

\setuphead[awikipart]
          [%
            number=no,
            footer=empty,
            style=\bfd,
            before={\blank[force,2*big]},
            align={middle,nothyphenated,verytolerant,stretch},
            after={\page[yes]}
          ]

% h3
\setuphead[chapter]
          [style=\bfc]

\setuphead[title]
          [style=\bfc]


% h4
\setuphead[section]
          [style=\bfb]

% h5
\setuphead[subsection]
          [style=\bfa]

% h6
\setuphead[subsubsection]
          [style=bold]


\setuplist[awikipart]
          [alternative=b,
            interaction=all,
            width=0mm,
            distance=0mm,
            before={\blank[medium]},
            after={\blank[small]},
            style=\bfa,
            criterium=all]
\setuplist[chapter]
          [alternative=c,
            interaction=all,
            width=1mm,
            before={\blank[small]},
            style=bold,
            criterium=all]
\setuplist[section]
          [alternative=c,
            interaction=all,
            width=1mm,
            style=\tf,
            criterium=all]
\setuplist[subsection]
          [alternative=c,
            interaction=all,
            width=8mm,
            distance=0mm,
            style=\tf,
            criterium=all]
\setuplist[subsubsection]
          [alternative=c,
            interaction=all,
            width=15mm,
            style=\tf,
            criterium=all]


% center

\definestartstop
  [awikicenter]
  [before={\blank[line]\startalignment[middle]},
   after={\stopalignment\blank[line]}]

% right

\definestartstop
  [awikiright]
  [before={\blank[line]\startalignment[flushright]},
   after={\stopalignment\blank[line]}]


% blockquote

\definestartstop
  [blockquote]
  [before={\blank[big]
    \setupnarrower[middle=1em]
    \startnarrower
    \setupindenting[no]
    \setupwhitespace[medium]},
  after={\stopnarrower
    \blank[big]}]

% verse

\definestartstop
  [awikiverse]
  [before={\blank[big]
      \setupnarrower[middle=2em]
      \startnarrower
      \startlines},
    after={\stoplines
      \stopnarrower
      \blank[big]}]

\definestartstop
  [awikibiblio]
  [before={%
      \blank[big]
      \setupnarrower[left=1em]
      \startnarrower[left]
        \setupindenting[yes,-1em,first]},
    after={\stopnarrower
      \blank[big]}]
                
% same as above, but with no spacing around
\definestartstop
  [awikiplay]
  [before={%
      \setupnarrower[left=1em]
      \startnarrower[left]
        \setupindenting[yes,-1em,first]},
    after={\stopnarrower}]



% interaction
% we start the interaction only if it's not an imposed format.
\startnotmode[a4imposed,a4imposedbc,letterimposed,letterimposedbc,a5imposed,%
  a5imposedbc,halfletterimposed,halfletterimposedbc]
  \setupinteraction[state=start,color=black,contrastcolor=black,style=bold]
  \placebookmarks[awikipart,chapter,section,subsection,subsubsection][force=yes]
  \setupinteractionscreen[option=bookmark]
\stopnotmode



\setupexternalfigures[%
  maxwidth=\textwidth,
  maxheight=\textheight,
  factor=fit]

\setupitemgroup[itemize][each][packed][indenting=no]

\definemakeup[titlepage][pagestate=start,doublesided=no]

%%%%%%%%%%%%%%%%%%%%%%%%%%%%%%%%%%%%%%%%%%%%%%%%%%%%%%%%%%%%%%%%%%%%%%%%%%%%%%%%
%                                IMPOSER                                       %
%%%%%%%%%%%%%%%%%%%%%%%%%%%%%%%%%%%%%%%%%%%%%%%%%%%%%%%%%%%%%%%%%%%%%%%%%%%%%%%%

\startusercode

function optimize_signature(pages,min,max)
   local minsignature = min or 40
   local maxsignature = max or 80
   local originalpages = pages

   -- here we want to be sure that the max and min are actual *4
   if (minsignature%4) ~= 0 then
      global.texio.write_nl('term and log', "The minsig you provided is not a multiple of 4, rounding up")
      minsignature = minsignature + (4 - (minsignature % 4))
   end
   if (maxsignature%4) ~= 0 then
      global.texio.write_nl('term and log', "The maxsig you provided is not a multiple of 4, rounding up")
      maxsignature = maxsignature + (4 - (maxsignature % 4))
   end
   global.assert((minsignature % 4) == 0, "I suppose something is wrong, not a n*4")
   global.assert((maxsignature % 4) == 0, "I suppose something is wrong, not a n*4")

   --set needed pages to and and signature to 0
   local neededpages, signature = 0,0

   -- this means that we have to work with n*4, if not, add them to
   -- needed pages 
   local modulo = pages % 4
   if modulo==0 then
      signature=pages
   else
      neededpages = 4 - modulo
   end

   -- add the needed pages to pages
   pages = pages + neededpages
   
   if ((minsignature == 0) or (maxsignature == 0)) then 
      signature = pages -- the whole text
   else
      -- give a try with the signature
      signature = find_signature(pages, maxsignature)
      
      -- if the pages, are more than the max signature, find the right one
      if pages>maxsignature then
	 while signature<minsignature do
	    pages = pages + 4
	    neededpages = 4 + neededpages
	    signature = find_signature(pages, maxsignature)
	    --         global.texio.write_nl('term and log', "Trying signature of " .. signature)
	 end
      end
      global.texio.write_nl('term and log', "Parameters:: maxsignature=" .. maxsignature ..
		   " minsignature=" .. minsignature)

   end
   global.texio.write_nl('term and log', "ImposerMessage:: Original pages: " .. originalpages .. "; " .. 
	 "Signature is " .. signature .. ", " ..
	 neededpages .. " pages are needed, " .. 
	 pages ..  " of output")
   -- let's do it
   tex.print("\\dorecurse{" .. neededpages .. "}{\\page[empty]}")

end

function find_signature(number, maxsignature)
   global.assert(number>3, "I can't find the signature for" .. number .. "pages")
   global.assert((number % 4) == 0, "I suppose something is wrong, not a n*4")
   local i = maxsignature
   while i>0 do
      -- global.texio.write_nl('term and log', "Trying " .. i  .. "for max of " .. maxsignature)
      if (number % i) == 0 then
	 return i
      end
      i = i - 4
   end
end

\stopusercode

\define[1]\fillthesignature{
  \usercode{optimize_signature(#1, 40, 80)}}


\define\alibraryflushpages{
  \page[yes] % reset the page
  \fillthesignature{\the\realpageno}
}


% various papers 
\definepapersize[halfletter][width=5.5in,height=8.5in]
\definepapersize[halfafour][width=148.5mm,height=210mm]
\definepapersize[quarterletter][width=4.25in,height=5.5in]
\definepapersize[halfafive][width=105mm,height=148mm]
\definepapersize[generic][width=210mm,height=279.4mm]

%% this is the default ``paper'' which should work with both letter and a4

\setuppapersize[generic][generic]
\setuplayout[%
  backspace=42mm,
  topspace=31mm,% 176 / 15
  height=195mm,%130mm,
  footer=9mm, %
  header=0pt, % no header
  width=126mm] % 10.5 x 11

\startmode[libertine]
  \usetypescript[libertine]
  \setupbodyfont[libertine,11pt]
\stopmode

\startmode[pagella]
  \setupbodyfont[pagella,11pt]
\stopmode

\startmode[antykwa]
  \setupbodyfont[antykwa-poltawskiego,11pt]
\stopmode

\startmode[iwona]
  \setupbodyfont[iwona-medium,11pt]
\stopmode

\startmode[helvetica]
  \setupbodyfont[heros,11pt]
\stopmode

\startmode[century]
  \setupbodyfont[schola,11pt]
\stopmode

\startmode[modern]
  \setupbodyfont[modern,11pt]
\stopmode

\startmode[charis]
  \setupbodyfont[charis,11pt]
\stopmode        

\startmode[mini]
  \setuppapersize[S33][S33] % 176 × 176 mm
  \setuplayout[%
    backspace=20pt,
    topspace=15pt,% 176 / 15
    height=280pt,%130mm,
    footer=20pt, %
    header=0pt, % no header
    width=260pt] % 10.5 x 11
\stopmode

% for the plain A4 and letter, we use the classic LaTeX dimensions
% from the article class
\startmode[a4]
  \setuppapersize[A4][A4]
  \setuplayout[%
    backspace=42mm,
    topspace=45mm,
    height=218mm,
    footer=10mm,
    header=0pt, % no header
    width=126mm]
\stopmode

\startmode[letter]
  \setuppapersize[letter][letter]
  \setuplayout[%
    backspace=44mm,
    topspace=46mm,
    height=199mm,
    footer=10mm,
    header=0pt, % no header
    width=126mm]
\stopmode


% A4 imposed (A5), with no bc

\startmode[a4imposed]
% DIV=15 148 × 210: these are meant not to have binding correction,
  % but just to play safe, let's say 1mm => 147x210
  \setuppapersize[halfafour][halfafour]
  \setuplayout[%
    backspace=10.8mm, % 146/15 = 9.8 + 1
    topspace=14mm, % 210/15 =  14
    height=182mm, % 14 x 12 + 14 of the footer
    footer=14mm, %
    header=0pt, % no header
    width=117.6mm] % 9.8 x 12
\stopmode

% A4 imposed (A5), with bc
\startmode[a4imposedbc]
  \setuppapersize[halfafour][halfafour]
  \setuplayout[% 14 mm was a bit too near to the spine, using the glue binding
    backspace=17.3mm,  % 140/15 + 8 =
    topspace=14mm, % 210/15 =  14
    height=182mm, % 14 x 12 + 14 of the footer
    footer=14mm, %
    header=0pt, % no header
    width=112mm] % 9.333 x 12
\stopmode


\startmode[letterimposedbc] % 139.7mm x 215.9 mm
  \setuppapersize[halfletter][halfletter]
  % DIV=15 8mm binding corr, => 132 x 216
  \setuplayout[%
    backspace=16.8mm, % 8.8 + 8
    topspace=14.4mm, % 216/15 =  14.4
    height=187.2mm, % 15.4 x 11 + 15 of the footer
    footer=14.4mm, %
    header=0pt, % no header
    width=105.6mm] % 8.8 x 12
\stopmode

\startmode[letterimposed] % 139.7mm x 215.9 mm
  \setuppapersize[halfletter][halfletter]
  % DIV=15, 1mm binding correction. => 138.7x215.9
  \setuplayout[%
    backspace=10.3mm, % 9.24 + 1
    topspace=14.4mm, % 216/15 =  14.4
    height=187.2mm, % 15.4 x 11 + 15 of the footer
    footer=14.4mm, %
    header=0pt, % no header
    width=111mm] % 9.24 x 12
\stopmode

%%% new formats for mini books
%%% \definepapersize[halfafive][width=105mm,height=148mm]

\startmode[a5imposed]
% DIV=12 105x148 : these are meant not to have binding correction,
  % but just to play safe, let's say 1mm => 104x148
  \setuppapersize[halfafive][halfafive]
  \setuplayout[%
    backspace=9.6mm,
    topspace=12.3mm,
    height=123.5mm, % 14 x 12 + 14 of the footer
    footer=12.3mm, %
    header=0pt, % no header
    width=78.8mm] % 9.8 x 12
\stopmode

% A5 imposed (A6), with bc
\startmode[a5imposedbc]
% DIV=12 105x148 : with binding correction,
  % let's say 8mm => 96x148
  \setuppapersize[halfafive][halfafive]
  \setuplayout[%
    backspace=16mm,
    topspace=12.3mm,
    height=123.5mm, % 14 x 12 + 14 of the footer
    footer=12.3mm, %
    header=0pt, % no header
    width=72mm] % 9.8 x 12
\stopmode

%%% \definepapersize[quarterletter][width=4.25in,height=5.5in]

% DIV=12 width=4.25in (108mm),height=5.5in (140mm) 
\startmode[halfletterimposed] % 107x140
  \setuppapersize[quarterletter][quarterletter]
  \setuplayout[%
    backspace=10mm,
    topspace=11.6mm,
    height=116mm,
    footer=11.6mm,
    header=0pt, % no header
    width=80mm] % 9.24 x 12
\stopmode

\startmode[halfletterimposedbc]
  \setuppapersize[quarterletter][quarterletter]
  \setuplayout[%
    backspace=15.4mm,
    topspace=11.6mm,
    height=116mm,
    footer=11.6mm,
    header=0pt, % no header
    width=76mm] % 9.24 x 12
\stopmode

\startmode[quickimpose]
  \setuppapersize[A5][A4,landscape]
  \setuparranging[2UP]
  \setuppagenumbering[alternative=doublesided]
  \setuplayout[% 14 mm was a bit too near to the spine, using the glue binding
    backspace=17.3mm,  % 140/15 + 8 =
    topspace=14mm, % 210/15 =  14
    height=182mm, % 14 x 12 + 14 of the footer
    footer=14mm, %
    header=0pt, % no header
    width=112mm] % 9.333 x 12
\stopmode

\startmode[tenpt]
  \setupbodyfont[10pt]
\stopmode

\startmode[twelvept]
  \setupbodyfont[12pt]
\stopmode

%%%%%%%%%%%%%%%%%%%%%%%%%%%%%%%%%%%%%%%%%%%%%%%%%%%%%%%%%%%%%%%%%%%%%%%%%%%%%%%%
%                            DOCUMENT BEGINS                                   %
%%%%%%%%%%%%%%%%%%%%%%%%%%%%%%%%%%%%%%%%%%%%%%%%%%%%%%%%%%%%%%%%%%%%%%%%%%%%%%%%


\mainlanguage[en]


\starttext

\starttitlepagemakeup
  \startalignment[middle,nothanging,nothyphenated,stretch]


  \switchtobodyfont[18pt] % author
  {\bf \em

The Anarchist FAQ Editorial Collective  \par}
  \blank[2*big]
  \switchtobodyfont[24pt] % title
  {\bf

An Anarchist FAQ (12/17)

\par}
  \blank[big]
  \switchtobodyfont[20pt] % subtitle
  {\bf 

\par}
  \vfill
  \stopalignment
  \startalignment[middle,bottom,nothyphenated,stretch,nothanging]
  \switchtobodyfont[global]

June 18, 2009. Version 13.1

  \stopalignment
\stoptitlepagemakeup



\title{Contents}

\placelist[awikipart,chapter,section,subsection]



\page[yes,right]

\awikipart{Appendix: Anarchism and “anarcho”-capitalism 
}

This appendix exists for one reason, namely to explain why the idea of “anarcho”-capitalism is a bogus one. While we have covered this topic in section F, we thought that this appendix should be created in order to discuss in more detail why anarchists reject both “anarcho”-capitalism and its claims to being anarchist.


This appendix has three parts. The first two sections are our critique of Bryan Caplan’s “anarcho”-capitalist “Anarchist Theory FAQ.” Caplan’s FAQ is the main on-line attempt to give the oxymoron of “anarcho”-capitalism some form of justification and so it is worthwhile explaining, using his FAQ as the base, why such an attempt fails. The last part of this appendix is the original version of section F,


As we will prove, Caplan’s FAQ fails in its attempt to show that “anarcho” capitalism can be considered as part of the anarchist movement and in fact his account involves extensive re-writing of history. This appendix is in two parts, a reply to Caplan’s most recent FAQ release (version 5.2) and an older reply to version 4.1.1 (which was originally section F.10 of the FAQ). The introduction to the reply to version 4.1.1 indicates what most anarchists think of Caplan’s FAQ and its claims of “objectivity” as so we will not repeat ourselves here.


We decided to replace the original version of section F with an edited version simply because the original section was too long in respect to the rest of the FAQ. While this FAQ may have started out as a rebuttal to “anarcho”-capitalist claims of being anarchist, it no longer is. As such, in an {\bf {\em anarchist}} FAQ section F became redundant as “anarcho”-capitalism is a fringe ideology even within the USA. If it were not for their presence on the web and some academics taking their claims to being anarchists at face value, we would only mention them in passing.


We have decided to include this appendix as it is really an addition to the main body of the FAQ. Parties interested in why “anarcho”-capitalist claims are false can explore this appendix, those who are interested in anarchist politics can read the FAQ without having to also read too many arguments between anarchists and capitalists. We should, perhaps, thank Caplan for allowing us an opportunity of explaining the ideas of such people as Proudhon and Tucker, allowing us to quote them and so bring their ideas to a wider audience and for indicating that anarchism, in all its forms, has always opposed capitalism and always will.


\chapter{Replies to Some Errors and Distortions in Bryan Caplan’s “Anarchist Theory FAQ” version 5.2
}

\section{1. Individualist Anarchists and the socialist movement.
}

Caplan, in his FAQ, attempts to rewrite anarchist history by trying to claim that the individualist anarchists were forerunners of the so-called “anarcho-capitalist” school. However, as is so often the case with Caplan’s FAQ, nothing could be further from the truth.


In section 5 (What major subdivisions may be made among anarchists?) of his FAQ, Caplan writes that:



\startblockquote
{\em “A large segment of left-anarchists is extremely sceptical about the anarchist credentials of anarcho-capitalists, arguing that the anarchist movement has historically been clearly leftist. In my own view, it is necessary to re-write a great deal of history to maintain this claim.”}



\stopblockquote
He quotes Carl Landauer’s {\bf European Socialism: A History of Ideas and Movements} as evidence:



\startblockquote
{\em “To be sure, there is a difference between individualistic anarchism and collectivistic or communistic anarchism; Bakunin called himself a communist anarchist. But the communist anarchists also do not acknowledge any right to society to force the individual. They differ from the anarchistic individualists in their belief that men, if freed from coercion, will enter into voluntary associations of a communistic type, while the other wing believes that the free person will prefer a high degree of isolation. The communist anarchists repudiate the right of private property which is maintained through the power of the state. The individualist anarchists are inclined to maintain private property as a necessary condition of individual independence, without fully answering the question of how property could be maintained without courts and police.”}



\stopblockquote
Caplan goes on to state that {\em “the interesting point is that before the emergence of modern anarcho-capitalism Landauer found it necessary to distinguish two strands of anarchism, only one of which he considered to be within the broad socialist tradition.”}


However, what Caplan seems to ignore is that both individualist and social anarchists agree that there {\bf is} a difference between the two schools of anarchist thought! Some insight. Of course, Caplan tries to suggest that Landauer’s non-discussion of the individualist anarchists is somehow “evidence” that their ideas are not socialistic. Firstly, Landauer’s book is about {\bf European} Socialism. Individualist anarchism was almost exclusively based in America and so hardly falls within the book’s subject area. Secondly, from the index Kropotkin is mentioned on {\bf two} pages (one of which a footnote). Does that mean Kropotkin was not a socialist? Of course not. It seems likely, therefore, that Landauer is using the common Marxist terminology of defining Marxism as Socialism, while calling other parts of the wider socialist movement by their self-proclaimed names of anarchism, syndicalism and so on. Hardly surprising that Kropotkin is hardly mentioned in a history of “Socialism” (i.e. Marxism).


As noted above, both schools of anarchism knew there was a difference between their ideas. Kropotkin and Tucker, for example, both distinguished between two types of anarchism as well as two types of socialism. Thus Caplan’s {\em “interesting point”} is just a banality, a common fact which anyone with a basic familiarity of anarchist history would know. Kropotkin in his justly famous essay on Anarchism for {\bf The Encyclopaedia Britannica} {\bf also} found it necessary to distinguish two strands of anarchism. As regards Caplan’s claims that only one of these strands of anarchism is {\em “within the broad socialist tradition”} all we can say is that both Kropotkin {\bf and} Tucker considered their ideas and movement to be part of the broader socialist tradition. According to an expert on Individualist Anarchism, Tucker {\em “looked upon anarchism as a branch of the general socialist movement”} [James J. Martin, {\bf Men Against the State}, pp. 226–7]. Other writers on Individualist Anarchism have noted the same fact (for example, Tucker {\em “definitely thought of himself a socialist”} [William O. Reichart, {\bf Partisans of Freedom: A Study in American Anarchism}, p. 156]). As evidence of the anti-socialist nature of individualist anarchism, Caplan’s interpretation of Landauer’s words is fundamentally nonsense. If you look at the writings of people like Tucker you will see that they called themselves socialists and considered themselves part of the wider socialist movement. No one familiar with Tucker’s works could overlook this fact.


Interestingly, Landauer includes Proudhon in his history and states that he was {\em “the most profound thinker among pre-Marxian socialists.”} [p. 67] Given that Caplan elsewhere in his FAQ tries to co-opt Proudhon into the “anarcho”-capitalist school as well as Tucker, his citing of Landauer seems particularly dishonest. Landauer presents Proudhon’s ideas in some depth in his work within a chapter headed {\bf {\em “The three Anticapitalistic Movements.”}} Indeed, he starts his discussion of Proudhon’s ideas with the words {\em “In France, post-Utopian socialism begins with Peter Joseph Proudhon.”} [p. 59] Given that both Kropotkin and Tucker indicated that Individualist Anarchism followed Proudhon’s economic and political ideas the fact that Landauer states that Proudhon was a socialist implies that Individualist Anarchism is also socialist (or “Leftist” to use Caplan’s term).


Tucker and the other individualist anarchists considered themselves as followers of Proudhon’s ideas (as did Bakunin and Kropotkin). For example, Tucker stated that his journal {\bf Liberty} was {\em “brought into existence as a direct consequence of the teachings of Proudhon”} and {\em “lives principally to spread them.”} [cited by Paul Avrich in his {\em “Introduction”} to {\bf Proudhon and his {\em “Bank of the People”}} by Charles A. Dana]


Obviously Landauer considered Proudhon a socialist and if Individualist Anarchism follows Proudhon’s ideas then it, too, must be socialist.


Unsurprisingly, then, Tucker also considered himself a socialist. To state the obvious, Tucker and Bakunin both shared Proudhon’s opposition to {\bf private} property (in the capitalist sense of the word), although Tucker confused this opposition (and possibly the casual reader) by talking about possession as “property.”


So, it appears that Caplan is the one trying to rewrite history.


\section{2. Why is Caplan’s definition of socialism wrong?
}

Perhaps the problem lies with Caplan’s “definition” of socialism. In section 7 (Is anarchism the same thing as socialism?) he states:



\startblockquote
{\em “If we accept one traditional definition of socialism — ‘advocacy of government ownership of the means of production’ — it seems that anarchists are not socialists by definition. But if by socialism we mean something more inclusive, such as ‘advocacy of the strong restriction or abolition of private property,’ then the question becomes more complex.”}



\stopblockquote
Which are hardly traditional definitions of socialism unless you are ignorant of socialist ideas! By definition one, Bakunin and Kropotkin are not socialists. As far as definition two goes, all anarchists were opposed to (capitalist) private property and argued for its abolition and its replacement with possession. The actual forms of possession differed from between anarchist schools of thought, but the common aim to end private property (capitalism) was still there. To quote Dana, in a pamphlet called {\em “a really intelligent, forceful, and sympathetic account of mutual banking”} by Tucker, individualist anarchists desire to {\em “destroy the tyranny of capital,- that is, of property”} by mutual credit. [Charles A. Dana, {\bf Proudhon and his {\em “Bank of the People”}}, p. 46]


Interestingly, this second definition of socialism brings to light a contradiction in Caplan’s account. Elsewhere in the FAQ he notes that Proudhon had {\em “ideas on the desirability of a modified form of private property.”} In fact, Proudhon did desire to restrict private property to that of possession, as Caplan himself seems aware. In other words, even taking his own definitions we find that Proudhon would be considered a socialist! Indeed, according to Proudhon, {\em “all accumulated capital is collective property, no one may be its exclusive owner.”} [{\bf Selected Writings of Pierre-Joseph Proudhon}, p. 44] Thus Jeremy Jennings’ summary of the anarchist position on private property:



\startblockquote
{\em “The point to stress is that all anarchists [including Spooner and Tucker], and not only those wedded to the predominant twentieth-century strain of anarchist communism have been critical of private property to the extent that it was a source of hierarchy and privilege.”}



\stopblockquote
He goes on to state that anarchists like Tucker and Spooner {\em “agreed with the proposition that property was legitimate only insofar as it embraced no more than the total product of individual labour.”} [{\em “Anarchism”}, {\bf Contemporary Political Ideologies}, Roger Eatwell and Anthony Wright (eds.), p. 132]


The idea that socialism can be defined as state ownership or even opposition to, or “abolition” of, all forms of property is not one which is historically accurate for all forms of socialism. Obviously communist-anarchists and syndicalists would dismiss out of hand the identification of socialism as state ownership, as would Individualist Anarchists like Tucker and Joseph Labadie. As for opposition or abolition of all forms of “private property” as defining socialism, such a position would have surprised communist-anarchists like Kropotkin (and, obviously, such self-proclaimed socialists as Tucker and Labadie).


For example, in {\bf Act for Yourselves} Kropotkin explicitly states that a peasant {\em “who is in possession of just the amount of land he can cultivate”} would not be expropriated in an anarchist revolution. Similarly for the family {\em “inhabiting a house which affords them just enough space \unknown{} considered necessary for that number of people”} and the artisan {\em “working with their own tools or handloom”} would be left alone [pp. 104–5]. He makes the same point in {\bf The Conquest of Bread} [p. 61] Thus, like Proudhon, Kropotkin replaces {\bf private property} with {\bf possession} as the former is {\em “theft”} (i.e. it allows exploitation, which {\em “indicate[s] the scope of Expropriation”} namely {\em “to everything that enables any man [or woman]\unknown{} to appropriate the product of other’s toil”} [{\bf The Conquest of Bread}, p. 61])


Even Marx and Engels did not define socialism in terms of the abolition of all forms of “private property.” Like anarchists, they distinguished between that property which allows exploitation to occur and that which did not. Looking at the {\bf Communist Manifesto} we find them arguing that the {\em “distinguishing feature of Communism is not the abolition of property generally, but the abolition of bourgeois property”} and that {\em “Communism deprives no man of the power to appropriate the products of society; all that it does is to deprive him of the power to subjugate the labour of others by means of such appropriation.”} Moreover, they correctly note that “property” has meant different things at different times and that the {\em “abolition of existing property relations is not at all a distinctive feature of Communism”} as {\em “[a]ll property relations in the past have continually been subject to historical change consequent upon the change in historical conditions.”} As an example, they argue that the French Revolution {\em “abolished feudal property in favour of bourgeois property.”} [{\bf The Manifesto of the Communist Party}, p.47, p. 49 and p. 47]


Which means that the idea that socialism means abolishing “private property” is {\bf only} true for those kinds of property that are used to exploit the labour of others. Nicholas Walter sums up the anarchist position when he wrote that anarchists {\em “are in favour of the private property which cannot be used by one person to exploit another.”} [{\bf Reinventing Anarchy}, p. 49] In other words, property which is no longer truly {\bf private} as it is used by those who do not own it. In effect, the key point of Proudhon’s {\bf What is Property?}, namely the difference between possession and property. Which means that rather than desire the abolition of all forms of “private property,” socialists (of all kinds, libertarian and authoritarian) desire the abolition of a specific kind of property, namely that kind which allows the exploitation and domination of others. To ignore this distinction is to paint a very misleading picture of what socialism stands for.


This leaves the {\em “the strong restriction \unknown{} of private property”} definition of socialism. Here Caplan is on stronger ground. Unfortunately, by using that definition the Individualist Anarchists, like the Social Anarchists, are included in socialist camp, a conclusion he is trying to avoid. As {\bf every} anarchist shares Proudhon’s analysis that {\em “property is theft”} and that {\bf possession} would be the basis of anarchism, it means that every anarchist is a socialist (as Labadie always claimed). This includes Tucker and the other Individualist Anarchists. For example, Joseph Labadie stated that {\em “the two great sub-divisions of Socialists”} (anarchists and State Socialists) both {\em “agree that the resources of nature — land, mines, and so forth — should not be held as private property and subject to being held by the individual for speculative purposes, that use of these things shall be the only valid title, and that each person has an equal right to the use of all these things. They all agree that the present social system is one composed of a class of slaves and a class of masters, and that justice is impossible under such conditions.”} [{\bf What is Socialism?}] Tucker himself argued that the anarchists’ {\em “occupancy and use”} title to land and other scare material would involve a change (and, in effect, {\em “restriction”}) of current (i.e. capitalist) property rights:



\startblockquote
{\em “It will be seen from this definition that Anarchistic property concerns only products. But anything is a product upon which human labour has been expended. It should be stated, however, that in the case of land, or of any other material the supply of which is so limited that all cannot hold it in unlimited quantities, Anarchism undertakes to protect no titles except such as are based on actual occupancy and use.”} [{\bf Instead of a Book}, p. 61]



\stopblockquote
and so:



\startblockquote
{\em “no advocate of occupancy and use believes that it can be put in force until as a theory it has been accepted as generally \unknown{} seen and accepted as is the prevailing theory of ordinary private property.”} [{\bf Occupancy and Use versus the Single Tax}]



\stopblockquote
So, as can be seen, Individualist Anarchism rejected important aspects of capitalist property rights. Given that the Individualist Anarchists were writing at a time when agriculture was still the largest source of employment this position on land is much more significant than it first appears. In effect, Tucker and the other American Anarchists were advocating a {\bf massive} and {\bf fundamental} change in property-rights, in the social relationships they generated and in American society. This is, in other words, a very {\em “strong restriction”} in capitalist property rights (and it is {\bf this} type of property Caplan is referring to, rather than “property” in the abstract).


However, such a “definition” of socialism as “restricting” private property is flawed as it does not really reflect anarchist ideas on the subject. Anarchists, in effect, reject the simplistic analysis that because a society (or thinker) accepts “property” that it (or he/she) is capitalistic. This is for two reasons. Firstly, the term “property” has been used to describe a wide range of situations and institutions. Thus Tucker used the term “property” to describe a society in which capitalist property rights were {\bf not} enforced. Secondly, and far more importantly, concentrating on “property” rights in the abstract ignores the social relationships it generates. Freedom is product of social interaction, not one of isolation. This means that the social relationships generated in a given society are the key to evaluating it — not whether it has “property” or not. To look at “property” in the abstract is to ignore people and the relationships they create between each other. And it is these relationships which determine whether they are free or not (and so exploited or not). Caplan’s use of the anti-property rights “definition” of socialism avoids the central issue of freedom, of whether a given society generates oppression and exploitation or not. By looking at “property” Caplan ignores liberty, a strange but unsurprising position for a self-proclaimed “libertarian” to take.


Thus both of Caplan’s “definitions” of socialism are lacking. A {\em “traditional”} one of government ownership is hardly that and the one based on “property” rights avoids the key issue while, in its own way, includes {\bf all} the anarchists in the socialist camp (something Caplan, we are sure, did not intend).


So what would be a useful definition of socialism? From our discussion on property we can instantly reject Caplan’s biased and simplistic starting points. In fact, a definition of socialism which most socialists would agree with would be one that stated that {\em “the whole produce of labour ought to belong to the labourer”} (to use words Thomas Hodgskin, an early English socialist, from his essay {\bf Labour Defended against the Claims of Capital}). Tucker stated that {\em “the bottom claim of Socialism”} was {\em “that labour should be put in possession of its own,”} that {\em “the natural wage of labour is its product”} (see his essay {\bf State Socialism and Anarchism}). This definition also found favour with Kropotkin who stated that socialism {\em “in its wide, generic, and true sense”} was an {\em “effort to {\bf abolish} the exploitation of labour by capital.”} [{\bf Kropotkin’s Revolutionary Pamphlets}, p. 169]


From this position, socialists soon realised that (to again quote Kropotkin) {\em “the only guarantee not to by robbed of the fruits of your labour is to possess the instruments of labour.”} [{\bf The Conquest of Bread}, p. 145] Because of this socialism also could be defined as {\em “the workers shall own the means of production,”} as this automatically meant that the product would go to the producer, and, in fact, this could also be a definition of socialism most socialists would agree with. The form of this ownership, however, differed from socialist tendency to socialist tendency (some, like Proudhon, proposed co-operative associations, others like Kropotkin communal ownership, others like the Social Democrats state ownership and so on). Moreover, as the economy changed in the 19\high{th} century, so did socialist ideas. Murray Bookchin gives a good summary of this process:



\startblockquote
{\em “Th[e] growing shift from artisanal to an industrial economy gave rise to a gradual but major shift in socialism itself. For the artisan, socialism meant producers’ co-operatives composed of men who worked together in small shared collectivist associations \unknown{} For the industrial proletarian, by contrast, socialism came to mean the formation of a mass organisation that gave factory workers the collective power to expropriate a plant that no single worker could properly own\unknown{} They advocated {\bf public} ownership of the means of production, whether by the state or by the working class organised in trade unions.”} [{\bf The Third Revolution}, vol. 2, p. 262]



\stopblockquote
So, in this evolution of socialism we can place the various brands of anarchism. Individualist anarchism is clearly a form of artisanal socialism (which reflects its American roots) while communist anarchism and anarcho-syndicalism are forms of industrial (or proletarian) socialism (which reflects its roots in Europe). Proudhon’s mutualism bridges these extremes, advocating as it does artisan socialism for small-scale industry and agriculture and co-operative associations for large-scale industry (which reflects the state of the French economy in the 1840s to 1860s). The common feature of all these forms of anarchism is opposition to usury and the notion that {\em “workers shall own the means of production.”} Or, in Proudhon’s words, {\em “abolition of the proletariat.”} [{\bf Op. Cit.}, p. 179] As one expert on Proudhon points out, Proudhon’s support for {\em “association”} (or {\em “associative socialism”}) {\em “anticipated all those later movements”} which demanded {\em “that the economy be controlled neither by private enterprise nor by the state \unknown{} but by the producers”} such as {\em “the revolutionary syndicalists”} and {\em “the students of 1968.”} [K. Steven Vincent, {\bf Pierre-Joseph Proudhon and the Rise of French Republican Socialism}, p. 165] {\em “Industrial Democracy must\unknown{} succeed Industrial Feudalism,”} to again quote Proudhon. [{\bf Op. Cit.}, p. 167]


Thus the common agreement between all socialists was that capitalism was based upon exploitation and wage slavery, that workers did not have access to the means of production and so had to sell themselves to the class that did. Thus we find Individualist Anarchists arguing that the whole produce of labour ought to belong to the labourer and opposing the exploitation of labour by capital. To use Tucker’s own words:



\startblockquote
{\em “the fact that one class of men are dependent for their living upon the sale of their labour, while another class of men are relieved of the necessity of labour by being legally privileged to sell something that is not labour\unknown{} And to such a state of things I am as much opposed as any one. But the minute you remove privilege \unknown{} every man will be a labourer exchanging with fellow-labourers \unknown{} What Anarchistic-Socialism aims to abolish is usury \unknown{} it wants to deprive capital of its reward.”} [{\bf Instead of a Book}, p. 404]



\stopblockquote
By ending wage labour, anarchist socialism would ensure {\em “The land to the cultivator. The mine to the miner. The tool to the labourer. The product to the producer”} and so {\em “everyone [would] be a proprietor”} and so there would be {\em “no more proletaires”} (in the words of Ernest Lesigne, quoted favourably by Tucker as part of what he called a {\em “summary exposition of Socialism from the standpoint of Anarchism”} [{\bf Op. Cit.}, p. 17, p. 16]). Wage labour, and so capitalism, would be no more and {\em “the product [would go] to the producer.”} The Individualist Anarchists, as Wm. Gary Kline correctly points out, {\em “expected a society of largely self-employed workmen with no significant disparity of wealth between any of them.”} [{\bf The Individualist Anarchists}, p. 104] In other words, the {\em “abolition of the proletariat”} as desired by Proudhon.


Therefore, like all socialists, Tucker wanted to end usury, ensure the {\em “product to the producer”} and this meant workers owning and controlling the means of production they used ({\em “no more proletaires”}). He aimed to do this by reforming capitalism away by creating mutual banks and other co-operatives (he notes that Individualist Anarchists followed Proudhon, who {\em “would individualise and associate”} the productive and distributive forces in society [as quoted by James J. Martin, {\bf Men Against the State}, p. 228]). Here is Kropotkin on Proudhon’s reformist mutualist-socialism:



\startblockquote
{\em “When he proclaimed in his first memoir on property that ‘Property is theft’, he meant only property in its present, Roman-law, sense of ‘right of use and abuse’; in property-rights, on the other hand, understood in the limited sense of {\bf possession}, he saw the best protection against the encroachments of the state. At the same time he did not want violently to dispossess the present owners of land, dwelling-houses, mines, factories and so on. He preferred to {\bf attain the same end} by rendering capital incapable of earning interest.”} [{\bf Kropotkin’s Revolutionary Pamphlet’s}, pp. 290–1 — emphasis added]



\stopblockquote
In other words, like all anarchists, Proudhon desired to see a society without capitalists and wage slaves ({\em “the same end”}) but achieved by different means. When Proudhon wrote to Karl Marx in 1846 he made the same point:



\startblockquote
{\em “through Political Economy we must turn the theory of Property against Property in such a way as to create what you German socialists call {\bf community} and which for the moment I will only go so far as calling {\bf liberty} or {\bf equality.}”} [{\bf Selected Writings of Pierre-Joseph Proudhon}, p. 151]



\stopblockquote
In other words, Proudhon shared the common aim of all socialists (namely to abolish capitalism, wage labour and exploitation) but disagreed with the means. As can be seen, Tucker placed himself squarely in this tradition and so could (and did) call himself a socialist. Little wonder Joseph Labadie often said that {\em “All anarchists are socialists, but not all socialists are anarchists.”} That Caplan tries to ignore this aspect of Individualist Anarchism in an attempt to co-opt it into “anarcho”-capitalism indicates well that his FAQ is not an objective or neutral work.


Caplan states that the {\em “United States has been an even more fertile ground for individualist anarchism: during the 19\high{th}-century, such figures as Josiah Warren, Lysander Spooner, and Benjamin Tucker gained prominence for their vision of an anarchism based upon freedom of contract and private property.”}


However, as indicated, Tucker and Spooner did {\bf not} support private property in the capitalist sense of the word and Kropotkin and Bakunin, no less than Tucker and Spooner, supported free agreement between individuals and groups. What does that prove? That Caplan seems more interested in the words Tucker and Proudhon used rather than the meanings {\bf they} attached to them. Hardly convincing.


Perhaps Caplan should consider Proudhon’s words on the subject of socialism:



\startblockquote
{\em “Modern Socialism was not founded as a sect or church; it has seen a number of different schools.”} [{\bf Selected Writings of Pierre-Joseph Proudhon}, p. 177]



\stopblockquote
If he did perhaps he would who see that the Individualist Anarchists were a school of socialism, given their opposition to exploitation and the desire to see its end via their political, economic and social ideas.


\section{3. Was Proudhon a socialist or a capitalist?
}

In section 8 (Who are the major anarchist thinkers?), Caplan tries his best to claim that Proudhon was not really a socialist at all. He states that {\em “Pierre[-Joseph] Proudhon is also often included [as a “left anarchist”] although his ideas on the desirability of a modified form of private property would lead some to exclude him from the leftist camp altogether.”}


“Some” of which group? Other anarchists, like Bakunin and Kropotkin? Obviously not — Bakunin claimed that {\em “Proudhon was the master of us all.”} According to George Woodcock Kropotkin was one of Proudhon’s {\em “confessed disciples.”} Perhaps that makes Bakunin and Kropotkin proto-capitalists? Obviously not. What about Tucker? He called Proudhon {\em “the father of the Anarchistic school of Socialism.”} [{\bf Instead of a Book}, p. 381] And, as we noted above, the socialist historian Carl Launder considered Proudhon a socialist, as did the noted British socialist G.D.H. Cole in his {\bf History of Socialist Thought} (and in fact called him one of the {\em “major prophets of Socialism.”}). What about Marx and Engels, surely they would be able to say if he was a socialist or not? According to Engels, Proudhon was {\em “the Socialist of the small peasant and master-craftsman.”} [Marx and Engels, {\bf Selected Works}, p. 260]


In fact, the only “left” (i.e. social) anarchist of note who seems to place Proudhon outside of the “leftist” (i.e. anarchist) camp is Murray Bookchin. In the second volume of {\bf The Third Revolution} Bookchin argues that {\em “Proudhon was no socialist”} simply because he favoured {\em “private property.”} [p. 39] However, he does note the {\em “one moral provision [that] distinguished the Proudhonist contract from the capitalist contract”} namely {\em “it abjured profit and exploitation.”} [{\bf Op. Cit.}, pp. 40–41] — which, of course, places him in the socialist tradition (see last section). Unfortunately, Bookchin fails to acknowledge this or that Proudhon was totally opposed to wage labour along with usury, which, again, instantly places him in ranks of socialism (see, for example, the {\bf General Idea of the Revolution}, p. 98, pp. 215–6 and pp. 221–2, and his opposition to state control of capital as being {\em “more wage slavery”} and, instead, urging whatever capital required collective labour to be {\em “democratically organised workers’ associations”} [{\bf No Gods, No Masters}, vol. 1, p. 62]).


Bookchin (on page 78) quotes Proudhon as arguing that {\em “association”} was {\em “a protest against the wage system”} which suggests that Bookchin’s claims that Proudhonian {\em “analysis minimised the social relations embodied in the capitalist market and industry”} [p. 180] is false. Given that wage labour is {\bf the} unique social relationship within capitalism, it is clear from Proudhon’s works that he did not “minimise” the social relations created by capitalism, rather the opposite. Proudhon’s opposition to wage labour clearly shows that he focused on the {\bf key} social relation which capitalism creates — namely the one of domination of the worker by the capitalist.


Bookchin {\bf does} mention that Proudhon was {\em “obliged in 1851, in the wake of the associationist ferment of 1848 and after, to acknowledge that association of some sort was unavoidable for large-scale enterprises.”} [p. 78] However, Proudhon’s support of industrial democracy pre-dates 1851 by some 11 years. He stated in {\bf What is Property?} that he {\em “preach[ed] emancipation to the proletaires; association to the labourers”} and that {\em “leaders”} within industry {\em “must be chosen from the labourers by the labourers themselves.”} [p. 137 and p. 414] It is significant that the first work to call itself anarchist opposed property along with the state, exploitation along with oppression and supported self-management against hierarchical relationships within production (“anarcho”-capitalists take note!). Proudhon also called for {\em “democratically organised workers’ associations”} to run large-scale industry in his 1848 Election Manifesto. [{\bf No Gods, No Masters}, vol. 1, p. 62] Given that Bookchin considers as {\em “authentic artisanal socialists”} those who called for {\bf collective} ownership of the means of production, but {\em “exempted from collectivisation the peasantry”} [p. 4] we have to conclude that Proudhon was such an “authentic” artisanal socialist! Indeed, at one point Bookchin mentions the {\em “individualistic artisanal socialism of Proudhon”} [p. 258] which suggests a somewhat confused approach to Proudhon’s ideas!


In effect, Bookchin makes the same mistake as Caplan; but, unlike Caplan, he should know better. Rather than not being a socialist, Proudhon is obviously an example of what Bookchin himself calls {\em “artisanal socialism”} (as Marx and Engels recongised). Indeed, he notes that Proudhon was its {\em “most famous advocate”} and that {\em “nearly all so-called ‘utopian’ socialists, even [Robert] Owen — the most labour-orientated — as well as Proudhon — essentially sought the equitable distribution of property.”} [p. 273] Given Proudhon’s opposition to wage labour and capitalist property and his support for industrial democracy as an alternative, Bookchin’s position is untenable — he confuses socialism with communism, rejecting as socialist all views which are not communism (a position he shares with right-libertarians).


He did not always hold this position, though. He writes in {\bf The Spanish Anarchists} that:



\startblockquote
{\em “Proudhon envisions a free society as one in which small craftsmen, peasants, and collectively owned industrial enterprises negotiate and contract with each other to satisfy their material needs. Exploitation is brought to an end\unknown{} Although these views involve a break with capitalism, by no means can they be regarded as communist ideas\unknown{}”} [p. 18]



\stopblockquote
In contrast to some of Bookchin’s comments (and Caplan) K. Steven Vincent is correct to argue that, for Proudhon, justice {\em “applied to the economy was associative socialism”} and so Proudhon is squarely in the socialist camp [{\bf Pierre-Joseph Proudhon and the Rise of French Republican Socialism}, p. 228].


However, perhaps all these “leftists” are wrong (bar Bookchin, who {\bf is} wrong, at least some of the time). Perhaps they just did not understand what socialism actually is (and as Proudhon stated {\em “I am socialist”} [{\bf Selected Writing of Pierre-Joseph Proudhon}, p. 195] and described himself as a socialist many times this also applies to Proudhon himself!). So the question arises, did Proudhon support private property in the capitalist sense of the word? The answer is no. To quote George Woodcock summary of Proudhon’s ideas on this subject we find:



\startblockquote
{\em “He [Proudhon] was denouncing the property of a man who uses it to exploit the labour of others, without an effort on his own part, property distinguished by interest and rent, by the impositions of the non-producer on the producer. Towards property regarded as ‘possession,’ the right of a man to control his dwelling and the land and tools he needs to live, Proudhon had no hostility; indeed he regarded it as the cornerstone of liberty.”} [{\em “On Proudhon’s ‘What is Property?’”}, {\bf The Raven} No. 31, pp. 208–9]



\stopblockquote
George Crowder makes the same point:



\startblockquote
{\em “The ownership he opposes is basically that which is unearned \unknown{} including such things as interest on loans and income from rent. This is contrasted with ownership rights in those goods either produced by the work of the owner or necessary for that work, for example his dwelling-house, land and tools. Proudhon initially refers to legitimate rights of ownership of these goods as ‘possession,’ and although in his latter work he calls {\bf this} ‘property,’ the conceptual distinction remains the same.”} [{\bf Classical Anarchism}, pp. 85–86]



\stopblockquote
Indeed, according to Proudhon himself, the {\em “accumulation of capital and instrument is what the capitalist owes to the producer, but he never pays him for it. It is this fraudulent deprivation which causes the poverty of the worker, the opulence of the idle and the inequality of their conditions. And it is this, above all, which has so aptly been called the exploitation of man by man.”} [{\bf Selected Writings of Pierre-Joseph Proudhon}, p. 43]


He called his ideas on possession a {\em “third form of society, the synthesis of communism and property”} and calls it {\em “liberty.”} [{\bf The Anarchist Reader}, p. 68]. He even goes so far as to say that property {\em “by its despotism and encroachment, soon proves itself oppressive and anti-social.”} [{\bf Op. Cit.}, p. 67] Opposing private property he thought that {\em “all accumulated capital is collective property, no one may be its exclusive owner.”} Indeed, he considered the aim of his economic reforms {\em “was to rescue the working masses from capitalist exploitation.”} [{\bf Selected Writings of Pierre-Joseph Proudhon}, p. 44, p. 80]


In other words, Proudhon considered capitalist property to be the source of exploitation and oppression and he opposed it. He explicitly contrasts his ideas to that of capitalist property and {\bf rejects} it as a means of ensuring liberty.


Caplan goes on to claim that {\em “[s]ome of Proudhon’s other heterodoxies include his defence of the right of inheritance and his emphasis on the genuine antagonism between state power and property rights.”}


However, this is a common anarchist position. Anarchists are well aware that possession is a source of independence within capitalism and so should be supported. As Albert Meltzer puts it:



\startblockquote
{\em “All present systems of ownership mean that some are deprived of the fruits of their labour. It is true that, in a competitive society, only the possession of independent means enables one to be free of the economy (that is what Proudhon meant when, addressing himself to the self-employed artisan, he said ‘property is liberty’, which seems at first sight a contradiction with his dictum that it was theft)”}[{\bf Anarchism: Arguments For and Against}, pp. 12–13]



\stopblockquote
Malatesta makes the same point:



\startblockquote
{\em “Our opponents \unknown{} are in the habit of justifying the right to private property by stating that property is the condition and guarantee of liberty.}


{\em “And we agree with them. Do we not say repeatedly that poverty is slavery?}


{\em “But then why do we oppose them?}


{\em “The reason is clear: in reality the property that they defend is capitalist property\unknown{} which therefore depends on the existence of a class of the disinherited and dispossessed, forced to sell their labour to the property owners for a wage below its real value\unknown{} This means that workers are subjected to a kind of slavery.”} [{\bf The Anarchist Revolution}, p. 113]



\stopblockquote
As does Kropotkin:



\startblockquote
{\em “the only guarantee not to be robbed of the fruits of your labour is to possess the instruments of labour\unknown{} man really produces most when he works in freedom, when he has a certain choice in his occupations, when he has no overseer to impede him, and lastly, when he sees his work bringing profit to him and to others who work like him, but bringing in little to idlers.”} [{\bf The Conquest of Bread}, p. 145]



\stopblockquote
Perhaps this makes these three well known anarcho-communists “really” proto-“anarcho”-capitalists as well? Obviously not. Instead of wondering if his ideas on what socialism is are wrong, he tries to rewrite history to fit the anarchist movement into his capitalist ideas of what anarchism, socialism and whatever are actually like.


In addition, we must point out that Proudhon’s {\em “emphasis on the genuine antagonism between state power and property rights”} came from his later writings, in which he argued that property rights were required to control state power. In other words, this {\em “heterodoxy”} came from a period in which Proudhon did not think that state could be abolished and so {\em “property is the only power that can act as a counterweight to the State.”} [{\bf Selected Writings of Pierre-Joseph Proudhon}, p. 140] Of course, this “later” Proudhon also acknowledged that property was {\em “an absolutism within an absolutism,”} {\em “by nature autocratic”} and that its {\em “politics could be summed up in a single word,”} namely {\em “exploitation.”} [p. 141, p. 140, p. 134]


Moreover, Proudhon argues that {\em “spread[ing] it more equally and establish[ing] it more firmly in society”} is the means by which {\em “property”} {\em “becomes a guarantee of liberty and keeps the State on an even keel.”} [p. 133, p. 140] In other words, rather than “property” {\bf as such} limiting the state, it is “property” divided equally through society which is the key, without concentrations of economic power and inequality which would result in exploitation and oppression. Therefore, {\em “[s]imple justice\unknown{} requires that equal division of land shall not only operate at the outset. If there is to be no abuse, it must be maintained from generation to generation.”} [{\bf Op. Cit.}, p. 141, p. 133, p. 130].


Interestingly, one of Proudhon’s {\em “other heterodoxies”} Caplan does not mention is his belief that “property” was required not only to defend people against the state, but also capitalism. He saw society dividing into {\em “two classes, one of employed workers, the other of property-owners, capitalists, entrepreneurs.”} He thus recognised that capitalism was just as oppressive as the state and that it assured {\em “the victory of the strong over the weak, of those who property over those who own nothing.”} [as quoted by Alan Ritter, {\bf The Political Thought of Pierre-Joseph Proudhon}, p. 121] Thus Proudhon’s argument that {\em “property is liberty”} is directed not only against the state, but also against social inequality and concentrations of economic power and wealth.


Indeed, he considered that {\em “companies of capitalists”} were the {\em “exploiters of the bodies and souls of their wage earners”} and an outrage on {\em “human dignity and personality.”} Instead of wage labour he thought that the {\em “industry to be operated, the work to be done, are the common and indivisible property of all the participant workers.”} In other words, self-management and workers’ control. In this way there would be {\em “no more government of man by man, by means of accumulation of capital”} and the {\em “social republic”} established. Hence his support for co-operatives:



\startblockquote
{\em “The importance of their work lies not in their petty union interests, but in their denial of the rule of capitalists, usurers, and governments, which the first [French] revolution left undisturbed. Afterwards, when they have conquered the political lie\unknown{} the groups of workers should take over the great departments of industry which are their natural inheritance.”} [cited in {\bf Pierre-Joseph Proudhon}, E. Hymans, pp. 190–1, and {\bf Anarchism}, George Woodcock, p. 110, 112]



\stopblockquote
In other words, a {\bf socialist} society as workers would no longer be separated from the means of production and they would control their own work (the {\em “abolition of the proletariat,”} to use Proudhon’s expression). This would mean recognising that {\em “the right to products is exclusive — jus in re; the right to means is common — jus ad rem”} [cited by Woodcock, {\bf Anarchism}, p. 96] which would lead to self-management:



\startblockquote
{\em “In democratising us, revolution has launched us on the path of industrial democracy.”} [{\bf Selected Writings of Pierre-Joseph Proudhon}, p. 63]



\stopblockquote
As Woodcock points out, in Proudhon’s {\em “picture of the ideal society of the ideal society it is this predominance of the small proprietor, the peasant or artisan, that immediately impresses one”} with {\em “the creation of co-operative associations for the running of factories and railways.”} [{\em “On Proudhon’s ‘What is Property?’”}, {\bf Op. Cit.}, p. 209, p. 210]


All of which hardly supports Caplan’s attempts to portray Proudhon as “really” a capitalist all along. Indeed, the “later” Proudhon’s support for protectionism [{\bf Selected Writings of Pierre-Joseph Proudhon}, p. 187], the {\em “fixing after amicable discussion of a {\bf maximum} and {\bf minimum} profit margin,”} {\em “the organising of regulating societies”} and that mutualism would {\em “regulate the market”} [{\bf Op. Cit.}, p. 70] and his obvious awareness of economic power and that capitalism exploited and oppressed the wage-worker suggests that rather than leading some to exclude Proudhon from the “leftist camp” altogether, it is a case of excluding him utterly from the “rightist camp” (i.e. “anarcho”-capitalism). Therefore Caplan’s attempt to claim (co-opt would be better) Proudhon for “anarcho”-capitalism indicates how far Caplan will twist (or ignore) the evidence. As would quickly become obvious when reading his work, Proudhon would (to use Caplan’s words) {\em “normally classify government, property, hierarchical organisations \unknown{} as ‘rulership.’”}


To summarise, Proudhon was a socialist and Caplan’s attempts to rewrite anarchist and socialist history fails. Proudhon was the fountainhead for both wings of the anarchist movement and {\bf What is Property?} {\em “embraces the core of nineteenth century anarchism\unknown{} [bar support for revolution] all the rest of later anarchism is there, spoken or implied: the conception of a free society united by association, of workers controlling the means of production\unknown{} [this book] remains the foundation on which the whole edifice of nineteenth century anarchist theory was to be constructed.”} [{\bf Op. Cit.}, p. 210]


Little wonder Bakunin stated that his ideas were Proudhonism {\em “widely developed and pushed to these, its final consequences.”} [{\bf Michael Bakunin: Selected Writings}, p. 198]


\section{4. Tucker on Property, Communism and Socialism.
}

That Tucker called himself a socialist is quickly seen from {\bf Instead of A Book} or any of the books written about Tucker and his ideas. That Caplan seeks to deny this means that either Caplan has not looked at either {\bf Instead of a Book} or the secondary literature (with obvious implications for the accuracy of his FAQ) or he decided to ignore these facts in favour of his own ideologically tainted version of history (again with obvious implications for the accuracy and objectivity of his FAQ).


Caplan, in an attempt to deny the obvious, quotes Tucker from 1887 as follows in section 14 (What are the major debates between anarchists? What are the recurring arguments?):



\startblockquote
{\em “It will probably surprise many who know nothing of Proudhon save his declaration that ‘property is robbery’ to learn that he was perhaps the most vigorous hater of Communism that ever lived on this planet. But the apparent inconsistency vanishes when you read his book and find that by property he means simply legally privileged wealth or the power of usury, and not at all the possession by the labourer of his products.”}



\stopblockquote
You will instantly notice that Proudhon does not mean by property {\em “the possession of the labourer of his products.”} However, Proudhon did include in his definition of “property” the possession of the capital to steal profits from the work of the labourers. As is clear from the quote, Tucker and Proudhon was opposed to capitalist property ({\em “the power of usury”}). From Caplan’s own evidence he proves that Tucker was not a capitalist!


But lets quote Tucker on what he meant by {\em “usury”}:



\startblockquote
{\em “There are three forms of usury, interest on money, rent on land and houses, and profit in exchange. Whoever is in receipt of any of these is a usurer.”} [cited in {\bf Men against the State} by James J. Martin, p. 208]



\stopblockquote
Which can hardly be claimed as being the words of a person who supports capitalism!


And we should note that Tucker considered both government and capital oppressive. He argued that anarchism meant {\em “the restriction of power to self and the abolition of power over others. Government makes itself felt alike in country and in city, capital has its usurious grip on the farm as surely as on the workshop and the oppressions and exactions of neither government nor capital can be avoided by migration.”} [{\bf Instead of a Book}, p. 114]


And, we may add, since when was socialism identical to communism? Perhaps Caplan should actually read Proudhon and the anarchist critique of private property before writing such nonsense? We have indicated Proudhon’s ideas above and will not repeat ourselves. However, it is interesting that this passes as “evidence” of “anti-socialism” for Caplan, indicating that he does not know what socialism or anarchism actually is. To state the obvious, you can be a hater of “communism” and still be a socialist!


So this, his one attempt to prove that Tucker, Spooner and even Proudhon were really capitalists by quoting the actual people involved is a failure.


He asserts that for any claim that “anarcho”-capitalism is not anarchist is wrong because {\em “the factual supporting arguments are often incorrect. For example, despite a popular claim that socialism and anarchism have been inextricably linked since the inception of the anarchist movement, many 19\high{th}-century anarchists, not only Americans such as Tucker and Spooner, but even Europeans like Proudhon, were ardently in favour of private property (merely believing that some existing sorts of property were illegitimate, without opposing private property as such).”}


The facts supporting the claim of anarchists being socialists, however, are not “incorrect.” It is Caplan’s assumption that socialism is against all forms of “property” which is wrong. To state the obvious, socialism does not equal communism (and anarcho-communists support the rights of workers to own their own means of production if they do not wish to join communist communes — see above). Thus Proudhon was renown as the leading French Socialist theorist when he was alive. His ideas were widely known in the socialist movement and in many ways his economic theories were similar to the ideas of such well known early socialists as Robert Owen and William Thompson. As Kropotkin notes:



\startblockquote
{\em “It is worth noticing that French mutualism had its precursor in England, in William Thompson, who began by mutualism before he became a communist, and in his followers John Gray (A Lecture on Human Happiness, 1825; The Social System, 1831) and J. F. Bray (Labour’s Wrongs and Labour’s Remedy, 1839).”} [{\bf Kropotkin’s Revolutionary Pamphlets}, p. 291]



\stopblockquote
Perhaps Caplan will now claim Robert Owen and William Thompson as capitalists?


Tucker called himself a socialist on many different occasions and stated that there were {\em “two schools of Socialistic thought \unknown{} State Socialism and Anarchism.”} And stated in very clear terms that:



\startblockquote
{\em “liberty insists on Socialism\unknown{} — true Socialism, Anarchistic Socialism: the prevalence on earth of Liberty, Equality, and Solidarity.”} [{\bf Instead of a Book}, p. 363]



\stopblockquote
And like all socialists, he opposed capitalism (i.e. usury and wage slavery) and wished that {\em “there should be no more proletaires.”} [see the essay {\em “State Socialism and Anarchism”} in {\bf Instead of a Book}, p. 17]


Caplan, of course, is well aware of Tucker’s opinions on the subject of capitalism and private property. In section 13 (What moral justifications have been offered for anarchism?) he writes:



\startblockquote
{\em “Still other anarchists, such as Lysander Spooner and Benjamin Tucker as well as Proudhon, have argued that anarchism would abolish the exploitation inherent in interest and rent simply by means of free competition. In their view, only labour income is legitimate, and an important piece of the case for anarchism is that without government-imposed monopolies, non-labour income would be driven to zero by market forces. It is unclear, however, if they regard this as merely a desirable side effect, or if they would reject anarchism if they learned that the predicted economic effect thereof would not actually occur.”}



\stopblockquote
Firstly, we must point that Proudhon, Tucker and Spooner considered {\bf profits} to be exploitative as well as interest and rent. Hence we find Tucker arguing that a {\em “just distribution of the products of labour is to be obtained by destroying all sources of income except labour. These sources may be summed up in one word, — usury; and the three principle forms of usury are interest, rent and profit.”} [{\bf Instead of a Book}, p. 474] To ignore the fact that Tucker also considered profit as exploitative seems strange, to say the least, when presenting an account of his ideas.


Secondly, rather than it being {\em “unclear”} whether the end of usury was {\em “merely a desirable side effect”} of anarchism, the opposite is the case. Anyone reading Tucker (or Proudhon) would quickly see that their politics were formulated with the express aim of ending usury. Just one example from hundreds:



\startblockquote
{\em “Liberty will abolish interest; it will abolish profit; it will abolish monopolistic rent; it will abolish taxation; it will abolish the exploitation of labour; it will abolish all means whereby any labourer can be deprived of any of his product.”} [{\bf Instead of a Book}, p. 347]



\stopblockquote
While it is fair to wonder whether these economic effects would result from the application of Tucker’s ideas, it {\bf is} distinctly incorrect to claim that the end of usury was considered in any way as a {\em “desirable side effect”} of them. Rather, in {\bf their} eyes, the end of usury was one of {\bf the} aims of Individualist Anarchism, as can be clearly seen. As Wm. Gary Kline points out in his excellent account of Individualist Anarchism:



\startblockquote
{\em “the American anarchists exposed the tension existing in liberal thought between private property and the ideal of equal access. The Individualist Anarchists were, at least, aware that existing conditions were far from ideal, that the system itself worked against the majority of individuals in their efforts to attain its promises. Lack of capital, the means to creation and accumulation of wealth, usually doomed a labourer to a life of exploitation. This the anarchists knew and they abhorred such a system.”} [{\bf The Individualist Anarchists}, p. 102]



\stopblockquote
This is part of the reason why they considered themselves socialists and, equally as important, they were considered socialists by {\bf other} socialists such as Kropotkin and Rocker. The Individualist Anarchists, as can be seen, fit very easily into Kropotkin’s comments that {\em “the anarchists, in common with all socialists\unknown{} maintain that the now prevailing system of private ownership in land, and our capitalist production for the sake of profits, represent a monopoly which runs against both the principles of justice and the dictates of utility.”} [{\bf Kropotkin’s Revolutionary Pamphlets}, p. 285] Given that they considered profits as usury and proposed {\em “occupancy and use”} in place of the prevailing land ownership rights they are obviously socialists.


That the end of usury was considered a clear aim of his politics explains Tucker’s 1911 postscript to his famous essay {\em “State Socialism and Anarchism”} in which he argues that {\em “concentrated capital”} {\bf itself} was a barrier towards anarchy. He argued that the {\em “trust is now a monster which\unknown{} even the freest competition, could it be instituted, would be unable to destroy.”} While, in an earlier period, big business {\em “needed the money monopoly for its sustenance and its growth”} its size now ensured that it {\em “sees in the money monopoly a convenience, to be sure, but no longer a necessity. It can do without it.”} This meant that the way was now {\em “not so clear.”} Indeed, he argued that the problem of the trusts {\em “must be grappled with for a time solely by forces political or revolutionary”} as the trust had moved beyond the reach of {\em “economic forces”} simply due to the concentration of resources in its hands. [{\em “Postscript”} to {\bf State Socialism and Anarchism}]


If the end of {\em “usury”} {\bf was} considered a {\em “side-effect”} rather than an objective, then the problems of the trusts and economic inequality/power ({\em “enormous concentration of wealth”}) would not have been an issue. That the fact of economic power {\bf was} obviously considered a hindrance to anarchy suggests the end of usury was a key aim, an aim which “free competition” in the abstract could not achieve. Rather than take the “anarcho”-capitalist position that massive inequality did not affect “free competition” or individual liberty, Tucker obviously thought it did and, therefore, “free competition” (and so the abolition of the public state) in conditions of massive inequality would not create an anarchist society.


By trying to relegate an aim to a {\em “side-effect,”} Caplan distorts the ideas of Tucker. Indeed, his comments on trusts, {\em “concentrated capital”} and the {\em “enormous concentration of wealth”} indicates how far Individualist Anarchism is from “anarcho”-capitalism (which dismisses the question of economic power Tucker raises out of hand). It also indicates the unity of political and economic ideas, with Tucker being aware that without a suitable economic basis individual freedom was meaningless. That an economy (like capitalism) with massive inequalities in wealth and so power was not such a basis is obvious from Tucker’s comments.


Thirdly, what did Tucker consider as a government-imposed monopoly? Private property, particularly in land! As he states {\em “Anarchism undertakes to protect no titles except such as are based upon actual occupancy and use”} and that anarchism {\em “means the abolition of landlordism and the annihilation of rent.”} [{\bf Instead of a Book}, p. 61, p. 300] This, to state the obvious, is a restriction on “private property” (in the capitalist sense), which, if we use Caplan’s definition of socialism, means that Tucker was obviously part of the “Leftist camp” (i.e. socialist camp). In other words, Tucker considered capitalism as the product of statism while socialism (libertarian of course) would be the product of anarchy.


So, Caplan’s historical argument to support his notion that anarchism is simply anti-government fails. Anarchism, in all its many forms, have distinct economic as well as political ideas and these cannot be parted without loosing what makes anarchism unique. In particular, Caplan’s attempt to portray Proudhon as an example of a “pure” anti-government anarchism also fails, and so his attempt to co-opt Tucker and Spooner also fails (as noted, Tucker cannot be classed as a “pure” anti-government anarchist either). If Proudhon was a socialist, then it follows that his self-proclaimed followers will also be socialists — and, unsurprisingly, Tucker called himself a socialist and considered anarchism as part of the wider socialist movement.



\startblockquote
{\em “Like Proudhon, Tucker was an ‘un-marxian socialist’”} [William O. Reichart, {\bf Partisans of Freedom: A Study in American Anarchism}, p. 157]



\stopblockquote
\section{5. Anarchism and “anarcho”-capitalism
}

Caplan tries to build upon the non-existent foundation of Tucker’s and Proudhon’s “capitalism” by stating that:



\startblockquote
{\em “Nor did an ardent anarcho-communist like Kropotkin deny Proudhon or even Tucker the title of ‘anarchist.’ In his Modern Science and Anarchism, Kropotkin discusses not only Proudhon but ‘the American anarchist individualists who were represented in the fifties by S.P. Andrews and W. Greene, later on by Lysander Spooner, and now are represented by Benjamin Tucker, the well-known editor of the New York Liberty.’ Similarly in his article on anarchism for the 1910 edition of the Encyclopedia Britannica, Kropotkin again freely mentions the American individualist anarchists, including ‘Benjamin Tucker, whose journal Liberty was started in 1881 and whose conceptions are a combination of those of Proudhon with those of Herbert Spencer.’”}



\stopblockquote
There is a nice historical irony in Caplan’s attempts to use Kropotkin to prove the historical validity of “anarcho”-capitalism. This is because while Kropotkin was happy to include Tucker into the anarchist movement, Tucker often claimed that an anarchist could not be a communist! In {\bf State Socialism and Anarchism} he stated that anarchism was {\em “an ideal utterly inconsistent with that of those Communists who falsely call themselves Anarchists while at the same time advocating a regime of Archism fully as despotic as that of the State Socialists themselves.”} [{\em “State Socialism and Anarchism”}, {\bf Instead of a Book}, pp. 15–16]


While modern social anarchists follow Kropotkin in not denying Proudhon or Tucker as anarchists, we do deny the anarchist title to supporters of capitalism. Why? Simply because anarchism as a {\bf political} movement (as opposed to a dictionary definition) has always been anti-capitalist and against capitalist wage slavery, exploitation and oppression. In other words, anarchism (in all its forms) has always been associated with specific political {\bf and} economic ideas. Both Tucker and Kropotkin defined their anarchism as an opposition to both state and capitalism. To quote Tucker on the subject:



\startblockquote
{\em “Liberty insists\unknown{} [on] the abolition of the State and the abolition of usury; on no more government of man by man, and no more exploitation of man by man.”} [cited in {\bf Native American Anarchism — A Study of Left-Wing American Individualism} by Eunice Schuster, p. 140]



\stopblockquote
Kropotkin defined anarchism as {\em “the no-government system of socialism.”} [{\bf Kropotkin’s Revolutionary Pamphlets}, p. 46] Malatesta argued that {\em “when [people] sought to overthrow both State and property — then it was anarchy was born”} and, like Tucker, aimed for {\em “the complete destruction of the domination and exploitation of man by man.”} [{\bf Life and Ideas}, p. 19, pp. 22–28] Indeed {\bf every} leading anarchist theorist defined anarchism as opposition to government {\bf and} exploitation. Thus Brain Morris’ excellent summary:



\startblockquote
{\em “Another criticism of anarchism is that it has a narrow view of politics: that it sees the state as the fount of all evil, ignoring other aspects of social and economic life. This is a misrepresentation of anarchism. It partly derives from the way anarchism has been defined [in dictionaries, for example], and partly because Marxist historians have tried to exclude anarchism from the broader socialist movement. But when one examines the writings of classical anarchists\unknown{} as well as the character of anarchist movements\unknown{} it is clearly evident that it has never had this limited vision. It has always challenged all forms of authority and exploitation, and has been equally critical of capitalism and religion as it has been of the state.”} [{\em “Anthropology and Anarchism,”} {\bf Anarchy: A Journal of Desire Armed} no. 45, p. 40]



\stopblockquote
Therefore anarchism was never purely a political concept, but always combined an opposition to oppression with an opposition to exploitation. Little wonder, then, that both strands of anarchism have declared themselves “socialist” and so it is {\em “conceptually and historically misleading”} to {\em “create a dichotomy between socialism and anarchism.”} [Brian Morris, {\bf Op. Cit.}, p. 39] Needless to say, anarchists oppose {\bf state} socialism just as much as they oppose capitalism. All of which means that anarchism and capitalism are two {\bf different} political ideas with specific (and opposed) meanings — to deny these meanings by uniting the two terms creates an oxymoron, one that denies the history and the development of ideas as well as the whole history of the anarchist movement itself.


As Kropotkin knew Proudhon to be an anti-capitalist, a socialist (but not a communist) it is hardly surprising that he mentions him. Again, Caplan’s attempt to provide historical evidence for a “right-wing” anarchism fails. Funny that the followers of Kropotkin are now defending individualist anarchism from the attempted “adoption” by supporters of capitalism! That in itself should be enough to indicate Caplan’s attempt to use Kropotkin to give credence to “anarcho”-capitalist co-option of Proudhon, Tucker and Spooner fails.


Interestingly, Caplan admits that “anarcho”-capitalism has recent origins. In section 8 (Who are the major anarchist thinkers?) he states:



\startblockquote
{\em “Anarcho-capitalism has a much more recent origin in the latter half of the 20\high{th} century. The two most famous advocates of anarcho-capitalism are probably Murray Rothbard and David Friedman. There were however some interesting earlier precursors, notably the Belgian economist Gustave de Molinari. Two other 19\high{th}-century anarchists who have been adopted by modern anarcho-capitalists with a few caveats are Benjamin Tucker and Lysander Spooner. (Some left-anarchists contest the adoption, but overall Tucker and Spooner probably have much more in common with anarcho-capitalists than with left-anarchists.)”}



\stopblockquote
Firstly, as he states, Tucker and Spooner have been {\em “adopted”} by the “anarcho”-capitalist school. Being dead they have little chance to protest such an adoption, but it is clear that they considered themselves as socialists, against capitalism (it may be claimed that Spooner never called himself a socialist, but then again he never called himself an anarchist either; it is his strong opposition to wage labour that places him in the socialist camp). Secondly, Caplan lets the cat out the bag by noting that this “adoption” involved a few warnings — more specifically, the attempt to rubbish or ignore the underlying socio-economic ideas of Tucker and Spooner and the obvious anti-capitalist nature of their vision of a free society.


Individualist anarchists are, indeed, more similar to classical liberals than social anarchists. Similarly, social anarchists are more similar to Marxists than Individualist anarchists. But neither statement means that Individualist anarchists are capitalists, or social anarchists are state socialists. It just means some of their ideas overlap — and we must point out that Individualist anarchist ideas overlap with Marxist ones, and social anarchist ones with liberal ones (indeed, one interesting overlap between Marxism and Individualist Anarchism can be seen from Marx’s comment that abolishing interest and interest-bearing capital {\em “means the abolition of capital and of capitalist production itself.”} [{\bf Theories of Surplus Value}, vol. 3, p. 472] Given that Individualist Anarchism aimed to abolish interest (along with rent and profit) it would suggest, from a Marxist position, that it is a socialist theory).


So, if we accept Kropotkin’s summary that Individualist Anarchism ideas are {\em “partly those of Proudhon, but party those of Herbert Spencer”} [{\bf Kropotkins’ Revolutionary Pamphlets}, p. 173], what the “anarcho”-capitalist school is trying to is to ignore the Proudhonian (i.e. socialist) aspect of their theories. However, that just leaves Spencer and Spencer was not an anarchist, but a right-wing Libertarian, a supporter of capitalism (a {\em “champion of the capitalistic class”} as Tucker put it). In other words, to ignore the socialist aspect of Individualist Anarchism (or anarchism in general) is to reduce it to liberalism, an extreme version of liberalism, but liberalism nevertheless — and liberalism is not anarchism. To reduce anarchism so is to destroy what makes anarchism a unique political theory and movement:



\startblockquote
{\em “anarchism does derive from liberalism and socialism both historically and ideologically \unknown{} In a sense, anarchists always remain liberals and socialists, and whenever they reject what is good in either they betray anarchism itself \unknown{} We are liberals but more so, and socialists but more so.”} [Nicholas Walter, {\bf Reinventing Anarchy}, p. 44]



\stopblockquote
In other words, “anarcho”-capitalism is a development of ideas which have little in common with anarchism. Jeremy Jennings, in his overview of anarchist theory and history, agrees:



\startblockquote
{\em “It is hard not to conclude that these ideas [“anarcho”-capitalism] — with roots deep in classical liberalism — are described as anarchist only on the basis of a misunderstanding of what anarchism is.”} [{\bf Contemporary Political Ideologies}, Roger Eatwell and Anthony Wright (eds.), p. 142]



\stopblockquote
Barbara Goodwin also agrees that the “anarcho”-capitalists’ {\em “true place is in the group of right-wing libertarians”} not in anarchism [{\bf Using Political Ideas}, p. 148]. Indeed, that “anarcho”-capitalism is an off-shoot of classical liberalism is a position Murray Rothbard would agree with, as he states that right-wing Libertarians constitute {\em “the vanguard of classical liberalism.”} [quoted by Ulrike Heider, {\bf Anarchism: Left, Right and Green}, p. 95] Unfortunately for this perspective anarchism is not liberalism and liberalism is not anarchism. And equally as unfortunate (this time for the anarchist movement!) “anarcho”-capitalism {\em “is judged to be anarchism largely because some anarcho-capitalists {\bf say} they are ‘anarchists’ and because they criticise the State.”} [Peter Sabatini, {\bf Social Anarchism}, no. 23, p. 100] However, being opposed to the state is a necessary but not sufficient condition for being an anarchist (as can be seen from the history of the anarchist movement). Brian Morris puts it well when he writes:



\startblockquote
{\em “The term anarchy comes from the Greek, and essentially means ‘no ruler.’ Anarchists are people who reject all forms of government or coercive authority, all forms of hierarchy and domination. They are therefore opposed to what the Mexican anarchist Flores Magon called the ‘sombre trinity’ — state, capital and the church. Anarchists are thus opposed to both capitalism and to the state, as well as to all forms of religious authority. But anarchists also seek to establish or bring about by varying means, a condition of anarchy, that is, a decentralised society without coercive institutions, a society organised through a federation of voluntary associations. Contemporary ‘right-wing’ libertarians \unknown{} who are often described as ‘anarchocapitalists’ and who fervently defend capitalism, are not in any real sense anarchists.”} [{\bf Op. Cit.}, p. 38]



\stopblockquote
Rather than call themselves by a name which reflects their origins in liberalism (and {\bf not} anarchism), the “anarcho”-capitalists have instead seen fit to try and appropriate the name of anarchism and, in order to do so, ignore key aspects of anarchist theory in the process. Little wonder, then, they try and prove their anarchist credentials via dictionary definitions rather than from the anarchist movement itself (see next section).


Caplan’s attempt in his FAQ is an example to ignore individualist anarchist theory and history. Ignored is any attempt to understand their ideas on property and instead Caplan just concentrates on the fact they use the word. Caplan also ignores:



\startitemize[1]\relax
\item[] their many statements on being socialists and part of the wider socialist movement.




 \item[] their opposition to capitalist property-rights in land and other scarce resources.




 \item[] their recognition that capitalism was based on usury and that it was exploitation.




 \item[] their attacks on government {\bf and} capital, rather than just government.




 \item[] their support for strikes and other forms of direct action by workers to secure the full product of their labour.




 
\stopitemize
In fact, the only things considered useful seems to be the individualist anarchist’s support for free agreement (something Kropotkin also agreed with) and their use of the word “property.” But even a cursory investigation indicates the non-capitalist nature of their ideas on property and the socialistic nature of their theories.


Perhaps Caplan should ponder these words of Kropotkin supporters of the {\em “individualist anarchism of the American Proudhonians \unknown{} soon realise that the individualisation they so highly praise is not attainable by individual efforts, and \unknown{} abandon the ranks of the anarchists, and are driven into the liberal individualism of the classical economist.”} [{\bf Kropotkin’s Revolutionary Pamphlets}, p. 297]


Caplan seems to confuse the end of the ending place of ex-anarchists with their starting point. As can be seen from his attempt to co-opt Proudhon, Spooner and Tucker he has to ignore their ideas and rewrite history.


\section{6. Appendix: Defining Anarchism
}

In his Appendix {\em “Defining Anarchism”} we find that Caplan attempts to defend his dictionary definition of anarchism. He does this by attempting to refute two arguments, The Philological Argument and the Historical Argument.


Taking each in turn we find:


Caplan’s definition of {\em “The Philological Argument”} is as follows:



\startblockquote
{\em “Several critics have noted the origin of the term ‘anarchy,’ which derives from the Greek ‘arkhos,’ meaning ‘ruler,’ and the prefix an-,’ meaning ‘without.’ It is therefore suggested that in my definition the word ‘government’ should be replaced with the word ‘domination’ or ‘rulership’; thus re-written, it would then read: ‘The theory or doctrine that all forms of rulership are unnecessary, oppressive, and undesirable and should be abolished.’”}



\stopblockquote
Caplan replies by stating that:



\startblockquote
{\em “This is all good and well, so long as we realise that various groups of anarchists will radically disagree about what is or is not an instance of ‘rulership.’”}



\stopblockquote
However, in order to refute this argument by this method, he has to ignore his own methodology. A dictionary definition of ruler is {\em “a person who rules by authority.”} and {\em “rule”} is defined as {\em “to have authoritative control over people”} or {\em “to keep (a person or feeling etc.) under control, to dominate”} [{\bf The Oxford Study Dictionary}]


Hierarchy by its very nature is a form of rulership (hier-{\bf {\em archy}}) and is so opposed by anarchists. Capitalism is based upon wage labour, in which a worker follows the rules of their boss. This is obviously a form of hierarchy, of domination. Almost all people (excluding die-hard supporters of capitalism) would agree that being told what to do, when to do and how to do by a boss is a form of rulership. Anarchists, therefore, argue that {\em “economic exploitation and political domination \unknown{} [are] two continually interacting aspects of the same thing — the subjection of man by man.”} [Errico Malatesta, {\bf Life and Ideas}, p. 147] Rocker made the same point, arguing that the {\em “exploitation of man by man and the domination of man over man are inseparable, and each is the condition of the other.”} [{\bf Anarcho-Syndicalism}, p. 18]


Thus Caplan is ignoring the meaning of words to state that {\em “on its own terms this argument fails to exclude anarcho-capitalists”} because they define rulership to exclude most forms of archy! Hardly convincing.


Strangely enough, “anarcho”-capitalist icon Murray Rothbard actually provides evidence that the anarchist position {\bf is} correct. He argues that the state {\em “arrogates to itself a monopoly of force, of ultimate decision-making power, over a given area territorial area.”} [{\bf The Ethics of Liberty}, p. 170] This is obviously a form of rulership. However, he also argues that {\em “[o]bviously, in a free society, Smith has the ultimate decision-making power over his own just property, Jones over his, etc.”} [{\bf Op. Cit.}, p. 173] Which, to state the obvious, means that {\bf both} the state and property is marked by an {\em “ultimate decision-making power”} over a given territory. The only “difference” is that Rothbard claims the former is “just” (i.e. “justly” acquired) and the latter is “unjust” (i.e. acquired by force). In reality of course, the modern distribution of property is just as much a product of past force as is the modern state. In other words, the current property owners have acquired their property in the same unjust fashion as the state has its. If one is valid, so is the other. Rothbard (and “anarcho”-capitalists in general) are trying to have it both ways.


Rothbard goes on to show why statism and private property are essentially the same thing:



\startblockquote
{\em “If the State may be said too properly own its territory, then it is proper for it to make rules for everyone who presumes to live in that area. It can legitimately seize or control private property because there is no private property in its area, because it really owns the entire land surface. So long as the State permits its subjects to leave its territory, then, it can be said to act as does any other owner who sets down rules for people living on his property.”} [{\bf Op. Cit.}, p. 170]



\stopblockquote
Of course Rothbard does not draw the obvious conclusion. He wants to maintain that the state is bad and property is good while drawing attention to their obvious similarities! Ultimately Rothbard is exposing the bankruptcy of his own politics and analysis. According to Rothbard, something can look like a state (i.e. have the {\em “ultimate decision-making power”} over an area) and act like a state (i.e. {\em “make rules for everyone”} who lives in an area, i.e. govern them) but not be a state. This not a viable position for obvious reasons.


Thus to claim, as Caplan does, that property does not generate “rulership” is obviously nonsense. Not only does it ignore the dictionary definition of rulership (which, let us not forget, is Caplan’s {\bf own} methodology) as well as commonsense, it obviously ignores what the two institutions have in common. {\bf If} the state is to be condemned as “rulership” then so must property — for reasons, ironically enough, Rothbard makes clear!


Caplan’s critique of the {\em “Philological Argument”} fails because he tries to deny that the social relationship between worker and capitalist and tenant and landlord is based upon {\bf archy,} when it obviously is. To quote Proudhon, considered by Tucker as {\em “the Anarchist {\bf par excellence,}”} the employee {\em “is subordinated, exploited: his permanent condition is one of obedience.”} Without {\em “association”} (i.e. co-operative workplaces, workers’ self-management) there would be {\em “two industrial castes of masters and wage-workers which is repugnant to a free and democratic society,”} castes {\em “related as subordinates and superiors.”} [{\bf The General Idea of the Revolution}, p. 216]


Moving on, Caplan defines the Historical Argument as:



\startblockquote
{\em “A second popular argument states that historically, the term ‘anarchism’ has been clearly linked with anarcho-socialists, anarcho-communists, anarcho-syndicalists, and other enemies of the capitalist system. Hence, the term ‘anarcho-capitalism’ is a strange oxymoron which only demonstrates ignorance of the anarchist tradition.”}



\stopblockquote
He argues that {\em “even if we were to accept the premise of this argument — to wit, that the meaning of a word is somehow determined by its historical usage — the conclusion would not follow because the minor premise is wrong. It is simply not true that from its earliest history, all anarchists were opponents of private property, free markets, and so on.”}


Firstly, anarchism is not just a word, but a political idea and movement and so the word used in a political context is associated with a given body of ideas. You cannot use the word to describe something which has little or nothing in common with that body of ideas. You cannot call Marxism “anarchism” simply because they share the anarchist opposition to capitalist exploitation and aim for a stateless society, for example.


Secondly, it is true that anarchists like Tucker were not against the free market, but they did not consider capitalism to be defined by the free market but by exploitation and wage labour (as do all socialists). In this they share a common ground with Market Socialists who, like Tucker and Proudhon, do not equate socialism with opposition to the market or capitalism with the “free market.” The idea that socialists oppose {\em “private property, free markets, and so on”} is just an assumption by Caplan. Proudhon, for example, was not opposed to competition, “property” (in the sense of possession) and markets but during his lifetime and up to the present date he is acknowledged as a socialist, indeed one of the greatest in French (if not European) history. Similarly we find Rudolf Rocker writing that the Individualist Anarchists {\em “all agree on the point that man be given the full reward of his labour and recognised in this right the economic basis of all personal liberty. They regard free competition \unknown{} as something inherent in human nature \unknown{} They answered the {\bf socialists of other schools} [emphasis added] who saw in {\bf free competition} one of the destructive elements of capitalistic society that the evil lies in the fact that today we have too little rather than too much competition.”} [quoted by Herbert Read, {\bf A One-Man Manifesto}, p. 147] Rocker obviously considered support for free markets as compatible with socialism. In other words, Caplan’s assumption that all socialists oppose free markets, competition and so on is simply false — as can be seen from the history of the socialist movement. What socialists {\bf do} oppose is capitalist exploitation — socialism {\em “in its wide, generic, and true sense”} was an {\em “effort to {\bf abolish} the exploitation of labour by capital.”} [Peter Kropotkin, {\bf Kropotkin’s Revolutionary Pamphlets}, p. 169] In this sense the Individualist Anarchists are obviously socialists, as Tucker and Labadie constantly pointed out.


In addition, as we have proved elsewhere, Tucker was opposed to capitalist private property just as much as Kropotkin was. Moreover, it is clear from Tucker’s works that he considered himself an enemy of the capitalist system and called himself a socialist. Thus Caplan’s attempt to judge the historical argument on its own merits fails because he has to rewrite history to do so.


Caplan is right to state that the meaning of words change over time, but this does not mean we should run to use dictionary definitions. Dictionaries rarely express political ideas well — for example, most dictionaries define the word “anarchy” as “chaos” and “disorder.” Does that mean anarchists aim to create chaos? Of course not. Therefore, Caplan’s attempt to use dictionary definitions is selective and ultimately useless — anarchism as a political movement cannot be expressed by dictionary definitions and any attempt to do so means to ignore history.


The problems in using dictionary definitions to describe political ideas can best be seen from the definition of the word “Socialism.” According to the {\bf Oxford Study Dictionary} Socialism is {\em “a political and economic theory advocating that land, resources, and the chief industries should be owned and managed by the State.”} The {\bf Webster’s Ninth New Collegiate Dictionary}, conversely, defines socialism as {\em “any of various economic and political theories advocating collective or government ownership and administration of the means of production and distribution of goods.”}


Clearly the latter source has a more accurate definition of socialism than the former, by allowing for “collective” versus solely “State” control of productive means. Which definition would be better? It depends on the person involved. A Marxist, for example, could prefer the first one simply to exclude anarchism from the socialist movement, something they have continually tried to do. A right-libertarian could, again, prefer the first, for obvious reasons. Anarchists would prefer the second, again for obvious reasons. However neither definition does justice to the wide range of ideas that have described themselves as socialist.


Using dictionaries as the basis of defining political movements ensures that one’s views depend on {\bf which} dictionary one uses, and {\bf when} it was written, and so on. This is why they are not the best means of resolving disputes — if resolution of disputes is, in fact, your goal.


Both Kropotkin and Tucker stated that they were socialists and that anarchism was socialistic. If we take the common modern meaning of the word as state ownership as the valid one then Tucker and Kropotkin are {\bf not} socialists and no form of anarchism is socialist. This is obviously nonsense and it shows the limitations of using dictionary definitions on political theories.


Therefore Caplan’s attempt to justify using the dictionary definition fails. Firstly, because the definitions used would depend which dictionary you use. Secondly, dictionary definitions cannot capture the ins and outs of a {\bf political} theory or its ideas on wider subjects.


Ironically enough, Caplan is repeating an attempt made by State Socialists to deny Individualist Anarchism its socialist title (see {\em “Socialism and the Lexicographers”} in {\bf Instead of a Book}). In reply to this attempt, Tucker noted that:



\startblockquote
{\em “The makers of dictionaries are dependent upon specialists for their definitions. A specialist’s definition may be true or it may be erroneous. But its truth cannot be increased or its error diminished by its acceptance by the lexicographer. Each definition must stand on its own merits.”} [{\bf Instead of a Book}, p. 369]



\stopblockquote
And Tucker provided many quotes from {\bf other} dictionaries to refute the attempt by the State Socialists to define Individualist Anarchism outside the Socialist movement. He also notes that any person trying such a method will {\em “find that the Anarchistic Socialists are not to be stripped of one half of their title by the mere dictum of the last lexicographer.”} [{\bf Op. Cit.}, p. 365]


Caplan should take note. His technique been tried before and it failed then and it will fail again for the same reasons.


As far as his case against the Historical Argument goes, this is equally as flawed. Caplan states that:



\startblockquote
{\em “Before the Protestant Reformation, the word ‘Christian,’ had referred almost entirely to Catholics (as well as adherents of the Orthodox Church) for about one thousand years. Does this reveal any linguistic confusion on the part of Lutherans, Calvinists, and so on, when they called themselves ‘Christians’? Of course not. It merely reveals that a word’s historical usage does not determine its meaning.”}



\stopblockquote
However, as analogies go this is pretty pathetic. Both the Protestants and Catholics followed the teachings of Christ but had different interpretations of it. As such they could both be considered Christians — followers of the Bible. In the case of anarchism, there are two main groupings — individualist and social. Both Tucker and Bakunin claimed to follow, apply and develop Proudhon’s ideas (and share his opposition to both state and capitalism) and so are part of the anarchist tradition.


The anarchist movement was based upon applying the core ideas of Proudhon (his anti-statism and socialism) and developing them in the same spirit, and these ideas find their roots in {\bf socialist} history and theory. For example, William Godwin was claimed as an anarchist after his death by the movement because of his opposition to both state and private property, something all anarchists oppose. Similarly, Max Stirner’s opposition to both state and capitalist property places him within the anarchist tradition.


Given that we find fascists and Nazis calling themselves “republicans,” “democrats,” even “liberals” it is worthwhile remembering that the names of political theories are defined not by who use them, but by the ideas associated with the name. In other words, a fascist cannot call themselves a “liberal” any more than a capitalist can call themselves an “anarchist.” To state, as Caplan does, that the historical usage of a word does not determine its meaning results in utter confusion and the end of meaningful political debate. If the historical usage of a name is meaningless will we soon see fascists as well as capitalists calling themselves anarchists? In other words, the label “anarcho-capitalism” is a misnomer, pure and simple, as {\bf all} anarchists have opposed capitalism as an authoritarian system based upon exploitation and wage slavery.


To ignore the historical usage of a word means to ignore what the movement that used that word stood for. Thus, if Caplan is correct, an organisation calling itself the “Libertarian National Socialist Party,” for example, can rightly call itself libertarian for {\em “a word’s historical usage does not determine its meaning.”} Given that right-libertarians in the USA have tried to steal the name “libertarian” from anarchists and anarchist influenced socialists, such a perspective on Caplan’s part makes perfect sense. How ironic that a movement that defends private property so strongly continually tries to steal names from other political tendencies.


Perhaps a better analogy for the conflict between anarchism and “anarcho”- capitalism would be between Satanists and Christians. Would we consider as Christian a Satanist grouping claiming to be Christian? A grouping that rejects everything that Christians believe but who like the name? Of course not. Neither would we consider as a right-libertarian someone who is against the free market or someone as a Marxist who supports capitalism. However, that is what Caplan and other “anarcho”-capitalists want us to do with anarchism.


Both social and individualist anarchists defined their ideas in terms of both political (abolition of the state) {\bf and} economic (abolition of exploitation) ideas. Kropotkin defined anarchism as {\em “the no-government form of socialism”} while Tucker insisted that anarchism was {\em “the abolition of the State and the abolition of usury.”} In this they followed Proudhon who stated that {\em “[w]e do not admit the government of man by man any more than the exploitation of man by man.”} [quoted by Peter Marshall, {\bf Demanding the Impossible}, p. 245]


In other words, a political movement’s economic ideas are just as much a part of its theories as their political ideas. Any attempt to consider one in isolation from the other kills what defines the theory and makes it unique. And, ultimately, any such attempt, is a lie:



\startblockquote
{\em “[classical liberalism] is in theory a kind of anarchy without socialism, and therefore simply a lie, for freedom is impossible without equality, and real anarchy cannot exist without solidarity, without socialism.”} [Errico Malatesta, {\bf Anarchy}, p. 46]



\stopblockquote
Therefore Caplan’s case against the Historical Argument also fails — “anarcho-capitalism” is a misnomer because anarchism has always, in all its forms, opposed capitalism. Denying and re-writing history is hardly a means of refuting the historical argument.


Caplan ends by stating:



\startblockquote
{\em “Let us designate anarchism (1) anarchism as you define it. Let us designate anarchism (2) anarchism as I and the American Heritage College Dictionary define it. This is a FAQ about anarchism (2).”}



\stopblockquote
Note that here we see again how the dictionary is a very poor foundation upon to base an argument. Again using {\bf Webster’s Ninth New Collegiate Dictionary}, we find under “anarchist” — {\em “one who rebels against any authority, established order, or ruling power.”} This definition is very close to that which “traditional” anarchists have — which is the basis for our own opposition to the notion that anarchism is merely rebellion against {\bf State} authority.


Clearly this definition is at odds with Caplan’s own view; is Webster’s then wrong, and Caplan’s view right? Which view is backed by the theory and history of the movement? Surely that should be the basis of who is part of the anarchist tradition and movement and who is not? Rather than do this, Caplan and other “anarcho”-capitalists rush to the dictionary (well, those that do not define anarchy as “disorder”). This is for a reason as anarchism as a political movement as always been explicitly anti-capitalist and so the term “anarcho”-capitalism is an oxymoron.


What Caplan fails to even comprehend is that his choices are false. Anarchism can be designated in two ways:



\startitemize[N]\relax
\item[] Anarchism as you define it




 \item[] Anarchism as the anarchist movement defines it and finds expression in the theories developed by that movement.




 
\stopitemize
Caplan chooses anarchism (1) and so denies the whole history of the anarchist movement. Anarchism is not a word, it is a political theory with a long history which dictionaries cannot cover. Therefore any attempt to define anarchism by such means is deeply flawed and ultimately fails.


That Caplan’s position is ultimately false can be seen from the “anarcho”-capitalists themselves. In many dictionaries anarchy is defined as {\em “disorder,”} {\em “a state of lawlessness”} and so on. Strangely enough, no “anarcho”-capitalist ever uses {\bf these} dictionary definitions of “anarchy”! Thus appeals to dictionaries are just as much a case of defining anarchism as you desire as not using dictionaries. Far better to look at the history and traditions of the anarchist movement itself, seek out its common features and apply {\bf those} as criteria to those seeking to include themselves in the movement. As can be seen, “anarcho”-capitalism fails this test and, therefore, are not part of the anarchist movement. Far better for us all if they pick a new label to call themselves rather than steal our name.


Although most anarchists disagree on many things, the denial of our history is not one of them.


\chapter{ Replies to Some Errors and Distortions in Bryan Caplan’s “Anarchist Theory FAQ” version 4.1.1.
}

There have been a few “anarchist” FAQ’s produced before. Bryan Caplan’s anarchism FAQ is one of the more recent. While appearing to be a “neutral” statement of anarchist ideas, it is actually in large part an “anarcho”-capitalist FAQ. This can be seen by the fact that anarchist ideas (which he calls “left-anarchist”) are given less than half the available space while “anarcho”-capitalist dogma makes up the majority of it. Considering that anarchism has been around far longer than “anarcho”-capitalism and is the bigger and better established movement, this is surprising. Even his use of the term “left anarchist” is strange as it is never used by anarchists and ignores the fact that Individualist Anarchists like Tucker called themselves “socialists” and considered themselves part of the wider socialist movement. For anarchists, the expression “left anarchist” is meaningless as all anarchists are anti-capitalist. Thus the terms used to describe each “school” in his FAQ are biased (those whom Caplan calls “Left anarchists” do not use that term, usually preferring “social anarchist” to distinguish themselves from individualist anarchists like Tucker).


Caplan also frames the debate only around issues which he is comfortable with. For example, when discussing “left anarchist” ideas he states that {\em “A key value in this line of anarchist thought is egalitarianism, the view that inequalities, especially of wealth and power, are undesirable, immoral, and socially contingent.”} This, however, is {\bf not} why anarchists are egalitarians. Anarchists oppose inequalities because they undermine and restrict individual and social freedom.


Taking another example, under the question, {\em “How would left-anarchy work?”}, Caplan fails to spell out some of the really obvious forms of anarchist thought. For example, the works of Bookchin, Kropotkin, Bakunin and Proudhon are not discussed in any detail. His vague and confusing prose would seem to reflect the amount of thought that he has put into it. Being an “anarcho”-capitalist, Caplan concentrates on the economic aspect of anarchism and ignores its communal side. The economic aspect of anarchism he discusses is anarcho-syndicalism and tries to contrast the confederated economic system explained by one anarcho-syndicalist with Bakunin’s opposition to Marxism. Unfortunately for Caplan, Bakunin is the source of anarcho-syndicalism’s ideas on a confederation of self-managed workplaces running the economy. Therefore, to state that {\em “many”} anarchists {\em “have been very sceptical of setting up any overall political structure, even a democratic one, and focused instead on direct worker control at the factory level”} is simply {\bf false}. The idea of direct local control within a confederated whole is a common thread through anarchist theory and activity, as any anarchist could tell you.


Lastly, we must note that after Caplan posted his FAQ to the “anarchy-list,” many of the anarchists on that list presented numerous critiques of the “anarcho”-capitalist theories and of the ideas (falsely) attributed to social anarchists in the FAQ, which he chose to ignore (that he was aware of these postings is asserted by the fact he e-mailed one of the authors of this FAQ on the issue that anarchists never used or use the term “left-anarchist” to describe social anarchism. He replied by arguing that the term “left-anarchist” had been used by Michel Foucault, who never claimed to be an anarchist, in one of his private letters! Strangely, he never posted his FAQ to the list again).


Therefore, as can be seen from these few examples, Caplan’s “FAQ” is blatantly biased towards “anarcho-capitalism” and based on the mis-characterisations and the dis-emphasis on some of the most important issues between “anarcho-capitalists” and anarchists. It is clear that his viewpoint is anything but impartial.


This section will highlight some of the many errors and distortions in that FAQ. Numbers in square brackets refer to the corresponding sections Caplan’s FAQ.


\section{1 Is anarchism purely negative?
}

[1.] Caplan, consulting his {\bf American Heritage Dictionary}, claims: {\em “Anarchism is a negative; it holds that one thing, namely government, is bad and should be abolished. Aside from this defining tenet, it would be difficult to list any belief that all anarchists hold.”}


The last sentence is ridiculous. If we look at the works of Tucker, Kropotkin, Proudhon and Bakunin (for example) we discover that we can, indeed list one more {\em “belief that all anarchists hold.”} This is opposition to exploitation, to usury (i.e. profits, interest and rent). For example, Tucker argued that {\em “Liberty insists\unknown{} [on] the abolition of the State and the abolition of usury; on no more government of man by man, and no more exploitation of man by man.”} [cited in {\bf Native American Anarchism — A Study of Left-Wing American Individualism} by Eunice Schuster, p. 140] Such a position is one that Proudhon, Bakunin and Kropotkin would agree with.


In other words, anarchists hold two beliefs — opposition to government {\bf and} opposition to exploitation. Any person which rejects either of these positions cannot be part of the anarchist movement. In other words, an anarchist must be against capitalism in order to be a true anarchist.


Moreover it is not at all difficult to find a more fundamental {\em “defining tenet”} of anarchism. We can do so merely by analysing the term {\em “an-archy,”} which is composed of the Greek words {\bf an}, meaning {\em “no”} or {\em “without,”} and {\bf arche}, meaning literally {\em “a ruler,”} but more generally referring to the {\bf principle} of rulership, i.e. hierarchical authority. Hence an anarchist is someone who advocates abolishing the principle of hierarchical authority — not just in government but in all institutions and social relations.


Anarchists oppose the principle of hierarchical authority because it is the basis of domination, which is not only degrading in itself but generally leads to exploitation and all the social evils which follow from exploitation, from poverty, hunger and homelessness to class struggle and armed conflict.


Because anarchists oppose hierarchical authority, domination, and exploitation, they naturally seek to eliminate all hierarchies, as the very purpose of hierarchy is to facilitate the domination and (usually) exploitation of subordinates.


The reason anarchists oppose government, then, is because government is {\bf one manifestation} of the evils of hierarchical authority, domination, and exploitation. But the capitalist workplace is another. In fact, the capitalist workplace is where most people have their most frequent and unpleasant encounters with these evils. Hence workers’ control — the elimination of the hierarchical workplace through democratic self-management — has been central to the agenda of classical and contemporary anarchism from the 19\high{th} century to the present. Indeed, anarchism was born out of the struggle of workers against capitalist exploitation.


To accept Caplan’s definition of anarchism, however, would mean that anarchists’ historical struggle for workers’ self-management has never been a “genuine” anarchist activity. This is clearly a {\bf reductio ad absurdum} of that definition.


Caplan has confused a necessary condition with a sufficient condition. Opposition to government is a necessary condition of anarchism, but not a sufficient one. To put it differently, all anarchists oppose government, but opposition to government does not automatically make one an anarchist. To be an anarchist one must oppose government for anarchist reasons and be opposed to all other forms of hierarchical structure.


To understand why let use look to capitalist property. Murray Rothbard argues that {\em “[o]bviously, in a free society, Smith has the ultimate decision-making power over his own just property, Jones over his, etc.”} [{\bf The Ethics of Liberty}, p. 173] Defence firms would be employed to enforce those decisions (i.e. laws and rules). No real disagreement there. What {\bf is} illuminating is Rothbard’s comments that the state {\em “arrogates to itself a monopoly of force, of ultimate decision-making power, over a given area territorial area”} [{\bf Op. Cit.} , p. 170] Which, to state the obvious, means that both the state and property is marked by an {\em “ultimate decision-making power”} over their territory. The only “difference” is that Rothbard claims the former is “just” (i.e. “justly” acquired) and the latter is “unjust” (i.e. acquired by force). In reality of course, the modern distribution of property is just as much a product of past force as is the modern state. In other words, the current property owners have acquired their property in the same unjust fashion as the state has its. If one is valid, so is the other. Rothbard (and “anarcho”-capitalists in general) are trying to have it both ways.


Rothbard goes on to show why statism and private property are essentially the same thing:



\startblockquote
{\em “{\bf If} the State may be said too properly {\bf own} its territory, then it is proper for it to make rules for everyone who presumes to live in that area. It can legitimately seize or control private property because there {\bf is} no private property in its area, because it really owns the entire land surface. {\bf So long} as the State permits its subjects to leave its territory, then, it can be said to act as does any other owner who sets down rules for people living on his property.”} [{\bf Op. Cit.}, p. 170]



\stopblockquote
Of course Rothbard does not draw the obvious conclusion. He wants to maintain that the state is bad and property is good while drawing attention to their obvious similarities! Ultimately Rothbard is exposing the bankruptcy of his own politics and analysis. According to Rothbard, something can look like a state (i.e. have the {\em “ultimate decision-making power”} over an area) and act like a state (i.e. {\em “make rules for everyone”} who lives in an area, i.e. govern them) but not be a state. This not a viable position for obvious reasons.


In capitalism, property and possession are opposites — as Proudhon argued in {\bf What is Property?}. Under possession, the “property” owner exercises {\em “ultimate decision-making power”} over themselves as no-one else uses the resource in question. This is non-hierarchical. Under capitalism, however, use and ownership are divided. Landlords and capitalists give others access to their property while retaining power over it and so the people who use it. This is by nature hierarchical. Little wonder Noam Chomsky argued that a {\em “consistent anarchist must oppose private ownership of the means of production and the wage slavery which is a component of this system as incompatible with the principle that labour must be freely undertaken and under the control of the producer.”} [{\em “Notes on Anarchism”}, {\bf For Reasons of State}, p. 158]


Thus a true anarchist must oppose both state and capitalism as they generate the same hierarchical social relationships (as recognised by Rothbard but apparently subjected to “doublethink”). As “anarcho”-capitalists do not oppose capitalist property they cannot be anarchists — they support a very specific form of {\bf archy,} that of the capitalist/landlord over working class people.


Self-styled “anarcho”-capitalists do not oppose government for anarchist reasons. That is, they oppose it not because it is a manifestation of hierarchical authority, but because government authority often {\bf conflicts} with capitalists’ authority over the enterprises they control. By getting rid of government with its minimum wage laws, health and safety requirements, union rights laws, environmental standards, child labour laws, and other inconveniences, capitalists would have even more power to exploit workers than they already do. These consequences of “anarcho”-capitalism are diametrically opposed to the historically central objective of the anarchist movement, which is to eliminate capitalist exploitation.


We must conclude, then, that “anarcho”-capitalists are not anarchists at all. In reality they are capitalists {\bf posing} as anarchists in order to attract support for their laissez-faire economic project from those who are angry at government. This scam is only possible on the basis of the misunderstanding perpetrated by Caplan: that anarchism means nothing more than opposition to government.


Better definitions of anarchism can be found in other reference works. For example, in {\bf Grollier’s Online Encyclopedia} we read: {\em “Anarchism rejects all forms of hierarchical authority, social and economic as well as political.”} According to this more historically and etymologically accurate definition, “anarcho”-capitalism is not a form of anarchism, since it does not reject hierarchical authority in the economic sphere (which has been the area of prime concern to anarchists since day one). Hence it is {\bf bogus} anarchism.


\section{2 Anarchism and Equality
}

[5.] On the question “What major subdivisions may be made among anarchists?” Caplan writes:



\startblockquote
{\em “Unlike the left-anarchists, anarcho-capitalists generally place little or no value on equality, believing that inequalities along all dimensions — including income and wealth — are not only perfectly legitimate so long as they ‘come about in the right way,’ but are the natural consequence of human freedom.”}



\stopblockquote
This statement is not inaccurate as a characterisation of “anarcho”-capitalist ideas, but its implications need to be made clear. “Anarcho”-capitalists generally place little or no value on equality — particularly economic equality — because they know that under their system, where capitalists would be completely free to exploit workers to the hilt, wealth and income inequalities would become even greater than they are now. Thus their references to “human freedom” as the way in which such inequalities would allegedly come about means “freedom of capitalists to exploit workers;” it does not mean “freedom of workers {\bf from} capitalist exploitation.”


But “freedom to exploit workers” has historically been the objective only of capitalists, not anarchists. Therefore, “anarcho”-capitalism again shows itself to be nothing more than capitalism attempting to pass itself off as part of the anarchist movement — a movement that has been dedicated since its inception to the destruction of capitalism! One would have to look hard to find a more audacious fraud.


As we argue in section 2.1 of the appendix “Is ‘anarcho’-capitalism a type of anarchism?” the claim that inequalities are irrelevant if they {\em “come about the right way”} ignores the reality of freedom and what is required to be free. To see way we have to repeat part of our argument from that section and look at Murray Rothbard’s (a leading “anarcho”-capitalist icon) analysis of the situation after the abolition of serfdom in Russia and slavery in America. He writes:



\startblockquote
{\em “The {\bf bodies} of the oppressed were freed, but the property which they had worked and eminently deserved to own, remained in the hands of their former oppressors. With economic power thus remaining in their hands, the former lords soon found themselves virtual masters once more of what were now free tenants or farm labourers. The serfs and slaves had tasted freedom, but had been cruelly derived of its fruits.”} [{\bf The Ethics of Liberty}, p. 74]



\stopblockquote
However, contrast this with Rothbard’s (and Caplan’s) claims that if market forces (“voluntary exchanges”) result in the creation of free tenants or wage-labourers then these labourers and tenants are free (see, for example, {\bf The Ethics of Liberty}, pp. 221–2 on why “economic power” within capitalism does not, in fact, exist). But the labourers dispossessed by market forces are in {\bf exactly} the same situation as the former serfs and slaves. Rothbard sees the obvious “economic power” in the later case, but denies it in the former. But the {\bf conditions} of the people in question are identical and it is these conditions that horrify us and create social relationships because on subordination, authority and oppression rather than freedom. It is only ideology that stops Rothbard and Caplan drawing the obvious conclusion — identical conditions produce identical social relationships and so if the formally “free” ex-serfs are subject to “economic power” and “masters” then so are the formally “free” labourers within capitalism! Both sets of workers may be formally free, but their circumstances are such that they are “free” to “consent” to sell their freedom to others (i.e. economic power produces relationships of domination and unfreedom between formally free individuals).


Thus inequalities that {\em “come about in the right way”} restrict freedom just as much as inequalities that do not. If the latter restricts liberty and generate oppressive and exploitative social relationships then so do the former. Thus, if we are serious about individuality liberty (rather than property) we must look at inequalities and what generate them.


One last thing. Caplan states that inequalities in capitalism are {\em “the natural consequence of human freedom.”} They are not, unless you subscribe to the idea that capitalist property rights are the basis of human freedom. However, the assumption that capitalist property rights are the best means to defend individual liberty can be easily seen to be flawed just from the example of the ex-slaves and ex-serfs we have just described. Inequalities resulting from “voluntary exchanges” in the capitalist market can and do result in the denial of freedom, thus suggesting that “property” and liberty are not natural consequences of each other.


To state the obvious, private property (rather than possession) means that the non-property owner can gain access to the resource in question only when they agree to submit to the property owner’s authority (and pay tribute for the privilege of being bossed about). This aspect of property (rightly called {\em “despotism”} by Proudhon) is one which right-libertarians continually fail to highlight when they defend it as the paradigm of liberty.


\section{3 Is anarchism the same thing as socialism?
}

[7.] In this section (“Is anarchism the same thing as socialism?”) Caplan writes:



\startblockquote
{\em “Outside of the Anglo-American political culture, there has been a long and close historical relationship between the more orthodox socialists who advocate a socialist government, and the anarchist socialists who desire some sort of decentralised, voluntary socialism. The two groups both want to severely limit or abolish private property\unknown{}”}



\stopblockquote
For Caplan to claim that anarchism is not the same thing as socialism, he has to ignore anarchist history. For example, the Individualist anarchists called themselves {\em “socialists,”} as did social anarchists. Indeed, Individualist Anarchists like Joseph Labadie stated that {\em “Anarchism is voluntary socialism”} [{\bf Anarchism: What it is and What it is Not}) and wanted to limit private property in many ways (for example, {\em “the resources of nature — land, mines, and so forth — should not be held as private property and subject to being held by the individual for speculative purposes, that use of these things shall be the only valid title, and that each person has an equal right to the use of all these things.”} [{\bf What is Socialism?}]). Therefore, {\bf within} the {\em “Anglo-American political culture,”} {\bf all} types of anarchists considered themselves part of the socialist movement. This can be seen not only from Kropotkin’s or Bakunin’s work, but also in Tucker’s (see {\bf Instead of a Book}). So to claim that the {\em “Anglo-American”} anarchists did not have {\em “a long and close historical relationship”} with the wider socialist movement is simply {\bf false.}


The statement that anarchists want to severely limit or abolish “private property” is misleading if it is not further explained. For the way it stands, it sounds like anarchism is just another form of coercive “state” (i.e. a political entity that forcibly prevents people from owning private property), whereas this is far from the case.


Firstly, anarchists are {\bf not} against “private property” in the sense personal belongings. {\em “Anarchists,”} points out Nicholas Walter, {\em “are in favour of the private property which cannot be used by one person to exploit another — those personal possessions which we accumulate from childhood and which become part of ours.”} [{\em “About Anarchism”}, in {\bf Reinventing Anarchy}, p. 49] Kropotkin makes the anarchist position clear when he wrote that we {\em “do not want to rob any one of his coat”} but expropriation {\em “must apply to everything that enables any man [or woman] — by he financier, mill owner, or landlord — to appropriate the product of others’ toil.”} [{\bf The Conquest of Bread}, p. 61]


In effect, Caplan is confusing two very different kinds of “private property”, of which one rests on usefulness to an individual, the other on the employment (and so exploitation) of the labour of others. The latter produces social relations of domination between individuals, while the former is a relationship between people and things. As Proudhon argued, possession becomes property only when it also serves as means of exploitation and subjection of other people. But failing to distinguish these radically different forms of “private property” Caplan distorts the anarchist position.


Secondly, it is not that anarchists want to pass laws making private property (in the second, exploitative, sense) illegal. Rather they want to restructure society in such a way that the means of production are freely available for workers to use. This does not mean “anarchist police” standing around with guns to prohibit people from owning private property. Rather, it means dismantling the coercive state agencies that make private property possible, i.e., the departments of real police who now stand around with guns protecting private property.


Once that occurs, anarchists maintain that capitalism would be impossible, since capitalism is essentially a monopoly of the means of production, which can only be maintained by organised coercion. For suppose that in an anarchist society someone (call him Bob) somehow acquires certain machinery needed to produce widgets (a doubtful supposition if widget-making machines are very expensive, as there will be little wealth disparity in an anarchist society). And suppose Bob offers to let workers with widget-making skills use his machines if they will pay him “rent,” i.e. allow him to appropriate a certain amount of the value embodied in the widgets they produce. The workers will simply refuse, choosing instead to join a widget-making collective where they have free access to widget-making machinery, thus preventing Bob from living parasitically on their labour. Thus Kropotkin:



\startblockquote
{\em “Everywhere you will find that the wealth of the wealthy springs from the poverty of the poor. That is why an anarchist society need not fear the advent of a Rothschild [or any other millionaire] who would settle in its midst. If every member of the community knows that after a few hours of productive toil he [or she] will have a right to all the pleasures that civilisation procures, and to those deeper sources of enjoyment which art and science offer to all who seek them, he [or she] will not sell his strength\unknown{} No one will volunteer to work for the enrichment of your Rothschild.”} [{\bf Op. Cit.}, p. 61]



\stopblockquote
In this scenario, private property was “abolished,” but not through coercion. Indeed, it was precisely the abolition of organised coercion that allowed private property to be abolished.


\section{4 Anarchism and dissidents
}

[9.] On the question “How would left-anarchy work?” Caplan writes:



\startblockquote
{\em “Some other crucial features of the left-anarchist society are quite unclear. Whether dissidents who despised all forms of communal living would be permitted to set up their own inegalitarian separatist societies is rarely touched upon. Occasionally left-anarchists have insisted that small farmers and the like would not be forcibly collectivised, but the limits of the right to refuse to adopt an egalitarian way of life are rarely specified.”}



\stopblockquote
This is a straw man. “Left” (i.e. real) anarchist theory clearly implies and {\bf explicitly states} the answer to these questions.


Firstly, on the issue of “separatist” societies. Anarchist thinkers have always acknowledged that there would be multitude of different communities after a revolution (and not just Caplan’s “inegalitarian” ones). Marx, for example, mocked Bakunin for arguing that only revolutionary communes would federate together and that this would not claim any right to govern others (see Bakunin’s {\em “Letter to Albert Richards”}, {\bf Michael Bakunin: Selected Writings}, p. 179] Kropotkin stated that {\em “the point attained in the socialisation of wealth will not be everywhere the same”} and {\em “[s]ide by side with the revolutionised communes \unknown{} places would remain in an expectant attitude, and would go on living on the Individualist system.”} [{\bf The Conquest of Bread}, p. 81] While he was hopeful that {\em “everywhere [would be] more or less Socialism”} he recognised that the revolution would not conform to {\em “any particular rule”} and would differ in different areas — {\em “in one country State Socialist, in another Federation”} and so on. [{\bf Op. Cit.}, p. 82] Malatesta made the same point, arguing that {\em “after the revolution”} there would be {\em “relations between anarchist groupings and those living under some kind of authority, between communist collectives and those living in an individualistic way.”} This is because anarchism {\em “cannot be imposed”}. [{\bf Life and Ideas}, p. 173, p. 21]


Needless to say, these “separatist societies” (which may or may not be “inegalitarian”) would not be anarchist societies. If a group of people wanted to set up a capitalist, Marxist, Georgist or whatever kind of community then their right would be respected (although, of course, anarchists would seek to convince those who live in such a regime of the benefits of anarchism!). As Malatesta pointed out, {\em “free and voluntary communism is ironical if one has not the right and the possibility to live in a different regime, collectivist, mutualist, individualist — as one wishes, always on condition that there is no oppression or exploitation of other”} as {\em “it is clear that all, and only, those ways of life which respect freedom, and recognise that each individual has an equal right to the means of production and to the full enjoyment of the product of his own labour, have anything in common with anarchism.”} [{\bf Op. Cit.}, p. 103 and p. 33]


Ultimately, {\em “it is not a question of right and wrong; it is a question of freedom for everybody\unknown{} None can judge with certainty who is right and who is wrong, who is nearest to the truth, or which is the best way to achieve the greatest good for each and everyone. Freedom coupled with experience, is the only way of discovering the truth and what is best; and there can be no freedom if there is the denial of the freedom to err.”} [{\bf Op. Cit.}, p. 49]


Secondly, regarding {\em “dissidents”} who wanted to set up their own {\em “inegalitarian separatist societies,”} if the term “inegalitarian” implies economic inequalities due to private property, the answer is that private property requires some kind of state, if not a public state then private security forces (“private-state capitalism”), as advocated by “anarcho”-capitalists, in order to protect private property. Therefore, “anarcho”-capitalists are asking if an anarchist society will allow the existence of states. Of course, in the territory that used to be claimed by a nation state a whole host of communities and societies will spring up — but that does not make the non-anarchist ones anarchist!


Thus suppose that in a hypothetical libertarian socialist society, Bob tries to set up private security forces to protect certain means of production, e.g. farmland. By the hypothesis, if Bob merely wanted to work the land himself, there would be no reason for him go to the trouble of creating a private state to guard it, because use-rights guarantee that he has free access to the productive assets he needs to make a living. Thus, the only plausible reason Bob could have for claiming and guarding more farmland than he could use himself would be a desire to create a monopoly of land in order to exact tribute from others for the privilege of using it. But this would be an attempt to set up a system of feudal exploitation in the midst of a free community. Thus the community is justified in disarming this would-be parasite and ignoring his claims to “own” more land than he can use himself.


In other words, there is no “right” to adopt an “inegalitarian way of life” within a libertarian community, since such a right would have to be enforced by the creation of a coercive system of enslavement, which would mean the end of the “libertarian” community. To the contrary, the members of such a community have a right, guaranteed by “the people in arms,” to resist such attempts to enslave them.


The statement that “left” anarchists have {\em “occasionally”} insisted that small farmers and the like would not be forcibly collectivised is a distortion of the facts. No responsible left libertarian advocates forced collectivisation, i.e. compelling others to join collectives. Self-employment is always an option. This can be seen from Bakunin’s works [{\bf Bakunin on Anarchism}, p. 200], Kropotkin’s [{\bf The Conquest of Bread}, p. 61 and {\bf Act for Yourselves}, pp. 104–5] and Malatesta’s [{\bf Life and Ideas}, p. 99, p. 103]. So the anarchist opposition to forced collectivisation has always existed and, for anyone familiar with the ideas of social anarchism, very well know. Thus during the Spanish Revolution, small farmers who did not wish to join collective farms were allowed to keep as much land as they could work themselves. After perceiving the advantages of collectives, however, many joined them voluntarily (see Sam Dolgoff, ed., {\bf The Anarchist Collectives}).


To claim that social anarchists {\em “occasionally”} oppose forced collectivisation is a smear, pure and simple, with little basis in anarchist activity and even less in anarchist theory. Anyone remotely familiar with the literature could not make such a mistake.


Finally, we should point out that under “anarcho”-capitalism there would be, according to Murray Rothbard, a {\em “basic libertarian law code.”} Which means that under “anarcho”-capitalism, “egalitarian” communities could only come about within a “inegalitarian” legal framework! Thus, given that everything would be privatised, dissenters could only experiment if they could afford it {\bf and} accepted the legal system based on capitalist property rights (and, of course, survive the competition of capitalist companies within the capitalist framework). As we have argued in sections B.4 and F.3


\section{5 How would anarcho-capitalism work?
}

[10.] This section (How would anarcho-capitalism work?) contains Caplan’s summary of arguments for “anarcho”-capitalism, which he describes as an offshoot of Libertarianism. Thus:



\startblockquote
{\em “So-called ‘minarchist’ libertarians such as Nozick have argued that the largest justified government was one which was limited to the protection of individuals and their private property against physical invasion; accordingly, they favour a government limited to supplying police, courts, a legal code, and national defence.”}



\stopblockquote
The first thing to note about this argument is that it is stated in such a way as to prejudice the reader against the left-libertarian critique of private property. The minarchist right-“libertarian,” it is said, only wants to protect individuals and their private property against “physical invasion.” But, because of the loose way in which the term “property” is generally used, the “private property” of most “individuals” is commonly thought of as {\bf personal possessions,} i.e. cars, houses, clothing, etc. (For the left-libertarian distinction between private property and possessions, see section B.3.1.) Therefore the argument makes it appear that right libertarians are in favour of protecting personal possessions whereas left-libertarians are not, thus conjuring up a world where, for example, there would be no protection against one’s house being “physically invaded” by an intruder or a stranger stealing the shirt off one’s back!


By lumping the protection of “individuals” together with the protection of their “private property,” the argument implies that right libertarians are concerned with the welfare of the vast majority of the population, whereas in reality, the vast majority of “individuals” {\bf do not own} any private property (i.e. means of production) — only a handful of capitalists do. Moreover, these capitalists use their private property to exploit the working class, leading to impoverishment, alienation, etc., and thus {\bf damaging} most individuals rather than “protecting” them.


Caplan goes on:



\startblockquote
{\em “This normative theory is closely linked to laissez-faire economic theory, according to which private property and unregulated competition generally lead to both an efficient allocation of resources and (more importantly) a high rate of economic progress.”}



\stopblockquote
Caplan does not mention the obvious problems with this “theory,” e.g. that during the heyday of laissez-faire capitalism in the US there was vast wealth disparity, with an enormous mass of impoverished people living in slums in the major cities — hardly an “efficient” allocation of resources or an example of “progress.” Of course, if one defines “efficiency” as “the most effective means of exploiting the working class” and “progress” as “a high rate of profit for investors,” then the conclusion of the “theory” does indeed follow.


And let us not forget that it is general equilibrium theory which predicts that unregulated competition will produce an efficient allocation of resources. However, as we noted in section C.1, such a model has little to do with any real economy. This means that there is no real reason to assume an efficient outcome of capitalist economies. Concentrations of economic power and wealth can easily skew outcomes to favour the haves over the have-nots (as history again and again shows).


Moreover, the capitalism can easily lead to resources being allocated to the most profitable uses rather than those which are most needed by individuals. A classic example is in the case of famines. Amartya Sen (who won the 1998 Nobel Prize for economics) developed an {\bf {\em “entitlement”}} approach to the study of famine. This approach starts with the insight that having food available in a country or region does not mean everyone living there is “entitled” to it. In market economies, people are entitled to food according to their ability to produce it for themselves or to pay or swap for it. In capitalist economies, most people are entitled to food only if they can sell their labour/liberty to those who own the means of life (which increases the economic insecurity of wage workers).


If some group loses its entitlement to food, whether there is a decline in the available supply or not, a famine can occur. This may seem obvious, yet before — and after — Sen, famine studies have remained fixated on the drop in food available instead of whether specific social groups are entitled to it. Thus even a relatively success economy can price workers out of the food market (a depressed economy brings the contradiction between need and profit — use value and exchange value — even more to the forefront). This {\em “pricing out”} can occur especially if food can get higher prices (and so profits) elsewhere — for example the Irish famine of 1848 and sub-Saharan famines of the 1980s saw food being exported from famine areas to areas where it could fetch a higher price. In other words, market forces can skew resource allocation away from where it is most needed to where it can generate a profit. As anarchist George Barret noted decades before Sen:



\startblockquote
{\em “Today the scramble is to compete for the greatest profits. If there is more profit to be made in satisfying my lady’s passing whim than there is in feeding hungry children, then competition brings us in feverish haste to supply the former, whilst cold charity or the poor law can supply the latter, or leave it unsupplied, just as it feels disposed. That is how it works out.”} [{\bf Objectives to Anarchism}]



\stopblockquote
In other words, inequality skews resource allocation towards the wealthy. While such a situation may be {\em “efficient allocation of resources”} from the perspective of the capitalist, it is hardly so from a social perspective (i.e. one that considers {\bf all} individual needs rather than “effective demand”).


Furthermore, if we look at the stock market (a key aspect of any capitalist system) we discover a strong tendencies {\bf against} the efficient allocation of resources. The stock market often experiences “bubbles” and becomes significantly over-valued. An inflated stock market badly distorts investment decisions. For example, if Internet companies are wildly over-valued then the sale of shares of new Internet companies or the providing of start-up capital will drain away savings that could be more productively used elsewhere. The real economy will pay a heavy price from such misdirected investment and, more importantly, resources are {\bf not} efficiency allocated as the stock market skews resources into the apparently more profitable areas and away from where they could be used to satisfy other needs.


The stock market is also a source of other inefficiencies. Supporters of “free-market” capitalism always argued that the Stalinist system of central planning created a perverse set of incentives to managers. In effect, the system penalised honest managers and encouraged the flow of {\bf dis}-information. This lead to information being distorted and resources inefficiently allocated and wasted. Unfortunately the stock market also creates its own set of perverse responses and mis-information. Doug Henwood argues that {\em “something like a prisoners’ dilemma prevails in relations between managers and the stock market. Even if participants are aware of an upward bias to earnings estimates, and even if they correct for it, managers still have an incentive to try and fool the market. If you tell the truth, your accurate estimates will be marked down by a sceptical market. So its entirely rational for managers to boost profits in the short term, either through accounting gimmickry or by making only investments with quick paybacks.”} He goes on to note that {\em “[i]f the markets see high costs as bad, and low costs as good, then firms may shun expensive investments because they will be taken as signs of managerial incompetence. Throughout the late 1980s and early 1990s, the stock market rewarded firms announcing write-offs and mass firings — a bulimic strategy of management — since the cost cutting was seen as contributing rather quickly to profits. Firms and economies can’t get richer by starving themselves, but stock market investors can get richer when the companies they own go hungry. As for the long term, well, that’s someone else’s problem.”} [{\bf Wall Street}, p. 171]


This means that resources are allocated to short term projects, those that enrich the investors now rather than produce long term growth and benefits later. This results in slower and more unstable investment than less market centred economies, as well as greater instability over the business cycle [{\bf Op. Cit.}, pp. 174–5] Thus the claim that capitalism results in the “efficient” allocation of resources is only true if we assume “efficient” equals highest profits for capitalists. As Henwood summarises, {\em “the US financial system performs dismally at its advertised task, that of efficiently directing society’s savings towards their optimal investment pursuits. The system is stupefyingly expensive, gives terrible signals, and has surprisingly little to do with real investment.”} [{\bf Op. Cit.}, p. 3]


Moreover, the claim that laissez-faire economies produce a high rate of economic progress can be questioned on the empirical evidence available. For example, from the 1970s onwards there has been a strong tendency towards economic deregulation. However, this tendency has been associated with a {\bf slow down} of economic growth. For example, {\em “[g]rowth rates, investment rates and productivity rates are all lower now than in the [Keynesian post-war] Golden Age, and there is evidence that the trend rate of growth — the underlying growth rate — has also decreased.”} Before the Thatcher pro-market reforms, the British economy grew by 2.4\% in the 1970s. After Thatcher’s election in 1979, growth decreased to 2\% in the 1980s and to 1.2\% in the 1990s. In the USA, we find a similar pattern. Growth was 4.4\% in the 1960s, 3.2\% in the 1970s, 2.8\% in the 1980s and 1.9\% in the first half of the 1990s [Larry Elliot and Dan Atkinson, {\bf The Age of Insecurity}, p. 236]. Moreover, in terms of inflation-adjusted GDP per capita and productivity, the US had the worse performance out of the US, UK, Japan, Italy, France, Canada and Australia between 1970 and 1995 [Marc-Anfre Pigeon and L. Randall Wray, {\bf Demand Constraints and Economic Growth}]. Given that the US is usually considered the most laissez-faire out of these 7 countries, Caplan’s claim of high progress for deregulated systems seems at odds with this evidence.


As far as technological innovation goes, it is also not clear that deregulation has aided that process. Much of our modern technology owns its origins to the US Pentagon system, in which public money is provided to companies for military R\&D purposes. Once the technology has been proven viable, the companies involved can sell their public subsidised products for private profit. The computer industry (as we point out in section J.4.7) is a classic example of this — indeed it is unlikely whether we would have computers or the internet if we had waited for capitalists to development them. So whether a totally deregulated capitalism would have as high a rate of technological progress is a moot point.


So, it seems likely that it is only the {\bf assumption} that the free capitalist market will generate {\em “an efficient allocation of resources and (more importantly) a high rate of economic progress.”} Empirical evidence points the other way — namely, that state aided capitalism provides an approximation of these claims. Indeed, if we look at the example of the British Empire (which pursued a strong free trade and laissez-faire policy over the areas it had invaded) we can suggest that the opposite may be true. After 25 prosperous years of fast growth (3.5 per cent), after 1873 Britain had 40 years of slow growth (1.5 per cent), the last 14 years of which were the worse — with productivity declining, GDP stagnant and home investment halved. [Nicholas Kaldor, {\bf Further Essays on Applied Economics}, p. 239] In comparison, those countries which embraced protectionism (such as Germany and the USA) industrialised successfully and become competitors with the UK. Indeed, these new competitors grew in time to be efficient competitors of Britain not only in foreign markets but also in Britain’s home market. The result was that {\em “for fifty years Britain’s GDP grew very slowly relative to the more successful of the newer industrialised countries, who overtook her, one after another, in the volume of manufacturing production and in exports and finally in real income per head.”} [{\bf Op. Cit.}, p. xxvi] Indeed, {\em “America’s growth and productivity rates were higher when tariffs were steep than when they came down.”} [Larry Elliot and Dan Atkinson, {\bf Op. Cit.}, p. 277]


It is possible to explain almost everything that has ever happened in the world economy as evidence not of the failure of markets but rather of what happens when markets are not able to operate freely. Indeed, this is the right-libertarian position in a nut shell. However, it does seem strange that movements towards increased freedom for markets produce worse results than the old, more regulated, way. Similarly it seems strange that the country that embraced laissez-faire and free trade (Britain) did {\bf worse} than those which embraced protectionism (USA, Germany, etc.).


It could always be argued that the protectionist countries had embraced free trade their economies would have done even better. This is, of course, a possibility — if somewhat unlikely. After all, the argument for laissez-faire and free trade is that it benefits all parties, even if it is embraced unilaterally. That Britain obviously did not benefit suggests a flaw in the theory (and that no country {\bf has} industrialised without protectionism suggests likewise). Unfortunately, free-market capitalist economics lends itself to a mind frame that ensures that nothing could happen in the real world that would could ever change its supporters minds about anything.


Free trade, it could be argued, benefits only those who have established themselves in the market — that is, have market power. Thus Britain could initially benefit from free trade as it was the only industrialised nation (and even {\bf its} early industrialisation cannot be divorced from its initial mercantilist policies). This position of strength allowed them to dominate and destroy possible competitors (as Kaldor points out, {\em “[w]here the British succeeded in gaining free entry for its goods\unknown{} it had disastrous effects on local manufactures and employment.”} [{\bf Op. Cit.}, p. xxvi]). This would revert the other country back towards agriculture, an industry with diminishing returns to scale (manufacturing, in contrast, has increasing returns) and ensure Britain’s position of power.


The use of protection, however, sheltered the home industries of other countries and gave them the foothold required to compete with Britain. In addition, Britains continual adherence to free trade meant that a lot of {\bf new} industries (such as chemical and electrical ones) could not be properly established. This combination contributed to free trade leading to stunted growth, in stark contrast to the arguments of neo-classical economics.


Of course, we will be accused of supporting protectionism by recounting these facts. That is not the case, as protectionism is used as a means of “proletarianising” a nation (as we discuss in section F.8). Rather we are presenting evidence to refute a claim that deregulated capitalism will lead to higher growth. Thus, we suggest, the history of “actually existing” capitalism indicates that Caplan’s claim that deregulated capitalism will result {\em “a high rate of economic progress”} may be little more than an assumption. True, it is an assumption of neo-classical economics, but empirical evidence suggests that assumption is as unfounded as the rest of that theory.


Next we get to the meat of the defence of “anarcho”-capitalism:



\startblockquote
{\em “Now the anarcho-capitalist essentially turns the minarchist’s own logic against him, and asks why the remaining functions of the state could not be turned over to the free market. And so, the anarcho-capitalist imagines that police services could be sold by freely competitive firms; that a court system would emerge to peacefully arbitrate disputes between firms; and that a sensible legal code could be developed through custom, precedent, and contract.”}



\stopblockquote
Indeed, the functions in question could certainly be turned over to the “free” market, as was done in certain areas of the US during the 19\high{th} century, e.g. the coal towns that were virtually owned by private coal companies. We have already discussed the negative impact of that experiment on the working class in section F.6.2. Our objection is not that such privatisation cannot be done, but that it is an error to call it a form of anarchism. In reality it is an extreme form of laissez-faire capitalism, which is the exact opposite of anarchism. The defence of private power by private police is hardly a move towards the end of authority, nor are collections of private states an example of anarchism.


Indeed, that “anarcho”-capitalism does not desire the end of the state, just a change in its form, can be seen from Caplan’s own arguments. He states that {\em “the remaining functions of the state”} should be {\em “turned over to the free market.”} Thus the state (and its functions, primarily the defence of capitalist property rights) is {\bf privatised} and not, in fact, abolished. In effect, the “anarcho”-capitalist seeks to abolish the state by calling it something else.


Caplan:



\startblockquote
{\em “The anarcho-capitalist typically hails modern society’s increasing reliance on private security guards, gated communities, arbitration and mediation, and other demonstrations of the free market’s ability to supply the defensive and legal services normally assumed to be of necessity a government monopoly.”}



\stopblockquote
It is questionable that {\em “modern society”} {\bf as such} has increased its reliance on {\em “private security guards, gated communities”} and so on. Rather, it is the {\bf wealthy} who have increased their reliance on these forms of private defence. Indeed it is strange to hear a right-libertarian even use the term “society” as, according to that ideology, society does not exist! Perhaps the term “society” is used to hide the class nature of these developments? As for “gated communities” it is clear that their inhabitants would object if the rest of society gated themselves from them! But such is the logic of such developments — but the gated communities want it both ways. They seek to exclude the rest of society from their communities while expected to be given access to that society. Needless to say, Caplan fails to see that liberty for the rich can mean oppression for the working class — {\em “we who belong to the proletaire class, property excommunicates us!”} [Proudhon, {\bf What is Property?}, p. 105]


That the law code of the state is being defended by private companies is hardly a step towards anarchy. This indicates exactly why an “anarcho”- capitalist system will be a collection of private states united around a common, capitalistic, and hierarchical law code. In addition, this system does not abolish the monopoly of government over society represented by the {\em “general libertarian law code,”} nor the monopoly of power that owners have over their property and those who use it. The difference between public and private statism is that the boss can select which law enforcement agents will enforce his or her power.


The threat to freedom and justice for the working class is clear. The thug-like nature of many private security guards enforcing private power is well documented. For example, the beating of protesters by “private cops” is a common sight in anti-motorway campaigns or when animal right activists attempt to disrupt fox hunts. The shooting of strikers during strikes occurred during the peak period of American laissez-faire capitalism. However, as most forms of protest involve the violation of “absolute” property rights, the “justice” system under “anarcho”-capitalism would undoubtedly fine the victims of such attacks by private cops.


It is also interesting that the “anarcho”-capitalist “hails” what are actually symptoms of social breakdown under capitalism. With increasing wealth disparity, poverty, and chronic high unemployment, society is becoming polarised into those who can afford to live in secure, gated communities and those who cannot. The latter are increasingly marginalised in ghettos and poor neighbourhoods where drug-dealing, prostitution, and theft become main forms of livelihood, with gangs offering a feudalistic type of “protection” to those who join or pay tribute to them. Under “anarcho”-capitalism, the only change would be that drug-dealing and prostitution would be legalised and gangs could start calling themselves “defence companies.”


Caplan:



\startblockquote
{\em “In his ideal society, these market alternatives to government services would take over {\bf all} legitimate security services. One plausible market structure would involve individuals subscribing to one of a large number of competing police services; these police services would then set up contracts or networks for peacefully handling disputes between members of each others’ agencies. Alternately, police services might be ‘bundled’ with housing services, just as landlords often bundle water and power with rental housing, and gardening and security are today provided to residents in gated communities and apartment complexes.”}



\stopblockquote
This is a scenario designed with the upper classes in mind and a few working class people, i.e. those with {\bf some} property (for example, a house) — sometimes labelled the “middle class”. But under capitalism, the tendency toward capital concentration leads to increasing wealth polarisation, which means a shrinking “middle class” (i.e. working class with decent jobs and their own homes) and a growing “underclass” (i.e. working class people without a decent job). Ironically enough, America (with one of the most laissez-faire capitalist systems) is also the Western nation with the {\bf smallest} “middle class” and wealth concentration has steadily increased since the 1970s. Thus the number of people who could afford to buy protection and “justice” from the best companies would continually decrease. For this reason there would be a growing number of people at the mercy of the rich and powerful, particularly when it comes to matters concerning employment, which is the main way in which the poor would be victimised by the rich and powerful (as is indeed the case now).


Of course, if landlords {\bf do} “bundle” police services in their contracts this means that they are determining the monopoly of force over the property in question. Tenants would “consent” to the police force and the laws of the landlord in exactly the same way emigrants “consent” to the laws and government of, say, the USA when they move there. Rather than show the difference between statism and capitalism, Caplan has indicated their essential commonality. For the proletarian, property is but another form of state. For this reason anarchists would agree with Rousseau when he wrote that:



\startblockquote
{\em “That a rich and powerful man, having acquired immense possessions in lands, should impose laws on those who want to establish themselves there, and that he should only allow them to do so on condition that they accept his supreme authority and obey all his wishes; that, I can still conceive. But how can I conceive such a treaty, which presupposes anterior rights, could be the first foundation of law? Would not this tyrannical act contain a double usurpation: that on the ownership of the land and that on the liberty of the inhabitants?”} [{\bf The Social Contract and Discourses}, p. 316]



\stopblockquote
Caplan:



\startblockquote
{\em “The underlying idea is that contrary to popular belief, private police would have strong incentives to be peaceful and respect individual rights. For first of all, failure to peacefully arbitrate will yield to jointly destructive warfare, which will be bad for profits. Second, firms will want to develop long-term business relationships, and hence be willing to negotiate in good faith to insure their long-term profitability. And third, aggressive firms would be likely to attract only high-risk clients and thus suffer from extraordinarily high costs (a problem parallel to the well-known ‘adverse selection problem’ in e.g. medical insurance — the problem being that high-risk people are especially likely to seek insurance, which drives up the price when riskiness is hard for the insurer to discern or if regulation requires a uniform price regardless of risk).”}



\stopblockquote
The theory that {\em “failure to peacefully arbitrate will yield to jointly destructive warfare, which will be bad for profits”} can be faulted in two ways. Firstly, if warfare would be bad for profits, what is to stop a large “defence association” from ignoring a smaller one’s claim? If warfare were “bad for business,” it would be even worse for a small company without the capital to survive a conflict, which could give big “defence associations” the leverage to force compliance with their business interests. Price wars are often bad for business, but companies sometimes start them if they think they can win. Needless to say, demand would exist for such a service (unless you assume a transformation in the “human nature” generated by capitalism — an unlikely situation and one “anarcho”-capitalists usually deny is required for their system to work). Secondly — and this is equally, if not more, likely — a “balance of power” method to stop warfare has little to recommend it from history. This can be seen from the First World War and feudal society.


What the “anarcho”-capitalist is describing is essentially a system of “industrial feudalism” wherein people contract for “protection” with armed gangs of their choice. Feudal societies have never been known to be peaceful, even though war is always “unprofitable” for one side or the other or both. The argument fails to consider that “defence companies,” whether they be called police forces, paramilitaries or full-blown armies, tend to attract the “martial” type of authoritarian personality, and that this type of “macho” personality thrives on and finds its reason for existence in armed conflict and other forms of interpersonal violence and intimidation. Hence feudal society is continually wracked by battles between the forces of opposing warlords, because such conflicts allow the combatants a chance to “prove their manhood,” vent their aggression, obtain honours and titles, advance in the ranks, obtain spoils, etc. The “anarcho” capitalist has given no reason why warfare among legalised gangs would not continue under industrial feudalism, except the extremely lame reason that it would not be profitable — a reason that has never prevented war in any known feudal society.


It should be noted that the above is not an argument from “original sin.” Feudal societies are characterised by conflict between opposing “protection agencies” not because of the innate depravity of human beings but because of a social structure based on private property and hierarchy, which brings out the latent capacities for violence, domination, greed, etc. that humans have by creating a financial incentive to be so. But this is not to say that a different social structure would not bring out latent capacities for much different qualities like sharing, peaceableness, and co-operation, which human beings also have. In fact, as Kropotkin argued in {\bf Mutual Aid} and as recent anthropologists have confirmed in greater detail, ancient societies based on communal ownership of productive assets and little social hierarchy were basically peaceful, with no signs of warfare for thousands of years.


However, let us assume that such a competitive system does actually work as described. Caplan, in effect, argues that competition will generate co-operation. This is due to the nature of the market in question — defence (and so peace) is dependent on firms working together as the commodity “peace” cannot be supplied by one firm. However, this co-operation does not, for some reason, become {\bf collusion} between the firms in question. According to “anarcho”-capitalists this competitive system not only produces co-operation, it excludes “defence” firms making agreements to fix monopoly profits (i.e. co-operation that benefits the firms in question). Why does the market produce beneficial co-operation to everyone but not collusion for the firms in question? Collusion is when firms have “business relationships” and “negotiate in good faith” to insure their profitability by agreeing not to compete aggressively against each other in order to exploit the market. Obviously in “anarcho”-capitalism the firms in question only use their powers for good!


Needless to say, the “anarcho”-capitalist will object and argue that competition will ensure that collusion will not occur. However, given that co-operation is required between all firms in order to provide the commodity “peace” this places the “anarcho”-capitalist in a bind. As Caplan notes, “aggressive” firms are {\em “likely to attract only high-risk clients and thus suffer from extraordinarily high costs.”} From the perspective of the colluding firms, a new entry into their market is, by definition, aggressive. If the colluding firms do not co-operate with the new competitor, then it will suffer from {\em “extraordinarily high costs”} and either go out of business or join the co-operators. If the new entry could survive in the face of the colluding firms hostility then so could “bad” defence firms, ones that ignored the market standards.


So the “anarcho”-capitalist faces two options. Either an “aggressive” firm cannot survive or it can. If it cannot then the very reason why it cannot ensures that collusion is built into the market and while the system is peaceful it is based on an effective monopoly of colluding firms who charge monopoly profits. This, in effect, is a state under the “anarcho”-capitalist’s definition as a property owner cannot freely select their own “protection” — they are limited to the firms (and laws) provided by the co-operating firms. Or an “aggressive” firm can survive, violence is commonplace and chaos ensures.


Caplan’s passing reference to the {\em “adverse selection problem”} in medical insurance suggests another problem with “anarcho”-capitalism. The problem is that high-risk people are especially likely to seek protection, which drives up the price for, as “anarcho”-capitalists themselves note, areas with high crime levels “will be bad for profits,” as hardware and personnel costs will be correspondingly higher. This means that the price for “protection” in areas which need it most will be far higher than for areas which do not need it. As poor areas are generally more crime afflicted than rich areas, “anarcho”-capitalism may see vast sections of the population not able to afford “protection” (just as they may not be about to afford health care and other essential services). Indeed, “protection services” which try to provide cheap services to “high-risk” areas will be at an competitive disadvantage in relation to those who do not, as the “high-risk” areas will hurt profits and companies without “high-risk” “customers” could undercut those that have.


Caplan:



\startblockquote
{\em “Anarcho-capitalists generally give little credence to the view that their ‘private police agencies’ would be equivalent to today’s Mafia — the cost advantages of open, legitimate business would make ‘criminal police’ uncompetitive. (Moreover, they argue, the Mafia can only thrive in the artificial market niche created by the prohibition of alcohol, drugs, prostitution, gambling, and other victimless crimes. Mafia gangs might kill each other over turf, but liquor-store owners generally do not.)”}



\stopblockquote
As we have pointed out in section F.6, the “Mafia” objection to “anarcho”-capitalist defence companies is a red herring. The biggest problem would not be “criminal police” but the fact that working people and tenants would subject to the rules, power and laws of the property owners, the rich would be able to buy better police protection and “justice” than the poor and that the “general” law code these companies would defend would be slanted towards the interests and power of the capitalist class (defending capitalist property rights and the proprietors power). And as we also noted, such a system has already been tried in 19\high{th}-century and early 20\high{th} America, with the result that the rich reduced the working class to a serf-like existence, capitalist production undermined independent producers (to the annoyance of individualist anarchists at the time), and the result was the emergence of the corporate America that “anarcho”-capitalists say they oppose.


Caplan argues that “liquor-store owners” do not generally kill each other over turf. This is true (but then again they do not have access to their own private cops currently so perhaps this could change). But the company owners who created their own private police forces and armies in America’s past {\bf did} allow their goons to attack and murder union organisers and strikers. Let us look at Henry Ford’s Service Department (private police force) in action:



\startblockquote
{\em “In 1932 a hunger march of the unemployed was planned to march up to the gates of the Ford plant at Dearborn\unknown{} The machine guns of the Dearborn police and the Ford Motor Company’s Service Department killed [four] and wounded over a score of others\unknown{} Ford was fundamentally and entirely opposed to trade unions. The idea of working men questioning his prerogatives as an owner was outrageous\unknown{} [T]he River Rouge plant\unknown{} was dominated by the autocratic regime of Bennett’s service men. Bennett . . organise[d] and train[ed] the three and a half thousand private policemen employed by Ford. His task was to maintain discipline amongst the work force, protect Ford’s property [and power], and prevent unionisation\unknown{} Frank Murphy, the mayor of Detroit, claimed that ‘Henry Ford employs some of the worst gangsters in our city.’ The claim was well based. Ford’s Service Department policed the gates of his plants, infiltrated emergent groups of union activists, posed as workers to spy on men on the line\unknown{} Under this tyranny the Ford worker had no security, no rights. So much so that any information about the state of things within the plant could only be freely obtained from ex-Ford workers.”} [Huw Beynon, {\bf Working for Ford}, pp. 29–30]



\stopblockquote
The private police attacked women workers handing out pro-union handbills and gave them {\em “a serve beating.”} At Kansas and Dallas {\em “similar beatings were handed out to the union men.”} [{\bf Op. Cit.}, p. 34] This use of private police to control the work force was not unique. General Motors {\em “spent one million dollars on espionage, employing fourteen detective agencies and two hundred spies at one time [between 1933 and 1936]. The Pinkerton Detective Agency found anti-unionism its most lucrative activity.”} [{\bf Op. Cit.}, p. 32] We must also note that the Pinkerton’s had been selling their private police services for decades before the 1930s. In the 1870s, they had infiltrated and destroyed the Molly Maguires (a secret organisation Irish miners had developed to fight the coal bosses). For over 60 years the Pinkerton Detective Agency had {\em “specialised in providing spies, agent provocateurs, and private armed forces for employers combating labour organisations.”} By 1892 it {\em “had provided its services for management in seventy major labour disputes, and its 2 000 active agents and 30 000 reserves totalled more than the standing army of the nation.”} [Jeremy Brecher, {\bf Strike!}, p. 9 and p. 55] With this force available, little wonder unions found it so hard to survive in the USA. Given that unions could be considered as “defence” agencies for workers, this suggests a picture of how “anarcho”-capitalism may work in practice.


It could be argued that, in the end, the union was recognised by the Ford company. However, this occurred after the New Deal was in place (which helped the process), after years of illegal activity (by definition union activism on Ford property was an illegal act) and extremely militant strikes. Given that the union agreement occurred nearly 40 years after Ford was formed {\bf and} in a legal situation violently at odds with “anarcho”-capitalism (or even minimal statist capitalism), we would be justified in wondering if unionisation would ever have occurred at Ford and if Ford’s private police state would ever have been reformed.


Of course, from an “anarcho”-capitalist perspective the only limitation in the Ford workers’ liberty was the fact they had to pay taxes to the US government. The regime at Ford could {\bf not} restrict their liberty as no one forced them to work for the company. Needless to say, an “anarcho”-capitalist would reject out of hand the argument that no-one forced the citizen to entry or remain in the USA and so they consented to taxation, the government’s laws and so on.


This is more than a history lesson. Such private police forces are on the rise again (see {\em “Armed and Dangerous: Private Police on the March”} by Mike Zielinski, {\bf Covert Action Quarterly}, no. 54, Fall, 1995 for example). This system of private police (as demonstrated by Ford) is just one of the hidden aspects of Caplan’s comment that the “anarcho”-capitalist {\em “typically hails modern society’s increasing reliance on private security guards\unknown{} and other demonstrations of the free market’s ability to supply the defensive and legal services normally assumed to be of necessity a government monopoly.”}


Needless to say, private police states are not a step forward in anarchist eyes.


Caplan:



\startblockquote
{\em “Unlike some left-anarchists, the anarcho-capitalist has no objection to punishing criminals; and he finds the former’s claim that punishment does not deter crime to be the height of naivete. Traditional punishment might be meted out after a conviction by a neutral arbitrator; or a system of monetary restitution (probably in conjunction with a prison factory system) might exist instead.”}



\stopblockquote
Let us note first that in disputes between the capitalist class and the working class, there would be no {\em “neutral arbitrator,”} because the rich would either own the arbitration company or influence/control it through the power of the purse (see section F.6). In addition, “successful” arbitrators would also be wealthy, therefore making neutrality even more unlikely. Moreover, given that the laws the “neutral arbitrator” would be using are based on capitalist property rights, the powers and privileges of the owner are built into the system from the start.


Second, the left-libertarian critique of punishment does not rest, as “anarcho”-capitalists claim, on the naive view that intimidation and coercion aren’t effective in controlling behaviour. Rather, it rests on the premise that capitalist societies produce large numbers of criminals, whereas societies based on equality and community ownership of productive assets do not.


The argument for this is that societies based on private property and hierarchy inevitably lead to a huge gap between the haves and the have-nots, with the latter sunk in poverty, alienation, resentment, anger, and hopelessness, while at the same time such societies promote greed, ambition, ruthlessness, deceit, and other aspects of competitive individualism that destroy communal values like sharing, co-operation, and mutual aid. Thus in capitalist societies, the vast majority of “crime” turns out to be so-called “crimes against property,” which can be traced to poverty and the grossly unfair distribution of wealth. Where the top one percent of the population controls more wealth than the bottom 90 percent combined, it is no wonder that a considerable number of those on the bottom should try to recoup illegally some of the mal-distributed wealth they cannot obtain legally. (In this they are encouraged by the bad example of the ruling class, whose parasitic ways of making a living would be classified as criminal if the mechanisms for defining “criminal behaviour” were not controlled by the ruling class itself.) And most of the remaining “crimes against persons” can be traced to the alienation, dehumanisation, frustration, rage, and other negative emotions produced by the inhumane and unjust economic system.


Thus it is only in our societies like ours, with their wholesale manufacture of many different kinds of criminals, that punishment appears to be the only possible way to discourage “crime.” From the left-libertarian perspective, however, the punitive approach is a band-aid measure that does not get to the real root of the problem — a problem that lies in the structure of the system itself. The real solution is the creation of a non-hierarchical society based on communal ownership of productive assets, which, by eliminating poverty and the other negative effects of capitalism, would greatly reduce the incidence of criminal behaviour and so the need for punitive countermeasures.


Finally, two more points on private prisons. Firstly, as to the desirability of a “prison factory system,” we will merely note that, given the capitalist principle of “grow-or-die,” if punishing crime becomes a business, one can be sure that those who profit from it will find ways to ensure that the “criminal” population keeps expanding at a rate sufficient to maintain a high rate of profit and growth. After all, the logic of a “prison factory system” is self-defeating. If the aim of prison is to deter crime (as some claim) and if a private prison system will meet that aim, then a successful private prison system will stop crime, which, in turn, will put them out of business! Thus a “prison factory system” cannot aim to be efficient (i.e. stop crime).


Secondly, Caplan does not mention the effect of prison labour on the wages, job conditions and market position of workers. Having a sizeable proportion of the working population labouring in prison would have a serious impact on the bargaining power of workers. How could workers outside of prison compete with such a regime of labour discipline without submitting to prison-like conditions themselves? Unsurprisingly, US history again presents some insight into this. As Noam Chomsky notes, the {\em “rapid industrial development in the southeastern region [of America] a century ago was based on (Black) convict labour, leased to the highest bidder.”} Chomsky quotes expert Alex Lichtenstein comments that Southern Industrialists pointed out that convict labour was {\em “more reliable and productive than free labour”} and that it overcomes the problem of labour turnover and instability. It also {\em “remove[d] all danger and cost of strikes”} and that it lowers wages for {\em “free labour”} (i.e. wage labour). The US Bureau of Labor reported that {\em “mine owners [in Alabama] say they could not work at a profit without the lowering effect in wages of convict-labour competition.”} [{\bf The Umbrella of US Power}, p. 32]


Needless to say, Caplan fails to mention this aspect of “anarcho”-capitalism (just as he fails to mention the example of Ford’s private police state). Perhaps an “anarcho”-capitalist will say that prison labour will be less productive than wage labour and so workers have little to fear, but this makes little sense. If wage labour is more productive then prison labour will not find a market (and then what for the prisoners? Will profit-maximising companies {\bf really} invest in an industry with such high over-heads as maintaining prisoners for free?). Thus it seems more than likely that any “prison-factory system” will be as productive as the surrounding wage-labour ones, thus forcing down their wages and the conditions of labour. For capitalists this would be ideal, however for the vast majority a different conclusion must be drawn.


Caplan:



\startblockquote
{\em “Probably the main division between the anarcho-capitalists stems from the apparent differences between Rothbard’s natural-law anarchism, and David Friedman’s more economistic approach. Rothbard puts more emphasis on the need for a generally recognised libertarian legal code (which he thinks could be developed fairly easily by purification of the Anglo-American common law), whereas Friedman focuses more intently on the possibility of plural legal systems co-existing and responding to the consumer demands of different elements of the population. The difference, however, is probably overstated. Rothbard believes that it is legitimate for consumer demand to determine the philosophically neutral content of the law, such as legal procedure, as well as technical issues of property right definition such as water law, mining law, etc. And Friedman admits that ‘focal points’ including prevalent norms are likely to circumscribe and somewhat standardise the menu of available legal codes.”}



\stopblockquote
The argument that “consumer demand” would determine a “philosophically neutral” content of the law cannot be sustained. Any law code will reflect the philosophy of those who create it. Under “anarcho”-capitalism, as we have noted (see section F.6), the values of the capitalist rich will be dominant and will shape the law code and justice system, as they do now, only more so. The law code will therefore continue to give priority to the protection of private property over human values; those who have the most money will continue being able to hire the best lawyers; and the best (i.e. most highly paid) judges will be inclined to side with the wealthy and to rule in their interests, out of class loyalty (and personal interests).


Moreover, given that the law code exists to protect capitalist property rights, how can it be “philosophically neutral” with that basis? How would “competing” property frameworks co-exist? If a defence agency allowed squatting and another (hired by the property owner) did not, there is no way (bar force) a conflict could be resolved. Then the firm with the most resources would win. “Anarcho”-capitalism, in effect, smuggles into the foundation of their system a distinctly {\bf non}-neutral philosophy, namely capitalism. Those who reject such a basis may end up sharing the fate of tribal peoples who rejected that system of property rights, for example, the Native Americans.


In other words, in terms of outcome the whole system would favour {\bf capitalist} values and so not be “philosophically neutral.” The law would be favourable to employers rather than workers, manufacturers rather than consumers, and landlords rather than tenants. Indeed, from the “anarcho”-capitalist perspective the rules that benefit employers, landlords and manufacturers (as passed by progressive legislatures or enforced by direct action) simply define liberty and property rights whereas the rules that benefit workers, tenants and consumers are simply an interference with liberty. The rules one likes, in other words, are the foundations of sacred property rights (and so “liberty,” as least for the capitalist and landlord), those one does not like are meddlesome regulation. This is a very handy trick and would not be worth mentioning if it was not so commonplace in right-libertarian theory.


We should leave aside the fantasy that the law under “anarcho”-capitalism is a politically neutral set of universal rules deduced from particular cases and free from a particular instrumental or class agenda.


Caplan:



\startblockquote
{\em “Critics of anarcho-capitalism sometimes assume that communal or worker-owned firms would be penalised or prohibited in an anarcho-capitalist society. It would be more accurate to state that while individuals would be free to voluntarily form communitarian organisations, the anarcho-capitalist simply doubts that they would be widespread or prevalent.”}



\stopblockquote
There is good reason for this doubt. Worker co-operatives would not be widespread or prevalent in an “anarcho”-capitalist society for the same reason that they are not widespread or prevalent now: namely, that the socio-economic, legal, and political systems would be structured in such as way as to automatically discourage their growth (in addition, capitalist firms and the rich would also have an advantage in that they would still own and control the wealth they currently have which are a result of previous “initiations of force”. This would give them an obvious advantage on the “free-market” — an advantage which would be insurmountable).


As we explain in more detail in section J.5.11, the reason why there are not more producer co-operatives is partly structural, based on the fact that co-operatives have a tendency to grow at a slower rate than capitalist firms. This is a good thing if one’s primary concern is, say, protecting the environment, but fatal if one is trying to survive in a competitive capitalist environment.


Under capitalism, successful competition for profits is the fundamental fact of economic survival. This means that banks and private investors seeking the highest returns on their investments will favour those companies that grow the fastest. Moreover, in co-operatives returns to capital are less than in capitalist firms. Under such conditions, capitalist firms will attract more investment capital, allowing them to buy more productivity-enhancing technology and thus to sell their products more cheaply than co-operatives. Even though co-operatives are at least as efficient (usually more so) than their equivalent capitalist firms, the effect of market forces (particularly those associated with capital markets) will select against them. This bias against co-operatives under capitalism is enough to ensure that, despite their often higher efficiency, they cannot prosper under capitalism (i.e. capitalism selects the {\bf least efficient} way of producing). Hence Caplan’s comments hide how the effect of inequalities in wealth and power under capitalism determine which alternatives are “widespread” in the “free market”


Moreover, co-operatives within capitalism have a tendency to adapt to the dominant market conditions rather than undermining them. There will be pressure on the co-operatives to compete more effectively by adopting the same cost-cutting and profit-enhancing measures as capitalist firms. Such measures will include the deskilling of workers; squeezing as much “productivity” as is humanly possible from them; and a system of pay differentials in which the majority of workers receive low wages while the bulk of profits are reinvested in technology upgrades and other capital expansion that keeps pace with capitalist firms. But this means that in a capitalist environment, there tend to be few practical advantages for workers in collective ownership of the firms in which they work.


This problem can only be solved by eliminating private property and the coercive statist mechanisms required to protect it (including private states masquerading as “protection companies”), because this is the only way to eliminate competition for profits as the driving force of economic activity. In a libertarian socialist environment, federated associations of workers in co-operative enterprises would co-ordinate production for {\bf use} rather than profit, thus eliminating the competitive basis of the economy and so also the “grow-or-die” principle which now puts co-operatives at a fatal economic disadvantage. (For more on how such an economy would be organised and operated, as well as answers to objections, see section I.)


And let us not forget what is implied by Caplan’s statement that the “anarcho”-capitalist does not think that co-operative holding of “property” “would be widespread or prevalent.” It means that the vast majority would be subject to the power, authority and laws of the property owner and so would not govern themselves. In other words, it would a system of private statism rather than anarchy.


Caplan:



\startblockquote
{\em “However, in theory an ‘anarcho-capitalist’ society might be filled with nothing but communes or worker-owned firms, so long as these associations were formed voluntarily (i.e., individuals joined voluntarily and capital was obtained with the consent of the owners) and individuals retained the right to exit and set up corporations or other profit-making, individualistic firms.”}



\stopblockquote
It’s interesting that the “anarcho”-capitalists are willing to allow workers to set up “voluntary” co-operatives so long as the conditions are retained which ensure that such co-operatives will have difficulty surviving (i.e. private property and private states), but they are unwilling to allow workers to set up co-operatives under conditions that would ensure their success (i.e. the absence of private property and private states). This reflects the usual vacuousness of the right-libertarian concepts of “freedom” and “voluntarism.”


In other words, these worker-owned firms would exist in and be subject to the same capitalist {\em “general libertarian law code”} and work in the same capitalist market as the rest of society. So, not only are these co-operatives subject to capitalist market forces, they exist and operate in a society defined by capitalist laws. As discussed in section F.2, such disregard for the social context of human action shows up the “anarcho” capitalist’s disregard for meaningful liberty.


All Caplan is arguing here is that as long as people remain within the (capitalist) “law code,” they can do whatever they like. However, what determines the amount of coercion required in a society is the extent to which people are willing to accept the rules imposed on them. This is as true of an “anarcho”-capitalist society as it is of any other. In other words, if more and more people reject the basic assumptions of capitalism, the more coercion against anarchistic tendencies will be required. Saying that people would be free to experiment under “anarcho”-capitalist law (if they can afford it, of course) does not address the issue of changes in social awareness (caused, by example, by class struggle) which can make such “laws” redundant. So, when all is said and done, “anarcho”-capitalism just states that as long as you accept their rules, you are free to do what you like.


How generous of them!


Thus, while we would be allowed to be collective capitalists or property owners under “anarcho”-capitalists we would have no choice about living under laws based on the most rigid and extreme interpretation of property rights available. In other words, “anarcho”-capitalists recognise (at least implicitly) that there exists one collective need that needs collective support — a law system to define and protects people’s rights. Ultimately, as C.B. Macpherson argues, “Individualism” implies “collectivism” for the {\em “notion that individualism and ‘collectivism’ are the opposite ends of a scale along which states and theories of the state can be arranged \unknown{} is superficial and misleading\unknown{} [I]ndividualism \unknown{} does not exclude but on the contrary demands the supremacy of the state [or law] over the individual. It is not a question of the more individualism, the less collectivism; rather, the more through-going the individualism, the more complete the collectivism. Of this the supreme illustration is Hobbes’s theory.”} [{\bf The Political Theory of Possessive Individualism}, p.256] Under “anarcho”-capitalism the individual is subject to the laws regarding private property, laws decided in advance by a small group of ideological leaders. Then real individuals are expected to live with the consequences as best they can, with the law being placed ahead of these consequences for flesh and blood people. The abstraction of the law dominates and devours real individuals, who are considered below it and incapable of changing it (except for the worse). This, from one angle, shares a lot with theocracy and very little with liberty.


Needless to say, Caplan like most (if not all) “anarcho”-capitalists assume that the current property owners are entitled to their property. However, as John Stuart Mill pointed out over 100 years ago, the {\em “social arrangements”} existing today {\em “commenced from a distribution of property which was the result, not of a just partition, or acquisition by industry, but of conquest and violence \unknown{} [and] the system still retains many and large traces of its origin.”} [{\bf Principles of Political Economy}, p. 15] Given that (as we point out in section F.1) Murray Rothbard argues that the state cannot be claimed to own its territory simply because it did not acquire its property in a “just” manner, this suggests that “anarcho”-capitalism cannot actually argue against the state. After all, property owners today cannot be said to have received their property “justly” and {\bf if} they are entitled to it so is the state to {\bf its} “property”!


But as is so often the case, property owners are exempt from the analysis the state is subjected to by “anarcho”-capitalists. The state and property owners may do the same thing (such as ban freedom of speech and association or regulate individual behaviour) but only the state is condemned by “anarcho”-capitalism.


Caplan:



\startblockquote
{\em “On other issues, the anarcho-capitalist differs little if at all from the more moderate libertarian. Services should be privatised and opened to free competition; regulation of personal AND economic behaviour should be done away with.”}



\stopblockquote
The “anarcho”-capitalist’s professed desire to “do away” with the “regulation” of economic behaviour is entirely disingenuous. For, by giving capitalists the ability to protect their exploitative monopolies of social capital by the use of coercive private states, one is thereby “regulating” the economy in the strongest possible way, i.e. ensuring that it will be channelled in certain directions rather than others. For example, one is guaranteeing that production will be for profit rather than use; that there will consequently be runaway growth and an endless devouring of nature based on the principle of “grow or die;” and that the alienation and deskilling of the workforce will continue. What the “anarcho”-capitalist really means by “doing away with the regulation of economic behaviour” is that ordinary people will have even less opportunity than now to democratically control the rapacious behaviour of capitalists. Needless to say, the “regulation of personal” behaviour would {\bf not} be done away with in the workplace, where the authority of the bosses would still exist and you would have follow their petty rules and regulations.


Moreover, regardless of “anarcho”-capitalist claims, they do not, in fact, support civil liberties or oppose “regulation” of personal behaviour as such. Rather, they {\bf support} property owners suppressing civil liberties on their property and the regulation of personal behaviour by employers and landlords. This they argue is a valid expression of property rights. Indeed, any attempts to allow workers civil liberties or restrict employers demands on workers by state or union action is denounced as a violation of “liberty” (i.e. the power of the property owner). Those subject to the denial of civil liberties or the regulation of their personal behaviour by landlords or employees can “love it or leave it.” Of course, the same can be said to any objector to state oppression — and frequently is. This is an artificial double standard, which labels a restraint by one group or person in a completely different way than the same restraint by others simply because one is called “the government” and the other is not.


This denial of civil liberties can be seen from these words by Murray Rothbard:



\startblockquote
{\em “[I]n the profoundest sense there are no rights but property rights \unknown{} Freedom of speech is supposed to mean the right of everyone to say whatever he likes. But the neglected question is: Where? Where does a man have this right? He certainly does not have it on property on which he is trespassing. In short, he has this right only either on his own property or on the property of someone who has agreed, as a gift or in a rental contract, to allow him in the premises. In fact, then, there is no such thing as a separate ‘right to free speech’; there is only a man’s property right: the right to do as he wills with his own or to make voluntary agreements with other property owners.”} [Murray Rothbard, {\bf Power and Market}, p. 176]



\stopblockquote
Of course, Rothbard fails to see that for the property-less such a regime implies {\bf no} rights whatsoever. It also means the effective end of free speech and free association as the property owner can censor those on their property (such as workers or tenants) and ban their organisations (such as unions). Of course, in his example Rothbard looks at the “trespasser,” {\bf not} the wage worker or the tenant (two far more common examples in any modern society). Rothbard is proposing the dictatorship of the property owner and the end of civil liberties and equal rights (as property is unequally distributed). He gives this utter denial of liberty an Orwellian twist by proclaiming the end of civil liberties by property rights as “a new liberty.” Perhaps for the property-owner, but not the wage worker — {\em “We who belong to the proletaire class, property excommunicates us!”} [Proudhon, {\bf What is Property?}, p. 137]


In effect, right-Libertarians do not care how many restrictions are placed on you as long as it is not the government doing it. Of course it will be claimed that workers and tenants “consent” to these controls (although they reject the notion that citizens “consent” to government controls by not leaving their state). Here the libertarian case is so disingenuous as to be offensive. There is no symmetry in the situations facing workers and firms. To the worker, the loss of a job is often far more of a threat than the loss of one worker is to the firm. The reality of economic power leads people to contract into situations that, although they are indeed the “best” arrangements of those available, are nonetheless miserable. In any real economy — and, remember, the right-libertarian economy lacks any social safety net, making workers’ positions more insecure than now — the right-libertarian denial of economic power is a delusion.


Unlike anarchist theory, right-libertarian theory provides {\bf no} rationale to protest private power (or even state power if we accept the notion that the state owns its territory). Relations of domination and subjection are valid expressions of liberty in their system and, perversely, attempts to resist authority (by strikes, unions, resistance) are deemed “initiations of force” upon the oppressor! In contrast, anarchist theory provides a strong rationale for resisting private and public domination. Such domination violates freedom and any free association which dominates any within it violates the basis of that association in self-assumed obligation (see section A.2.11). Thus Proudhon:



\startblockquote
{\em “The social contract should increase the well-being and liberty of every citizen. — If any one-sided conditions should slip in; if one part of the citizens should find themselves, by the contract, subordinated and exploited by others, it would no longer be a contract; it would be a fraud, against which annulment might at any time by invoked justly.”} [{\bf The General Idea of the Revolution}, p. 114]



\stopblockquote
Caplan’s claim that right libertarians oppose regulation of individual behaviour is simply not true. They just oppose state regulation while supporting private regulation wholeheartedly. Anarchists, in contrast, reject both public and private domination.


Caplan:



\startblockquote
{\em “Poverty would be handled by work and responsibility for those able to care for themselves, and voluntary charity for those who cannot. (Libertarians hasten to add that a deregulated economy would greatly increase the economic opportunities of the poor, and elimination of taxation would lead to a large increase in charitable giving.)”}



\stopblockquote
Notice the implication that poverty is now caused by laziness and irresponsibility rather than by the inevitable workings of an economic system that {\bf requires} a large {\em “reserve army of the unemployed”} as a condition of profitability. The continuous “boom” economy of “anarcho”-capitalist fantasies is simply incompatible with the fundamental principles of capitalism. To re-quote Michael Kalecki (from section B.4.4), {\em “[l]asting full employment is not at all to [the] liking [of business leaders]. The workers would ‘get out of hand’ and the ‘captains of industry’ would be anxious ‘to teach them a lesson’”} as {\em “‘discipline in the factories’ and ‘political stability’ are more appreciated by business leaders than profits. Their class interest tells them that lasting full employment is unsound from their point of view and that unemployment is an integral part of the ‘normal’ capitalist system.”}. See section C.7 (“What causes the capitalist business cycle?”) for a fuller discussion of this point.


In addition, the claims that a “deregulated economy” would benefit the poor do not have much empirical evidence to back them up. If we look at the last quarter of the twentieth century we discover that a more deregulated economy has lead to massive increases in inequality and poverty. If a movement towards a deregulated economy has had the opposite effect than that predicted by Caplan, why should a totally deregulated economy have the opposite effect. It is a bit like claiming that while adding black paint to grey makes it more black, adding the whole tin will make it white!


The reason for increased inequality and poverty as a result of increased deregulation is simple. A “free exchange” between two people will benefit the stronger party. This is obvious as the economy is marked by power, regardless of “anarcho”-capitalist claims, and any “free exchange” will reflect difference in power. Moreover, a series of such exchanges will have an accumulative effect, with the results of previous exchanges bolstering the position of the stronger party in the current exchange.


Moreover, the claim that removing taxation will {\bf increase} donations to charity is someone strange. We doubt that the rich who object to money being taken from them to pay for welfare will {\bf increase} the amount of money they give to others if taxation {\bf was} abolished. As Peter Sabatini points out, “anarcho”-capitalists {\em “constantly rant and shriek about how the government, or the rabble, hinders their Lockean right to amass capital.”} [{\bf Social Anarchism}, no. 23, p.101] Caplan seems to expect them to turn over a new leaf and give {\bf more} to that same rabble!


\chapter{Is “anarcho”-capitalism a type of anarchism?
}

Anyone who has followed political discussion on the net has probably come across people calling themselves libertarians but arguing from a right-wing, pro-capitalist perspective. For most Europeans this is weird, as in Europe the term {\em “libertarian”} is almost always used in conjunction with {\em “socialist”} or {\em “communist.”} In the US, though, the Right has partially succeeded in appropriating this term for itself. Even stranger, however, is that a few of these right-wingers have started calling themselves “anarchists” in what must be one of the finest examples of an oxymoron in the English language: ‘Anarcho-capitalist’!!


Arguing with fools is seldom rewarded, but to allow their foolishness to go unchallenged risks allowing them to deceive those who are new to anarchism. That’s what this section of the anarchist FAQ is for, to show why the claims of these “anarchist” capitalists are false. Anarchism has always been anti-capitalist and any “anarchism” that claims otherwise cannot be part of the anarchist tradition. So this section of the FAQ does not reflect some kind of debate within anarchism, as many of these types like to pretend, but a debate between anarchism and its old enemy, capitalism. In many ways this debate mirrors the one between Peter Kropotkin and Herbert Spencer, an English pro-capitalist, minimal statist, at the turn the 19\high{th} century and, as such, it is hardly new.


The “anarcho”-capitalist argument hinges on using the dictionary definition of “anarchism” and/or “anarchy” — they try to define anarchism as being “opposition to government,” and nothing else. However, dictionaries are hardly politically sophisticated and their definitions rarely reflect the wide range of ideas associated with political theories and their history. Thus the dictionary “definition” is anarchism will tend to ignore its consistent views on authority, exploitation, property and capitalism (ideas easily discovered if actual anarchist texts are read). And, of course, many dictionaries “define” anarchy as “chaos” or “disorder” but we never see “anarcho”-capitalists use that particular definition!


And for this strategy to work, a lot of “inconvenient” history and ideas from all branches of anarchism must be ignored. From individualists like Spooner and Tucker to communists like Kropotkin and Malatesta, anarchists have always been anti-capitalist (see section G for more on the anti-capitalist nature of individualist anarchism). Therefore “anarcho”-capitalists are not anarchists in the same sense that rain is not dry.


Of course, we cannot stop the “anarcho”-capitalists using the words “anarcho”, “anarchism” and “anarchy” to describe their ideas. The democracies of the west could not stop the Chinese Stalinist state calling itself the People’s Republic of China. Nor could the social democrats stop the fascists in Germany calling themselves “National Socialists”. Nor could the Italian anarcho-syndicalists stop the fascists using the expression “National Syndicalism”. This does not mean that any of these movements actual name reflected their content — China is a dictatorship, not a democracy, the Nazi’s were not socialists (capitalists made fortunes in Nazi Germany because it crushed the labour movement), and the Italian fascist state had nothing in common with anarcho-syndicalists ideas of decentralised, “from the bottom up” unions and the abolition of the state and capitalism.


Therefore, just because someone uses a label it does not mean that they support the ideas associated with that label. And this is the case with “anarcho”-capitalism — its ideas are at odds with the key ideas associated with all forms of traditional anarchism (even individualist anarchism which is often claimed as being a forefather of the ideology).


All we can do is indicate {\bf why} “anarcho”-capitalism is not part of the anarchist tradition and so has falsely appropriated the name. This section of the FAQ aims to do just that — present the case why “anarcho”-capitalists are not anarchists. We do this, in part, by indicating where they differ from genuine anarchists (on such essential issues as private property, equality, exploitation and opposition to hierarchy) In addition, we take the opportunity to present a general critique of right-libertarian claims from an anarchist perspective. In this way we show up why anarchists reject that theory as being opposed to liberty and anarchist ideals.


We are covering this topic in an anarchist FAQ for three reasons. Firstly, the number of “libertarian” and “anarcho”-capitalists on the net means that those seeking to find out about anarchism may conclude that they are “anarchists” as well. Secondly, unfortunately, some academics and writers have taken their claims of being anarchists at face value and have included their ideology into general accounts of anarchism. These two reasons are obviously related and hence the need to show the facts of the matter. As we have extensively documented in earlier sections, anarchist theory has always been anti-capitalist. There is no relationship between anarchism and capitalism, in any form. Therefore, there is a need for this section in order to indicate exactly why “anarcho”-capitalism is not anarchist. As will be quickly seen from our discussion, almost all anarchists who become aware of “anarcho”-capitalism quickly reject it as a form of anarchism (the better academic accounts do note that anarchists generally reject the claim, though). The last reason is to provide other anarchists with arguments and evidence to use against “anarcho”-capitalism and its claims of being a new form of “anarchism.”


So this section of the FAQ does not, as we noted above, represent some kind of “debate” within anarchism. It reflects the attempt by anarchists to reclaim the history and meaning of anarchism from those who are attempting to steal its name (just as right-wingers in America have attempted to appropriate the name “libertarian” for their pro-capitalist views, and by so doing ignore over 100 years of anti-capitalist usage). However, this section also serves two other purposes. Firstly, critiquing right-libertarian and “anarcho”-capitalist theories allows us to explain anarchist ones at the same time and indicate why they are better. Secondly, and more importantly, the “ideas” and “ideals” that underlie “anarcho”-capitalism are usually identical (or, at the very least, similar) to those of neo-liberalism. This was noted by Bob Black in the early 1980s, when a {\em “wing of the Reaganist Right has obviously appropriated, with suspect selectivity, such libertarian themes as deregulation and voluntarism. Ideologues indignant that Reagan has travestied their principles. Tough shit! I notice that it’s their principles, not mine, that he found suitable to travesty.”} [{\bf The Libertarian As Conservative}] This was echoed by Noam Chomsky two decades later when while {\em “nobody takes [right-wing libertarianism] seriously”} as {\em “everybody knows that a society that worked by \unknown{} [its] principles would self-destruct in three seconds”} the {\em “only reason”} why some people {\em “pretend to take it seriously is because you can use it as a weapon.”} [{\bf Understanding Power}, p. 200] As neo-liberalism is being used as the ideological basis of the current attack on the working class, critiquing “anarcho”-capitalism and right-libertarianism also allows use to build theoretical weapons to use to resist this attack and aid the class struggle.


A few more points before beginning. When debating with “libertarian” or “anarchist” capitalists it’s necessary to remember that while they claim “real capitalism” does not exist (because all existing forms of capitalism are statist), they will claim that all the good things we have — advanced medical technology, consumer choice of products, etc. — are nevertheless due to “capitalism.” Yet if you point out any problems in modern life, these will be blamed on “statism.” Since there has never been and never will be a capitalist system without some sort of state, it’s hard to argue against this “logic.” Many actually use the example of the Internet as proof of the power of “capitalism,” ignoring the fact that the state paid for its development before turning it over to companies to make a profit from it. Similar points can be made about numerous other products of “capitalism” and the world we live in. To artificially separate one aspect of a complex evolution fails to understand the nature and history of the capitalist system.


In addition to this ability to be selective about the history and results of capitalism, their theory has a great “escape clause.” If wealthy employers abuse their power or the rights of the working class (as they have always done), then they have (according to “libertarian” ideology) ceased to be capitalists! This is based upon the misperception that an economic system that relies on force {\bf cannot} be capitalistic. This is {\bf very} handy as it can absolve the ideology from blame for any (excessive) oppression which results from its practice. Thus individuals are always to blame, {\bf not} the system that generated the opportunities for abuse they freely used.


Anarchism has always been aware of the existence of “free market” capitalism, particularly its extreme (minimal state) wing, and has always rejected it. As we discuss in section 7, anarchists from Proudhon onwards have rejected the idea of any similar aims and goals (and, significantly, vice versa). As academic Alan Carter notes, anarchist concern for equality as a necessary precondition for genuine freedom means {\em “that is one very good reason for not confusing anarchists with liberals or economic ‘libertarians’ — in other words, for not lumping together everyone who is in some way or another critical of the state. It is why calling the likes of Nozick ‘anarchists’ is highly misleading.”} [{\em “Some notes on ‘Anarchism’”}, pp. 141–5, {\bf Anarchist Studies}, vol. 1, no. 2, p. 143] So anarchists have evaluated “free market” capitalism and rejected it as non-anarchist for over 150 years. Attempts by “anarcho”-capitalism to say that their system is “anarchist” flies in the face of this long history of anarchist analysis. That some academics fall for their attempts to appropriate the anarchist label for their ideology is down to a false premise: it {\em “is judged to be anarchism largely because some anarcho-capitalists {\bf say} they are ‘anarchists’ and because they criticise the State.”} [Peter Sabatini, {\bf Social Anarchism}, no. 23, p. 100]


More generally, we must stress that most (if not all) anarchists do not want to live in a society {\bf just like this one} but without state coercion and (the initiation of) force. Anarchists do not confuse “freedom” with the “right” to govern and exploit others nor with being able to change masters. It is not enough to say we can start our own (co-operative) business in such a society. We want the abolition of the capitalist system of authoritarian relationships, not just a change of bosses or the possibility of little islands of liberty within a sea of capitalism (islands which are always in danger of being flooded and our activity destroyed). Thus, in this section of the FAQ, we analysis many “anarcho”-capitalist claims on their own terms (for example, the importance of equality in the market or why capitalism cannot be reformed away by exchanges on the capitalist market) but that does not mean we desire a society nearly identical to the current one. Far from it, we want to transform this society into one more suited for developing and enriching individuality and freedom. But before we can achieve that we must critically evaluate the current society and point out its basic limitations.


Finally, we dedicate this section of the FAQ to those who have seen the real face of “free market” capitalism at work: the working men and women (anarchist or not) murdered in the jails and concentration camps or on the streets by the hired assassins of capitalism.


\section{1 Are “anarcho”-capitalists really anarchists?
}

In a word, no. While “anarcho”-capitalists obviously try to associate themselves with the anarchist tradition by using the word “anarcho” or by calling themselves “anarchists”, their ideas are distinctly at odds with those associated with anarchism. As a result, any claims that their ideas are anarchist or that they are part of the anarchist tradition or movement are false.


“Anarcho”-capitalists claim to be anarchists because they say that they oppose government. As such, as noted in the last section, they use a dictionary definition of anarchism. However, this fails to appreciate that anarchism is a {\bf political theory}, not a dictionary definition. As dictionaries are rarely politically sophisticated things, this means that they fail to recognise that anarchism is more than just opposition to government, it is also marked a opposition to capitalism (i.e. exploitation and private property). Thus, opposition to government is a necessary but not sufficient condition for being an anarchist — you also need to be opposed to exploitation and capitalist private property. As “anarcho”-capitalists do not consider interest, rent and profits (i.e. capitalism) to be exploitative nor oppose capitalist property rights, they are not anarchists.


Moreover, “anarcho”-capitalism is inherently self-refuting. This can be seen from leading “anarcho”-capitalist Murray Rothbard. he thundered against the evil of the state, arguing that it {\em “arrogates to itself a monopoly of force, of ultimate decision-making power, over a given area territorial area.”} In and of itself, this definition is unremarkable. That a few people (an elite of rulers) claim the right to rule others must be part of any sensible definition of the state or government. However, the problems begin for Rothbard when he notes that {\em “[o]bviously, in a free society, Smith has the ultimate decision-making power over his own just property, Jones over his, etc.”} [{\bf The Ethics of Liberty}, p. 170 and p. 173] The logical contradiction in this position should be obvious, but not to Rothbard. It shows the power of ideology, the ability of means words (the expression {\em “private property”}) to turn the bad ({\em “ultimate decision-making power over a given area”}) into the good ({\em “ultimate decision-making power over a given area”}).


Now, this contradiction can be solved in only {\bf one} way — the owners of the {\em “given area”} are also its users. In other words, a system of possession (or “occupancy and use”) as favoured by anarchists. However, Rothbard is a capitalist and supports private property. In other words, wage labour and landlords. This means that he supports a divergence between ownership and use and this means that this {\em “ultimate decision-making power”} extends to those who {\bf use,} but do not own, such property (i.e. tenants and workers). The statist nature of private property is clearly indicated by Rothbard’s words — the property owner in an “anarcho”-capitalist society possesses the {\em “ultimate decision-making power”} over a given area, which is also what the state has currently. Rothbard has, ironically, proved by his own definition that “anarcho”-capitalism is not anarchist.


Rothbard does try to solve this obvious contradiction, but utterly fails. He simply ignores the crux of the matter, that capitalism is based on hierarchy and, therefore, cannot be anarchist. He does this by arguing that the hierarchy associated with capitalism is fine as long as the private property that produced it was acquired in a “just” manner. In so doing he yet again draws attention to the identical authority structures and social relationships of the state and property. As he puts it:



\startblockquote
{\em “{\bf If} the State may be said too properly {\bf own} its territory, then it is proper for it to make rules for everyone who presumes to live in that area. It can legitimately seize or control private property because there {\bf is} no private property in its area, because it really owns the entire land surface. {\bf So long} as the State permits its subjects to leave its territory, then, it can be said to act as does any other owner who sets down rules for people living on his property.”} [{\bf Op. Cit.}, p. 170]



\stopblockquote
Obviously Rothbard argues that the state does not “justly” own its territory — but given that the current distribution of property is just as much the result of violence and coercion as the state, his argument is seriously flawed. It amounts, as we note in section 4, to little more than an {\em {\bf “immaculate conception of property”}} unrelated to reality. Even assuming that private property was produced by the means Rothbard assumes, it does not justify the hierarchy associated with it as the current and future generations of humanity have, effectively, been excommunicated from liberty by previous ones. If, as Rothbard argues, property is a natural right and the basis of liberty then why should the many be excluded from their birthright by a minority? In other words, Rothbard denies that liberty should be universal. He chooses property over liberty while anarchists choose liberty over property.


Even worse, the possibility that private property can result in {\bf worse} violations of individual freedom (at least of workers) than the state of its citizens was implicitly acknowledged by Rothbard. He uses as a hypothetical example a country whose King is threatened by a rising “libertarian” movement. The King responses by {\em “employ[ing] a cunning stratagem,”} namely he {\em “proclaims his government to be dissolved, but just before doing so he arbitrarily parcels out the entire land area of his kingdom to the ‘ownership’ of himself and his relatives.”} Rather than taxes, his subjects now pay rent and he can {\em “regulate to regulate the lives of all the people who presume to live on”} his property as he sees fit. Rothbard then asks:



\startblockquote
{\em “Now what should be the reply of the libertarian rebels to this pert challenge? If they are consistent utilitarians, they must bow to this subterfuge, and resign themselves to living under a regime no less despotic than the one they had been battling for so long. Perhaps, indeed, {\bf more} despotic, for now the king and his relatives can claim for themselves the libertarians’ very principle of the absolute right of private property, an absoluteness which they might not have dared to claim before.”} [{\bf Op. Cit.}, pp. 54–5]



\stopblockquote
So not only does the property owner have the same monopoly of power over a given area as the state, it is {\bf more} despotic as it is based on the {\em “absolute right of private property”}! And remember, Rothbard is arguing {\bf in favour} of “anarcho”-capitalismAnd remember, Rothbard is arguing {\bf in favour} of “anarcho”-capitalism ({\em “if you have unbridled capitalism, you will have all kinds of authority: you will have {\bf extreme} authority.”} [Chomksy, {\bf Understanding Power}, p. 200]). So in practice, private property is a major source of oppression and authoritarianism within society — there is little or no freedom within capitalist production (as Bakunin noted, {\em “the worker sells his person and his liberty for a given time”}). So, in stark contrast to anarchists, “anarcho”-capitalists have no problem with factory fascism (i.e. wage labour), a position which seems highly illogical for a theory calling itself libertarian. If it were truly libertarian, it would oppose all forms of domination, not just statism. This position flows from the “anarcho”-capitalist definition of freedom as the absence of coercion and will be discussed in section 2 in more detail.


Of course, Rothbard has yet another means to escape the obvious, namely that the market will limit the abuses of the property owners. If workers do not like their ruler then they can seek another. However, this reply completely ignores the reality of economic and social power. Thus the “consent” argument fails because it ignores the social circumstances of capitalism which limit the choice of the many. Anarchists have long argued that, as a class, workers have little choice but to “consent” to capitalist hierarchy. The alternative is either dire poverty or starvation.


“Anarcho”-capitalists dismiss such claims by denying that there is such a thing as economic power. Rather, it is simply freedom of contract. Anarchists consider such claims as a joke. To show why, we need only quote (yet again) Rothbard on the abolition of slavery and serfdom in the 19\high{th} century. He argued, correctly, that the {\em “{\bf bodies} of the oppressed were freed, but the property which they had worked and eminently deserved to own, remained in the hands of their former oppressors. With economic power thus remaining in their hands, the former lords soon found themselves virtual masters once more of what were now free tenants or farm labourers. The serfs and slaves had tasted freedom, but had been cruelly derived of its fruits.”} [{\bf Op. Cit.}, p. 74]


To say the least, anarchists fail to see the logic in this position. Contrast this with the standard “anarcho”-capitalist claim that if market forces (“voluntary exchanges”) result in the creation of {\em “free tenants or farm labourers”} then they are free. Yet labourers dispossessed by market forces are in exactly the same social and economic situation as the ex-serfs and ex-slaves. If the latter do not have the fruits of freedom, neither do the former. Rothbard sees the obvious {\em “economic power”} in the latter case, but denies it in the former. It is only Rothbard’s ideology that stops him from drawing the obvious conclusion — identical economic conditions produce identical social relationships and so capitalism is marked by {\em “economic power”} and {\em “virtual masters.”} The only solution is for “anarcho”-capitalists to simply say the ex-serfs and ex-slaves were actually free to choose and, consequently, Rothbard was wrong. It might be inhuman, but at least it would be consistent!


Rothbard’s perspective is alien to anarchism. For example, as individualist anarchist William Bailie noted, under capitalism there is a class system marked by {\em “a dependent industrial class of wage-workers”} and {\em “a privileged class of wealth-monopolisers, each becoming more and more distinct from the other as capitalism advances.”} This has turned property into {\em “a social power, an economic force destructive of rights, a fertile source of injustice, a means of enslaving the dispossessed.”} He concludes: {\em “Under this system equal liberty cannot obtain.”} Bailie notes that the modern {\em “industrial world under capitalistic conditions”} have {\em “arisen under the {\bf regime} of status”} (and so {\em “law-made privileges”}) however, it seems unlikely that he would have concluded that such a class system would be fine if it had developed naturally or the current state was abolished while leaving the class structure intact (as we note in section G.4, Tucker recognised that even the {\em “freest competition”} was powerless against the {\em “enormous concentration of wealth”} associated with modern capitalism). [{\bf The Individualist Anarchists}, p. 121]


Therefore anarchists recognise that “free exchange” or “consent” in unequal circumstances will reduce freedom as well as increasing inequality between individuals and classes. In other words, as we discuss in section 3, inequality will produce social relationships which are based on hierarchy and domination, {\bf not} freedom. As Noam Chomsky put it:



\startblockquote
{\em “Anarcho-capitalism, in my opinion, is a doctrinal system which, if ever implemented, would lead to forms of tyranny and oppression that have few counterparts in human history. There isn’t the slightest possibility that its (in my view, horrendous) ideas would be implemented, because they would quickly destroy any society that made this colossal error. The idea of ‘free contract’ between the potentate and his starving subject is a sick joke, perhaps worth some moments in an academic seminar exploring the consequences of (in my view, absurd) ideas, but nowhere else.”} [{\bf Noam Chomsky on Anarchism}, interview with Tom Lane, December 23, 1996]



\stopblockquote
Clearly, then, by its own arguments “anarcho”-capitalism is not anarchist. This should come as no surprise to anarchists. Anarchism, as a political theory, was born when Proudhon wrote {\bf What is Property?} specifically to refute the notion that workers are free when capitalist property forces them to seek employment by landlords and capitalists. He was well aware that in such circumstances property {\em “violates equality by the rights of exclusion and increase, and freedom by despotism \unknown{} [and has] perfect identity with robbery.”} He, unsurprisingly, talks of the {\em “proprietor, to whom [the worker] has sold and surrendered his liberty.”} For Proudhon, anarchy was {\em “the absence of a master, of a sovereign”} while {\em “proprietor”} was {\em “synonymous”} with {\em “sovereign”} for he {\em “imposes his will as law, and suffers neither contradiction nor control.”} This meant that {\em “property engenders despotism,”} as {\em “each proprietor is sovereign lord within the sphere of his property.”} [{\bf What is Property}, p. 251, p. 130, p. 264 and pp. 266–7] It must also be stressed that Proudhon’s classic work is a lengthy critique of the kind of apologetics for private property Rothbard espouses to salvage his ideology from its obvious contradictions.


Ironically, Rothbard repeats the same analysis as Proudhon but draws the {\bf opposite} conclusions and expects to be considered an anarchist! Moreover, it seems equally ironic that “anarcho”-capitalism calls itself “anarchist” while basing itself on the arguments that anarchism was created in opposition to. As shown, “anarcho”-capitalism makes as much sense as “anarcho-statism” — an oxymoron, a contradiction in terms. The idea that “anarcho”-capitalism warrants the name “anarchist” is simply false. Only someone ignorant of anarchism could maintain such a thing. While you expect anarchist theory to show this to be the case, the wonderful thing is that “anarcho”-capitalism itself does the same.


Little wonder Bob Black argues that {\em “[t]o demonise state authoritarianism while ignoring identical albeit contract-consecrated subservient arrangements in the large-scale corporations which control the world economy is fetishism at its worst.”} [{\bf Libertarian as Conservative}] The similarities between capitalism and statism are clear — and so why “anarcho”-capitalism cannot be anarchist. To reject the authority (the {\em “ultimate decision-making power”}) of the state and embrace that of the property owner indicates not only a highly illogical stance but one at odds with the basic principles of anarchism. This whole-hearted support for wage labour and capitalist property rights indicates that “anarcho”-capitalists are not anarchists because they do not reject all forms of {\bf archy.} They obviously support the hierarchy between boss and worker (wage labour) and landlord and tenant. Anarchism, by definition, is against all forms of archy, including the hierarchy generated by capitalist property. To ignore the obvious archy associated with capitalist property is highly illogical.


In addition, we must note that such inequalities in power and wealth will need “defending” from those subject to them (“anarcho”-capitalists recognise the need for private police and courts to defend property from theft — and, anarchists add, to defend the theft and despotism associated with property!). Due to its support of private property (and thus authority), “anarcho”-capitalism ends up retaining a state in its “anarchy”; namely a {\bf private} state whose existence its proponents attempt to deny simply by refusing to call it a state, like an ostrich hiding its head in the sand (see section 6 for more on this and why “anarcho”-capitalism is better described as “private state” capitalism). As Albert Meltzer put it:



\startblockquote
{\em “Common-sense shows that any capitalist society might dispense with a ‘State’ \unknown{} but it could not dispense with organised government, or a privatised form of it, if there were people amassing money and others working to amass it for them. The philosophy of ‘anarcho-capitalism’ dreamed up by the ‘libertarian’ New Right, has nothing to do with Anarchism as known by the Anarchist movement proper. It is a lie \unknown{} Patently unbridled capitalism \unknown{} needs some force at its disposal to maintain class privileges, either form the State itself or from private armies. What they believe in is in fact a limited State — that us, one in which the State has one function, to protect the ruling class, does not interfere with exploitation, and comes as cheap as possible for the ruling class. The idea also serves another purpose \unknown{} a moral justification for bourgeois consciences in avoiding taxes without feeling guilty about it.”} [{\bf Anarchism: Arguments For and Against}, p. 50]



\stopblockquote
For anarchists, this need of capitalism for some kind of state is unsurprising. For {\em “Anarchy without socialism seems equally as impossible to us [as socialism without anarchy], for in such a case it could not be other than the domination of the strongest, and would therefore set in motion right away the organisation and consolidation of this domination; that is to the constitution of government.”} [Errico Malatesta, {\bf Life and Ideas}, p. 148] Because of this, the “anarcho”-capitalist rejection of anarchist ideas on capitalist property economics and the need for equality, they cannot be considered anarchists or part of the anarchist tradition.


Thus anarchism is far more than the common dictionary definition of “no government” — it also entails being against all forms of {\bf archy}, including those generated by capitalist property. This is clear from the roots of the word “anarchy.” As we noted in section A.1, the word anarchy means “no rulers” or “contrary to authority.” As Rothbard himself acknowledges, the property owner is the ruler of their property and, therefore, those who use it. For this reason “anarcho”-capitalism cannot be considered as a form of anarchism — a real anarchist must logically oppose the authority of the property owner along with that of the state. As “anarcho”-capitalism does not explicitly (or implicitly, for that matter) call for economic arrangements that will end wage labour and usury it cannot be considered anarchist or part of the anarchist tradition.


Political theories should be identified by their actual features and history rather than labels. Once we recognise that, we soon find out that “anarcho”-capitalism is an oxymoron. Anarchists and “anarcho”-capitalists are not part of the same movement or tradition. Their ideas and aims are in direct opposition to those of all kinds of anarchists.


While anarchists have always opposed capitalism, “anarcho”-capitalists have embraced it. And due to this embrace their “anarchy” will be marked by extensive differences in wealth and power, differences that will show themselves up in relationships based upon subordination and hierarchy (such as wage labour), {\bf not} freedom (little wonder that Proudhon argued that {\em “property is despotism”} — it creates authoritarian and hierarchical relationships between people in a similar way to statism).


Their support for “free market” capitalism ignores the impact of wealth and power on the nature and outcome of individual decisions within the market (see sections 2 and 3 for further discussion). For example, as we indicate in sections J.5.10, J.5.11 and J.5.12, wage labour is less efficient than self-management in production but due to the structure and dynamics of the capitalist market, “market forces” will actively discourage self-management due to its empowering nature for workers. In other words, a developed capitalist market will promote hierarchy and unfreedom in production in spite of its effects on individual workers and their wants (see also section 10.2). Thus “free market” capitalism tends to re-enforce inequalities of wealth and power, {\bf not} eliminate them.


Furthermore, any such system of (economic and social) power will require extensive force to maintain it and the “anarcho”-capitalist system of competing “defence firms” will simply be a new state, enforcing capitalist power, property rights and law.


Overall, the lack of concern for meaningful freedom within production and the effects of vast differences in power and wealth within society as a whole makes “anarcho”-capitalism little better than “anarchism for the rich.” Emma Goldman recognised this when she argued that {\em “‘Rugged individualism’ has meant all the ‘individualism’ for the masters \unknown{} in whose name political tyranny and social oppression are defended and held up as virtues while every aspiration and attempt of man to gain freedom \unknown{} is denounced as \unknown{} evil in the name of that same individualism.”} [{\bf Red Emma Speaks}, p. 112] And, as such, is no anarchism at all.


So, unlike anarchists, “anarcho”-capitalists do not seek the {\em “abolition of the proletariat”} (to use Proudhon’s expression) via changing capitalist property rights and institutions. Thus the “anarcho”-capitalist and the anarchist have different starting positions and opposite ends in mind and so they cannot be considered part of the same (anarchist) tradition. As we discuss further in later sections, the “anarcho”-capitalist claims to being anarchists are bogus simply because they reject so much of the anarchist tradition as to make what they do accept non-anarchist in theory and practice. Little wonder Peter Marshall said that {\em “few anarchists would accept the ‘anarcho-capitalists’ into the anarchist camp since they do not share a concern for economic equality and social justice.”} [{\bf Demanding the Impossible}, p. 565]


\subsection{1.1 Why is the failure to renounce hierarchy the Achilles Heel of right-wing libertarianism
}

Any capitalist system will produce vast differences in economic (and social) wealth and power. As we argue in section 3.1, such differences will reflect themselves in the market and any “free” contracts agreed there will create hierarchical relationships. Thus capitalism is marked by hierarchy (see section B.1.2) and, unsurprisingly, right-libertarians and “anarcho”-capitalists fail to oppose such “free market” generated hierarchy.


Both groups approve of it in the capitalist workplace or rented accommodation and the right-Libertarians also approve of it in a ‘minimal’ state to protect private property (“anarcho”-capitalists, in contrast, approve of the use of private defence firms to protect property). But the failure of these two movements to renounce hierarchy is their weakest point. For anti-authoritarianism has sunk deep roots into the modern psyche, as a legacy of the sixties.


Many people who do not even know what anarchism is have been profoundly affected by the personal liberation and counterculture movements of the past thirty years, epitomised by the popular bumper sticker, {\em “Question Authority.”} As a result, society now tolerates much more choice than ever before in matters of religion, sexuality, art, music, clothing, and other components of lifestyle. We need only recall the conservatism that reigned in such areas during the fifties to see that the idea of liberty has made tremendous advances in just a few decades.


Although this liberatory impulse has so far been confined almost entirely to the personal and cultural realms, it may yet be capable of spilling over and affecting economic and political institutions, provided it continues to grow. The Right is well aware of this, as seen in its ongoing campaigns for “family values,” school prayer, suppression of women’s rights, fundamentalist Christianity, sexual abstinence before marriage, and other attempts to revive the Ozzie-and-Harriet mindset of the Good Old Days. This is where the efforts of “cultural anarchists” — artists, musicians, poets, and others — are important in keeping alive the ideal of personal freedom and resistance to authority as a necessary foundation for economic and political restructuring.


Indeed, the libertarian right (as a whole) support restrictions on freedom {\bf as long as its not the state that is doing it}! Their support for capitalism means that they have no problem with bosses dictating what workers do during working hours (nor outside working hours, if the job requires employees to take drug tests or not be gay in order to keep it). If a private landlord or company decrees a mandatory rule or mode of living, workers/tenets must “love it or leave it!” Of course, that the same argument also applies to state laws is one hotly denied by right-Libertarians — a definite case of not seeing the wood for the trees (see section 2.3).


Of course, the “anarcho”-capitalist will argue, workers and tenants can find a more liberal boss or landlord. This, however, ignores two key facts. Firstly, being able to move to a more liberal state hardly makes state laws less offensive (as they themselves will be the first to point out). Secondly, looking for a new job or home is not that easy. Just a moving to a new state can involve drastic upheavals, so change changing jobs and homes. Moreover, the job market is usually a buyers market (it has to be in capitalism, otherwise profits are squeezed — see sections C.7 and 10.2) and this means that workers are not usually in a position (unless they organise) to demand increased liberties at work.


It seems somewhat ironic, to say the least, that right-libertarians place rights of property over the rights of self-ownership, even though (according to their ideology) self-ownership is the foundational right from which property rights are derived. Thus in right-libertarianism the rights of property owners to discriminate and govern the property-less are more important than the freedom from discrimination (i.e. to be yourself) or the freedom to govern oneself at all times.


So, when it boils down to it, right-libertarians are not really bothered about restrictions on liberty and, indeed, they will defend private restrictions on liberty with all their might. This may seem a strange position for self-proclaimed “libertarians” to take, but it flows naturally from their definition of freedom (see section 2 for a full discussion of this). but by not attacking hierarchy beyond certain forms of statism, the ‘libertarian’ right fundamentally undermines its claim to be libertarian. Freedom cannot be compartmentalised, but is holistic. The denial of liberty in, say, the workplace, quickly results in its being denied elsewhere in society (due to the impact of the inequalities it would produce) , just as the degrading effects of wage labour and the hierarchies with which is it bound up are felt by the worker outside work.


Neither the Libertarian Party nor so-called “anarcho”-capitalism is {\bf genuinely} anti-authoritarian, as those who are truly dedicated to liberty must be.


\subsection{1.2 How libertarian is right-Libertarian theory?
}

The short answer is, not very. Liberty not only implies but also requires independent, critical thought (indeed, anarchists would argue that critical thought requires free development and evolution and that it is precisely {\bf this} which capitalist hierarchy crushes). For anarchists a libertarian theory, if it is to be worthy of the name, must be based upon critical thought and reflect the key aspect that characterises life — change and the ability to evolve. To hold up dogma and base “theory” upon assumptions (as opposed to facts) is the opposite of a libertarian frame of mind. A libertarian theory must be based upon reality and recognise the need for change and the existence of change. Unfortunately, right-Libertarianism is marked more by ideology than critical analysis.


Right-Libertarianism is characterised by a strong tendency of creating theories based upon assumptions and deductions from these axioms (for a discussion on the pre-scientific nature of this methodology and of its dangers, see the next section). Robert Nozick, for example, in {\bf Anarchy, State, and Utopia} makes no attempt to provide a justification of the property rights his whole theory is based upon. His main assumption is that {\em “[i]ndividuals have rights, and there are certain things no person or group may do to them (without violating their rights).”} [{\bf Anarchy, State and Utopia}, p. ix] While this does have its intuitive appeal, it is not much to base a political ideology upon. After all, what rights people consider as valid can be pretty subjective and have constantly evolved during history. To say that “individuals have rights” is to open up the question “what rights?” Indeed, as we argue in greater length in section 2, such a rights based system as Nozick desires can and does lead to situations developing in which people “consent” to be exploited and oppressed and that, intuitively, many people consider supporting the “violation” of these “certain rights” (by creating other ones) simply because of their evil consequences.


In other words, starting from the assumption “people have [certain] rights” Nozick constructs a theory which, when faced with the reality of unfreedom and domination it would create for the many, justifies this unfreedom as an expression of liberty. In other words, regardless of the outcome, the initial assumptions are what matter. Nozick’s intuitive rights system can lead to some very non-intuitive outcomes.


And does Nozick prove the theory of property rights he assumes? He states that {\em “we shall not formulate [it] here.”} [{\bf Op. Cit.}, p. 150] Moreover, it is not formulated anywhere else in his book. And if it is not formulated, what is there to defend? Surely this means that his Libertarianism is without foundations? As Jonathan Wolff notes, Nozick’s {\em “Libertarian property rights remain substantially undefended.”} [{\bf Robert Nozick: Property, Justice and the Minimal State}, p. 117] Given that the right to acquire property is critical to his whole theory you would think it important enough to go into in some detail (or at least document). After all, unless he provides us with a firm basis for property rights then his entitlement theory is nonsense as no one has the right to (private) property.


It could be argued that Nozick {\bf does} present enough information to allow us to piece together a possible argument in favour of property rights based on his modification of the {\em “Lockean Proviso”} (although he does not point us to these arguments). However, assuming this is the case, such a defence actually fails (see section B.3.4 for more on this). If individuals {\bf do} have rights, these rights do not include property rights in the form Nozick assumes (but does not prove). Nozick appears initially convincing because what he assumes with regards to property is a normal feature of the society we are in (we would be forgiven when we note here that feeble arguments pass for convincing when they are on the same side as the prevailing sentiment).


Similarly, both Murray Rothbard and Ayn Rand (who is infamous for repeating {\em “A is A”} ad infinitum) do the same — base their ideologies on assumptions (see section 11 for more on this).


Therefore, we see that most of the leading right-Libertarian ideologues base themselves on assumptions about what “Man” is or the rights they should have (usually in the form that people have (certain) rights because they are people). From these theorems and assumptions they build their respective ideologies, using logic to deduce the conclusions that their assumptions imply. Such a methodology is unscientific and, indeed, a relic of religious (pre-scientific) society (see next section) but, more importantly, can have negative effects on maximising liberty. This is because this “methodology” has distinct problems. Murray Bookchin argues:



\startblockquote
{\em “Conventional reason rests on identity, not change; its fundamental principle is that {\bf A equals A,} the famous ‘principle of identity,’ which means that any given phenomenon can be only itself and cannot be other than what we immediately perceive it to be at a given moment in time. It does not address the problem of change. A human being is an infant at one time, a child at another, an adolescent at still another, and finally a youth and an adult. When we analyse an infant by means of conventional reason, we are not exploring what it is {\bf becoming} in the process of developing into a child.”} [{\em “A Philosophical Naturalism”}, {\bf Society and Nature} No.2, p. 64]



\stopblockquote
In other words, right-Libertarian theory is based upon ignoring the fundamental aspect of life — namely {\bf change} and {\bf evolution.} Perhaps it will be argued that identity also accounts for change by including potentiality — which means, that we have the strange situation that A can {\bf potentially} be A! If A is not actually A, but only has the potential to be A, then A is not A. Thus to include change is to acknowledge that A does not equal A — that individuals and humanity evolves and so what constitutes A also changes. To maintain identity and then to deny it seems strange.


That change is far from the “A is A” mentality can be seen from Murray Rothbard who goes so far as to state that {\em “one of the notable attributes of natural law”} is {\em “its applicability to all men [sic!], regardless of time or place. Thus ethical law takes its place alongside physical or ‘scientific’ natural laws.”} [{\bf The Ethics of Liberty}, p. 42] Apparently the “nature of man” is the only living thing in nature that does not evolve or change! Of course, it could be argued that by “natural law” Rothbard is only referring to his method of deducing his (and, we stress, they are just his — not natural) “ethical laws” — but his methodology starts by assuming certain things about “man.” Whether these assumptions seem far or not is besides the point, by using the term “natural law” Rothbard is arguing that any actions that violate {\bf his} ethical laws are somehow “against nature” (but if they were against nature, they could not occur — see section 11 for more on this). Deductions from assumptions is a Procrustean bed for humanity (as Rothbard’s ideology shows).


So, as can be seen, many leading right-Libertarians place great store by the axiom “A is A” or that “man” has certain rights simply because “he” is a “man”. And as Bookchin points out, such conventional reason {\em “doubtless plays an indispensable role in mathematical thinking and mathematical sciences \unknown{} and in the nuts-and-bolts of dealing with everyday life”} and so is essential to {\em “understand or design mechanical entities.”} [{\bf Ibid.}, p.67] But the question arises, is such reason useful when considering people and other forms of life?


Mechanical entities are but one (small) aspect of human life. Unfortunately for right-Libertarians (and fortunately for the rest of humanity), human beings are {\bf not} mechanical entities but instead are living, breathing, feeling, hoping, dreaming, {\bf changing} living organisms. They are not mechanical entities and any theory that uses reason based on such (non-living) entities will flounder when faced with living ones. In other words, right-Libertarian theory treats people as the capitalist system tries to — namely as commodities, as things. Instead of human beings, whose ideas, ideals and ethics change, develop and grow, capitalism and capitalist ideologues try to reduce human life to the level of corn or iron (by emphasising the unchanging “nature” of man and their starting assumptions/rights).


This can be seen from their support for wage labour, the reduction of human activity to a commodity on the market. While paying lip service to liberty and life, right-libertarianism justifies the commodification of labour and life, which within a system of capitalist property rights can result in the treating of people as means to an end as opposed to an end in themselves (see sections 2 and 3.1).


And as Bookchin points out, {\em “in an age of sharply conflicting values and emotionally charges ideals, such a way of reasoning is often repellent. Dogmatism, authoritarianism, and fear seem all-pervasive.”} [{\bf Ibid.}, p. 68] Right-Libertarianism provides more than enough evidence for Bookchin’s summary with its support for authoritarian social relationships, hierarchy and even slavery (see section 2).


This mechanical viewpoint is also reflected in their lack of appreciation that social institutions and relationships evolve over time and, sometimes, fundamentally change. This can best be seen from property. Right-libertarians fail to see that over time (in the words of Proudhon) property {\em “changed its nature.”} Originally, {\em “the word {\bf property} was synonymous with \unknown{} {\bf individual possession}”} but it became more {\em “complex”} and turned into {\bf private property} — {\em “the right to use it by his neighbour’s labour.”} The changing of use-rights to (capitalist) property rights created relations of domination and exploitation between people absent before. For the right-Libertarian, both the tools of the self-employed artisan and the capital of a transnational corporation are both forms of “property” and (so) basically identical. In practice, of course, the social relations they create and the impact they have on society are totally different. Thus the mechanical mind-set of right-Libertarianism fails to understand how institutions, like property, evolve and come to replace whatever freedom enhancing features they had with oppression (indeed, von Mises argued that {\em “[t]here may possibly be a difference of opinion about whether a particular institution is socially beneficial or harmful. But once it has been judged [by whom, we ask] beneficial, one can no longer contend that, for some inexplicable reason, it must be condemned as immoral”} [{\bf Liberalism}, p. 34] So much for evolution and change!).


Anarchism, in contrast, is based upon the importance of critical thought informed by an awareness that life is in a constant process of change. This means that our ideas on human society must be informed by the facts, not by what we wish was true. For Bookchin, an evaluation of conventional wisdom (as expressed in {\em “the law of identity”}) is essential and its conclusions have {\em “enormous importance for how we behave as ethical beings, the nature of nature, and our place in the natural world. Moreover\unknown{} these issues directly affect the kind of society, sensibility, and lifeways we wish to foster.”} [Bookchin, {\bf Op. Cit.}, p. 69–70]


Bookchin is correct. While anarchists oppose hierarchy in the name of liberty, right-libertarians support authority and hierarchy, all of which deny freedom and restrict individual development. This is unsurprising because the right-libertarian ideology rejects change and critical thought based upon the scientific method and so is fundamentally {\bf anti-life} in its assumptions and {\bf anti-human} in its method. Far from being a libertarian set of ideas, right-Libertarianism is a mechanical set of dogmas that deny the fundamental nature of life (namely change) and of individuality (namely critical thought and freedom). Moreover, in practice their system of (capitalist) rights would soon result in extensive restrictions on liberty and authoritarian social relationships (see sections 2 and 3) — a strange result of a theory proclaiming itself “libertarian” but one consistent with its methodology.


From a wider viewpoint, such a rejection of liberty by right-libertarians is unsurprising. They do, after all, support capitalism. Capitalism produces an inverted set of ethics, one in which capital (dead labour) is more important that people (living labour). After all, workers are usually easier to replace than investments in capital and the person who owns capital commands the person who “only” owns his life and productive abilities. And as Oscar Wilde once noted, crimes against property {\em “are the crimes that the English law, valuing what a man has more than what a man is, punishes with the harshest and most horrible severity.”} [{\bf The Soul of Man Under Socialism}]


This mentality is reflected in right-libertarianism when it claims that stealing food is a crime while starving to death (due to the action of market forces/power and property rights) is no infringement of your rights (see section 4.2 for a similar argument with regards to water). It can also be seen when right-libertarian’s claim that the taxation {\em “of earnings from labour”} (e.g. of one dollar from a millionaire) is {\em “{\bf on a par with} forced labour”} [Nozick, {\bf Op. Cit.}, p. 169] while working in a sweatshop for 14 hours a day (enriching said millionaire) does not affect your liberty as you “consent” to it due to market forces (although, of course, many rich people have earned their money {\bf without} labouring themselves — their earnings derive from the wage labour of others so would taxing those, non-labour, earnings be “forced labour”?) Interestingly, the Individualist Anarchist Ben Tucker argued that an income tax was {\em “a recognition of the fact that industrial freedom and equality of opportunity no longer exist here [in the USA in the 1890s] even in the imperfect state in which they once did exist”} [quoted by James Martin, {\bf Men Against the State}, p. 263] which suggests a somewhat different viewpoint on this matter than Nozick or Rothbard.


That capitalism produces an inverted set of ethics can be seen when the Ford produced the Pinto. The Pinto had a flaw in it which meant that if it was hit in a certain way in a crash the fuel tank exploded. The Ford company decided it was more “economically viable” to produce that car and pay damages to those who were injured or the relatives of those who died than pay to change the invested capital. The needs for the owners of capital to make a profit came before the needs of the living. Similarly, bosses often hire people to perform unsafe work in dangerous conditions and fire them if they protest. Right-libertarian ideology is the philosophical equivalent. Its dogma is “capital” and it comes before life (i.e. “labour”).


As Bakunin once put it, {\em “you will always find the idealists in the very act of practical materialism, while you will see the materialists pursuing and realising the most grandly ideal aspirations and thoughts.”} [{\bf God and the State}, p. 49] Hence we see right “libertarians” supporting sweat shops and opposing taxation — for, in the end, money (and the power that goes with it) counts far more in that ideology than ideals such as liberty, individual dignity, empowering, creative and productive work and so forth for all. The central flaw of right-libertarianism is that it does not recognise that the workings of the capitalist market can easily ensure that the majority end up becoming a resource for others in ways far worse than that associated with taxation. The legal rights of self-ownership supported by right-libertarians does not mean that people have the ability to avoid what is in effect enslavement to another (see sections 2 and 3).


Right-Libertarian theory is not based upon a libertarian methodology or perspective and so it is hardly surprising it results in support for authoritarian social relationships and, indeed, slavery (see section 2.6).


\subsection{1.3 Is right-Libertarian theory scientific in nature?
}

Usually, no. The scientific approach is {\bf inductive,} much of the right-libertarian approach is {\bf deductive.} The first draws generalisations from the data, the second applies preconceived generalisations to the data. A completely deductive approach is pre-scientific, however, which is why many right-Libertarians cannot legitimately claim to use a scientific method. Deduction does occur in science, but the generalisations are primarily based on other data, not {\em a priori} assumptions, and are checked against data to see if they are accurate. Anarchists tend to fall into the inductive camp, as Kropotkin put it:



\startblockquote
{\em “Precisely this natural-scientific method applied to economic facts, enables us to prove that the so-called ‘laws’ of middle-class sociology, including also their political economy, are not laws at all, but simply guesses, or mere assertions which have never been verified at all.”} [{\bf Kropotkin’s Revolutionary Pamphlets}, p. 153]



\stopblockquote
The idea that natural-scientific methods can be applied to economic and social life is one that many right-libertarians reject. Instead they favour the deductive (pre-scientific) approach (this we must note is not limited purely to Austrian economists, many more mainstream capitalist economists also embrace deduction over induction).


The tendency for right-Libertarianism to fall into dogmatism (or {\em a priori} theorems, as they call it) and its implications can best be seen from the work of Ludwig von Mises and other economists from the right-Libertarian “Austrian school.” Of course, not all right-libertarians necessarily subscribe to this approach (Murray Rothbard for one did) but its use by so many leading lights of both schools of thought is significant and worthy of comment. And as we are concentrating on {\bf methodology} it is not essential to discuss the starting assumptions. The assumptions (such as, to use Rothbard’s words, the Austrian’s {\em “fundamental axiom that individual human beings act”}) may be correct, incorrect or incomplete — but the method of using them advocated by von Mises ensures that such considerations are irrelevant.


Von Mises (a leading member of the Austrian school of economics) begins by noting that social and economic theory {\em “is not derived from experience; it is prior to experience\unknown{}”} Which is back to front. It is obvious that experience of capitalism is necessary in order to develop a viable theory about how it works. Without the experience, any theory is just a flight of fantasy. The actual specific theory we develop is therefore derived from experience, informed by it and will have to get checked against reality to see if it is viable. This is the scientific method — any theory must be checked against the facts. However, von Mises goes on to argue at length that {\em “no kind of experience can ever force us to discard or modify {\bf a priori} theorems; they are logically prior to it and cannot be either proved by corroborative experience or disproved by experience to the contrary \unknown{}”}


And if this does not do justice to a full exposition of the phantasmagoria of von Mises’ {\em a priorism}, the reader may take some joy (or horror) from the following statement:



\startblockquote
{\em “If a contradiction appears between a theory and experience, {\bf we must always assume} that a condition pre-supposed by the theory was not present, or else there is some error in our observation. The disagreement between the theory and the facts of experience frequently forces us to think through the problems of the theory again. {\bf But so long as a rethinking of the theory uncovers no errors in our thinking, we are not entitled to doubt its truth}”} [emphasis added — the quotes presented here are cited in {\bf Ideology and Method in Economics} by Homa Katouzian, pp. 39–40]



\stopblockquote
In other words, if reality is in conflict with your ideas, do not adjust your views because reality must be at fault! The scientific method would be to revise the theory in light of the facts. It is not scientific to reject the facts in light of the theory! This anti-scientific perspective is at the heart of his economics as experience {\em “can never \unknown{} prove or disprove any particular theorem”}:



\startblockquote
{\em  “What assigns economics to its peculiar and unique position in the orbit of pure knowledge and of the practical utilisation of knowledge is the fact that its particular theorems are not open to any verification or falsification on the grounds of experience \unknown{}.. The ultimate yardstick of an economic theorem’s correctness or incorrectness is solely reason unaided by experience.”} [{\bf Human Action}, p. 858]



\stopblockquote
Von Mises rejects the scientific approach as do all Austrian Economists. Murray Rothbard states approvingly that {\em “Mises indeed held not only that economic theory does not need to be ‘tested’ by historical fact but also that it {\bf cannot} be so tested.”} [{\em “Praxeology: The Methodology of Austrian Economics”} in {\bf The Foundation of Modern Austrian Economics}, p. 32] Similarly, von Hayek wrote that economic theories can {\em “never be verified or falsified by reference to facts. All that we can and must verify is the presence of our assumptions in the particular case.”} [{\bf Individualism and Economic Order}, p. 73]


This may seen somewhat strange to non-Austrians. How can we ignore reality when deciding whether a theory is a good one or not? If we cannot evaluate our ideas, how can we consider them anything bar dogma? The Austrian’s maintain that we cannot use historical evidence because every historical situation is unique. Thus we cannot use {\em “complex heterogeneous historical facts as if they were repeatable homogeneous facts”} like those in a scientist’s experiment [Rothbard, {\bf Op. Cit.}, p. 33]. While such a position {\bf does} have an element of truth about it, the extreme {\em a priorism} that is drawn from this element is radically false (just as extreme empiricism is also false, but for different reasons).


Those who hold such a position ensure that their ideas cannot be evaluated beyond logical analysis. As Rothbard makes clear, {\em “since praxeology begins with a true axiom, A, all that can be deduced from this axiom must also be true. For if A implies be, and A is true, then B must also be true.”} [{\bf Op. Cit.}, pp. 19–20] But such an approach makes the search for truth a game without rules. The Austrian economists (and other right-libertarians) who use this method are free to theorise anything they want, without such irritating constrictions as facts, statistics, data, history or experimental confirmation. Their only guide is logic. But this is no different from what religions do when they assert the logical existence of God. Theories ungrounded in facts and data are easily spun into any belief a person wants. Starting assumptions and trains of logic may contain inaccuracies so small as to be undetectable, yet will yield entirely false conclusions.


In addition, trains of logic may miss things which are only brought to light by actual experiences (after all, the human mind is not all knowing or all seeing). To ignore actual experience is to loose that input when evaluating a theory. Hence our comments on the irrelevance of the assumptions used — the methodology is such that incomplete or incorrect assumptions or steps cannot be identified in light of experience. This is because one way of discovering if a given chain of logic requires checking is to test its conclusions against available evidence (although von Mises did argue that the {\em “ultimate yardstick”} was {\em “solely reason unaided by experience”}). If we {\bf do} take experience into account and rethink a given theory in the light of contradictory evidence, the problem remains that a given logical chain may be correct, but incomplete or concentrate on or stress inappropriate factors. In other words, our logical deductions may be correct but our starting place or steps wrong and as the facts are to be rejected in the light of the deductive method, we cannot revise our ideas.


Indeed, this approach could result in discarding (certain forms of) human behaviour as irrelevant (which the Austrian system claims using empirical evidence does). For there are too many variables that can have an influence upon individual acts to yield conclusive results explaining human behaviour. Indeed, the deductive approach may ignore as irrelevant certain human motivations which have a decisive impact on an outcome. There could be a strong tendency to project “right-libertarian person” onto the rest of society and history, for example, and draw inappropriate insights into the way human society works or has worked. This can be seen, for example, in attempts to claim pre-capitalist societies as examples of “anarcho”-capitalism in action.


Moreover, deductive reasoning cannot indicate the relative significance of assumptions or theoretical factors. That requires empirical study. It could be that a factor considered important in the theory actually turns out to have little effect in practice and so the derived axioms are so weak as to be seriously misleading.


In such a purely ideal realm, observation and experience are distrusted (when not ignored) and instead theory is the lodestone. Given the bias of most theorists in this tradition, it is unsurprising that this style of economics can always be trusted to produce results proving free markets to be the finest principle of social organisation. And, as an added bonus, reality can be ignored as it is {\bf never} “pure” enough according to the assumptions required by the theory. It could be argued, because of this, that many right-libertarians insulate their theories from criticism by refusing to test them or acknowledge the results of such testing (indeed, it could also be argued that much of right-libertarianism is more a religion than a political theory as it is set-up in such a way that it is either true or false, with this being determined not by evaluating facts but by whether you accept the assumptions and logical chains presented with them).


Strangely enough, while dismissing the “testability” of theories many right-Libertarians (including Murray Rothbard) {\bf do} investigate historical situations and claim them as examples of how well their ideas work in practice. But why does historical fact suddenly become useful when it can be used to bolster the right-Libertarian argument? Any such example is just as “complex” as any other and the good results indicated may not be accountable to the assumptions and steps of the theory but to other factors totally ignored by it. If economic (or other) theory is untestable then {\bf no} conclusions can be drawn from history, including claims for the superiority of laissez-faire capitalism. You cannot have it both ways — although we doubt that right-libertarians will stop using history as evidence that their ideas work.


Perhaps the Austrian desire to investigate history is not so strange after all. Clashes with reality make a-priori deductive systems implode as the falsifications run back up the deductive changes to shatter the structure built upon the original axioms. Thus the desire to find {\bf some} example which proves their ideology must be tremendous. However, the deductive a-priori methodology makes them unwilling to admit to being mistaken — hence their attempts to downplay examples which refute their dogmas. Thus we have the desire for historical examples while at the same time they have extensive ideological justifications that ensure reality only enters their world-view when it agrees with them. In practice, the latter wins as real-life refuses to be boxed into their dogmas and deductions.


Of course it is sometimes argued that it is {\bf complex} data that is the problem. Let use assume that this is the case. It is argued that when dealing with complex information it is impossible to use aggregate data without first having more simple assumptions (i.e. that “humans act”). Due to the complexity of the situation, it is argued, it is impossible to aggregate data because this hides the individual activities that creates it. Thus “complex” data cannot be used to invalidate assumptions or theories. Hence, according to Austrians, the axioms derived from the “simple fact” that “humans act” are the only basis for thinking about the economy.


Such a position is false in two ways.


Firstly, the aggregation of data {\bf does} allow us to understand complex systems. If we look at a chair, we cannot find out whether it is comfortable, its colour, whether it is soft or hard by looking at the atoms that make it up. To suggest that you can is to imply the existence of green, soft, comfortable atoms. Similarly with gases. They are composed to countless individual atoms but scientists do not study them by looking at those atoms and their actions. Within limits, this is also valid for human action. For example, it would be crazy to maintain from historical data that interest rates will be a certain percentage a week but it is valid to maintain that interest rates are known to be related to certain variables in certain ways. Or that certain experiences will tend to result in certain forms of psychological damage. General tendencies and “rules of thumb” can be evolved from such study and these can be used to {\bf guide} current practice and theory. By aggregating data you can produce valid information, rules of thumb, theories and evidence which would be lost if you concentrated on “simple data” (such as “humans act”). Therefore, empirical study produces facts which vary across time and place, and yet underlying and important patterns can be generated (patterns which can be evaluated against {\bf new} data and improved upon).


Secondly, the simple actions themselves influence and are influenced in turn by overall (complex) facts. People act in different ways in different circumstances (something we can agree with Austrians about, although we refuse to take it to their extreme position of rejecting empirical evidence as such). To use simple acts to understand complex systems means to miss the fact that these acts are not independent of their circumstances. For example, to claim that the capitalist market is “just” the resultant of bilateral exchanges ignores the fact that the market activity shapes the nature and form of these bilateral exchanges. The “simple” data is dependent on the “complex” system — and so the complex system {\bf cannot} be understood by looking at the simple actions in isolation. To do so would be to draw incomplete and misleading conclusions (and it is due to these interrelations that we argue that aggregate data should be used critically). This is particularly important when looking at capitalism, where the “simple” acts of exchange in the labour market are dependent upon and shaped by circumstances outside these acts.


So to claim that (complex) data cannot be used to evaluate a theory is false. Data can be useful when seeing whether a theory is confirmed by reality. This is the nature of the scientific method — you compare the results expected by your theory to the facts and if they do not match you check your facts {\bf and} check your theory. This may involve revising the assumptions, methodology and theories you use if the evidence is such as to bring them into question. For example, if you claim that capitalism is based on freedom but that the net result of capitalism is to produce relations of domination between people then it would be valid to revise, for example, your definition of freedom rather than deny that domination restricts freedom (see section 2 on this). But if actual experience is to be distrusted when evaluating theory, we effectively place ideology above people — after all, how the ideology affects people in {\bf practice} is irrelevant as experiences cannot be used to evaluate the (logically sound but actually deeply flawed) theory.


Moreover, there is a slight arrogance in the “Austrian” dismissal of empirical evidence. If, as they argue, the economy is just too complex to allow us to generalise from experience then how can one person comprehend it sufficiently to create an economic ideology as the Austrian’s suggest? Surely no one mind (or series of minds) can produce a model which accurately reflects such a complex system? To suggest that one can deduce a theory for an exceedingly complex social system from the theoretical work based on an analysis technique which deliberately ignores that reality as being unreliable seems to require a deliberate suspension of one’s reasoning faculties. Of course, it may be argued that such a task is possible, given a small enough subset of economic activity. However, such a process is sure to lead its practitioners astray as the subset is not independent of the whole and, consequently, can be influenced in ways the ideologist does not (indeed, cannot) take into account. Simply put, even the greatest mind cannot comprehend the complexities of real life and so empirical evidence needs to inform any theory seeking to describe and explain it. To reject it is simply to retreat into dogmatism and ideology, which is precisely what right-wing libertarians generally do.


Ultimately, this dismissal of empirical evidence seems little more than self-serving. It’s utility to the ideologist is obvious. It allows them to speculate to their hearts content, building models of the economy with no bearing to reality. Their models and the conclusions it generates need never be bothered with reality — nor the effects of their dogma. Which shows its utility to the powerful. It allows them to spout comments like “the free market benefits all” while the rich get richer and allows them to brush aside any one who points out such troublesome facts.


That this position is self-serving can be seen from the fact that most right libertarians are very selective about applying von Mises’ argument. As a rule of thumb, it is only applied when the empirical evidence goes against capitalism. In such circumstances the fact that the current system is not a free market will also be mentioned. However, if the evidence seems to bolster the case for propertarianism then empirical evidence becomes all the rage. Needless to say, the fact that we do not have a free market will be conveniently forgotten. Depending on the needs of the moment, fundamental facts are dropped and retrieved to bolster the ideology.


As we indicated above (in section 1.2) and will discuss in more depth later (in section 11) most of the leading right-Libertarian theorists base themselves on such deductive methodologies, starting from assumptions and “logically” drawing conclusions from them. The religious undertones of such methodology can best be seen from the roots of right-Libertarian “Natural law” theory.


Carole Pateman, in her analysis of Liberal contract theory, indicates the religious nature of the “Natural Law” argument so loved by the theorists of the “Radical Right.” She notes that for Locke (the main source of the Libertarian Right’s Natural Law cult) {\em “natural law”} was equivalent of {\em “God’s Law”} and that {\em “God’s law exists externally to and independently of individuals.”} [{\bf The Problem of Political Obligation}, p. 154] No role for critical thought there, only obedience. Most modern day “Natural Law” supporters forget to mention this religious undercurrent and instead talk of about “Nature” (or “the market”) as the deity that creates Law, not God, in order to appear “rational.” So much for science.


Such a basis in dogma and religion can hardly be a firm foundation for liberty and indeed “Natural Law” is marked by a deep authoritarianism:



\startblockquote
{\em “Locke’s traditional view of natural law provided individual’s with an external standard which they could recognise, but which they did not voluntarily choose to order their political life.”} [Pateman, {\bf Op. Cit.}, p. 79]



\stopblockquote
In section 11 we discuss the authoritarian nature of “Natural Law” and will not do so here. However, here we must point out the political conclusions Locke draws from his ideas. In Pateman’s words, Locke believed that {\em “obedience lasts only as long as protection. His individuals are able to take action themselves to remedy their political lot\unknown{} but this does not mean, as is often assumed, that Locke’s theory gives direct support to present-day arguments for a right of civil disobedience\unknown{} His theory allows for two alternatives only: either people go peacefully about their daily affairs under the protection of a liberal, constitutional government, or they are in revolt against a government which has ceased to be ‘liberal’ and has become arbitrary and tyrannical, so forfeiting its right to obedience.”} [{\bf Op. Cit.}, p. 77]


Locke’s “rebellion” exists purely to reform a {\bf new} ‘liberal’ government, not to change the existing socio-economic structure which the ‘liberal’ government exists to protect. His theory, therefore, indicates the results of a priorism, namely a denial of any form of social dissent which may change the “natural law” as defined by Locke. This perspective can be found in Rothbard who lambasted the individualist anarchists for arguing that juries should judge the law as well as the facts. For Rothbard, the law would be drawn up by jurists and lawyers, not ordinary people (see section 1.4 for details). The idea that those subject to laws should have a say in forming them is rejected in favour of elite rule. As von Mises put it:



\startblockquote
{\em  “The flowering of human society depends on two factors: the intellectual power of outstanding men to conceive sound social and economic theories, and the ability of these or other men to make these ideologies palatable to the majority.”} [{\bf Human Action}, p. 864]



\stopblockquote
Yet such a task would require massive propaganda work and would only, ultimately, succeed by removing the majority from any say in the running of society. Once that is done then we have to believe that the ruling elite will be altruistic in the extreme and not abuse their position to create laws and processes which defended what {\bf they} thought was “legitimate” property, property rights and what constitutes “aggression.” Which, ironically, contradicts the key capitalist notion that people are driven by self-gain. The obvious conclusion from such argument is that any right-libertarian regime would have to exclude change. If people can change the regime they are under they may change it in ways that right libertarian’s do not support. The provision for ending amendments to the regime or the law would effectively ban most opposition groups or parties as, by definition, they could do nothing once in office (for minimal state “libertarians”) or in the market for “defence” agencies (for “anarcho”-capitalists). How this differs from a dictatorship is hard to say — after all, most dictatorships have parliamentary bodies which have no power but which can talk a lot. Perhaps the knowledge that it is {\bf private} police enforcing {\bf private} power will make those subject to the regime maximise their utility by keeping quiet and not protesting. Given this, von Mises’ praise for fascism in the 1920s may be less contradictory than it first appears (see section 6.5) as it successfully “deterred democracy” by crushing the labour, socialist and anarchist movements across the world.


So, von Mises, von Hayek and most right-libertarians reject the scientific method in favour of ideological correctness — if the facts contradict your theory then they can be dismissed as too “complex” or “unique”. Facts, however, should inform theory and any theory’s methodology should take this into account. To dismiss facts out of hand is to promote dogma. This is not to suggest that a theory should be modified very time new data comes along — that would be crazy as unique situations {\bf do} exist, data can be wrong and so forth — but it does suggest that if your theory {\bf continually} comes into conflict with reality, its time to rethink the theory and not assume that facts cannot invalidate it. A true libertarian would approach a contradiction between reality and theory by evaluating the facts available and changing the theory is this is required, not by ignoring reality or dismissing it as “complex”.


Thus, much of right-Libertarian theory is neither libertarian nor scientific. Much of right-libertarian thought is highly axiomatic, being logically deduced from such starting axioms as {\em “self-ownership”} or {\em “no one should initiate force against another”}. Hence the importance of our discussion of von Mises as this indicates the dangers of this approach, namely the tendency to ignore/dismiss the consequences of these logical chains and, indeed, to justify them in terms of these axioms rather than from the facts. In addition, the methodology used is such as that it would be fair to argue that right-libertarians get to critique reality but reality can never be used to critique right-libertarianism — for any empirical data presented as evidence as be dismissed as “too complex” or “unique” and so irrelevant (unless it can be used to support their claims, of course).


Hence W. Duncan Reekie’s argument (quoting leading Austrian economist Israel Kirzner) that {\em “empirical work ‘has the function of establishing the {\bf applicability} of particular theorems, and thus {\bf illustrating} their operation’ \unknown{} Confirmation of theory is not possible because there is no constants in human action, nor is it necessary because theorems themselves describe relationships logically developed from hypothesised conditions. Failure of a logically derived axiom to fit the facts does not render it invalid, rather it ‘might merely indicate inapplicability’ to the circumstances of the case.’”} [{\bf Markets, Entrepreneurs and Liberty}, p. 31]


So, if facts confirm your theory, your theory is right. If facts do not confirm your theory, it is still right but just not applicable in this case! Which has the handy side effect of ensuring that facts can {\bf only} be used to support the ideology, {\bf never} to refute it (which is, according to this perspective, impossible anyway). As Karl Popper argued, a {\em “theory which is not refutable by any conceivable event is non-scientific.”} [{\bf Conjectures and Refutations}, p. 36] In other words (as we noted above), if reality contradicts your theory, ignore reality!


Kropotkin hoped {\em “that those who believe in [current economic doctrines] will themselves become convinced of their error as soon as they come to see the necessity of verifying their quantitative deductions by quantitative investigation.”} [{\bf Op. Cit.}, p. 178] However, the Austrian approach builds so many barriers to this that it is doubtful that this will occur. Indeed, right-libertarianism, with its focus on exchange rather than its consequences, seems to be based upon justifying domination in terms of their deductions than analysing what freedom actually means in terms of human existence (see section 2 for a fuller discussion).


The real question is why are such theories taken seriously and arouse such interest. Why are they not simply dismissed out of hand, given their methodology and the authoritarian conclusions they produce? The answer is, in part, that feeble arguments can easily pass for convincing when they are on the same side as the prevailing sentiment and social system. And, of course, there is the utility of such theories for ruling elites — {\em “[a]n ideological defence of privileges, exploitation, and private power will be welcomed, regardless of its merits.”} [Noam Chomsky, {\bf The Chomsky Reader}, p. 188]


\subsection{ 1.4 Is “anarcho”-capitalism a new form of individualist anarchism?
}

Some “anarcho”-capitalists shy away from the term, preferring such expressions as “market anarchist” or “individualist anarchist.” This suggests that there is some link between their ideology and that of Tucker. However, the founder of “anarcho”-capitalism, Murray Rothbard, refused that label for, while {\em “strongly tempted,”} he could not do so because {\em “Spooner and Tucker have in a sense pre-empted that name for their doctrine and that from that doctrine I have certain differences.”} Somewhat incredibly Rothbard argued that on the whole politically {\em “these differences are minor,”} economically {\em “the differences are substantial, and this means that my view of the consequences of putting our more of less common system into practice is very far from theirs.”} [{\em “The Spooner-Tucker Doctrine: An Economist’s View”}, {\bf Journal of Libertarian Studies}, vol. 20, no. 1, p. 7]


What an understatement! Individualist anarchists advocated an economic system in which there would have been very little inequality of wealth and so of power (and the accumulation of capital would have been minimal without profit, interest and rent). Removing this social and economic basis would result in {\bf substantially} different political regimes. This can be seen from the fate of Viking Iceland, where a substantially communal and anarchistic system was destroyed from within by increasing inequality and the rise of tenant farming (see section 9 for details). In other words, politics is not isolated from economics. As David Wieck put it, Rothbard {\em “writes of society as though some part of it (government) can be extracted and replaced by another arrangement while other things go on before, and he constructs a system of police and judicial power without any consideration of the influence of historical and economic context.”} [{\em “Anarchist Justice,”} in {\bf Nomos XIX}, Pennock and Chapman, eds., p. 227]


Unsurprisingly, the political differences he highlights {\bf are} significant, namely {\em “the role of law and the jury system”} and {\em “the land question.”} The former difference relates to the fact that the individualist anarchists {\em “allow[ed] each individual free-market court, and more specifically, each free-market jury, totally free rein over judicial decision.”} This horrified Rothbard. The reason is obvious, as it allows real people to judge the law as well as the facts, modifying the former as society changes and evolves. For Rothbard, the idea that ordinary people should have a say in the law is dismissed. Rather, {\em “it would not be a very difficult task for Libertarian lawyers and jurists to arrive at a rational and objective code of libertarian legal principles and procedures.”} [{\bf Op. Cit.}, p. 7–8] Of course, the fact that {\em “lawyers”} and {\em “jurists”} may have a radically different idea of what is just than those subject to their laws is not raised by Rothbard, never mind answered. While Rothbard notes that juries may defend the people against the state, the notion that they may defend the people against the authority and power of the rich is not even raised. That is why the rich have tended to oppose juries as well as popular assemblies.


Unsurprisingly, the few individualist anarchists that remained pointed this out. Laurance Labadie, the son of Tucker associate Joseph Labadie, argued in response to Rothbard as follows:



\startblockquote
{\em “Mere common sense would suggest that any court would be influenced by experience; and any free-market court or judge would in the very nature of things have some precedents guiding them in their instructions to a jury. But since no case is exactly the same, a jury would have considerable say about the heinousness of the offence in each case, realising that circumstances alter cases, and prescribing penalty accordingly. This appeared to Spooner and Tucker to be a more flexible and equitable administration of justice possible or feasible, human beings being what they are\unknown{}}


{\em “But when Mr. Rothbard quibbles about the jurisprudential ideas of Spooner and Tucker, and at the same time upholds {\bf presumably in his courts} the very economic evils which are at bottom the very reason for human contention and conflict, he would seem to be a man who chokes at a gnat while swallowing a camel.”} [quoted by Mildred J. Loomis and Mark A. Sullivan, {\em “Laurance Labadie: Keeper Of The Flame”}, pp. 116–30, {\bf Benjamin R. Tucker and the Champions of Liberty}, Coughlin, Hamilton and Sullivan (eds.), p. 124]



\stopblockquote
In other words, to exclude the general population from any say in the law and how it changes is hardly a {\em “minor”} difference! Particularly if you are proposing an economic system which is based on inequalities of wealth, power and influence and the means of accumulating more. It is like a supporter of the state saying that it is a {\em “minor”} difference if you favour a dictatorship rather than a democratically elected government. As Tucker argued, {\em “it is precisely in the tempering of the rigidity of enforcement that one of the chief excellences of Anarchism consists \unknown{} under Anarchism all rules and laws will be little more than suggestions for the guidance of juries, and that all disputes \unknown{} will be submitted to juries which will judge not only the facts but the law, the justice of the law, its applicability to the given circumstances, and the penalty or damage to be inflicted because of its infraction \unknown{} under Anarchism the law \unknown{} will be regarded as {\bf just} in proportion to its flexibility, instead of now in proportion to its rigidity.”} [{\bf The Individualist Anarchists}, pp. 160–1] In others, the law will evolve to take into account changing social circumstances and, as a consequence, public opinion on specific events and rights. Tucker’s position is fundamentally {\bf democratic} and evolutionary while Rothbard’s is autocratic and fossilised.


On the land question, Rothbard opposed the individualist position of “occupancy and use” as it {\em “would automatically abolish all rent payments for land.”} Which was {\bf precisely} why the individualist anarchists advocated it! In a predominantly rural economy, this would result in a significant levelling of income and social power as well as bolstering the bargaining position of non-land workers by reducing unemployment. He bemoans that landlords cannot charge rent on their {\em “justly-acquired private property”} without noticing that is begging the question as anarchists deny that this is {\em “justly-acquired”} land. Unsurprising, Rothbard considers {\em “the property theory”} of land ownership as John Locke’s, ignoring the fact that the first self-proclaimed anarchist book was written to refute that kind of theory. His argument simply shows how far from anarchism his ideology is. For Rothbard, it goes without saying that the landlord’s {\em “freedom of contract”} tops the worker’s freedom to control their own work and live and, of course, their right to life. [{\bf Op. Cit.}, p. 8 and p. 9] However, for anarchists, {\em “the land is indispensable to our existence, consequently a common thing, consequently insusceptible of appropriation.”} [Proudhon, {\bf What is Property?}, p. 107]


The reason question is why Rothbard considers this a {\bf political} difference rather than an economic one. Unfortunately, he does not explain. Perhaps because of the underlying {\bf socialist} perspective behind the anarchist position? Or perhaps the fact that feudalism and monarchism was based on the owner of the land being its ruler suggests a political aspect to the ideology best left unexplored? Given that the idea of grounding rulership on land ownership receded during the Middle Ages, it may be unwise to note that under “anarcho”-capitalism the landlord and capitalist would, likewise, be sovereign over the land {\bf and} those who used it? As we noted in section 1, this is the conclusion that Rothbard does draw. As such, there {\bf is} a political aspect to this difference.


Moreover. {\em “the expropriation of the mass of the people from the soil forms the basis of the capitalist mode of production.”} [Marx, {\bf Capital}, vol. 1, p. 934] For there are {\em “two ways of oppressing men: either directly by brute force, by physical violence; or indirectly by denying them the means of life and this reducing them to a state of surrender.”} In the second case, government is {\em “an organised instrument to ensure that dominion and privilege will be in the hands of those who \unknown{} have cornered all the means of life, first and foremost the land, which they make use of to keep the people in bondage and to make them work for their benefit.”} [Malatesta, {\bf Anarchy}, p. 21] Privatising the coercive functions of said government hardly makes much difference.


Of course, Rothbard is simply skimming the surface. There are two main ways “anarcho”-capitalists differ from individualist anarchists. The first one is the fact that the individualist anarchists are socialists. The second is on whether equality is essential or not to anarchism. Each will be discussed in turn.


Unlike both Individualist (and social) anarchists, “anarcho”-capitalists support capitalism (a “pure” free market type, which has never existed although it has been approximated occasionally). This means that they reject totally the ideas of anarchists with regards to property and economic analysis. For example, like all supporters of capitalists they consider rent, profit and interest as valid incomes. In contrast, all Anarchists consider these as exploitation and agree with the Individualist Anarchist Benjamin Tucker when he argued that {\em “{\bf [w]hoever} contributes to production is alone entitled. {\bf What} has no rights that {\bf who} is bound to respect. {\bf What} is a thing. {\bf Who} is a person. Things have no claims; they exist only to be claimed. The possession of a right cannot be predicted of dead material, but only a living person.”}[quoted by Wm. Gary Kline, {\bf The Individualist Anarchists}, p. 73]


This, we must note, is the fundamental critique of the capitalist theory that capital is productive. In and of themselves, fixed costs do not create value. Rather value is creation depends on how investments are developed and used once in place. Because of this the Individualist Anarchists, like other anarchists, considered non-labour derived income as usury, unlike “anarcho”-capitalists. Similarly, anarchists reject the notion of capitalist property rights in favour of possession (including the full fruits of one’s labour). For example, anarchists reject private ownership of land in favour of a “occupancy and use” regime. In this we follow Proudhon’s {\bf What is Property?} and argue that {\em “property is theft”}. Rothbard, as noted, rejected this perspective.


As these ideas are an {\bf essential} part of anarchist politics, they cannot be removed without seriously damaging the rest of the theory. This can be seen from Tucker’s comments that {\em “{\bf Liberty} insists\unknown{} [on] the abolition of the State and the abolition of usury; on no more government of man by man, and no more exploitation of man by man.”} [cited by Eunice Schuster in {\bf Native American Anarchism}, p. 140]. He indicates that anarchism has specific economic {\bf and} political ideas, that it opposes capitalism along with the state. Therefore anarchism was never purely a “political” concept, but always combined an opposition to oppression with an opposition to exploitation. The social anarchists made exactly the same point. Which means that when Tucker argued that {\em “{\bf Liberty} insists on Socialism\unknown{} — true Socialism, Anarchistic Socialism: the prevalence on earth of Liberty, Equality, and Solidarity”} he knew exactly what he was saying and meant it wholeheartedly. [{\bf Instead of a Book}, p. 363]


So because “anarcho”-capitalists embrace capitalism and reject socialism, they cannot be considered anarchists or part of the anarchist tradition.


Which brings us nicely to the second point, namely a lack of concern for equality. In stark contrast to anarchists of all schools, inequality is not seen to be a problem with “anarcho”-capitalists (see section 3). However, it is a truism that not all “traders” are equally subject to the market (i.e. have the same market power). In many cases, a few have sufficient control of resources to influence or determine price and in such cases, all others must submit to those terms or not buy the commodity. When the commodity is labour power, even this option is lacking — workers have to accept a job in order to live. As we argue in section 10.2, workers are usually at a disadvantage on the labour market when compared to capitalists, and this forces them to sell their liberty in return for making profits for others. These profits increase inequality in society as the property owners receive the surplus value their workers produce. This increases inequality further, consolidating market power and so weakens the bargaining position of workers further, ensuring that even the freest competition possible could not eliminate class power and society (something B. Tucker recognised as occurring with the development of trusts within capitalism — see section G.4).


By removing the underlying commitment to abolish non-labour income, any “anarchist” capitalist society would have vast differences in wealth and so power. Instead of a government imposed monopolies in land, money and so on, the economic power flowing from private property and capital would ensure that the majority remained in (to use Spooner’s words) {\em “the condition of servants”} (see sections 2 and 3.1 for more on this). The Individualist Anarchists were aware of this danger and so supported economic ideas that opposed usury (i.e. rent, profit and interest) and ensured the worker the full value of her labour. While not all of them called these ideas “socialist” it is clear that these ideas {\bf are} socialist in nature and in aim (similarly, not all the Individualist Anarchists called themselves anarchists but their ideas are clearly anarchist in nature and in aim).


This combination of the political and economic is essential as they mutually reinforce each other. Without the economic ideas, the political ideas would be meaningless as inequality would make a mockery of them. As Kline notes, the Individualist Anarchists’ {\em “proposals were designed to establish true equality of opportunity \unknown{} and they expected this would result in a society without great wealth or poverty. In the absence of monopolistic factors which would distort competition, they expected a society largely of self-employed workmen with no significant disparity of wealth between any of them since all would be required to live at their own expense and not at the expense of exploited fellow human beings.”} [{\bf Op. Cit.}, pp. 103–4]


Because of the evil effects of inequality on freedom, both social and individualist anarchists desired to create an environment in which circumstances would not drive people to sell their liberty to others at a disadvantage. In other words, they desired an equalisation of market power by opposing interest, rent and profit and capitalist definitions of private property. Kline summarises this by saying {\em “the American [individualist] anarchists exposed the tension existing in liberal thought between private property and the ideal of equal access. The Individual Anarchists were, at least, aware that existing conditions were far from ideal, that the system itself working against the majority of individuals in their efforts to attain its promises. Lack of capital, the means to creation and accumulation of wealth, usually doomed a labourer to a life of exploitation. This the anarchists knew and they abhorred such a system.”} [{\bf Op. Cit.}, p. 102]


And this desire for bargaining equality is reflected in their economic ideas and by removing these underlying economic ideas of the individualist anarchists, “anarcho”-capitalism makes a mockery of any ideas they do appropriate. Essentially, the Individualist Anarchists agreed with Rousseau that in order to prevent extreme inequality of fortunes you deprive people of the means to accumulate in the first place and {\bf not} take away wealth from the rich. An important point which “anarcho”-capitalism fails to understand or appreciate.


There are, of course, overlaps between individualist anarchism and “anarcho”-capitalism, just as there are overlaps between it and Marxism (and social anarchism, of course). However, just as a similar analysis of capitalism does not make individualist anarchists Marxists, so apparent similarities between individualist anarchism does not make it a forerunner of “anarcho”-capitalism. For example, both schools support the idea of “free markets.” Yet the question of markets is fundamentally second to the issue of property rights for what is exchanged on the market is dependent on what is considered legitimate property. In this, as Rothbard notes, individualist anarchists and “anarcho”-capitalists differ and different property rights produce different market structures and dynamics. This means that capitalism is not the only economy with markets and so support for markets cannot be equated with support for capitalism. Equally, opposition to markets is {\bf not} the defining characteristic of socialism (as we note in section G.2.1). As such, it {\bf is} possible to be a market socialist (and many socialist are). This is because “markets” and “property” do not equate to capitalism:



\startblockquote
{\em “Political economy confuses, on principle, two very different kinds of private property, one of which rests on the labour of the producers himself, and the other on the exploitation of the labour of others. It forgets that the latter is not only the direct antithesis of the former, but grows on the former’s tomb and nowhere else.}


{\em “In Western Europe, the homeland of political economy, the process of primitive accumulation is more of less accomplished\unknown{}}


{\em “It is otherwise in the colonies. There the capitalist regime constantly comes up against the obstacle presented by the producer, who, as owner of his own conditions of labour, employs that labour to enrich himself instead of the capitalist. The contradiction of these two diametrically opposed economic systems has its practical manifestation here in the struggle between them.”} [Karl Marx, {\bf Capital}, vol. 1, p. 931]



\stopblockquote
Individualist anarchism is obviously an aspect of this struggle between the system of peasant and artisan production of early America and the state encouraged system of private property and wage labour. “Anarcho”-capitalists, in contrast, assume that generalised wage labour would remain under their system (while paying lip-service to the possibilities of co-operatives — and if an “anarcho”-capitalist thinks that co-operative will become the dominant form of workplace organisation, then they are some kind of market socialist, {\bf not} a capitalist). It is clear that their end point (a pure capitalism, i.e. generalised wage labour) is directly the opposite of that desired by anarchists. This was the case of the Individualist Anarchists who embraced the ideal of (non-capitalist) laissez faire competition — they did so, as noted, to {\bf end} exploitation, {\bf not} to maintain it. Indeed, their analysis of the change in American society from one of mainly independent producers into one based mainly upon wage labour has many parallels with, of all people, Karl Marx’s presented in chapter 33 of {\bf Capital}. Marx, correctly, argues that {\em “the capitalist mode of production and accumulation, and therefore capitalist private property, have for their fundamental condition the annihilation of that private property which rests on the labour of the individual himself; in other words, the expropriation of the worker.”} [{\bf Op. Cit.}, p. 940] He notes that to achieve this, the state is used:



\startblockquote
{\em “How then can the anti-capitalistic cancer of the colonies be healed? \unknown{} Let the Government set an artificial price on the virgin soil, a price independent of the law of supply and demand, a price that compels the immigrant to work a long time for wages before he can earn enough money to buy land, and turn himself into an independent farmer.”} [{\bf Op. Cit.}, p. 938]



\stopblockquote
Moreover, tariffs are introduced with {\em “the objective of manufacturing capitalists artificially”} for the {\em “system of protection was an artificial means of manufacturing manufacturers, or expropriating independent workers, of capitalising the national means of production and subsistence, and of forcibly cutting short the transition \unknown{} to the modern mode of production,”} to capitalism [{\bf Op. Cit.}, p. 932 and pp. 921–2]


It is this process which Individualist Anarchism protested against, the use of the state to favour the rising capitalist class. However, unlike social anarchists, many individualist anarchists were not consistently against wage labour. This is the other significant overlap between “anarcho”-capitalism and individualist anarchism. However, they were opposed to exploitation and argued (unlike “anarcho”-capitalism) that in their system workers bargaining powers would be raised to such a level that their wages would equal the full product of their labour. However, as we discuss in section G.1.1 the social context the individualist anarchists lived in must be remembered. America at the times was a predominantly rural society and industry was not as developed as it is now wage labour would have been minimised (Spooner, for example, explicitly envisioned a society made up mostly entirely of self-employed workers). As Kline argues:



\startblockquote
{\em “Committed as they were to equality in the pursuit of property, the objective for the anarchist became the construction of a society providing equal access to those things necessary for creating wealth. The goal of the anarchists who extolled mutualism and the abolition of all monopolies was, then, a society where everyone willing to work would have the tools and raw materials necessary for production in a non-exploitative system \unknown{} the dominant vision of the future society \unknown{} [was] underpinned by individual, self-employed workers.”} [{\bf Op. Cit.}, p. 95]



\stopblockquote
As such, a limited amount of wage labour within a predominantly self-employed economy does not make a given society capitalist any more than a small amount of governmental communities within an predominantly anarchist world would make it statist. As Marx argued. when {\em “the separation of the worker from the conditions of labour and from the soil \unknown{} does not yet exist, or only sporadically, or on too limited a scale \unknown{} Where, amongst such curious characters, is the ‘field of abstinence’ for the capitalists? \unknown{} Today’s wage-labourer is tomorrow’s independent peasant or artisan, working for himself. He vanishes from the labour-market — but not into the workhouse.”} There is a {\em “constant transformation of wage-labourers into independent producers, who work for themselves instead of for capital”} and so {\em “the degree of exploitation of the wage-labourer remain[s] indecently low.”} In addition, the {\em “wage-labourer also loses, along with the relation of dependence, the feeling of dependence on the abstemious capitalist.”} [{\bf Op. Cit.}, pp. 935–6]


Saying that, as we discuss in section G.4, individualist anarchist support for wage labour is at odds with the ideas of Proudhon and, far more importantly, in contradiction to many of the stated principles of the individualist anarchists themselves. In particular, wage labour violates “occupancy and use” as well as having more than a passing similarity to the state. However, these problems can be solved by consistently applying the principles of individualist anarchism, unlike “anarcho”-capitalism, and that is why it is a real school of anarchism. In other words, a system of {\bf generalised} wage labour would not be anarchist nor would it be non-exploitative. Moreover, the social context these ideas were developed in and would have been applied ensure that these contradictions would have been minimised. If they had been applied, a genuine anarchist society of self-employed workers would, in all likelihood, have been created (at least at first, whether the market would increase inequalities is a moot point — see section G.4).


We must stress that the social situation is important as it shows how apparently superficially similar arguments can have radically different aims and results depending on who suggests them and in what circumstances. As noted, during the rise of capitalism the bourgeoisie were not shy in urging state intervention against the masses. Unsurprisingly, working class people generally took an anti-state position during this period. The individualist anarchists were part of that tradition, opposing what Marx termed {\em “primitive accumulation”} in favour of the pre-capitalist forms of property and society it was destroying.


However, when capitalism found its feet and could do without such obvious intervention, the possibility of an “anti-state” capitalism could arise. Such a possibility became a definite once the state started to intervene in ways which, while benefiting the system as a whole, came into conflict with the property and power of individual members of the capitalist and landlord class. Thus social legislation which attempted to restrict the negative effects of unbridled exploitation and oppression on workers and the environment were having on the economy were the source of much outrage in certain bourgeois circles:



\startblockquote
{\em “Quite independently of these tendencies [of individualist anarchism] \unknown{} the anti-state bourgeoisie (which is also anti-statist, being hostile to any social intervention on the part of the State to protect the victims of exploitation — in the matter of working hours, hygienic working conditions and so on), and the greed of unlimited exploitation, had stirred up in England a certain agitation in favour of pseudo-individualism, an unrestrained exploitation. To this end, they enlisted the services of a mercenary pseudo-literature \unknown{} which played with doctrinaire and fanatical ideas in order to project a species of ‘individualism’ that was absolutely sterile, and a species of ‘non-interventionism’ that would let a man die of hunger rather than offend his dignity.”} [Max Nettlau, {\bf A Short History of Anarchism}, p. 39]



\stopblockquote
This perspective can be seen when Tucker denounced Herbert Spencer as a champion of the capitalistic class for his vocal attacks on social legislation which claimed to benefit working class people but stays strangely silent on the laws passed to benefit (usually indirectly) capital and the rich. “Anarcho”-capitalism is part of that tradition, the tradition associated with a capitalism which no longer needs obvious state intervention as enough wealth as been accumulated to keep workers under control by means of market power.


As with the original nineteenth century British “anti-state” capitalists like Spencer and Herbert, Rothbard {\em “completely overlooks the role of the state in building and maintaining a capitalist economy in the West. Privileged to live in the twentieth century, long after the battles to establish capitalism have been fought and won, Rothbard sees the state solely as a burden on the market and a vehicle for imposing the still greater burden of socialism. He manifests a kind of historical nearsightedness that allows him to collapse many centuries of human experience into one long night of tyranny that ended only with the invention of the free market and its ‘spontaneous’ triumph over the past. It is pointless to argue, as Rothbard seems ready to do, that capitalism would have succeeded without the bourgeois state; the fact is that all capitalist nations have relied on the machinery of government to create and preserve the political and legal environments required by their economic system.”} That, of course, has not stopped him {\em “critis[ing] others for being unhistorical.”} [Stephen L Newman, {\bf Liberalism at Wit’s End}, pp. 77–8 and p. 79]


In other words, there is substantial differences between the victims of a thief trying to stop being robbed and be left alone to enjoy their property and the successful thief doing the same! Individualist Anarchist’s were aware of this. For example, Victor Yarros stressed this key difference between individualist anarchism and the proto-“libertarian” capitalists of “voluntaryism”:



\startblockquote
{\em “[Auberon Herbert] believes in allowing people to retain all their possessions, no matter how unjustly and basely acquired, while getting them, so to speak, to swear off stealing and usurping and to promise to behave well in the future. We, on the other hand, while insisting on the principle of private property, in wealth honestly obtained under the reign of liberty, do not think it either unjust or unwise to dispossess the landlords who have monopolised natural wealth by force and fraud. We hold that the poor and disinherited toilers would be justified in expropriating, not alone the landlords, who notoriously have no equitable titles to their lands, but {\bf all} the financial lords and rulers, all the millionaires and very wealthy individuals\unknown{} Almost all possessors of great wealth enjoy neither what they nor their ancestors rightfully acquired (and if Mr. Herbert wishes to challenge the correctness of this statement, we are ready to go with him into a full discussion of the subject)\unknown{}}


{\em “If he holds that the landlords are justly entitled to their lands, let him make a defence of the landlords or an attack on our unjust proposal.”} [quoted by Carl Watner, {\em “The English Individualists As They Appear In Liberty,”} pp. 191–211, {\bf Benjamin R. Tucker and the Champions of Liberty}, Coughlin, Hamilton and Sullivan (eds.), pp. 199–200]



\stopblockquote
Significantly, Tucker and other individualist anarchists saw state intervention has a result of capital manipulating legislation to gain an advantage on the so-called free market which allowed them to exploit labour and, as such, it benefited the {\bf whole} capitalist class. Rothbard, at best, acknowledges that {\bf some} sections of big business benefit from the current system and so fails to have the comprehensive understanding of the dynamics of capitalism as a {\bf system} (rather as an ideology). This lack of understanding of capitalism as a historic and dynamic system rooted in class rule and economic power is important in evaluating “anarcho”-capitalist claims to anarchism. Marxists are not considered anarchists as they support the state as a means of transition to an anarchist society. Much the same logic can be applied to right-wing libertarians (even if they do call themselves “anarcho”-capitalists). This is because they do not seek to correct the inequalities produced by previous state action before ending it nor do they seek to change the definitions of “private property” imposed by the state. In effect, they argue that the “dictatorship of the bourgeoisie” should “wither away” and be limited to defending the property accumulated in a few hands. Needless to say, starting from the current (coercively produced) distribution of property and then eliminating “force” simply means defending the power and privilege of ruling minorities:



\startblockquote
{\em “The modern Individualism initiated by Herbert Spencer is, like the critical theory of Proudhon, a powerful indictment against the dangers and wrongs of government, but its practical solution of the social problem is miserable — so miserable as to lead us to inquire if the talk of ‘No force’ be merely an excuse for supporting landlord and capitalist domination.”} [{\bf Act For Yourselves}, p. 98]



\stopblockquote
\section{2 What do “anarcho”-capitalists mean by “freedom”?
}

For “anarcho”-capitalists, the concept of freedom is limited to the idea of {\em “freedom from.”} For them, freedom means simply freedom from the {\em “initiation of force,”} or the {\em “non-aggression against anyone’s person and property.”} [Murray Rothbard, {\bf For a New Liberty}, p. 23] The notion that real freedom must combine both freedom {\em “to”} {\bf and} freedom {\em “from”} is missing in their ideology, as is the social context of the so-called freedom they defend.


Before starting, it is useful to quote Alan Haworth when he notes that {\em “[i]n fact, it is surprising how {\bf little} close attention the concept of freedom receives from libertarian writers. Once again {\bf Anarchy, State, and Utopia} is a case in point. The word ‘freedom’ doesn’t even appear in the index. The word ‘liberty’ appears, but only to refer the reader to the ‘Wilt Chamberlain’ passage. In a supposedly ‘libertarian’ work, this is more than surprising. It is truly remarkable.”} [{\bf Anti-Libertarianism}, p. 95]


Why this is the case can be seen from how the “anarcho”-capitalist defines freedom.


In a right-libertarian or “anarcho”-capitalist society, freedom is considered to be a product of property. As Murray Rothbard puts it, {\em “the libertarian defines the concept of ‘freedom’ or ‘liberty’\unknown{}[as a] condition in which a person’s ownership rights in his body and his legitimate material property rights are not invaded, are not aggressed against\unknown{} Freedom and unrestricted property rights go hand in hand.”} [{\bf Op. Cit.}, p.41]


This definition has some problems, however. In such a society, one cannot (legitimately) do anything with or on another’s property if the owner prohibits it. This means that an individual’s only {\bf guaranteed} freedom is determined by the amount of property that he or she owns. This has the consequence that someone with no property has no guaranteed freedom at all (beyond, of course, the freedom not to be murdered or otherwise harmed by the deliberate acts of others). In other words, a distribution of property is a distribution of freedom, as the right-libertarians themselves define it. It strikes anarchists as strange that an ideology that claims to be committed to promoting freedom entails the conclusion that some people should be more free than others. However, this is the logical implication of their view, which raises a serious doubt as to whether “anarcho”-capitalists are actually interested in freedom.


Looking at Rothbard’s definition of “liberty” quoted above, we can see that freedom is actually no longer considered to be a fundamental, independent concept. Instead, freedom is a derivative of something more fundamental, namely the {\em “legitimate rights”} of an individual, which are identified as property rights. In other words, given that “anarcho”-capitalists and right libertarians in general consider the right to property as “absolute,” it follows that freedom and property become one and the same. This suggests an alternative name for the right Libertarian, namely {\bf {\em “Propertarian.”}} And, needless to say, if we do not accept the right-libertarians’ view of what constitutes “legitimate” “rights,” then their claim to be defenders of liberty is weak.


Another important implication of this “liberty as property” concept is that it produces a strangely alienated concept of freedom. Liberty, as we noted, is no longer considered absolute, but a derivative of property — which has the important consequence that you can “sell” your liberty and still be considered free by the ideology. This concept of liberty (namely “liberty as property”) is usually termed “self-ownership.” But, to state the obvious, I do not “own” myself, as if were an object somehow separable from my subjectivity — I {\bf am} myself. However, the concept of “self-ownership” is handy for justifying various forms of domination and oppression — for by agreeing (usually under the force of circumstances, we must note) to certain contracts, an individual can “sell” (or rent out) themselves to others (for example, when workers sell their labour power to capitalists on the “free market”). In effect, “self-ownership” becomes the means of justifying treating people as objects — ironically, the very thing the concept was created to stop! As L. Susan Brown notes, {\em “[a]t the moment an individual ‘sells’ labour power to another, he/she loses self-determination and instead is treated as a subjectless instrument for the fulfilment of another’s will.”} [{\bf The Politics of Individualism}, p. 4]


Given that workers are paid to obey, you really have to wonder which planet Murray Rothbard is on when he argues that a person’s {\em “labour service is alienable, but his {\bf will} is not”} and that he [sic!] {\em “cannot alienate his {\bf will}, more particularly his control over his own mind and body.”} [{\bf The Ethics of Liberty}, p. 40, p. 135] He contrasts private property and self-ownership by arguing that {\em “[a]ll physical property owned by a person is alienable \unknown{} I can give away or sell to another person my shoes, my house, my car, my money, etc. But there are certain vital things which, in natural fact and in the nature of man, are {\bf in}alienable \unknown{} [his] will and control over his own person are inalienable.”} [{\bf Op. Cit.}, pp. 134–5]


But {\em “labour services”} are unlike the private possessions Rothbard lists as being alienable. As we argued in section B.1 (“Why do anarchists oppose hierarchy”) a person’s {\em “labour services”} and {\em “will”} cannot be divided — if you sell your labour services, you also have to give control of your body and mind to another person! If a worker does not obey the commands of her employer, she is fired. That Rothbard denies this indicates a total lack of common-sense. Perhaps Rothbard will argue that as the worker can quit at any time she does not alienate their will (this seems to be his case against slave contracts — see section 2.6). But this ignores the fact that between the signing and breaking of the contract and during work hours (and perhaps outside work hours, if the boss has mandatory drug testing or will fire workers who attend union or anarchist meetings or those who have an “unnatural” sexuality and so on) the worker {\bf does} alienate his will and body. In the words of Rudolf Rocker, {\em “under the realities of the capitalist economic form \unknown{} there can be no talk of a ‘right over one’s own person,’ for that ends when one is compelled to submit to the economic dictation of another if he does not want to starve.”} [{\bf Anarcho-Syndicalism}, p. 17]


Ironically, the rights of property (which are said to flow from an individual’s self-ownership of themselves) becomes the means, under capitalism, by which self-ownership of non-property owners is denied. The foundational right (self-ownership) becomes denied by the derivative right (ownership of things). Under capitalism, a lack of property can be just as oppressive as a lack of legal rights because of the relationships of domination and subjection this situation creates.


So Rothbard’s argument (as well as being contradictory) misses the point (and the reality of capitalism). Yes, {\bf if} we define freedom as {\em “the absence of coercion”} then the idea that wage labour does not restrict liberty is unavoidable, but such a definition is useless. This is because it hides structures of power and relations of domination and subordination. As Carole Pateman argues, {\em “the contract in which the worker allegedly sells his labour power is a contract in which, since he cannot be separated from his capacities, he sells command over the use of his body and himself\unknown{} To sell command over the use of oneself for a specified period \unknown{} is to be an unfree labourer.”} [{\bf The Sexual Contract}, p. 151]


In other words, contracts about property in the person inevitably create subordination. “Anarcho”-capitalism defines this source of unfreedom away, but it still exists and has a major impact on people’s liberty. Therefore freedom is better described as “self-government” or “self-management” — to be able to govern ones own actions (if alone) or to participate in the determination of join activity (if part of a group). Freedom, to put it another way, is not an abstract legal concept, but the vital concrete possibility for every human being to bring to full development all their powers, capacities, and talents which nature has endowed them. A key aspect of this is to govern one own actions when within associations (self-management). If we look at freedom this way, we see that coercion is condemned but so is hierarchy (and so is capitalism for during working hours, people are not free to make their own plans and have a say in what affects them. They are order takers, {\bf not} free individuals).


It is because anarchists have recognised the authoritarian nature of capitalist firms that they have opposed wage labour and capitalist property rights along with the state. They have desired to replace institutions structured by subordination with institutions constituted by free relationships (based, in other words, on self-management) in {\bf all} areas of life, including economic organisations. Hence Proudhon’s argument that the {\em “workmen’s associations \unknown{} are full of hope both as a protest against the wage system, and as an affirmation of {\bf reciprocity}”} and that their importance lies {\em “in their denial of the rule of capitalists, money lenders and governments.”} [{\bf The General Idea of the Revolution}, pp. 98–99]


Unlike anarchists, the “anarcho”-capitalist account of freedom allows an individual’s freedom to be rented out to another while maintaining that the person is still free. It may seem strange that an ideology proclaiming its support for liberty sees nothing wrong with the alienation and denial of liberty but, in actual fact, it is unsurprising. After all, contract theory is a {\em “theoretical strategy that justifies subjection by presenting it as freedom”} and nothing more. Little wonder, then, that contract {\em “creates a relation of subordination”} and not of freedom [Carole Pateman, {\bf Op. Cit.}, p. 39, p. 59]


Any attempt to build an ethical framework starting from the abstract individual (as Rothbard does with his {\em “legitimate rights”} method) will result in domination and oppression between people, {\bf not} freedom. Indeed, Rothbard provides an example of the dangers of idealist philosophy that Bakunin warned about when he argued that while {\em “[m]aterialism denies free will and ends in the establishment of liberty; idealism, in the name of human dignity, proclaims free will, and on the ruins of every liberty founds authority.”} [{\bf God and the State}, p. 48] This is the case with “anarcho”-capitalism can be seen from Rothbard’s wholehearted support for wage labour and the rules imposed by property owners on those who use, but do not own, their property. Rothbard, basing himself on abstract individualism, cannot help but justify authority over liberty.


Overall, we can see that the logic of the right-libertarian definition of “freedom” ends up negating itself, because it results in the creation and encouragement of {\bf authority,} which is an {\bf opposite} of freedom. For example, as Ayn Rand points out, {\em “man has to sustain his life by his own effort, the man who has no right to the product of his effort has no means to sustain his life. The man who produces while others dispose of his product, is a slave.”} [{\bf The Ayn Rand Lexicon: Objectivism from A to Z}, pp. 388–9] But, as was shown in section C, capitalism is based on, as Proudhon put it, workers working {\em “for an entrepreneur who pays them and keeps their products,”} and so is a form of {\bf theft.} Thus, by “libertarian” capitalism’s {\bf own} logic, capitalism is based not on freedom, but on (wage) slavery; for interest, profit and rent are derived from a worker’s {\bf unpaid} labour, i.e. {\em “others dispose of his [sic] product.”}


And if a society {\bf is} run on the wage- and profit-based system suggested by the “anarcho” and “libertarian” capitalists, freedom becomes a commodity. The more money you have, the more freedom you get. Then, since money is only available to those who earn it, Libertarianism is based on that classic saying {\em “work makes one free!”} ({\bf {\em Arbeit macht frei!}}), which the Nazis placed on the gates of their concentration camps. Of course, since it is capitalism, this motto is somewhat different for those at the top. In this case it is {\em “other people’s work makes one free!”} — a truism in any society based on private property and the authority that stems from it.


Thus it is debatable that a libertarian or “anarcho” capitalist society would have less unfreedom or coercion in it than “actually existing capitalism.” In contrast to anarchism, “anarcho”-capitalism, with its narrow definitions, restricts freedom to only a few aspects of social life and ignores domination and authority beyond those aspects. As Peter Marshall points out, the right-libertarian’s {\em “definition of freedom is entirely negative. It calls for the absence of coercion but cannot guarantee the positive freedom of individual autonomy and independence.”} [{\bf Demanding the Impossible}, p. 564] By confining freedom to such a narrow range of human action, “anarcho”-capitalism is clearly {\bf not} a form of anarchism. Real anarchists support freedom in every aspect of an individual’s life.


\subsection{2.1 What are the implications of defining liberty in terms of (property) rights?
}

The change from defending liberty to defending (property) rights has important implications. For one thing, it allows right libertarians to imply that private property is similar to a “fact of nature,” and so to conclude that the restrictions on freedom produced by it can be ignored. This can be seen in Robert Nozick’s argument that decisions are voluntary if the limitations on one’s actions are not caused by human action which infringe the rights of others. Thus, in a “pure” capitalist society the restrictions on freedom caused by wage slavery are not really restrictions because the worker voluntarily consents to the contract. The circumstances that drive a worker to make the contract are irrelevant because they are created by people exercising their rights and not violating other peoples’ ones (see the section on {\em “Voluntary Exchange”} in {\bf Anarchy, State, and Utopia}, pp. 262–265).


This means that within a society {\em “[w]hether a person’s actions are voluntary depends on what limits his alternatives. If facts of nature do so, the actions are voluntary. (I may voluntarily walk to someplace I would prefer to fly to unaided).”} [{\bf Anarchy, State, and Utopia}, p. 262] Similarly, the results of voluntary actions and the transference of property can be considered alongside the “facts of nature” (they are, after all, the resultants of “natural rights”). This means that the circumstances created by the existence and use of property can be considered, in essence, as a “natural” fact and so the actions we take in response to these circumstances are therefore “voluntary” and we are “free” (Nozick presents the example [p. 263] of someone who marries the only available person — all the more attractive people having already chosen others — as a case of an action that is voluntary despite removal of all but the least attractive alternative through the legitimate actions of others. Needless to say, the example can be — and is — extended to workers on the labour market — although, of course, you do not starve to death if you decide not to marry).


However, such an argument fails to notice that property is different from gravity or biology. Of course not being able to fly does not restrict freedom. Neither does not being able to jump 10 feet into the air. But unlike gravity (for example), private property has to be protected by laws and the police. No one stops you from flying, but laws and police forces must exist to ensure that capitalist property (and the owners’ authority over it) is respected. The claim, therefore, that private property in general, and capitalism in particular, can be considered as “facts of nature,” like gravity, ignores an important fact: namely that the people involved in an economy must accept the rules of its operation — rules that, for example, allow contracts to be enforced; forbid using another’s property without his or her consent (“theft,” trespass, copyright infringement, etc.); prohibit “conspiracy,” unlawful assembly, rioting, and so on; and create monopolies through regulation, licensing, charters, patents, etc. This means that capitalism has to include the mechanisms for deterring property crimes as well as mechanisms for compensation and punishment should such crimes be committed. In other words, capitalism is in fact far more than “voluntary bilateral exchange,” because it {\bf must} include the policing, arbitration, and legislating mechanisms required to ensure its operation. Hence, like the state, the capitalist market is a social institution, and the distributions of goods that result from its operation are therefore the distributions sanctioned by a capitalist society. As Benjamin Franklin pointed out, {\em “Private property \unknown{} is a Creature of Society, and is subject to the Calls of that Society.”}


Thus, to claim with Sir Isaiah Berlin (the main, modern, source of the concepts of {\em “negative”} and {\em “positive”} freedom — although we must add that Berlin was not a right-Libertarian), that {\em “[i]f my poverty were a kind of disease, which prevented me from buying bread \unknown{} as lameness prevents me from running, this inability would not naturally be described as a lack of freedom”} totally misses the point [{\em “Two Concepts of Liberty”}, in {\bf Four Essays on Liberty}, p. 123]. If you are lame, police officers do not come round to stop you running. They do not have to. However, they {\bf are} required to protect property against the dispossessed and those who reject capitalist property rights.


This means that by using such concepts as “negative” liberty and ignoring the social nature of private property, right-libertarians are trying to turn the discussion away from liberty toward “biology” and other facts of nature. And conveniently, by placing property rights alongside gravity and other natural laws, they also succeed in reducing debate even about rights.


Of course, coercion and restriction of liberty {\bf can} be resisted, unlike “natural forces” like gravity. So if, as Berlin argues, {\em “negative”} freedom means that you {\em “lack political freedom only if you are prevented from attaining a goal by human beings,”} then capitalism is indeed based on such a lack, since property rights need to be enforced by human beings ({\em “I am prevented by others from doing what I could otherwise do”}). After all, as Proudhon long ago noted, the market is manmade, hence any constraint it imposes is the coercion of man by man and so economic laws are not as inevitable as natural ones [see Alan Ritter’s {\bf The Political Thought of Pierre-Joseph Proudhon}, p. 122]. Or, to put it slightly differently, capitalism requires coercion in order to work, and hence, is {\bf not} similar to a “fact of nature,” regardless of Nozick’s claims (i.e. property rights have to be defined and enforced by human beings, although the nature of the labour market resulting from capitalist property definitions is such that direct coercion is usually not needed). This implication is actually recognised by right-libertarians, because they argue that the rights-framework of society should be set up in one way rather than another. In other words, they recognise that society is not independent of human interaction, and so can be changed.


Perhaps, as seems the case, the “anarcho”-capitalist or right-Libertarian will claim that it is only {\bf deliberate} acts which violate your (libertarian defined) rights by other humans beings that cause unfreedom ({\em “we define freedom \unknown{} as the {\bf absence of invasion} by another man of an man’s person or property”} [Rothbard, {\bf The Ethics of Liberty}, p. 41]) and so if no-one deliberately coerces you then you are free. In this way the workings of the capitalist market can be placed alongside the “facts of nature” and ignored as a source of unfreedom. However, a moments thought shows that this is not the case. Both deliberate and non-deliberate acts can leave individuals lacking freedom.


Let us assume (in an example paraphrased from Alan Haworth’s excellent book {\bf Anti-Libertarianism}, p. 49) that someone kidnaps you and places you down a deep (naturally formed) pit, miles from anyway, which is impossible to climb up. No one would deny that you are unfree. Let us further assume that another person walks by and accidentally falls into the pit with you.


According to right-libertarianism, while you are unfree (i.e. subject to deliberate coercion) your fellow pit-dweller is perfectly free for they have subject to the “facts of nature” and not human action (deliberate or otherwise). Or, perhaps, they “voluntarily choose” to stay in the pit, after all, it is “only” the “facts of nature” limiting their actions. But, obviously, both of you are in {\bf exactly the same position,} have {\bf exactly the same choices} and so are {\bf equally} unfree! Thus a definition of “liberty” that maintains that only deliberate acts of others — for example, coercion — reduces freedom misses the point totally.


Why is this example important? Let us consider Murray Rothbard’s analysis of the situation after the abolition of serfdom in Russia and slavery in America. He writes:



\startblockquote
{\em “The {\bf bodies} of the oppressed were freed, but the property which they had worked and eminently deserved to own, remained in the hands of their former oppressors. With economic power thus remaining in their hands, the former lords soon found themselves virtual masters once more of what were now free tenants or farm labourers. The serfs and slaves had tasted freedom, but had been cruelly derived of its fruits.”} [{\bf The Ethics of Liberty}, p. 74]



\stopblockquote
However, contrast this with Rothbard’s claims that if market forces (“voluntary exchanges”) result in the creation of free tenants or labourers then these labourers and tenants are free (see, for example, {\bf The Ethics of Liberty}, pp. 221–2 on why “economic power” within capitalism does not exist). But the labourers dispossessed by market forces are in {\bf exactly} the same situation as the former serfs and slaves. Rothbard sees the obvious {\em “economic power”} in the later case, but denies it in the former. But the {\bf conditions} of the people in question are identical and it is these conditions that horrify us. It is only his ideology that stops Rothbard drawing the obvious conclusion — identical conditions produce identical social relationships and so if the formally “free” ex-serfs are subject to {\em “economic power”} and {\em “masters”} then so are the formally “free” labourers within capitalism! Both sets of workers may be formally free, but their circumstances are such that they are “free” to “consent” to sell their freedom to others (i.e. economic power produces relationships of domination and unfreedom between formally free individuals).


Thus Rothbard’s definition of liberty in terms of rights fails to provide us with a realistic and viable understanding of freedom. Someone can be a virtual slave while still having her rights non-violated (conversely, someone can have their property rights violated and still be free; for example, the child who enters your backyard without your permission to get her ball hardly violates your liberty — indeed, you would never know that she has entered your property unless you happened to see her do it). So the idea that freedom means non-aggression against person and their legitimate material property justifies extensive {\bf non-freedom} for the working class. The non-violation of property rights does {\bf not} imply freedom, as Rothbard’s discussion of the former slaves shows. Anyone who, along with Rothbard, defines freedom {\em “as the {\bf absence of invasion} by another man of any man’s person or property”} in a deeply inequality society is supporting, and justifying, capitalist and landlord domination. As anarchists have long realised, in an unequal society, a contractarian starting point implies an absolutist conclusion.


Why is this? Simply because freedom is a result of {\bf social} interaction, not the product of some isolated, abstract individual (Rothbard uses the model of Robinson Crusoe to construct his ideology). But as Bakunin argued, {\em “the freedom of the individual is a function of men in society, a necessary consequence of the collective development of mankind.”} He goes on to argue that {\em “man in isolation can have no awareness of his liberty \unknown{} Liberty is therefore a feature not of isolation but of interaction, not of exclusion but rather of connection.”} [{\bf Selected Writings}, p. 146, p. 147] Right Libertarians, by building their definition of freedom from the isolated person, end up by supporting restrictions of liberty due to a neglect of an adequate recognition of the actual interdependence of human beings, of the fact what each person does is effected by and affects others. People become aware of their humanity (liberty) in society, not outside it. It is the {\bf social relationships} we take part in which determine how free we are and any definition of freedom which builds upon an individual without social ties is doomed to create relations of domination, not freedom, between individuals — as Rothbard’s theory does (to put it another way, voluntary association is a necessary, but not sufficient, condition for freedom. Which is why anarchists have always stressed the importance of equality — see section 3 for details).


So while facts of nature can restrict your options and freedom, it is the circumstances within which they act and the options they limit that are important (a person trapped at the bottom of a pit is unfree as the options available are so few; the lame person is free because their available options are extensive). In the same manner, the facts of society can and do restrict your freedom because they are the products of human action and are defined and protected by human institutions, it is the circumstances within which individuals make their decisions and the social relationships these decisions produce that are important (the worker driven by poverty to accept a slave contract in a sweat shop is unfree because the circumstances he faces have limited his options and the relations he accepts are based upon hierarchy; the person who decides to join an anarchist commune is free because the commune is non-hierarchical and she has the option of joining another commune, working alone and so forth).


All in all, the right-Libertarian concept of freedom is lacking. For an ideology that takes the name “Libertarianism” it is seems happy to ignore actual liberty and instead concentrate on an abstract form of liberty which ignores so many sources of unfreedom as to narrow the concept until it becomes little more than a justification for authoritarianism. This can be seen from right-Libertarian attitudes about private property and its effects on liberty (as discussed in the next section).


\subsection{2.2 How does private property affect freedom?
}

The right-libertarian does not address or even acknowledge that the (absolute) right of private property may lead to extensive control by property owners over those who use, but do not own, property (such as workers and tenants). Thus a free-market capitalist system leads to a very selective and class-based protection of “rights” and “freedoms.” For example, under capitalism, the “freedom” of employers inevitably conflicts with the “freedom” of employees. When stockholders or their managers exercise their “freedom of enterprise” to decide how their company will operate, they violate their employee’s right to decide how their labouring capacities will be utilised. In other words, under capitalism, the “property rights” of employers will conflict with and restrict the “human right” of employees to manage themselves. Capitalism allows the right of self-management only to the few, not to all. Or, alternatively, capitalism does not recognise certain human rights as {\bf universal} which anarchism does.


This can be seen from Austrian Economist W. Duncan Reekie’s defence of wage labour. While referring to {\em “intra-firm labour markets”} as {\em “hierarchies”}, Reekie (in his best {\em ex cathedra} tone) states that {\em “[t]here is nothing authoritarian, dictatorial or exploitative in the relationship. Employees order employers to pay them amounts specified in the hiring contract just as much as employers order employees to abide by the terms of the contract.”} [{\bf Markets, Entrepreneurs and Liberty}, p. 136, p. 137]. Given that {\em “the terms of contract”} involve the worker agreeing to obey the employers orders and that they will be fired if they do not, its pretty clear that the ordering that goes on in the {\em “intra-firm labour market”} is decidedly {\bf one way}. Bosses have the power, workers are paid to obey. And this begs the question, {\bf if} the employment contract creates a free worker, why must she abandon her liberty during work hours?


Reekie actually recognises this lack of freedom in a “round about” way when he notes that {\em “employees in a firm at any level in the hierarchy can exercise an entrepreneurial role. The area within which that role can be carried out increases the more authority the employee has.”} [{\bf Op. Cit.}, p. 142] Which means workers {\bf are} subject to control from above which restricts the activities they are allowed to do and so they are {\bf not} free to act, make decisions, participate in the plans of the organisation, to create the future and so forth within working hours. And it is strange that while recognising the firm as a hierarchy, Reekie tries to deny that it is authoritarian or dictatorial — as if you could have a hierarchy without authoritarian structures or an unelected person in authority who is not a dictator. His confusion is shared by Austrian guru Ludwig von Mises, who asserts that the {\em “entrepreneur and capitalist are not irresponsible autocrats”} because they are {\em “unconditionally subject to the sovereignty of the consumer”} while, {\bf on the next page}, admitting there is a {\em “managerial hierarchy”} which contains {\em “the average subordinate employee.”} [{\bf Human Action}, p. 809 and p. 810] It does not enter his mind that the capitalist may be subject to some consumer control while being an autocrat to their subordinated employees. Again, we find the right-“libertarian” acknowledging that the capitalist managerial structure is a hierarchy and workers are subordinated while denying it is autocratic to the workers! Thus we have “free” workers within a relationship distinctly {\bf lacking} freedom (in the sense of self-government) — a strange paradox. Indeed, if your personal life were as closely monitored and regulated as the work life of millions of people across the world, you would rightly consider it oppression.


Perhaps Reekie (like most right-libertarians) will maintain that workers voluntarily agree (“consent”) to be subject to the bosses dictatorship (he writes that {\em “each will only enter into the contractual agreement known as a firm if each believes he will be better off thereby. The firm is simply another example of mutually beneficial exchange”} [{\bf Op. Cit.}, p. 137]). However, this does not stop the relationship being authoritarian or dictatorial (and so exploitative as it is {\bf highly} unlikely that those at the top will not abuse their power). And as we argue further in the next section (and also see sections B.4, 3.1 and 10.2), in a capitalist society workers have the option of finding a job or facing abject poverty and/or starvation.


Little wonder, then, that people “voluntarily” sell their labour and “consent” to authoritarian structures! They have little option to do otherwise. So, {\bf within} the labour market, workers {\bf can} and {\bf do} seek out the best working conditions possible, but that does not mean that the final contract agreed is “freely” accepted and not due to the force of circumstances, that both parties have equal bargaining power when drawing up the contract or that the freedom of both parties is ensured. Which means to argue (as many right-libertarians do) that freedom cannot be restricted by wage labour because people enter into relationships they consider will lead to improvements over their initial situation totally misses the points. As the initial situation is not considered relevant, their argument fails. After all, agreeing to work in a sweatshop 14 hours a day {\bf is} an improvement over starving to death — but it does not mean that those who so agree are free when working there or actually {\bf want} to be there. They are not and it is the circumstances, created and enforced by the law, that have ensured that they “consent” to such a regime (given the chance, they would desire to {\bf change} that regime but cannot as this would violate their bosses property rights and they would be repressed for trying).


So the right-wing “libertarian” right is interested only in a narrow concept of freedom (rather than in “freedom” or “liberty” as such). This can be seen in the argument of Ayn Rand (a leading ideologue of “libertarian” capitalism) that {\em “{\bf Freedom}, in a political context, means freedom from government coercion. It does {\bf not} mean freedom from the landlord, or freedom from the employer, or freedom from the laws of nature which do not provide men with automatic prosperity. It means freedom from the coercive power of the state — and nothing else!”} [{\bf Capitalism: The Unknown Ideal}, p. 192] By arguing in this way, right libertarians ignore the vast number of authoritarian social relationships that exist in capitalist society and, as Rand does here, imply that these social relationships are like “the laws of nature.” However, if one looks at the world without prejudice but with an eye to maximising freedom, the major coercive institution is seen to be not the state but capitalist social relationships (as indicated in section B.4).


The right “libertarian,” then, far from being a defender of freedom, is in fact a keen defender of certain forms of authority and domination. As Peter Kropotkin noted, the {\em “modern Individualism initiated by Herbert Spencer is, like the critical theory of Proudhon, a powerful indictment against the dangers and wrongs of government, but its practical solution of the social problem is miserable — so miserable as to lead us to inquire if the talk of ‘No force’ be merely an excuse for supporting landlord and capitalist domination.”} [{\bf Act For Yourselves}, p. 98]


To defend the “freedom” of property owners is to defend authority and privilege — in other words, statism. So, in considering the concept of liberty as “freedom from,” it is clear that by defending private property (as opposed to possession) the “anarcho”-capitalist is defending the power and authority of property owners to govern those who use “their” property. And also, we must note, defending all the petty tyrannies that make the work lives of so many people frustrating, stressful and unrewarding.


However, anarchism, by definition, is in favour of organisations and social relationships which are non-hierarchical and non-authoritarian. Otherwise, some people are more free than others. Failing to attack hierarchy leads to massive contradiction. For example, since the British Army is a volunteer one, it is an “anarchist” organisation! (see next section for a discussion on why the “anarcho”-capitalism concept of freedom also allows the state to appear “libertarian”).


In other words, “full capitalist property rights” do not protect freedom, in fact they actively deny it. But this lack of freedom is only inevitable if we accept capitalist private property rights. If we reject them, we can try and create a world based on freedom in all aspects of life, rather than just in a few.


\subsection{2.3 Can “anarcho”-capitalist theory justify the state?
}

Ironically enough, “anarcho”-capitalist ideology actually allows the state to be justified along with capitalist hierarchy. This is because the reason why capitalist authority is acceptable to the “anarcho”-capitalist is because it is “voluntary” — no one forces the worker to join or remain within a specific company (force of circumstances are irrelevant in this viewpoint). Thus capitalist domination is not really domination at all. But the same can be said of all democratic states as well. Few such states bar exit for its citizens — they are free to leave at any time and join any other state that will have them (exactly as employees can with companies). Of course there {\bf are} differences between the two kinds of authority — anarchists do not deny that — but the similarities are all too clear.


The “anarcho”-capitalist could argue that changing jobs is easier than changing states and, sometimes, this is correct — but not always. Yes, changing states does require the moving of home and possessions over great distances but so can changing job (indeed, if a worker has to move half-way across a country or even the world to get a job “anarcho”-capitalists would celebrate this as an example of the benefits of a “flexible” labour market). Yes, states often conscript citizens and send them into dangerous situations but bosses often force their employees to accept dangerous working environments on pain of firing. Yes, many states do restrict freedom of association and speech, but so do bosses. Yes, states tax their citizens but landlords and companies only let others use their property if they get money in return (i.e. rent or profits). Indeed, if the employee or tenant does not provide the employer or landlord with enough profits, they will quickly be shown the door. Of course employees can start their own companies but citizens can start their own state if they convince an existing state (the owner of a set of resources) to sell/give land to them. Setting up a company also requires existing owners to sell/give resources to those who need them. Of course, in a democratic state citizens can influence the nature of laws and orders they obey. In a capitalist company, this is not the case.


This means that, logically, “anarcho”-capitalism must consider a series of freely exitable states as “anarchist” and not a source of domination. If consent (not leaving) is what is required to make capitalist domination not domination then the same can be said of statist domination. Stephen L. Newman makes the same point:



\startblockquote
{\em “The emphasis [right-wing] libertarians place on the opposition of liberty and political power tends to obscure the role of authority in their worldview \unknown{} the authority exercised in private relationships, however — in the relationship between employer and employee, for instance — meets with no objection\unknown{} [This] reveals a curious insensitivity to the use of private authority as a means of social control. Comparing public and private authority, we might well ask of the [right-wing] libertarians: When the price of exercising one’s freedom is terribly high, what practical difference is there between the commands of the state and those issued by one’s employer? \unknown{} Though admittedly the circumstances are not identical, telling disgruntled empowers that they are always free to leave their jobs seems no different in principle from telling political dissidents that they are free to emigrate.”} [{\bf Liberalism at Wit’s End}, pp. 45–46]



\stopblockquote
Murray Rothbard, in his own way, agrees:



\startblockquote
{\em “{\bf If} the State may be said too properly {\bf own} its territory, then it is proper for it to make rules for everyone who presumes to live in that area. It can legitimately seize or control private property because there {\bf is} no private property in its area, because it really owns the entire land surface. {\bf So long} as the State permits its subjects to leave its territory, then, it can be said to act as does any other owner who sets down rules for people living on his property.”} [{\bf The Ethics of Liberty}, p. 170]



\stopblockquote
Rothbard’s argues that this is {\bf not} the case simply because the state did not acquire its property in a {\em “just”} manner and that it claims rights over virgin land (both of which violates Rothbard’s “homesteading” theory of property — see section 4.1 for details and a critique). Rothbard argues that this defence of statism (the state as property owner) is unrealistic and ahistoric, but his account of the origins of property is equally unrealistic and ahistoric and that does not stop him supporting capitalism. People in glass houses should not throw stones!


Thus he claims that the state is evil and its claims to authority/power false simply because it acquired the resources it claims to own {\em “unjustly”} — for example, by violence and coercion (see {\bf The Ethics of Liberty}, pp. 170–1, for Rothbard’s attempt to explain why the state should not be considered as the owner of land). And even {\bf if} the state {\bf was} the owner of its territory, it cannot appropriate virgin land (although, as he notes elsewhere, the {\em “vast”} US frontier no longer exists {\em “and there is no point crying over the fact”} [{\bf Op. Cit.}, p. 240]).


So what makes hierarchy legitimate for Rothbard is whether the property it derives from was acquired justly or unjustly. Which leads us to a few {\bf very} important points.


Firstly, Rothbard is explicitly acknowledging the similarities between statism and capitalism. He is arguing that {\bf if} the state had developed in a {\em “just”} way, then it is perfectly justifiable in governing ({\em “set[ting] down rules”}) those who “consent” to live on its territory in {\bf exactly} the same why a property owner does. In other words, private property can be considered as a “justly” created state! These similarities between property and statism have long been recognised by anarchists and that is why we reject private property along with the state (Proudhon did, after all, note that {\em “property is despotism”} and well as {\em “theft”}). But, according to Rothbard, something can look like a state (i.e. be a monopoly of decision making over an area) and act like a state (i.e. set down rules for people, govern them, impose a monopoly of force) but not be a state. But if it looks like a duck and sounds like a duck, it is a duck. Claiming that the origins of the thing are what counts is irrelevant — for example, a cloned duck is just as much a duck as a naturally born one. A statist organisation is authoritarian whether it comes from {\em “just”} or {\em “unjust”} origins. Does transforming the ownership of the land from states to capitalists {\bf really} make the relations of domination created by the dispossession of the many less authoritarian and unfree? Of course not.


Secondly, much property in “actually existing” capitalism is the product (directly or indirectly) of state laws and violence ({\em “the emergence of both agrarian and industrial capitalism in Britain [and elsewhere, we must add] \unknown{} could not have got off the ground without resources to state violence — legal or otherwise”} [Brian Morris, {\bf Ecology \& Anarchism}, p. 190]). If state claims of ownership are invalid due to their history, then so are many others (particularly those which claim to own land). As the initial creation was illegitimate, so are the transactions which have sprung from it. Thus if state claims of property rights are invalid, so are most (if not all) capitalist claims. If the laws of the state are illegitimate, so are the rules of the capitalist. If taxation is illegitimate, then so are rent, interest and profit. Rothbard’s “historical” argument against the state can also be applied to private property and if the one is unjustified, then so is the other.


Thirdly, {\bf if} the state had evolved “justly” then Rothbard would actually have nothing against it! A strange position for an anarchist to take. Logically this means that if a system of corporate states evolved from the workings of the capitalist market then the “anarcho”-capitalist would have nothing against it. This can be seen from “anarcho”-capitalist support for company towns even though they have correctly been described as {\em “industrial feudalism”} (see section 6 for more on this).


Fourthly, Rothbard’s argument implies that similar circumstances producing similar relationships of domination and unfreedom are somehow different if they are created by {\em “just”} and {\em “unjust”} means. Rothbard claims that because the property is {\em “justly”} acquired it means the authority a capitalist over his employees is totally different from that of a state over its subject. But such a claim is false — both the subject/citizen and the employee are in a similar relationship of domination and authoritarianism. As we argued in section 2.2, how a person got into a situation is irrelevant when considering how free they are. Thus, the person who “consents” to be governed by another because all available resources are privately owned is in exactly the same situation as a person who has to join a state because all available resources are owned by one state or another. Both are unfree and are part of authoritarian relationships based upon domination.


And, lastly, while “anarcho”-capitalism may be a “just” society, it is definitely {\bf not} a free one. It will be marked by extensive hierarchy, unfreedom and government, but these restrictions of freedom will be of a private nature. As Rothbard indicates, the property owner and the state create/share the same authoritarian relationships. If statism is unfree, then so is capitalism. And, we must add, how “just” is a system which undermines liberty. Can “justice” ever be met in a society in which one class has more power and freedom than another. If one party is in an inferior position, then they have little choice but to agree to the disadvantageous terms offered by the superior party (see section 3.1). In such a situation, a “just” outcome will be unlikely as any contract agreed will be skewed to favour one side over the other.


The implications of these points are important. We can easily imagine a situation within “anarcho”-capitalism where a few companies/people start to buy up land and form company regions and towns. After all, this {\bf has} happened continually throughout capitalism. Thus a “natural” process may develop where a few owners start to accumulate larger and larger tracks of land “justly”. Such a process does not need to result in {\bf one} company owning the world. It is likely that a few hundred, perhaps a few thousand, could do so. But this is not a cause for rejoicing — after all the current “market” in “unjust” states also has a few hundred competitors in it. And even if there is a large multitude of property owners, the situation for the working class is exactly the same as the citizen under current statism! Does the fact that it is “justly” acquired property that faces the worker really change the fact she must submit to the government and rules of another to gain access to the means of life?


When faced with anarchist criticisms that {\bf circumstances} force workers to accept wage slavery the “anarcho”-capitalist claims that these are to be considered as objective facts of nature and so wage labour is not domination. However, the same can be said of states — we are born into a world where states claim to own all the available land. If states are replaced by individuals or groups of individuals does this change the essential nature of our dispossession? Of course not.


Rothbard argues that {\em “[o]bviously, in a free society, Smith has the ultimate decision-making power over his own just property, Jones over his, etc.”} [{\bf Op. Cit.}, p. 173] and, equally obviously, this ultimate-decision making power extends to those who {\bf use,} but do not own, such property. But how “free” is a free society where the majority have to sell their liberty to another in order to live? Rothbard (correctly) argues that the State {\em “uses its monopoly of force \unknown{} to control, regulate, and coerce its hapless subjects. Often it pushes its way into controlling the morality and the very lives of its subjects.”} [{\bf Op. Cit.}, p. 171] However he fails to note that employers do exactly the same thing to their employees. This, from an anarchist perspective, is unsurprising, for (after all) the employer {\bf is} {\em “the ultimate decision-making power over his just property”} just as the state is over its “unjust” property. That similar forms of control and regulation develop is not a surprise given the similar hierarchical relations in both structures.


That there is a choice in available states does not make statism any less unjust and unfree. Similarly, just because we have a choice between employers does not make wage labour any less unjust or unfree. But trying to dismiss one form of domination as flowing from “just” property while attacking the other because it flows from “unjust” property is not seeing the wood for the trees. If one reduces liberty, so does the other. Whether the situation we are in resulted from “just” or “unjust” steps is irrelevant to the restrictions of freedom we face because of them (and as we argue in section 2.5, “unjust” situations can easily flow from “just” steps).


The “anarcho”-capitalist insistence that the voluntary nature of an association determines whether it is anarchistic is deeply flawed — so flawed in fact that states and state-like structures (such as capitalist firms) can be considered anarchistic! In contrast, anarchists think that the hierarchical nature of the associations we join is equally as important as its voluntary nature when determining whether it is anarchistic or statist. However this option is not available to the “anarcho”-capitalist as it logically entails that capitalist companies are to be opposed along with the state as sources of domination, oppression and exploitation.


\subsection{2.4 But surely transactions on the market are voluntary? 
}

Of course, it is usually maintained by “anarcho”-capitalists that no-one puts a gun to a worker’s head to join a specific company. Yes, indeed, this is true — workers can apply for any job they like. But the point is that the vast majority cannot avoid having to sell their liberty to others (self-employment and co-operatives {\bf are} an option, but they account for less than 10\% of the working population and are unlikely to spread due to the nature of capitalist market forces — see sections J.5.11 and J.5.12 for details). And as Bob Black pointed out, right libertarians argue that {\em “‘one can at least change jobs.’ but you can’t avoid having a job — just as under statism one can at least change nationalities but you can’t avoid subjection to one nation-state or another. But freedom means more than the right to change masters.”} [{\bf The Libertarian as Conservative}]


So why do workers agree to join a company? Because circumstances force them to do so — circumstances created, we must note, by {\bf human} actions and institutions and not some abstract “fact of nature.” And if the world that humans create by their activity is detrimental to what we should value most (individual liberty and individuality) then we should consider how to {\bf change that world for the better.} Thus “circumstances” (current “objective reality”) is a valid source of unfreedom and for human investigation and creative activity — regardless of the claims of right-Libertarians.


Let us look at the circumstances created by capitalism. Capitalism is marked by a class of dispossessed labourers who have nothing to sell by their labour. They are legally barred from access to the means of life and so have little option but to take part in the labour market. As Alexander Berkman put it:



\startblockquote
{\em “The law says your employer does not sell anything from you, because it is done with your consent. You have agreed to work for your boss for certain pay, he to have all that you produce \unknown{}}


{\em “But did you really consent?}


{\em “When the highway man holds his gun to your head, you turn your valuables over to him. You ‘consent’ all right, but you do so because you cannot help yourself, because you are {\bf compelled} by his gun.}


{\em “Are you not {\bf compelled} to work for an employer? Your need compels you just as the highwayman’s gun. You must live\unknown{} You can’t work for yourself \unknown{}The factories, machinery, and tools belong to the employing class, so you {\bf must} hire yourself out to that class in order to work and live. Whatever you work at, whoever your employer may be, it is always comes to the same: you must work {\bf for him}. You can’t help yourself. You are {\bf compelled}.”} [{\bf What is Communist Anarchism?}, p. 9]



\stopblockquote
Due to this class monopoly over the means of life, workers (usually) are at a disadvantage in terms of bargaining power — there are more workers than jobs (see sections B.4.3 and 10.2 for a discussion why this is the normal situation on the labour market).


As was indicated in section B.4 (How does capitalism affect liberty?), within capitalism there is no equality between owners and the dispossessed, and so property is a source of {\bf power.} To claim that this power should be “left alone” or is “fair” is {\em “to the anarchists\unknown{} preposterous. Once a State has been established, and most of the country’s capital privatised, the threat of physical force is no longer necessary to coerce workers into accepting jobs, even with low pay and poor conditions. To use Ayn Rand’s term, ‘initial force’ has {\bf already taken place,} by those who now have capital against those who do not\unknown{} In other words, if a thief died and willed his ‘ill-gotten gain’ to his children, would the children have a right to the stolen property? Not legally. So if ‘property is theft,’ to borrow Proudhon’s quip, and the fruit of exploited labour is simply legal theft, then the only factor giving the children of a deceased capitalist a right to inherit the ‘booty’ is the law, the State. As Bakunin wrote, ‘Ghosts should not rule and oppress this world, which belongs only to the living’”} [Jeff Draughn, {\bf Between Anarchism and Libertarianism}].


Or, in other words, right-Libertarianism fails to {\em “meet the charge that normal operations of the market systematically places an entire class of persons (wage earners) in circumstances that compel them to accept the terms and conditions of labour dictated by those who offer work. While it is true that individuals are formally free to seek better jobs or withhold their labour in the hope of receiving higher wages, in the end their position in the market works against them; they cannot live if they do not find employment. When circumstances regularly bestow a relative disadvantage on one class of persons in their dealings with another class, members of the advantaged class have little need of coercive measures to get what they want.”} [Stephen L. Newman, {\bf Liberalism at Wit’s End}, p. 130]


To ignore the circumstances which drive people to seek out the most “beneficial exchange” is to blind yourself to the power relationships inherent within capitalism — power relationships created by the unequal bargaining power of the parties involved (also see section 3.1). And to argue that “consent” ensures freedom is false; if you are “consenting” to be join a dictatorial organisation, you “consent” {\bf not} to be free (and to paraphrase Rousseau, a person who renounces freedom renounces being human).


Which is why circumstances are important — if someone truly wants to join an authoritarian organisation, then so be it. It is their life. But if circumstances ensure their “consent” then they are not free. The danger is, of course, that people become {\bf accustomed} to authoritarian relationships and end up viewing them as forms of freedom. This can be seen from the state, which the vast majority support and “consent” to. And this also applies to wage labour, which many workers today accept as a “necessary evil” (like the state) but, as we indicate in section 8.6, the first wave of workers viewed with horror as a form of (wage) slavery and did all that they could to avoid. In such situations all we can do is argue with them and convince them that certain forms of organisations (such as the state and capitalist firms) are an evil and urge them to change society to ensure their extinction.


So due to this lack of appreciation of circumstances (and the fact that people become accustomed to certain ways of life) “anarcho”-capitalism actively supports structures that restrict freedom for the many. And how is “anarcho”-capitalism {\bf anarchist} if it generates extensive amounts of archy? It is for this reason that all anarchists support self-management within free association — that way we maximise freedom both inside {\bf and} outside organisations. But only stressing freedom outside organisations, “anarcho”-capitalism ends up denying freedom as such (after all, we spend most of our waking hours at work). If “anarcho”-capitalists {\bf really} desired freedom, they would reject capitalism and become anarchists — only in a libertarian socialist society would agreements to become a wage worker be truly voluntary as they would not be driven by circumstances to sell their liberty.


This means that while right-Libertarianism appears to make “choice” an ideal (which sounds good, liberating and positive) in practice it has become a “dismal politics,” a politics of choice where most of the choices are bad. And, to state the obvious, the choices we are “free” to make are shaped by the differences in wealth and power in society (see section 3.1) as well as such things as “isolation paradoxes” (see section B.6) and the laws and other human institutions that exist. If we ignore the context within which people make their choices then we glorify abstract processes at the expense of real people. And, as importantly, we must add that many of the choices we make under capitalism (shaped as they are by the circumstances within which they are made), such as employment contracts, result in our “choice” being narrowed to “love it or leave it” in the organisations we create/join as a result of these “free” choices.


This ideological blind spot flows from the “anarcho”-capitalist definition of “freedom” as “absence of coercion” — as workers “freely consent” to joining a specific workplace, their freedom is unrestricted. But to defend {\bf only} “freedom from” in a capitalist society means to defend the power and authority of the few against the attempts of the many to claim their freedom and rights. To requote Emma Goldman, {\em “‘Rugged individualism’ has meant all the ‘individualism’ for the masters \unknown{} , in whose name political tyranny and social oppression are defended and held up as virtues’ while every aspiration and attempt of man to gain freedom \unknown{} is denounced as \unknown{} evil in the name of that same individualism.”} [{\bf Red Emma Speaks}, p. 112]


In other words, its all fine and well saying (as right-libertarians do) that you aim to abolish force from human relationships but if you support an economic system which creates hierarchy (and so domination and oppression) by its very workings, “defensive” force will always be required to maintain and enforce that domination. Moreover, if one class has extensive power over another due to the systematic (and normal) workings of the market, any force used to defend that power is {\bf automatically} “defensive”. Thus to argue against the use of force and ignore the power relationships that exist within and shape a society (and so also shape the individuals within it) is to defend and justify capitalist and landlord domination and denounce any attempts to resist that domination as “initiation of force.”


Anarchists, in contrast, oppose {\bf hierarchy} (and so domination within relationships — bar S\&M personal relationships, which are a totally different thing altogether; they are truly voluntary and they also do not attempt to hide the power relationships involved by using economic jargon). This opposition, while also including opposition to the use of force against equals (for example, anarchists are opposed to forcing workers and peasants to join a self-managed commune or syndicate), also includes support for the attempts of those subject to domination to end it (for example, workers striking for union recognition are not “initiating force”, they are fighting for their freedom).


In other words, apparently “voluntary” agreements can and do limit freedom and so the circumstances that drive people into them {\bf must} be considered when deciding whether any such limitation is valid. By ignoring circumstances, “anarcho”-capitalism ends up by failing to deliver what it promises — a society of free individuals — and instead presents us with a society of masters and servants. The question is, what do we feel moved to insist that people enjoy? Formal, abstract (bourgeois) self-ownership (“freedom”) or a more substantive control over one’s life (i.e. autonomy)?


\subsection{2.5 But surely circumstances are the result of liberty and so cannot be objected to?
}

It is often argued by right-libertarians that the circumstances we face within capitalism are the result of individual decisions (i.e. individual liberty) and so we must accept them as the expressions of these acts (the most famous example of this argument is in Nozick’s {\bf Anarchy, State, and Utopia} pp. 161–163 where he maintains that {\em “liberty upsets patterns”}). This is because whatever situation evolves from a just situation by just (i.e. non-coercive steps) is also (by definition) just.


However, it is not apparent that adding just steps to a just situation will result in a just society. We will illustrate with a couple of banal examples. If you add chemicals which are non-combustible together you can create a new, combustible, chemical (i.e. X becomes not-X by adding new X to it). Similarly, if you have an odd number and add another odd number to it, it becomes even (again, X becomes not-X by adding a new X to it). So it {\bf is} very possible to go from an just state to an unjust state by just step (and it is possible to remain in an unjust state by just acts; for example if we tried to implement “anarcho”-capitalism on the existing — unjustly created — situation of “actually existing” capitalism it would be like having an odd number and adding even numbers to it). In other words, the outcome of “just” steps can increase inequality within society and so ensure that some acquire an unacceptable amount of power over others, via their control over resources. Such an inequality of power would create an “unjust” situation where the major are free to sell their liberty to others due to inequality in power and resources on the “free” market.


Ignoring this objection, we could argue (as many “anarcho”-capitalists and right-libertarians do) that the unforeseen results of human action are fine unless we assume that these human actions are in themselves bad (i.e. that individual choice is evil).


Such an argument is false for three reasons.


First, when we make our choices the aggregate impact of these choices are unknown to us — and not on offer when we make our choices. Thus we cannot be said to “choose” these outcomes, outcomes which we may consider deeply undesirable, and so the fact that these outcomes are the result of individual choices is besides the point (if we knew the outcome we could refrain from doing them). The choices themselves, therefore, do not validate the outcome as the outcome was not part of the choices when they where made (i.e. the means do not justify the ends). In other words, private acts often have important public consequences (and “bilateral exchanges” often involve externalities for third parties). Secondly, if the outcome of individual choices is to deny or restrict individual choice on a wider scale at a later stage, then we are hardly arguing that individual choice is a bad thing. We want to arrange it so that the decisions we make now do not result in them restricting our ability to make choices in important areas of life at a latter stage. Which means we are in favour of individual choices and so liberty, not against them. Thirdly, the unforeseen or unplanned results of individual actions are not necessarily a good thing. If the aggregate outcome of individual choices harms individuals then we have a right to modify the circumstances within which choices are made and/or the aggregate results of these choices.


An example will show what we mean (again drawn from Haworth’s excellent {\bf Anti-Libertarianism}, p. 35). Millions of people across the world bought deodorants which caused a hole to occur in the ozone layer surrounding the Earth. The resultant of these acts created a situation in which individuals and the eco-system they inhabited were in great danger. The actual acts themselves were by no means wrong, but the aggregate impact was. A similar argument can apply to any form of pollution. Now, unless the right-Libertarian argues that skin cancer or other forms of pollution related illness are fine, its clear that the resultant of individual acts can be harmful to individuals.


The right-Libertarian could argue that pollution is an “initiation of force” against an individual’s property-rights in their person and so individuals can sue the polluters. But hierarchy also harms the individual (see section B.1) — and so can be considered as an infringement of their “property-rights” (i.e. liberty, to get away from the insane property fetish of right-Libertarianism). The loss of autonomy can be just as harmful to an individual as lung cancer although very different in form. And the differences in wealth resulting from hierarchy is well known to have serious impacts on life-span and health.


As noted in section 2.1, the market is just as man-made as pollution. This means that the “circumstances” we face are due to aggregate of millions of individual acts and these acts occur within a specific framework of rights, institutions and ethics. Anarchists think that a transformation of our society and its rights and ideals is required so that the resultant of individual choices does not have the ironic effect of limiting individual choice (freedom) in many important ways (such as in work, for example).


In other words, the {\bf circumstances} created by capitalist rights and institutions requires a {\bf transformation} of these rights and institutions in such a way as to maximise individual choice for all — namely, to abolish these rights and replace them with new ones (for example, replace property rights with use rights). Thus Nozick’s claims that {\em “Z does choose voluntarily if the other individuals A through Y each acted voluntarily and within their rights”} [{\bf Op. Cit.}, p. 263] misses the point — it is these rights that are in question (given that Nozick {\bf assumes} these rights then his whole thesis is begging the question).


And we must add (before anyone points it out) that, yes, we are aware that many decisions will unavoidably limit current and future choices. For example, the decision to build a factory on a green-belt area will make it impossible for people to walk through the woods that are no longer there. But such “limitations” (if they can be called that) of choice are different from the limitations we are highlighting here, namely the lose of freedom that accompanies the circumstances created via exchange in the market. The human actions which build the factory modify reality but do not generate social relationships of domination between people in so doing. The human actions of market exchange, in contrast, modify the relative strengths of everyone in society and so has a distinct impact on the social relationships we “voluntarily” agree to create. Or, to put it another way, the decision to build on the green-belt site does “limit” choice in the abstract but it does {\bf not} limit choice in the kind of relationships we form with other people nor create authoritarian relationships between people due to inequality influencing the content of the associations we form. However, the profits produced from using the factory increases inequality (and so market/economic power) and so weakens the position of the working class in respect to the capitalist class within society. This increased inequality will be reflected in the “free” contracts and working regimes that are created, with the weaker “trader” having to compromise far more than before.


So, to try and defend wage slavery and other forms of hierarchy by arguing that “circumstances” are created by individual liberty runs aground on its own logic. If the circumstances created by individual liberty results in pollution then the right-Libertarian will be the first to seek to change those circumstances. They recognise that the right to pollute while producing is secondary to our right to be healthy. Similarly, if the circumstances created by individual liberty results in hierarchy (pollution of the mind and our relationships with others as opposed to the body, although it affects that to) then we are entitled to change these circumstances too and the means by which we get there (namely the institutional and rights framework of society). Our right to liberty is more important than the rights of property — sadly, the right-Libertarian refuses to recognise this.


\subsection{2.6 Do Libertarian-capitalists support slavery?
}

Yes. It may come as a surprise to many people, but right-Libertarianism is one of the few political theories that justifies slavery. For example, Robert Nozick asks whether {\em “a free system would allow [the individual] to sell himself into slavery”} and he answers {\em “I believe that it would.”} [{\bf Anarchy, State and Utopia}, p. 371] While some right-Libertarians do not agree with Nozick, there is no logical basis in their ideology for such disagreement.


The logic is simple, you cannot really own something unless you can sell it. Self-ownership is one of the cornerstones of laissez-faire capitalist ideology. Therefore, since you own yourself you can sell yourself.


(For Murray Rothbard’s claims of the {\em “unenforceability, in libertarian theory, of voluntary slave contracts”} see {\bf The Ethics of Liberty}, pp. 134–135 — of course, {\bf other} libertarian theorists claim the exact opposite so {\em “libertarian theory”} makes no such claims, but nevermind! Essentially, his point revolves around the assertion that a person {\em “cannot, in nature, sell himself into slavery and have this sale enforced — for this would mean that his future will over his own body was being surrendered in advance”} and that if a {\em “labourer remains totally subservient to his master’s will voluntarily, he is not yet a slave since his submission is voluntary.”} [p. 40] However, as we noted in section 2, Rothbard emphasis on quitting fails to recognise that actual denial of will and control over ones own body that is explicit in wage labour. It is this failure that pro-slave contract “libertarians” stress — as we will see, they consider the slave contract as an extended wage contract. Moreover, a modern slave contract would likely take the form of a {\em “performance bond”} [p. 136] in which the slave agrees to perform X years labour or pay their master substantial damages. The threat of damages that enforces the contract and such a “contract” Rothbard does agree is enforceable — along with {\em “conditional exchange”} [p. 141] which could be another way of creating slave contracts.)


Nozick’s defence of slavery should not come as a surprise to any one familiar with classical liberalism. An elitist ideology, its main rationale is to defend the liberty and power of property owners and justify unfree social relationships (such as government and wage labour) in terms of “consent.” Nozick just takes it to its logical conclusion, a conclusion which Rothbard, while balking at the label used, does not actually disagree with.


This is because Nozick’s argument is not new but, as with so many others, can be found in John Locke’s work. The key difference is that Locke refused the term {\em “slavery”} and favoured {\em “drudgery”} as, for him, slavery mean a relationship {\em “between a lawful conqueror and a captive”} where the former has the power of life and death over the latter. Once a {\em “compact”} is agreed between them, {\em “an agreement for a limited power on the one side, and obedience on the other \unknown{} slavery ceases.”} As long as the master could not kill the slave, then it was {\em “drudgery.”} Like Nozick, he acknowledges that {\em “men did sell themselves; but, it is plain, this was only to drudgery, not to slavery: for, it is evident, the person sold was not under an absolute, arbitrary, despotical power: for the master could not have power to kill him, at any time, whom, at a certain time, he was obliged to let go free out of his service.”} [Locke, {\bf Second Treatise of Government}, Section 24] In other words, like Rothbard, voluntary slavery was fine but just call it something else.


Not that Locke was bothered by involuntary slavery. He was heavily involved in the slave trade. He owned shares in the “Royal Africa Company” which carried on the slave trade for England, making a profit when he sold them. He also held a significant share in another slave company, the “Bahama Adventurers.” In the {\em “Second Treatise”}, Locke justified slavery in terms of {\em “Captives taken in a just war.”} [Section 85] In other words, a war waged against aggressors. That, of course, had nothing to do with the {\bf actual} slavery Locke profited from (slave raids were common, for example). Nor did his “liberal” principles stop him suggesting a constitution that would ensure that {\em “every freeman of Carolina shall have absolute power and authority over his Negro slaves.”} The constitution itself was typically autocratic and hierarchical, designed explicitly to {\em “avoid erecting a numerous democracy.”} [{\bf The Works of John Locke}, vol. X, p. 196]


So the notion of contractual slavery has a long history within right-wing liberalism, although most refuse to call it by that name. It is of course simply embarrassment that stops Rothbard calling a spade a spade. He incorrectly assumes that slavery has to be involuntary. In fact, historically, voluntary slave contracts have been common (David Ellerman’s {\bf Property and Contract in Economics} has an excellent overview). Any new form of voluntary slavery would be a “civilised” form of slavery and could occur when an individual would “agree” to sell themselves to themselves to another (as when a starving worker would “agree” to become a slave in return for food). In addition, the contract would be able to be broken under certain conditions (perhaps in return for breaking the contract, the former slave would have pay damages to his or her master for the labour their master would lose — a sizeable amount no doubt and such a payment could result in debt slavery, which is the most common form of “civilised” slavery. Such damages may be agreed in the contract as a “performance bond” or “conditional exchange”).


In summary, right-Libertarians are talking about “civilised” slavery (or, in other words, civil slavery) and not forced slavery. While some may have reservations about calling it slavery, they agree with the basic concept that since people own themselves they can sell themselves as well as selling their labour for a lifetime.


We must stress that this is no academic debate. “Voluntary” slavery has been a problem in many societies and still exists in many countries today (particularly third world ones where bonded labour — i.e. where debt is used to enslave people — is the most common form). With the rise of sweat shops and child labour in many “developed” countries such as the USA, “voluntary” slavery (perhaps via debt and bonded labour) may become common in all parts of the world — an ironic (if not surprising) result of “freeing” the market and being indifferent to the actual freedom of those within it.


And it is interesting to note that even Murray Rothbard is not against the selling of humans. He argued that children are the property of their parents. They can (bar actually murdering them by violence) do whatever they please with them, even sell them on a {\em “flourishing free child market.”} [{\bf The Ethics of Liberty}, p. 102] Combined with a whole hearted support for child labour (after all, the child can leave its parents if it objects to working for them) such a “free child market” could easily become a “child slave market” — with entrepreneurs making a healthy profit selling infants to other entrepreneurs who could make profits from the toil of “their” children (and such a process did occur in 19\high{th} century Britain). Unsurprisingly, Rothbard ignores the possible nasty aspects of such a market in human flesh (such as children being sold to work in factories, homes and brothels). And, of course, such a market could see women “specialising” in producing children for it (the use of child labour during the Industrial Revolution actually made it economically sensible for families to have more children) and, perhaps, gluts and scarcities of babies due to changing market conditions. But this is besides the point.


Of course, this theoretical justification for slavery at the heart of an ideology calling itself “libertarianism” is hard for many right-Libertarians to accept. Some of the “anarcho”-capitalist type argue that such contracts would be very hard to enforce in their system of capitalism. This attempt to get out of the contradiction fails simply because it ignores the nature of the capitalist market. If there is a demand for slave contracts to be enforced, then companies will develop to provide that “service” (and it would be interesting to see how two “protection” firms, one defending slave contracts and another not, could compromise and reach a peaceful agreement over whether slave contracts were valid). Thus we could see a so-called “anarchist” or “free” society producing companies whose specific purpose was to hunt down escaped slaves (i.e. individuals in slave contracts who have not paid damages to their owners for freedom). Of course, perhaps Rothbard would claim that such slave contracts would be “outlawed” under his “general libertarian law code” but this is a denial of market “freedom”. If slave contracts {\bf are} “banned” then surely this is paternalism, stopping individuals from contracting out their “labour services” to whom and however long they “desire”. You cannot have it both ways.


So, ironically, an ideology proclaiming itself to support “liberty” ends up justifying and defending slavery. Indeed, for the right-libertarian the slave contract is an exemplification, not the denial, of the individual’s liberty! How is this possible? How can slavery be supported as an expression of liberty? Simple, right-Libertarian support for slavery is a symptom of a {\bf deeper} authoritarianism, namely their uncritical acceptance of contract theory. The central claim of contract theory is that contract is the means to secure and enhance individual freedom. Slavery is the antithesis to freedom and so, in theory, contract and slavery must be mutually exclusive. However, as indicated above, some contract theorists (past and present) have included slave contracts among legitimate contracts. This suggests that contract theory cannot provide the theoretical support needed to secure and enhance individual freedom. Why is this?


As Carole Pateman argues, {\em “contract theory is primarily about a way of creating social relations constituted by subordination, not about exchange.”} Rather than undermining subordination, contract theorists justify modern subjection — {\em “contract doctrine has proclaimed that subjection to a master — a boss, a husband — is freedom.”} [{\bf The Sexual Contract}, p. 40 and p. 146] The question central to contract theory (and so right-Libertarianism) is not “are people free” (as one would expect) but “are people free to subordinate themselves in any manner they please.” A radically different question and one only fitting to someone who does not know what liberty means.


Anarchists argue that not all contracts are legitimate and no free individual can make a contract that denies his or her own freedom. If an individual is able to express themselves by making free agreements then those free agreements must also be based upon freedom internally as well. Any agreement that creates domination or hierarchy negates the assumptions underlying the agreement and makes itself null and void. In other words, voluntary government is still government and the defining chararacteristic of an anarchy must be, surely, “no government” and “no rulers.”


This is most easily seen in the extreme case of the slave contract. John Stuart Mill stated that such a contract would be “null and void.” He argued that an individual may voluntarily choose to enter such a contract but in so doing {\em “he abdicates his liberty; he foregoes any future use of it beyond that single act. He therefore defeats, in his own case, the very purpose which is the justification of allowing him to dispose of himself\unknown{}The principle of freedom cannot require that he should be free not to be free. It is not freedom, to be allowed to alienate his freedom.”} He adds that {\em “these reasons, the force of which is so conspicuous in this particular case, are evidently of far wider application.”} [quoted by Pateman, {\bf Op. Cit.}, pp. 171–2]


And it is such an application that defenders of capitalism fear (Mill did in fact apply these reasons wider and unsurprisingly became a supporter of a market syndicalist form of socialism). If we reject slave contracts as illegitimate then, logically, we must also reject {\bf all} contracts that express qualities similar to slavery (i.e. deny freedom) including wage slavery. Given that, as David Ellerman points out, {\em “the voluntary slave \unknown{} and the employee cannot in fact take their will out of their intentional actions so that they could be ‘employed’ by the master or employer”} we are left with {\em “the rather implausible assertion that a person can vacate his or her will for eight or so hours a day for weeks, months, or years on end but cannot do so for a working lifetime.”} [{\bf Property and Contract in Economics}, p. 58]


The implications of supporting voluntary slavery is quite devastating for all forms of right-wing “libertarianism.” This was proven by Ellerman when he wrote an extremely robust defence of it under the pseudonym “J. Philmore” called {\bf The Libertarian Case for Slavery} (first published in {\bf The Philosophical Forum}, xiv, 1982). This classic rebuttal takes the form of “proof by contradiction” (or {\bf reductio ad absurdum}) whereby he takes the arguments of right-libertarianism to their logical end and shows how they reach the memorably conclusion that the {\em “time has come for liberal economic and political thinkers to stop dodging this issue and to critically re-examine their shared prejudices about certain voluntary social institutions \unknown{} this critical process will inexorably drive liberalism to its only logical conclusion: libertarianism that finally lays the true moral foundation for economic and political slavery.”}


Ellerman shows how, from a right-“libertarian” perspective there is a {\em “fundamental contradiction”} in a modern liberal society for the state to prohibit slave contracts. He notes that there {\em “seems to be a basic shared prejudice of liberalism that slavery is inherently involuntary, so the issue of genuinely voluntary slavery has received little scrutiny. The perfectly valid liberal argument that involuntary slavery is inherently unjust is thus taken to include voluntary slavery (in which case, the argument, by definition, does not apply). This has resulted in an abridgment of the freedom of contract in modern liberal society.”} Thus it is possible to argue for a {\em “civilised form of contractual slavery.”} [“J. Philmore,”, {\bf Op. Cit.}]


So accurate and logical was Ellerman’s article that many of its readers were convinced it {\bf was} written by a right-libertarian (including, we have to say, us!). One such writer was Carole Pateman, who correctly noted that {\em “[t]here is a nice historical irony here. In the American South, slaves were emancipated and turned into wage labourers, and now American contractarians argue that all workers should have the opportunity to turn themselves into civil slaves.”} [{\bf Op. Cit.}, p. 63]).


The aim of Ellerman’s article was to show the problems that employment (wage labour) presents for the concept of self-government and how contract need not result in social relationships based on freedom. As “Philmore” put it, {\em “[a]ny thorough and decisive critique of voluntary slavery or constitutional nondemocratic government would carry over to the employment contract — which is the voluntary contractual basis for the free-market free-enterprise system. Such a critique would thus be a {\bf reductio ad absurdum}.”} As {\em “contractual slavery”} is an {\em “extension of the employer-employee contract,”} he shows that the difference between wage labour and slavery is the time scale rather than the principle or social relationships involved. [{\bf Op. Cit.}] This explains, firstly, the early workers’ movement called capitalism {\em {\bf “wage slavery”}} (anarchists still do) and, secondly, why capitalists like Rothbard support the concept but balk at the name. It exposes the unfree nature of the system they support! While it is possible to present wage labour as “freedom” due to its “consensual” nature, it becomes much harder to do so when talking about slavery or dictatorship. Then the contradictions are exposed for all to see and be horrified by.


All this does not mean that we must reject free agreement. Far from it! Free agreement is {\bf essential} for a society based upon individual dignity and liberty. There are a variety of forms of free agreement and anarchists support those based upon co-operation and self-management (i.e. individuals working together as equals). Anarchists desire to create relationships which reflect (and so express) the liberty that is the basis of free agreement. Capitalism creates relationships that deny liberty. The opposition between autonomy and subjection can only be maintained by modifying or rejecting contract theory, something that capitalism cannot do and so the right-wing Libertarian rejects autonomy in favour of subjection (and so rejects socialism in favour of capitalism).


The real contrast between anarchism and right-Libertarianism is best expressed in their respective opinions on slavery. Anarchism is based upon the individual whose individuality depends upon the maintenance of free relationships with other individuals. If individuals deny their capacities for self-government from themselves through a contract the individuals bring about a qualitative change in their relationship to others — freedom is turned into mastery and subordination. For the anarchist, slavery is thus the paradigm of what freedom is {\bf not}, instead of an exemplification of what it is (as right-Libertarians state). As Proudhon argued:



\startblockquote
{\em “If I were asked to answer the following question: What is slavery? and I should answer in one word, It is murder, my meaning would be understood at once. No extended argument would be required to show that the power to take from a man his thought, his will, his personality, is a power of life and death; and that to enslave a man is to kill him.”} [{\bf What is Property?}, p. 37]



\stopblockquote
In contrast, the right-Libertarian effectively argues that “I support slavery because I believe in liberty.” It is a sad reflection of the ethical and intellectual bankruptcy of our society that such an “argument” is actually taken seriously by (some) people. The concept of “slavery as freedom” is far too Orwellian to warrant a critique — we will leave it up to right Libertarians to corrupt our language and ethical standards with an attempt to prove it.


From the basic insight that slavery is the opposite of freedom, the anarchist rejection of authoritarian social relations quickly follows (the right-wing Libertarians fear):



\startblockquote
{\em “Liberty is inviolable. I can neither sell nor alienate my liberty; every contract, every condition of a contract, which has in view the alienation or suspension of liberty, is null: the slave, when he plants his foot upon the soil of liberty, at that moment becomes a free man\unknown{} Liberty is the original condition of man; to renounce liberty is to renounce the nature of man: after that, how could we perform the acts of man?”} [P.J. Proudhon, {\bf Op. Cit.}, p. 67]



\stopblockquote
The employment contract (i.e. wage slavery) abrogates liberty. It is based upon inequality of power and {\em “exploitation is a consequence of the fact that the sale of labour power entails the worker’s subordination.”} [Carole Pateman, {\bf Op. Cit.}, P. 149] Hence Proudhon’s (and Mill’s) support of self-management and opposition to capitalism — any relationship that resembles slavery is illegitimate and no contract that creates a relationship of subordination is valid. Thus in a truly anarchistic society, slave contracts would be unenforceable — people in a truly free (i.e. non-capitalist) society would {\bf never} tolerate such a horrible institution or consider it a valid agreement. If someone was silly enough to sign such a contract, they would simply have to say they now rejected it in order to be free — such contracts are made to be broken and without the force of a law system (and private defence firms) to back it up, such contracts will stay broken.


The right-Libertarian support for slave contracts (and wage slavery) indicates that their ideology has little to do with liberty and far more to do with justifying property and the oppression and exploitation it produces. Their support and theoretical support for slavery indicates a deeper authoritarianism which negates their claims to be libertarians.


\subsection{2.7 But surely abolishing capitalism would restrict liberty?
}

Many “anarcho”-capitalists and other supporters of capitalism argue that it would be “authoritarian” to restrict the number of alternatives that people can choose between by abolishing capitalism. If workers become wage labourers, so it is argued, it is because they “value” other things more — otherwise they would not agree to the exchange. But such an argument ignores that reality of capitalism.


By {\bf maintaining} capitalist private property, the options available to people {\bf are} restricted. In a fully developed capitalist economy the vast majority have the “option” of selling their labour or starving/living in poverty — self-employed workers account for less than 10\% of the working population. Usually, workers are at a disadvantage on the labour market due to the existence of unemployment and so accept wage labour because otherwise they would starve (see section 10.2 for a discussion on why this is the case). And as we argue in sections J.5.11 and J.5.12, even {\bf if} the majority of the working population desired co-operative workplaces, a capitalist market will not provide them with that outcome due to the nature of the capitalist workplace (also see Juliet C. Schor’s excellent book {\bf The Overworked American} for a discussion of why workers desire for more free time is not reflected in the labour market). In other words, it is a myth to claim that wage labour exists or that workplaces are hierarchical because workers value other things — they are hierarchical because bosses have more clout on the market than workers and, to use Schor’s expression, workers end up wanting what they get rather than getting what they want.


Looking at the reality of capitalism we find that because of inequality in resources (protected by the full might of the legal system, we should note) those with property get to govern those without it during working hours (and beyond in many cases). If the supporters of capitalism were actually concerned about liberty (as opposed to property) that situation would be abhorrent to them — after all, individuals can no longer exercise their ability to make decisions, choices, and are reduced to being order takers. If choice and liberty are the things we value, then the ability to make choices in all aspects of life automatically follows (including during work hours). However, the authoritarian relationships and the continual violation of autonomy wage labour implies are irrelevant to “anarcho”-capitalists (indeed, attempts to change this situation are denounced as violations of the autonomy of the property owner!). By purely concentrating on the moment that a contract is signed they blind themselves to the restricts of liberty that wage contracts create.


Of course, anarchists have no desire to {\bf ban} wage labour — we aim to create a society within which people are not forced by circumstances to sell their liberty to others. In order to do this, anarchists propose a modification of property and property rights to ensure true freedom of choice (a freedom of choice denied to us by capitalism). As we have noted many times, “bilateral exchanges” can and do adversely effect the position of third parties if they result in the build-up of power/money in the hands of a few. And one of these adverse effects can be the restriction of workers options due to economic power. Therefore it is the supporter of capitalist who restricts options by supporting an economic system and rights framework that by their very workings reduce the options available to the majority, who then are “free to choose” between those that remain (see also section B.4). Anarchists, in contrast, desire to expand the available options by abolishing capitalist private property rights and removing inequalities in wealth and power that help restrict our options and liberties artificially.


So does an anarchist society have much to fear from the spread of wage labour within it? Probably not. If we look at societies such as the early United States or the beginnings of the Industrial Revolution in Britain, for example, we find that, given the choice, most people preferred to work for themselves. Capitalists found it hard to find enough workers to employ and the amount of wages that had to be offered to hire workers were so high as to destroy any profit margins. Moreover, the mobility of workers and their “laziness” was frequently commented upon, with employers despairing at the fact workers would just work enough to make end meet and then disappear. Thus, left to the actions of the “free market,” it is doubtful that wage labour would have spread. But it was not left to the “free market”.


In response to these “problems”, the capitalists turned to the state and enforced various restrictions on society (the most important being the land, tariff and money monopolies — see sections B.3 and 8). In free competition between artisan and wage labour, wage labour only succeeded due to the use of state action to create the required circumstances to discipline the labour force and to accumulate enough capital to give capitalists an edge over artisan production (see section 8 for more details).


Thus an anarchist society would not have to fear the spreading of wage labour within it. This is simply because would-be capitalists (like those in the early United States) would have to offer such excellent conditions, workers’ control and high wages as to make the possibility of extensive profits from workers’ labour nearly impossible. Without the state to support them, they will not be able to accumulate enough capital to give them an advantage within a free society. Moreover, it is somewhat ironic to hear capitalists talking about anarchism denying choice when we oppose wage labour considering the fact workers were not given any choice when the capitalists used the state to develop wage labour in the first place!


\subsection{2.8 Why should we reject the “anarcho”-capitalist definitions of freedom and justice?
}

Simply because they lead to the creation of authoritarian social relationships and so to restrictions on liberty. A political theory which, when consistently followed, has evil or iniquitous consequences, is a bad theory.


For example, any theory that can justify slavery is obviously a bad theory — slavery does not cease to stink the moment it is seen to follow your theory. As right-Libertarians can justify slave contracts as a type of wage labour (see section 2.6) as well as numerous other authoritarian social relationships, it is obviously a bad theory.


It is worth quoting Noam Chomsky at length on this subject:



\startblockquote
{\em “Consider, for example, the ‘entitlement theory of justice’\unknown{} [a]ccording to this theory, a person has a right to whatever he has acquired by means that are just. If, by luck or labour or ingenuity, a person acquires such and such, then he is entitled to keep it and dispose of it as he wills, and a just society will not infringe on this right.}


{\em “One can easily determine where such a principle might lead. It is entirely possible that by legitimate means — say, luck supplemented by contractual arrangements ‘freely undertaken’ under pressure of need — one person might gain control of the necessities of life. Others are then free to sell themselves to this person as slaves, if he is willing to accept them. Otherwise, they are free to perish. Without extra question-begging conditions, the society is just.}


{\em “The argument has all the merits of a proof that 2 + 2 = 5\unknown{} Suppose that some concept of a ‘just society’ is advanced that fails to characterise the situation just described as unjust\unknown{} Then one of two conclusions is in order. We may conclude that the concept is simply unimportant and of no interest as a guide to thought or action, since it fails to apply properly even in such an elementary case as this. Or we may conclude that the concept advanced is to be dismissed in that it fails to correspond to the pretheorectical notion that it intends to capture in clear cases. If our intuitive concept of justice is clear enough to rule social arrangements of the sort described as grossly unjust, then the sole interest of a demonstration that this outcome might be ‘just’ under a given ‘theory of justice’ lies in the inference by {\bf reductio ad absurdum} to the conclusion that the theory is hopelessly inadequate. While it may capture some partial intuition regarding justice, it evidently neglects others.}


{\em “The real question to be raised about theories that fail so completely to capture the concept of justice in its significant and intuitive sense is why they arouse such interest. Why are they not simply dismissed out of hand on the grounds of this failure, which is striking in clear cases? Perhaps the answer is, in part, the one given by Edward Greenberg in a discussion of some recent work on the entitlement theory of justice. After reviewing empirical and conceptual shortcomings, he observes that such work ‘plays an important function in the process of \unknown{} ‘blaming the victim,’ and of protecting property against egalitarian onslaughts by various non-propertied groups.’ An ideological defence of privileges, exploitation, and private power will be welcomed, regardless of its merits.}


{\em “These matters are of no small importance to poor and oppressed people here and elsewhere.”} [{\bf The Chomsky Reader}, pp. 187–188]



\stopblockquote
It may be argued that the reductions in liberty associated with capitalism is not really an iniquitous outcome, but such an argument is hardly fitting for a theory proclaiming itself “libertarian.” And the results of these authoritarian social relationships? To quote Adam Smith, under the capitalist division of labour the worker {\em “has no occasion to exert his understanding, or exercise his invention”} and {\em “he naturally loses, therefore, the habit of such exercise and generally becomes as stupid and ignorant as it is possible for a human creature to become.”} The worker’s mind falls {\em “into that drowsy stupidity, which, in a civilised society, seems to benumb the understanding of almost all of the inferior [sic!] ranks of people.”} [cited by Chomsky, {\bf Op. Cit.}, p. 186]


Of course, it may be argued that these evil effects of capitalist authority relations on individuals are also not iniquitous (or that the very real domination of workers by bosses is not really domination) but that suggests a desire to sacrifice real individuals, their hopes and dreams and lives to an abstract concept of liberty, the accumulative effect of which would be to impoverish all our lives. The kind of relationships we create {\bf within} the organisations we join are of as great an importance as their voluntary nature. Social relations {\bf shape} the individual in many ways, restricting their freedom, their perceptions of what freedom is and what their interests actually are. This means that, in order not to be farcical, any relationships we create must reflect in their internal workings the critical evaluation and self-government that created them in the first place. Sadly capitalist individualism masks structures of power and relations of domination and subordination within seemingly “voluntary” associations — it fails to note the relations of domination resulting from private property and so {\em “what has been called ‘individualism’ up to now has been only a foolish egoism which belittles the individual. Foolish because it was not individualism at all. It did not lead to what was established as a goal; that is the complete, broad, and most perfectly attainable development of individuality.”} [Peter Kropotkin, {\bf Selected Writings}, p. 297]


This right-Libertarian lack of concern for concrete individual freedom and individuality is a reflection of their support for “free markets” (or “economic liberty” as they sometimes phrase it). However, as Max Stirner noted, this fails to understand that {\em “[p]olitical liberty means that the {\bf polis,} the State, is free; \unknown{} not, therefore, that I am free of the State\unknown{} It does not mean {\bf my} liberty, but the liberty of a power that rules and subjugates me; it means that one of my {\bf despots} \unknown{} is free.”} [{\bf The Ego and Its Own}, p. 107] Thus the desire for “free markets” results in a blindness that while the market may be “free” the individuals within it may not be (as Stirner was well aware, {\em “[u]nder the {\bf regime} of the commonality the labourers always fall into the hands of the possessors \unknown{} of the capitalists, therefore.”} [{\bf Op. Cit.}, p. 115])


In other words, right-libertarians give the greatest importance to an abstract concept of freedom and fail to take into account the fact that real, concrete freedom is the outcome of self-managed activity, solidarity and voluntary co-operation. For liberty to be real it must exist in all aspects of our daily life and cannot be contracted away without seriously effecting our minds, bodies and lives. Thus, the right-Libertarian’s {\em “defence of freedom is undermined by their insistence on the concept of negative liberty, which all too easily translates in experience as the negation of liberty.”} [Stephan L. Newman, {\bf Liberalism as Wit’s End}, p. 161]


Thus right-Libertarian’s fundamental fallacy is that “contract” does not result in the end of power or domination (particularly when the bargaining power or wealth of the would-be contractors is not equal). As Carole Pateman notes, {\em “[i]ronically, the contractarian ideal cannot encompass capitalist employment. Employment is not a continual series of discrete contracts between employer and worker, but \unknown{} one contract in which a worker binds himself to enter an enterprise and follow the directions of the employer for the duration of the contract. As Huw Benyon has bluntly stated, ‘workers are paid to obey.’”} [{\bf The Sexual Contract}, p. 148] This means that {\em “the employment contract (like the marriage contract) is not an exchange; both contracts create social relations that endure over time — social relations of subordination.”} [{\bf Ibid.}]


Authority impoverishes us all and must, therefore, be combated wherever it appears. That is why anarchists oppose capitalism, so that there shall be {\em “no more government of man by man, by means of accumulation of capital.”} [P-J Proudhon, cited by Woodcock in {\bf Anarchism}, p. 110] If, as Murray Bookchin point it, {\em “the object of anarchism is to increase choice”} [{\bf The Ecology of Freedom}, p. 70] then this applies both to when we are creating associations/relationships with others and when we are {\bf within} these associations/relationships — i.e. that they are consistent with the liberty of all, and that implies participation and self-management {\bf not} hierarchy. “Anarcho”-capitalism fails to understand this essential point and by concentrating purely on the first condition for liberty ensures a society based upon domination, oppression and hierarchy and not freedom.


It is unsurprising, therefore, to find that the basic unit of analysis of the “anarcho”-capitalist/right-libertarian is the transaction (the “trade,” the “contract”). The freedom of the individual is seen as revolving around an act, the contract, and {\bf not} in our relations with others. All the social facts and mechanisms that precede, surround and result from the transaction are omitted. In particular, the social relations that result from the transaction are ignored (those, and the circumstances that make people contract, are the two unmentionables of right-libertarianism).


For anarchists it seems strange to concentrate on the moment that a contract is signed and ignore the far longer time the contract is active for (as we noted in section A.2.14, if the worker is free when they sign a contract, slavery soon overtakes them). Yes, the voluntary nature of a decision is important, but so are the social relationships we experience due to those decisions.


For the anarchist, freedom is based upon the insight that other people, apart from (indeed, {\bf because} of) having their own intrinsic value, also are “means to my end”, that it is through their freedom that I gain my own — so enriching my life. As Bakunin put it:



\startblockquote
{\em “I who want to be free cannot be because all the men around me do not yet want to be free, and consequently they become tools of oppression against me.”} [quoted by Errico Malatesta in {\bf Anarchy}, p. 27]



\stopblockquote
Therefore anarchists argue that we must reject the right-Libertarian theories of freedom and justice because they end up supporting the denial of liberty as the expression of liberty. What this fails to recognise is that freedom is a product of social life and that (in Bakunin’s words) {\em “[n]o man can achieve his own emancipation without at the same time working for the emancipation of all men around him. My freedom is the freedom of all since I am not truly free in thought and in fact, except when my freedom and my rights are confirmed and approved in the freedom and rights of all men who are my equals.”} [{\bf Ibid.}]


Other people give us the possibilities to develop our full human potentiality and thereby our freedom, so when we destroy the freedom of others we limit our own. {\em “To treat others and oneself as property,”} argues anarchist L. Susan Brown, {\em “objectifies the human individual, denies the unity of subject and object and is a negation of individual will \unknown{} even the freedom gained by the other is compromised by this relationship, for to negate the will of another to achieve one’s own freedom destroys the very freedom one sought in the first place.”} [{\bf The Politics of Individualism}, p. 3]


Fundamentally, it is for this reason that anarchists reject the right-Libertarian theories of freedom and justice — it just does not ensure individual freedom or individuality.


\section{3 Why do anarcho”-capitalists place little or no value on “equality”?
}

Murray Rothbard argues that {\em “the ‘rightist’ libertarian is not opposed to inequality.”} [{\bf For a New Liberty}, p. 47] In contrast, “leftist” libertarians oppose inequality because it has harmful effects on individual liberty.


Part of the reason “anarcho”-capitalism places little or no value on “equality” derives from their definition of that term. Murray Rothbard defines equality as:



\startblockquote
{\em “A and B are ‘equal’ if they are identical to each other with respect to a given attribute\unknown{} There is one and only one way, then, in which any two people can really be ‘equal’ in the fullest sense: they must be identical in {\bf all} their attributes.”}



\stopblockquote
He then points out the obvious fact that {\em “men are not uniform,\unknown{} the species, mankind, is uniquely characterised by a high degree of variety, diversity, differentiation: in short, inequality.”} [{\bf Egalitarianism as a Revolt against Nature and Other Essays}, p. 4, p.5]


In others words, every individual is unique. Something no egalitarian has ever denied. On the basis of this amazing insight, he concludes that equality is impossible (except “equality of rights”) and that the attempt to achieve “equality” is a “revolt against nature” — as if any anarchist had ever advocated such a notion of equality as being identical!


And so, because we are all unique, the outcome of our actions will not be identical and so social inequality flows from natural differences and not due to the economic system we live under. Inequality of endowment implies inequality of outcome and so social inequality. As individual differences are a fact of nature, attempts to create a society based on “equality” (i.e. making everyone identical in terms of possessions and so forth) is impossible and “unnatural.”


Before continuing, we must note that Rothbard is destroying language to make his point and that he is not the first to abuse language in this particular way. In George Orwell’s {\bf 1984}, the expression {\em “all men are created equal”} could be translated into Newspeak, but it would make as much sense as saying {\em “all men have red hair,”} an obvious falsehood (see {\em “The Principles of Newspeak”} Appendix). It’s nice to know that “Mr. Libertarian” is stealing ideas from Big Brother, and for the same reason: to make critical thought impossible by restricting the meaning of words.


“Equality,” in the context of political discussion, does not mean “identical,” it usually means equality of rights, respect, worth, power and so forth. It does not imply treating everyone identically (for example, expecting an eighty year old man to do identical work to an eighteen violates treating both with respect as unique individuals). For anarchists, as Alexander Berkman writes, {\em “equality does not mean an equal amount but equal {\bf opportunity}\unknown{} Do not make the mistake of identifying equality in liberty with the forced equality of the convict camp. True anarchist equality implies freedom, not quantity. It does not mean that every one must eat, drink, or wear the same things, do the same work, or live in the same manner. Far from it: the very reverse, in fact. Individual needs and tastes differ, as appetites differ. It is {\bf equal} opportunity to satisfy them that constitutes true equality. Far from levelling, such equality opens the door for the greatest possible variety of activity and development. For human character is diverse, and only the repression of this free diversity results in levelling, in uniformity and sameness. Free opportunity and acting out your individuality means development of natural dissimilarities and variations\unknown{} Life in freedom, in anarchy will do more than liberate man merely from his present political and economic bondage. That will be only the first step, the preliminary to a truly human existence.”} [{\bf The ABC of Anarchism}, p. 25]


Thus anarchists reject the Rothbardian-Newspeak definition of equality as meaningless within political discussion. No two people are identical and so imposing “identical” equality between them would mean treating them as {\bf unequals}, i.e. not having equal worth or giving them equal respect as befits them as human beings and fellow unique individuals.


So what should we make of Rothbard’s claim? It is tempting just to quote Rousseau when he argued {\em “it is \unknown{} useless to inquire whether there is any essential connection between the two inequalities [social and natural]; for this would be only asking, in other words, whether those who command are necessarily better than those who obey, and if strength of body or of mind, wisdom, or virtue are always found in particular individuals, in proportion to their power or wealth: a question fit perhaps to be discussed by slaves in the hearing of their masters, but highly unbecoming to reasonable and free men in search of the truth.”} [{\bf The Social Contract and Discourses}, p. 49] But a few more points should be raised.


The uniqueness of individuals has always existed but for the vast majority of human history we have lived in very egalitarian societies. If social inequality did, indeed, flow from natural inequalities then {\bf all} societies would be marked by it. This is not the case. Indeed, taking a relatively recent example, many visitors to the early United States noted its egalitarian nature, something that soon changed with the rise of wage labour and industrial capitalism (a rise dependent upon state action, we must add, — see section 8). This implies that the society we live in (its rights framework, the social relationships it generates and so forth) has a far more of a decisive impact on inequality than individual differences. Thus certain rights frameworks will tend to magnify “natural” inequalities (assuming that is the source of the initial inequality, rather than, say, violence and force). As Noam Chomsky argues:



\startblockquote
{\em “Presumably it is the case that in our ‘real world’ some combination of attributes is conducive to success in responding to ‘the demands of the economic system’ \unknown{} One might suppose that some mixture of avarice, selfishness, lack of concern for others, aggressiveness, and similar characteristics play a part in getting ahead [in capitalism]\unknown{} Whatever the correct collection of attributes may be, we may ask what follows from the fact, if it is a fact, that some partially inherited combination of attributes tends to material success? All that follows \unknown{} is a comment on our particular social and economic arrangements \unknown{} The egalitarian might responds, in all such cases, that the social order should be changes so that the collection of attributes that tends to bring success no longer do so \unknown{} “} [{\bf The Chomsky Reader}, p. 190]



\stopblockquote
So, perhaps, if we change society then the social inequalities we see today would disappear. It is more than probable that natural difference has been long ago been replaced with {\bf social} inequalities, especially inequalities of property (which will tend to increase, rather than decrease, inequality). And as we argue in section 8 these inequalities of property were initially the result of force, {\bf not} differences in ability. Thus to claim that social inequality flows from natural differences is false as most social inequality has flown from violence and force. This initial inequality has been magnified by the framework of capitalist property rights and so the inequality within capitalism is far more dependent upon, say, the existence of wage labour, rather than “natural” differences between individuals.


If we look at capitalism, we see that in workplaces and across industries many, if not most, unique individuals receive identical wages for identical work (although this often is not the case for women and blacks, who receive less wages than male, white workers). Similarly, capitalists have deliberately introduced wage inequalities and hierarchies for no other reason that to divide (and so rule) the workforce (see section D.10). Thus, if we assume egalitarianism {\bf is} a revolt against nature, then much of capitalist economic life is in such a revolt (and when it is not, the “natural” inequalities have been imposed artificially by those in power).


Thus “natural” differences do not necessarily result in inequality as such. Given a different social system, “natural” differences would be encouraged and celebrated far wider than they are under capitalism (where, as we argued in section B.1, hierarchy ensures the crushing of individuality rather than its encouragement) without any change in social equality. The claim that “natural” differences generates social inequalities is question begging in the extreme — it takes the rights framework of society as a given and ignores the initial source of inequality in property and power. Indeed, inequality of outcome or reward is more likely to be influenced by social conditions rather than individual differences (as would be the case in a society based on wage labour or other forms of exploitation).


Another reason for “anarcho”-capitalist lack of concern for equality is that they think that {\em “liberty upsets patterns”} (see section 2.5, for example). It is argued that equality can only be maintained by restricting individual freedom to make exchanges or by taxation of income. However, what this argument fails to acknowledge is that inequality also restricts individual freedom (see next section, for example) and that the capitalist property rights framework is not the only one possible. After all, money is power and inequalities in terms of power easily result in restrictions of liberty and the transformation of the majority into order takers rather than free producers. In other words, once a certain level of inequality is reached, property does not promote, but actually conflicts with, the ends which render private property legitimate. Moreover, Nozick (in his “liberty upsets patterns” argument) {\em “has produced \unknown{} an argument for unrestricted private property using unrestricted private property, and thus he begs the question he tries to answer.”} [Andrew Kerhohan, {\em “Capitalism and Self-Ownership”}, from {\bf Capitalism}, p. 71] For example, a worker employed by a capitalist cannot freely exchange the machines or raw materials they have been provided with to use but Nozick does not class this distribution of “restricted” property rights as infringing liberty (nor does he argue that wage slavery itself restricts freedom, of course).


So in response to the claim that equality could only be maintained by continuously interfering with people’s lives, anarchists would say that the inequalities produced by capitalist property rights also involve extensive and continuous interference with people’s lives. After all, as Bob Black notes {\em “[y]our foreman or supervisor gives you more or-else orders in a week than the police do in a decade”} nevermind the other effects of inequality such as stress, ill health and so on [{\bf Libertarian as Conservative}]. Thus claims that equality involves infringing liberty ignores the fact that inequality also infringes liberty. A reorganisation of society could effectively minimise inequalities by eliminating the major source of such inequalities (wage labour) by self-management (see section I.5.12 for a discussion of “capitalistic acts” within an anarchist society). We have no desire to restrict free exchanges (after all, most anarchists desire to see the “gift economy” become a reality sooner or later) but we argue that free exchanges need not involve the unrestricted property rights Nozick assumes. As we argue in sections 2 and 3.1, inequality can easily led to the situation where self-ownership is used to justify its own negation and so unrestricted property rights may undermine the meaningful self-determination (what anarchists would usually call “freedom” rather than self-ownership) which many people intuitively understand by the term “self-ownership”.


Thus, for anarchists, the “anarcho”-capitalist opposition to equality misses the point and is extremely question begging. Anarchists do not desire to make humanity “identical” (which would be impossible and a total denial of liberty {\bf and} equality) but to make the social relationships between individuals equal in {\bf power.} In other words, they desire a situation where people interact together without institutionalised power or hierarchy and are influenced by each other “naturally,” in proportion to how the (individual) {\bf differences} between (social) {\bf equals} are applicable in a given context. To quote Michael Bakunin, {\em “[t]he greatest intelligence would not be equal to a comprehension of the whole. Thence results\unknown{} the necessity of the division and association of labour. I receive and I give — such is human life. Each directs and is directed in his turn. Therefore there is no fixed and constant authority, but a continual exchange of mutual, temporary, and, above all, voluntary authority and subordination.”} [{\bf God and the State}, p. 33]


Such an environment can only exist within self-managed associations, for capitalism (i.e. wage labour) creates very specific relations and institutions of authority. It is for this reason anarchists are socialists (i.e. opposed to wage labour, the existence of a proletariat or working class). In other words, anarchists support equality precisely {\bf because} we recognise that everyone is unique. If we are serious about “equality of rights” or “equal freedom” then conditions must be such that people can enjoy these rights and liberties. If we assume the right to develop one’s capacities to the fullest, for example, then inequality of resources and so power within society destroys that right simply because people do not have the means to freely exercise their capacities (they are subject to the authority of the boss, for example, during work hours).


So, in direct contrast to anarchism, right-Libertarianism is unconcerned about any form of equality except “equality of rights”. This blinds them to the realities of life; in particular, the impact of economic and social power on individuals within society and the social relationships of domination they create. Individuals may be “equal” before the law and in rights, but they may not be free due to the influence of social inequality, the relationships it creates and how it affects the law and the ability of the oppressed to use it. Because of this, all anarchists insist that equality is essential for freedom, including those in the Individualist Anarchist tradition the “anarcho”-capitalist tries to co-opt — {\em “Spooner and Godwin insist that inequality corrupts freedom. Their anarchism is directed as much against inequality as against tyranny”} and {\em “[w]hile sympathetic to Spooner’s individualist anarchism, they [Rothbard and David Friedman] fail to notice or conveniently overlook its egalitarian implications.”} [Stephen L. Newman, {\bf Liberalism at Wit’s End}, p. 74, p. 76]


Why equality is important is discussed more fully in the next section. Here we just stress that without social equality, individual freedom is so restricted that it becomes a mockery (essentially limiting freedom of the majority to choosing {\bf which} employer will govern them rather than being free within and outside work).


Of course, by defining “equality” in such a restrictive manner, Rothbard’s own ideology is proved to be nonsense. As L.A. Rollins notes, {\em “Libertarianism, the advocacy of ‘free society’ in which people enjoy ‘equal freedom’ and ‘equal rights,’ is actually a specific form of egalitarianism. As such, Libertarianism itself is a revolt against nature. If people, by their very biological nature, are unequal in all the attributes necessary to achieving, and preserving ‘freedom’ and ‘rights’\unknown{} then there is no way that people can enjoy ‘equal freedom’ or ‘equal rights’. If a free society is conceived as a society of ‘equal freedom,’ then there ain’t no such thing as ‘a free society’.”} [{\bf The Myth of Natural Law}, p. 36]


Under capitalism, freedom is a commodity like everything else. The more money you have, the greater your freedom. “Equal” freedom, in the Newspeak-Rothbardian sense, {\bf cannot} exist! As for “equality before the law”, its clear that such a hope is always dashed against the rocks of wealth and market power (see next section for more on this). As far as rights go, of course, both the rich and the poor have an “equal right” to sleep under a bridge (assuming the bridge’s owner agrees of course!); but the owner of the bridge and the homeless have {\bf different} rights, and so they cannot be said to have “equal rights” in the Newspeak-Rothbardian sense either. Needless to say, poor and rich will not “equally” use the “right” to sleep under a bridge, either.


Bob Black observes in {\bf The Libertarian as Conservative} that {\em “[t]he time of your life is the one commodity you can sell but never buy back. Murray Rothbard thinks egalitarianism is a revolt against nature, but his day is 24 hours long, just like everybody else’s.”}


By twisting the language of political debate, the vast differences in power in capitalist society can be “blamed” not on an unjust and authoritarian system but on “biology” (we are all unique individuals, after all). Unlike genes (although biotechnology corporations are working on this, too!), human society {\bf can} be changed, by the individuals who comprise it, to reflect the basic features we all share in common — our humanity, our ability to think and feel, and our need for freedom.


\subsection{3.1 Why is this disregard for equality important?
}

Simply because a disregard for equality soon ends with liberty for the majority being negated in many important ways. Most “anarcho”-capitalists and right-Libertarians deny (or at best ignore) market power. Rothbard, for example, claims that economic power does not exist; what people call {\em “economic power”} is {\em “simply the right under freedom to refuse to make an exchange”} [{\bf The Ethics of Liberty}, p. 222] and so the concept is meaningless.


However, the fact is that there are substantial power centres in society (and so are the source of hierarchical power and authoritarian social relations) which are {\bf not the state.} The central fallacy of “anarcho”-capitalism is the (unstated) assumption that the various actors within an economy have relatively equal power. This assumption has been noted by many readers of their works. For example, Peter Marshall notes that {\em “‘anarcho-capitalists’ like Murray Rothbard assume individuals would have equal bargaining power in a [capitalist] market-based society”} [{\bf Demanding the Impossible}, p. 46] George Walford also makes this clear in his comments on David Friedman’s {\bf The Machinery of Freedom}:



\startblockquote
{\em “The private ownership envisages by the anarcho-capitalists would be very different from that which we know. It is hardly going too far to say that while the one is nasty, the other would be nice. In anarcho-capitalism there would be no National Insurance, no Social Security, no National Health Service and not even anything corresponding to the Poor Laws; there would be no public safety-nets at all. It would be a rigorously competitive society: work, beg or die. But as one reads on, learning that each individual would have to buy, personally, all goods and services needed, not only food, clothing and shelter but also education, medicine, sanitation, justice, police, all forms of security and insurance, even permission to use the streets (for these also would be privately owned), as one reads about all this a curious feature emerges: everybody always has enough money to buy all these things.}


{\em “There are no public casual wards or hospitals or hospices, but neither is there anybody dying in the streets. There is no public educational system but no uneducated children, no public police service but nobody unable to buy the services of an efficient security firm, no public law but nobody unable to buy the use of a private legal system. Neither is there anybody able to buy much more than anybody else; no person or group possesses economic power over others.}


{\em “No explanation is offered. The anarcho-capitalists simply take it for granted that in their favoured society, although it possesses no machinery for restraining competition (for this would need to exercise authority over the competitors and it is an {\bf anarcho}- capitalist society) competition would not be carried to the point where anybody actually suffered from it. While proclaiming their system to be a competitive one, in which private interest rules unchecked, they show it operating as a co-operative one, in which no person or group profits at the cost of another.”} [{\bf On the Capitalist Anarchists}]



\stopblockquote
This assumption of (relative) equality comes to the fore in Murray Rothbard’s “Homesteading” concept of property (discussed in section 4.1). “Homesteading” paints a picture of individuals and families doing into the wilderness to make a home for themselves, fighting against the elements and so forth. It does {\bf not} invoke the idea of transnational corporations employing tens of thousands of people or a population without land, resources and selling their labour to others. Indeed, Rothbard argues that economic power does not exist (at least under capitalism; as we saw in section 2.1 he does make — highly illogical — exceptions). Similarly, David Friedman’s example of a pro-death penalty and anti-death penalty “defence” firm coming to an agreement (see section 6.3) assumes that the firms have equal bargaining powers and resources — if not, then the bargaining process would be very one-sided and the smaller company would think twice before taking on the larger one in battle (the likely outcome if they cannot come to an agreement on this issue) and so compromise.


However, the right-libertarian denial of market power is unsurprising. The necessity, not the redundancy, of equality is required if the inherent problems of contract are not to become too obvious. If some individuals {\bf are} assumed to have significantly more power than others, and if they are always self-interested, then a contract that creates equal partners is impossible — the pact will establish an association of masters and servants. Needless to say, the strong will present the contract as being to the advantage of both: the strong no longer have to labour (and become rich, i.e. even stronger) and the weak receive an income and so do not starve.


If freedom is considered as a function of ownership then it is very clear that individuals lacking property (outside their own body, of course) loses effective control over their own person and labour (which was, lets not forget, the basis of their equal natural rights). When ones bargaining power is weak (which is typically the case in the labour market) exchanges tend to magnify inequalities of wealth and power over time rather than working towards an equalisation.


In other words, “contract” need not replace power if the bargaining position and wealth of the would-be contractors are not equal (for, if the bargainers had equal power it is doubtful they would agree to sell control of their liberty/time to another). This means that “power” and “market” are not antithetical terms. While, in an abstract sense, all market relations are voluntary in practice this is not the case within a capitalist market. For example, a large company has a comparative advantage over small ones and communities which will definitely shape the outcome of any contract. For example, a large company or rich person will have access to more funds and so stretch out litigations and strikes until their opponents resources are exhausted. Or, if a local company is polluting the environment, the local community may put up with the damage caused out of fear that the industry (which it depends upon) would relocate to another area. If members of the community {\bf did} sue, then the company would be merely exercising its property rights when it threatened to move to another location. In such circumstances, the community would “freely” consent to its conditions or face massive economic and social disruption. And, similarly, {\em “the landlords’ agents who threaten to discharge agricultural workers and tenants who failed to vote the reactionary ticket”} in the 1936 Spanish election were just exercising their legitimate property rights when they threatened working people and their families with economic uncertainty and distress. [Murray Bookchin, {\bf The Spanish Anarchists}, p. 260]


If we take the labour market, it is clear that the “buyers” and “sellers” of labour power are rarely on an equal footing (if they were, then capitalism would soon go into crisis — see section 10.2). In fact, competition {\em “in labour markets is typically skewed in favour of employers: it is a buyer’s market. And in a buyer’s, it is the sellers who compromise.”} [Juliet B. Schor, {\bf The Overworked American}, p. 129] Thus the ability to refuse an exchange weights most heavily on one class than another and so ensures that “free exchange” works to ensure the domination (and so exploitation) of one party by the other.


Inequality in the market ensures that the decisions of the majority of within it are shaped in accordance with that needs of the powerful, not the needs of all. It was for this reason that the Individual Anarchist J.K. Ingalls opposed Henry George’s proposal of nationalising the land. Ingalls was well aware that the rich could outbid the poor for leases on land and so the dispossession of the working classes would continue.


The market, therefore, does not end power or unfreedom — they are still there, but in different forms. And for an exchange to be truly voluntary, both parties must have equal power to accept, reject, or influence its terms. Unfortunately, these conditions are rarely meet on the labour market or within the capitalist market in general. Thus Rothbard’s argument that economic power does not exist fails to acknowledge that the rich can out-bid the poor for resources and that a corporation generally has greater ability to refuse a contract (with an individual, union or community) than vice versa (and that the impact of such a refusal is such that it will encourage the others involved to “compromise” far sooner). And in such circumstances, formally free individuals will have to “consent” to be unfree in order to survive.


As Max Stirner pointed out in the 1840s, free competition {\em “is not ‘free,’ because I lack the {\bf things} for competition.”} [{\bf The Ego and Its Own}, p. 262] Due to this basic inequality of wealth (of “things”) we find that {\em “[u]nder the {\bf regime} of the commonality the labourers always fall into the hands of the possessors \unknown{} of the capitalists, therefore. The labourer cannot {\bf realise} on his labour to the extent of the value that it has for the customer.”} [{\bf Op. Cit.}, p. 115] Its interesting to note that even Stirner recognises that capitalism results in exploitation. And we may add that value the labourer does not {\em “realise”} goes into the hands of the capitalists, who invest it in more “things” and which consolidates and increases their advantage in “free” competition.


To quote Stephan L. Newman:



\startblockquote
{\em “Another disquieting aspect of the libertarians’ refusal to acknowledge power in the market is their failure to confront the tension between freedom and autonomy\unknown{} Wage labour under capitalism is, of course, formally free labour. No one is forced to work at gun point. Economic circumstance, however, often has the effect of force; it compels the relatively poor to accept work under conditions dictated by owners and managers. The individual worker retains freedom [i.e. negative liberty] but loses autonomy [positive liberty].”} [{\bf Liberalism at Wit’s End}, pp. 122–123]



\stopblockquote
(As an aside, we should point out that the full Stirner quote cited above is {\em “[u]nder the {\bf regime} of the commonality the labourers always fall into the hands of the possessors, of those who have at their disposal some bit of the state domains (and everything possessible in State domain belongs to the State and is only a fief of the individual), especially money and land; of the capitalists, therefore. The labourer cannot {\bf realise} on his labour to the extent of the value that it has for the customer.”}


It could be argued that we misrepresenting Stirner by truncating the quote, but we feel that such a claim this is incorrect. Its clear from his book that Stirner is considering the “minimal” state ({\em “The State is a — commoners’ State \unknown{} It protects man \unknown{}according to whether the rights entrusted to him by the State are enjoyed and managed in accordance with the will, that is, laws, of the State.”} The State {\em “looks on indifferently as one grows poor and the other rich, unruffled by this alternation. As {\bf individuals} they are really equal before its face.”} [{\bf Op. Cit.}, p. 115, p. 252]). As “anarcho”-capitalists consider their system to be one of rights and laws (particularly property rights), we feel that its fair to generalise Stirner’s comments into capitalism {\bf as such} as opposed to “minimum state” capitalism. If we replace “State” by “libertarian law code” you will see what we mean. We have included this aside before any right-libertarians claim that we are misrepresenting Stirner’ argument.)


If we consider “equality before the law” it is obvious that this also has limitations in an (materially) unequal society. Brian Morris notes that for Ayn Rand, {\em “[u]nder capitalism \unknown{} politics (state) and economics (capitalism) are separated \unknown{} This, of course, is pure ideology, for Rand’s justification of the state is that it ‘protects’ private property, that is, it supports and upholds the economic power of capitalists by coercive means.”} [{\bf Ecology \& Anarchism}, p. 189] The same can be said of “anarcho”-capitalism and its “protection agencies” and “general libertarian law code.” If within a society a few own all the resources and the majority are dispossessed, then any law code which protects private property {\bf automatically} empowers the owning class. Workers will {\bf always} be initiating force if act against the code and so “equality before the law” reinforces inequality of power and wealth.


This means that a system of property rights protects the liberties of some people in a way which gives them an unacceptable degree of power over others. And this cannot be met merely by reaffirming the rights in question, we have to assess the relative importance of various kinds of liberty and other values we how dear.


Therefore right-libertarian disregard for equality is important because it allows “anarcho”-capitalism to ignore many important restrictions of freedom in society. In addition, it allows them to brush over the negative effects of their system by painting an unreal picture of a capitalist society without vast extremes of wealth and power (indeed, they often construe capitalist society in terms of an ideal — namely artisan production — that is really {\bf pre}-capitalist and whose social basis has been eroded by capitalist development). Inequality shapes the decisions we have available and what ones we make:



\startblockquote
{\em “An ‘incentive’ is always available in conditions of substantial social inequality that ensure that the ‘weak’ enter into a contract. When social inequality prevails, questions arises about what counts as voluntary entry into a contract \unknown{} Men and women \unknown{} are now juridically free and equal citizens, but, in unequal social conditions, the possibility cannot be ruled out that some or many contracts create relationships that bear uncomfortable resemblances to a slave contract.”} [Carole Pateman, {\bf The Sexual Contract}, p. 62]



\stopblockquote
This ideological confusion of right-libertarianism can also be seen from their opposition to taxation. On the one hand, they argue that taxation is wrong because it takes money from those who “earn” it and gives it to the poor. On the other hand, “free market” capitalism is assumed to be a more equal society! If taxation takes from the rich and gives to the poor, how will “anarcho”-capitalism be more egalitarian? That equalisation mechanism would be gone (of course, it could be claimed that all great riches are purely the result of state intervention skewing the “free market” but that places all their “rags to riches” stories in a strange position). Thus we have a problem, either we have relative equality or we do not. Either we have riches, and so market power, or we do not. And its clear from the likes of Rothbard, “anarcho”-capitalism will not be without its millionaires (there is, after all, apparently nothing un-libertarian about {\em “organisation, hierarchy, wage-work, granting of funds by libertarian millionaires, and a libertarian party”}). And so we are left with market power and so extensive unfreedom.


Thus, for a ideology that denounces egalitarianism as a {\em “revolt against nature”} it is pretty funny that they paint a picture of “anarcho”-capitalism as a society of (relative) equals. In other words, their propaganda is based on something that has never existed, and never will, namely an egalitarian capitalist society.


\subsection{3.2 But what about “anarcho”-capitalist support for charity?
}

Yes, while being blind to impact of inequality in terms of economic and social power and influence, most right-libertarians {\bf do} argue that the very poor could depend on charity in their system. But such a recognition of poverty does not reflect an awareness of the need for equality or the impact of inequality on the agreements we make. Quite the reverse in fact, as the existence of extensive inequality is assumed — after all, in a society of relative equals, poverty would not exist, nor would charity be needed.


Ignoring the fact that their ideology hardly promotes a charitable perspective, we will raise four points. Firstly, charity will not be enough to countermand the existence and impact of vast inequalities of wealth (and so power). Secondly, it will be likely that charities will be concerned with “improving” the moral quality of the poor and so will divide them into the “deserving” (i.e. obedient) and “undeserving” (i.e. rebellious) poor. Charity will be forthcoming to the former, those who agree to busy-bodies sticking their noses into their lives. In this way charity could become another tool of economic and social power (see Oscar Wilde’s {\bf The Soul of Man Under Socialism} for more on charity). Thirdly, it is unlikely that charity will be able to replace all the social spending conducted by the state — to do so would require a ten-fold increase in charitable donations (and given that most right-libertarians denounce the government for making them pay taxes to help the poor, it seems unlikely that they will turn round and {\bf increase} the amount they give). And, lastly, charity is an implicate recognition that, under capitalism, no one has the right of life — its a privilege you have to pay for. That in itself is enough to reject the charity option. And, of course, in a system designed to secure the life and liberty of each person, how can it be deemed acceptable to leave the life and protection of even one individual to the charitable whims of others? (Perhaps it will be argued that individual’s have the right to life, but not a right to be a parasite. This ignores the fact some people {\bf cannot} work — babies and some handicapped people — and that, in a functioning capitalist economy, many people cannot find work all the time. Is it this recognition of that babies cannot work that prompts many right-libertarians to turn them into property? Of course, rich folk who have never done a days work in their lives are never classed as parasites, even if they inherited all their money). All things considered, little wonder that Proudhon argued that:



\startblockquote
{\em “Even charitable institutions serve the ends of those in authority marvellously well.}


{\em “Charity is the strongest chain by which privilege and the Government, bound to protect them, holds down the lower classes. With charity, sweeter to the heart of men, more intelligible to the poor man than the abstruse laws of Political Economy, one may dispense with justice.”} [{\bf The General Idea of the Revolution}, pp. 69–70]



\stopblockquote
As noted, the right-libertarian (passing) acknowledgement of poverty does not mean that they recognise the existence of market power. They never ask themselves how can someone be free if their social situation is such that they are drowning in a see of usury and have to sell their labour (and so liberty) to survive.


\section{4 What is the right-libertarian position on private property?
}

Right libertarians are not interested in eliminating capitalist private property and thus the authority, oppression and exploitation which goes with it. It is true that they call for an end to the state, but this is not because they are concerned about workers being exploited or oppressed but because they don’t want the state to impede capitalists’ “freedom” to exploit and oppress workers even more than is the case now!


They make an idol of private property and claim to defend absolute, “unrestricted” property rights (i.e. that property owners can do anything they like with their property, as long as it does not damage the property of others. In particular, taxation and theft are among the greatest evils possible as they involve coercion against “justly held” property). They agree with John Adams that {\em “[t]he moment that idea is admitted into society that property is not as sacred as the Laws of God, and that there is not a force of law and public justice to protect it, anarchy and tyranny commence. Property must be sacred or liberty cannot exist.”}


But in their celebration of property as the source of liberty they ignore the fact that private property is a source of “tyranny” in itself (see sections B.1 and B.4, for example — and please note that anarchists only object to private property, {\bf not} individual possession, see section B.3.1). However, as much anarchists may disagree about other matters, they are united in condemning private property. Thus Proudhon argued that property was {\em “theft”} and {\em “despotism”} while Stirner indicated the religious and statist nature of private property and its impact on individual liberty when he wrote :



\startblockquote
{\em “Property in the civic sense means {\bf sacred} property, such that I must {\bf respect} your property\unknown{} Be it ever so little, if one only has somewhat of his own — to wit, a {\bf respected} property: The more such owners\unknown{} the more ‘free people and good patriots’ has the State.}


{\em “Political liberalism, like everything religious, counts on {\bf respect,} humaneness, the virtues of love\unknown{} For in practice people respect nothing, and everyday the small possessions are bought up again by greater proprietors, and the ‘free people’ change into day labourers.”} [{\bf The Ego and Its Own}, p. 248]



\stopblockquote
Thus “anarcho”-capitalists reject totally one of the common (and so defining) features of all anarchist traditions — the opposition to capitalist property. From Individualist Anarchists like Tucker to Communist-Anarchists like Bookchin, anarchists have been opposed to what Godwin termed {\em “accumulated property.”} This was because it was in {\em “direct contradiction”} to property in the form of {\em “the produce of his [the worker’s] own industry”} and so it allows {\em “one man\unknown{} [to] dispos[e] of the produce of another man’s industry.”} [{\bf The Anarchist Reader}, pp. 129–131] Thus, for anarchists, capitalist property is a source exploitation and domination, {\bf not} freedom (it undermines the freedom associated with possession by created relations of domination between owner and employee).


Hardly surprising then the fact that, according to Murray Bookchin, Murray Rothbard {\em “attacked me [Bookchin] as an anarchist with vigour because, as he put it, I am opposed to private property.”} [{\bf The Raven}, no. 29, p. 343]


We will discuss Rothbard’s “homesteading” justification of property in the next section. However, we will note here one aspect of right-libertarian defence of “unrestricted” property rights, namely that it easily generates evil side effects such as hierarchy and starvation. As famine expert Amartya Sen notes:



\startblockquote
{\em “Take a theory of entitlements based on a set of rights of ‘ownership, transfer and rectification.’ In this system a set of holdings of different people are judged to be just (or unjust) by looking at past history, and not by checking the consequences of that set of holdings. But what if the consequences are recognisably terrible? \unknown{}[R]efer[ing] to some empirical findings in a work on famines \unknown{} evidence [is presented] to indicate that in many large famines in the recent past, in which millions of people have died, there was no over-all decline in food availability at all, and the famines occurred precisely because of shifts in entitlement resulting from exercises of rights that are perfectly legitimate\unknown{} [Can] famines \unknown{} occur with a system of rights of the kind morally defended in various ethical theories, including Nozick’s. I believe the answer is straightforwardly yes, since for many people the only resource that they legitimately possess, viz. their labour-power, may well turn out to be unsaleable in the market, giving the person no command over food \unknown{} [i]f results such as starvations and famines were to occur, would the distribution of holdings still be morally acceptable despite their disastrous consequences? There is something deeply implausible in the affirmative answer.”} [{\bf Resources, Values and Development}, pp. 311–2]



\stopblockquote
Thus “unrestricted” property rights can have seriously bad consequences and so the existence of “justly held” property need not imply a just or free society — far from it. The inequalities property can generate can have a serious on individual freedom (see section 3.1). Indeed, Murray Rothbard argued that the state was evil not because it restricted individual freedom but because the resources it claimed to own were not “justly” acquired. Thus right-libertarian theory judges property {\bf not} on its impact on current freedom but by looking at past history. This has the interesting side effect of allowing its supporters to look at capitalist and statist hierarchies, acknowledge their similar negative effects on the liberty of those subjected to them but argue that one is legitimate and the other is not simply because of their history! As if this changed the domination and unfreedom that both inflict on people living today (see section 2.3 for further discussion and sections 2.8 and 4.2 for other examples of “justly acquired” property producing terrible consequences).


The defence of capitalist property does have one interesting side effect, namely the need arises to defend inequality and the authoritarian relationships inequality creates. In order to protect the private property needed by capitalists in order to continue exploiting the working class, “anarcho”-capitalists propose private security forces rather than state security forces (police and military) — a proposal that is equivalent to bringing back the state under another name.


Due to (capitalist) private property, wage labour would still exist under “anarcho”-capitalism (it is capitalism after all). This means that “defensive” force, a state, is required to “defend” exploitation, oppression, hierarchy and authority from those who suffer them. Inequality makes a mockery of free agreement and “consent” (see section 3.1). As Peter Kropotkin pointed out long ago:



\startblockquote
{\em “When a workman sells his labour to an employer \unknown{} it is a mockery to call that a free contract. Modern economists may call it free, but the father of political economy — Adam Smith — was never guilty of such a misrepresentation. As long as three-quarters of humanity are compelled to enter into agreements of that description, force is, of course, necessary, both to enforce the supposed agreements and to maintain such a state of things. Force — and a good deal of force — is necessary to prevent the labourers from taking possession of what they consider unjustly appropriated by the few\unknown{} The Spencerian party [proto-right-libertarians] perfectly well understand that; and while they advocate no force for changing the existing conditions, they advocate still more force than is now used for maintaining them. As to Anarchy, it is obviously as incompatible with plutocracy as with any other kind of -cracy.”} [{\bf Anarchism and Anarchist Communism}, pp. 52–53]



\stopblockquote
Because of this need to defend privilege and power, “anarcho”-capitalism is best called “private-state” capitalism. This will be discussed in more detail in section 6.


By advocating private property, right libertarians contradict many of their other claims. For example, they say that they support the right of individuals to travel where they like. They make this claim because they assume that only the state limits free travel. But this is a false assumption. Owners must agree to let you on their land or property ({\em “people only have the right to move to those properties and lands where the owners desire to rent or sell to them.”} [Murray Rothbard, {\bf The Ethics of Liberty}, p. 119]. There is no “freedom of travel” onto private property (including private roads). Therefore immigration may be just as hard under “anarcho”-capitalism as it is under statism (after all, the state, like the property owner, only lets people in whom it wants to let in). People will still have to get another property owner to agree to let them in before they can travel — exactly as now (and, of course, they also have to get the owners of the road to let them in as well). Private property, as can be seen from this simple example, is the state writ small.


One last point, this ignoring of (“politically incorrect”) economic and other views of dead political thinkers and activists while claiming them as “libertarians” seems to be commonplace in right-Libertarian circles. For example, Aristotle (beloved by Ayn Rand) {\em “thought that only living things could bear fruit. Money, not a living thing, was by its nature barren, and any attempt to make it bear fruit ({\bf tokos}, in Greek, the same word used for interest) was a crime against nature.”} [Marcello de Cecco, quoted by Doug Henwood, {\bf Wall Street}, p. 41] Such opposition to interest hardly fits well into capitalism, and so either goes unmentioned or gets classed as an “error” (although we could ask why Aristotle is in error while Rand is not). Similarly, individualist anarchist opposition to capitalist property and rent, interest and profits is ignored or dismissed as “bad economics” without realising that these ideas played a key role in their politics and in ensuring that an anarchy would not see freedom corrupted by inequality. To ignore such an important concept in a person’s ideas is to distort the remainder into something it is not.


\subsection{4.1 What is wrong with a “homesteading” theory of property?
}

So how do “anarcho”-capitalists justify property? Looking at Murray Rothbard, we find that he proposes a {\em “homesteading theory of property”}. In this theory it is argued that property comes from occupancy and mixing labour with natural resources (which are assumed to be unowned). Thus the world is transformed into private property, for {\em “title to an unowned resource (such as land) comes properly only from the expenditure of labour to transform that resource into use.”} [{\bf The Ethics of Liberty}, p. 63]


Rothbard paints a conceptual history of individuals and families forging a home in the wilderness by the sweat of their labour (its tempting to rename his theory the {\em “immaculate conception of property”} as his conceptual theory is somewhat at odds with actual historical fact).


Sadly for Murray Rothbard, his “homesteading” theory was refuted by Proudhon in {\bf What is Property?} in 1840 (along with many other justifications of property). Proudhon rightly argues that {\em “if the liberty of man is sacred, it is equally sacred in all individuals; that, if it needs property for its objective action, that is, for its life, the appropriation of material is equally necessary for all \unknown{} Does it not follow that if one individual cannot prevent another \unknown{} from appropriating an amount of material equal to his own, no more can he prevent individuals to come.”} And if all the available resources are appropriated, and the owner {\em “draws boundaries, fences himself in \unknown{} Here, then, is a piece of land upon which, henceforth, no one has a right to step, save the proprietor and his friends \unknown{} Let [this]\unknown{} multiply, and soon the people \unknown{} will have nowhere to rest, no place to shelter, no ground to till. They will die at the proprietor’s door, on the edge of that property which was their birthright.”} [{\bf What is Property?}, pp. 84–85, p. 118]


As Rothbard himself noted in respect to the aftermath of slavery (see section 2.1), not having access to the means of life places one the position of unjust dependency on those who do. Rothbard’s theory fails because for {\em “[w]e who belong to the proletaire class, property excommunicates us!”} [P-J Proudhon, {\bf Op. Cit.}, p. 105] and so the vast majority of the population experience property as theft and despotism rather than as a source of liberty and empowerment (which possession gives). Thus, Rothbard’s account fails to take into account the Lockean Proviso (see section B.3.4) and so, for all its intuitive appeal, ends up justifying capitalist and landlord domination (see next section on why the Lockean Proviso is important).


It also seems strange that while (correctly) attacking social contract theories of the state as invalid (because {\em “no past generation can bind later generations”} [{\bf Op. Cit.}, p. 145]) he fails to see he is doing {\bf exactly that} with his support of private property (similarly, Ayn Rand argued that {\em “[a]ny alleged ‘right’ of one man, which necessitates the violation of the right of another, is not and cannot be a right”} [{\bf Capitalism: The Unknown Ideal}, p. 325] but obviously appropriating land does violate the rights of others to walk, use or appropriate that land). Due to his support for appropriation and inheritance, he is clearly ensuring that future generations are {\bf not} born as free as the first settlers were (after all, they cannot appropriate any land, it is all taken!). If future generations cannot be bound by past ones, this applies equally to resources and property rights. Something anarchists have long realised — there is no defensible reason why those who first acquired property should control its use by future generations.


However, if we take Rothbard’s theory at face value we find numerous problems with it. If title to unowned resources comes via the {\em “expenditure of labour”} on it, how can rivers, lakes and the oceans be appropriated? The banks of the rivers can be transformed, but can the river itself? How can you mix your labour with water? “Anarcho”-capitalists usually blame pollution on the fact that rivers, oceans, and so forth are unowned, but how can an individual “transform” water by their labour? Also, does fencing in land mean you have “mixed labour” with it? If so then transnational corporations can pay workers to fence in vast tracks of virgin land (such as rainforest) and so come to “own” it. Rothbard argues that this is not the case (he expresses opposition to {\em “arbitrary claims”}). He notes that it is {\bf not} the case that {\em “the first discoverer \unknown{} could properly lay claim to [a piece of land] \unknown{} [by] laying out a boundary for the area.”} He thinks that {\em “their claim would still be no more than the boundary {\bf itself}, and not to any of the land within, for only the boundary will have been transformed and used by men”} [{\bf Op. Cit.}, p. 50f]


However, if the boundary {\bf is} private property and the owner refuses others permission to cross it, then the enclosed land is inaccessible to others! If an “enterprising” right-libertarian builds a fence around the only oasis in a desert and refuses permission to cross it to travellers unless they pay his price (which is everything they own) then the person {\bf has} appropriated the oasis without “transforming” it by his labour. The travellers have the choice of paying the price or dying (and the oasis owner is well within his rights letting them die). Given Rothbard’s comments, it is probable that he will claim that such a boundary is null and void as it allows “arbitrary” claims — although this position is not at all clear. After all, the fence builder {\bf has} transformed the boundary and “unrestricted” property rights is what right-libertarianism is all about.


And, of course, Rothbard ignores the fact of economic power — a transnational corporation can “transform” far more virgin resources in a day than a family could in a year. Transnational’s “mixing their labour” with the land does not spring into mind reading Rothbard’s account of property growth, but in the real world that is what will happen.


If we take the question of wilderness (a topic close to many eco-anarchists’ and deep ecologists’ hearts) we run into similar problems. Rothbard states clearly that {\em “libertarian theory must invalidate [any] claim to ownership”} of land that has {\em “never been transformed from its natural state”} (he presents an example of an owner who has left a piece of his {\em “legally owned”} land untouched). If another person appears who {\bf does} transform the land, it becomes {\em “justly owned by another”} and the original owner cannot stop her (and should the original owner {\em “use violence to prevent another settler from entering this never-used land and transforming it into use”} they also become a {\em “criminal aggressor”}). Rothbard also stresses that he is {\bf not} saying that land must continually be in use to be valid property [{\bf Op. Cit.}, pp. 63–64] (after all, that would justify landless workers seizing the land from landowners during a depression and working it themselves).


Now, where does that leave wilderness? In response to ecologists who oppose the destruction of the rainforest, “anarcho”-capitalists suggest that they put their money where their mouth is and {\bf buy} rainforest land. In this way, it is claimed, rainforest will be protected (see section B.5 for why such arguments are nonsense). As ecologists desire the rainforest {\bf because it is wilderness} they are unlikely to “transform” it by human labour (its precisely that they want to stop). From Rothbard’s arguments it is fair to ask whether logging companies have a right to “transform” the virgin wilderness owned by ecologists, after all it meets Rothbard’s criteria (it is still wilderness). Perhaps it will be claimed that fencing off land “transforms” it (hardly what you imagine “mixing labour” with to mean, but nevermind) — but that allows large companies and rich individuals to hire workers to fence in vast tracks of land (and recreate the land monopoly by a “libertarian” route). But as we noted above, fencing off land does not seem to imply that it becomes property in Rothbard’s theory. And, of course, fencing in areas of rainforest disrupts the local eco-system — animals cannot freely travel, for example — which, again, is what ecologists desire to stop. Would Rothbard accept a piece of paper as “transforming” land? We doubt it (after all, in his example the wilderness owner {\bf did} legally own it) — and so most ecologists will have a hard time in “anarcho”-capitalism (wilderness is just not an option).


As an aside, we must note that Rothbard fails to realise — and this comes from his worship of the market and his “Austrian economics” — is that people value many things which do not appear on the market. He claims that wilderness is {\em “valueless unused natural objects”} (for it people valued them, they would use — i.e. appropriate — them). But unused things may be of {\bf considerable} value to people, wilderness being a classic example. And if something {\bf cannot} be transformed into private property, does that mean people do not value it? For example, people value community, stress free working environments, meaningful work — if the market cannot provide these, does that mean they do not value them? Of course not (see Juliet Schor’s {\bf The Overworked American} on how working people’s desire for shorter working hours was not transformed into options on the market).


Moreover, Rothbard’s “homesteading” theory actually violates his support for unrestricted property rights. What if a property owner {\bf wants} part of her land to remain wilderness? Their desires are violated by the “homesteading” theory (unless, of course, fencing things off equals “transforming” them, which it apparently does not). How can companies provide wilderness holidays to people if they have no right to stop settlers (including large companies) “homesteading” that wilderness? And, of course, where does Rothbard’s theory leave hunter-gather or nomad societies. They {\bf use} the resources of the wilderness, but they do not “transform” them (in this case you cannot easily tell if virgin land is empty or being used as a resource). If a troop of nomads find its traditionally used, but natural, oasis appropriated by a homesteader what are they to do? If they ignore the homesteaders claims he can call upon his “defence” firm to stop them — and then, in true Rothbardian fashion, the homesteader can refuse to supply water to them unless they hand over all their possessions (see section 4.2 on this). And if the history of the United States (which is obviously the model for Rothbard’s theory) is anything to go by, such people will become “criminal aggressors” and removed from the picture.


Which is another problem with Rothbard’s account. It is completely ahistoric (and so, as we noted above, is more like an {\em “immaculate conception of property”}). He has transported “capitalist man” into the dawn of time and constructed a history of property based upon what he is trying to justify (not surprising, as he does this with his “Natural Law” theory too — see section 7). What {\bf is} interesting to note, though, is that the {\bf actual} experience of life on the US frontier (the historic example Rothbard seems to want to claim) was far from the individualistic framework he builds upon it and (ironically enough) it was destroyed by the development of capitalism.


As Murray Bookchin notes, {\em “the independence that the New England yeomanry enjoyed was itself a function of the co-operative social base from which it emerged. To barter home-grown goods and objects, to share tools and implements, to engage in common labour during harvesting time in a system of mutual aid, indeed, to help new-comers in barn-raising, corn-husking, log-rolling, and the like, was the indispensable cement that bound scattered farmsteads into a united community.”} [{\bf The Third Revolution}, vol. 1, p. 233] Bookchin quotes David P. Szatmary (author of a book on Shay’ Rebellion) stating that it was a society based upon {\em “co-operative, community orientated interchanges”} and not a {\em “basically competitive society.”} [{\bf Ibid.}]


Into this non-capitalist society came capitalist elements. Market forces and economic power soon resulted in the transformation of this society. Merchants asked for payment in specie which (and along with taxes) soon resulted in indebtedness and the dispossession of the homesteaders from their land and goods. In response Shay’s rebellion started, a rebellion which was an important factor in the centralisation of state power in America to ensure that popular input and control over government were marginalised and that the wealthy elite and their property rights were protected against the many (see Bookchin, {\bf Op. Cit.}, for details). Thus the homestead system was undermined, essentially, by the need to pay for services in specie (as demanded by merchants).


So while Rothbard’s theory as a certain appeal (reinforced by watching too many Westerns, we imagine) it fails to justify the “unrestricted” property rights theory (and the theory of freedom Rothbard derives from it). All it does is to end up justifying capitalist and landlord domination (which is probably what it was intended to do).


\subsection{4.2 Why is the “Lockean Proviso” important?
}

Robert Nozick, in his work {\bf Anarchy, State, and Utopia} presented a case for private property rights that was based on what he termed the {\em “Lockean Proviso”} — namely that common (or unowned) land and resources could be appropriated by individuals as long as the position of others is not worsen by so doing. However, if we {\bf do} take this Proviso seriously private property rights cannot be defined (see section B.3.4 for details). Thus Nozick’s arguments in favour of property rights fail.


Some right-libertarians, particularly those associated with the Austrian school of economics argue that we must reject the Lockean Proviso (probably due to the fact it can be used to undermine the case for absolute property rights). Their argument goes as follows: if an individual appropriates and uses a previously unused resource, it is because it has value to him/her, as an individual, to engage in such action. The individual has stolen nothing because it was previously unowned and we cannot know if other people are better or worse off, all we know is that, for whatever reason, they did not appropriate the resource ({\em “If latecomers are worse off, well then that is their proper assumption of risk in this free and uncertain world. There is no longer a vast frontier in the United States, and there is no point crying over the fact.”} [Murray Rothbard, {\bf The Ethics of Liberty}, p. 240]).


Hence the appropriation of resources is an essentially individualistic, asocial act — the requirements of others are either irrelevant or unknown. However, such an argument fails to take into account {\bf why} the Lockean Proviso has such an appeal. When we do this we see that rejecting it leads to massive injustice, even slavery.


However, let us start with a defence of rejecting the Proviso from a leading Austrian economist:



\startblockquote
{\em “Consider \unknown{} the case \unknown{} of the unheld sole water hole in the desert (which {\bf everyone} in a group of travellers knows about), which one of the travellers, by racing ahead of the others, succeeds in appropriating \unknown{} [This] clearly and unjustly violates the Lockean proviso\unknown{} For use, however, this view is by no means the only one possible. We notice that the energetic traveller who appropriated all the water was not doing anything which (always ignoring, of course, prohibitions resting on the Lockean proviso itself) the other travellers were not equally free to do. The other travellers, too, could have raced ahead \unknown{} [they] did {\bf not} bother to race for the water \unknown{} It does not seem obvious that these other travellers can claim that they were {\bf hurt} by an action which they could themselves have easily taken”} [Israel M. Kirzner, {\em “Entrepreneurship, Entitlement, and Economic Justice”}, pp. 385–413, in {\bf Reading Nozick}, p. 406]



\stopblockquote
Murray Rothbard, we should note, takes a similar position in a similar example, arguing that {\em “the owner [of the sole oasis] is scarcely being ‘coercive’; in fact he is supplying a vital service, and should have the right to refuse a sale or charge whatever the customers will pay. The situation may be unfortunate for the customers, as are many situations in life.”} [{\bf The Ethics of Liberty}, p. 221] (Rothbard, we should note, is relying to the right-libertarian von Hayek who — to his credit — does maintain that this is a coercive situation; but as others, including other right-libertarians, point out, he has to change his definition of coercion/freedom to do so — see Stephan L. Newman’s {\bf Liberalism at Wit’s End}, pp. 130–134 for an excellent summary of this debate).


Now, we could be tempted just to rant about the evils of the right libertarian mind-frame but we will try to present a clam analysis of this position. Now, what Kirzner (and Rothbard et al) fails to note is that without the water the other travellers will die in a matter of days. The monopolist has the power of life and death over his fellow travellers. Perhaps he hates one of them and so raced ahead to ensure their death. Perhaps he just recognised the vast power that his appropriation would give him and so, correctly, sees that the other travellers would give up all their possessions and property to him in return for enough water to survive.


Either way, its clear that perhaps the other travellers did not {\em “race ahead”} because they were ethical people — they would not desire to inflict such tyranny on others because they would not like it inflicted upon them.


Thus we can answer Kirzner’s question — {\em “What \unknown{} is so obviously acceptable about the Lockean proviso\unknown{} ?”} [{\bf Ibid.}]


It is the means by which human actions are held accountable to social standards and ethics. It is the means by which the greediest, most evil and debased humans are stopped from dragging the rest of humanity down to their level (via a “race to the bottom”) and inflicting untold tyranny and domination on their fellow humans. An ideology that could consider the oppression which could result from such an appropriation as “supplying a vital service” and any act to remove this tyranny as “coercion” is obviously a very sick ideology. And we may note that the right-libertarian position on this example is a good illustration of the dangers of deductive logic from assumptions (see section 1.3 for more on this right-libertarian methodology) — after all W. Duncan Reekie, in his introduction to Austrian Economics, states that {\em “[t]o be intellectually consistent one must concede his absolute right to the oasis.”} [{\bf Markets, Entrepreneurs and Liberty}, p. 181] To place ideology before people is to ensure humanity is placed on a Procrustean bed.


Which brings us to another point. Often right-libertarians say that anarchists and other socialists are “lazy” or “do not want to work”. You could interpret Kirzner’s example as saying that the other travellers are “lazy” for not rushing ahead and appropriating the oasis. But this is false. For under capitalism you can only get rich by exploiting the labour of others via wage slavery or, within a company, get better pay by taking “positions of responsibility” (i.e. management positions). If you have an ethical objection to treating others as objects (“means to an end”) then these options are unavailable to you. Thus anarchists and other socialists are not “lazy” because they are not rich — they just have no desire to get rich off the labour and liberty of others (as expressed in their opposition to private property and the relations of domination it creates). In other words, Anarchism is not the “politics of envy”; it is the politics of liberty and the desire to treat others as “ends in themselves”.


Rothbard is aware of what is involved in accepting the Lockean Proviso — namely the existence of private property ({\em “Locke’s proviso may lead to the outlawry of {\bf all} private property of land, since one can always say that the reduction of available land leaves everyone else \unknown{} worse off”}, {\bf The Ethics of Liberty}, p. 240 — see section B.3.4 for a discussion on why the Proviso {\bf does} imply the end of capitalist property rights). Which is why he, and other right-libertarians, reject it. Its simple. Either you reject the Proviso and embrace capitalist property rights (and so allow one class of people to be dispossessed and another empowered at their expense) or you reject private property in favour of possession and liberty. Anarchists, obviously, favour the latter option.


As an aside, we should point out that (following Stirner) the would-be monopolist is doing nothing wrong (as such) in attempting to monopolise the oasis. He is, after all, following his self-interest. However, what is objectionable is the right-libertarian attempt to turn thus act into a “right” which must be respected by the other travellers. Simply put, if the other travellers gang up and dispose of this would be tyrant then they are right to do so — to argue that this is a violation of the monopolists “rights” is insane and an indication of a slave mentality (or, following Rousseau, that the others are {\em “simple”}). Of course, if the would-be monopolist has the necessary {\bf force} to withstand the other travellers then his property then the matter is closed — might makes right. But to worship rights, even when they obviously result in despotism, is definitely a case of {\em “spooks in the head”} and “man is created for the Sabbath” not “the Sabbath is created for man.”


\subsection{4.3 How does private property effect individualism?
}

Private property is usually associated by “anarcho”-capitalism with individualism. Usually private property is seen as the key way of ensuring individualism and individual freedom (and that private property is the expression of individualism). Therefore it is useful to indicate how private property can have a serious impact on individualism.


Usually right-libertarians contrast the joys of “individualism” with the evils of “collectivism” in which the individual is sub-merged into the group or collective and is made to work for the benefit of the group (see any Ayn Rand book or essay on the evils of collectivism).


But what is ironic is that right-libertarian ideology creates a view of industry which would (perhaps) shame even the most die-hard fan of Stalin. What do we mean? Simply that right-libertarians stress the abilities of the people at the top of the company, the owner, the entrepreneur, and tend to ignore the very real subordination of those lower down the hierarchy (see, again, any Ayn Rand book on the worship of business leaders). In the Austrian school of economics, for example, the entrepreneur is considered the driving force of the market process and tend to abstract away from the organisations they govern. This approach is usually followed by right-libertarians. Often you get the impression that the accomplishments of a firm are the personal triumphs of the capitalists, as though their subordinates are merely tools not unlike the machines on which they labour.


We should not, of course, interpret this to mean that right-libertarians believe that entrepreneurs run their companies single-handedly (although you do get that impression sometimes!). But these abstractions help hide the fact that the economy is overwhelmingly interdependent and organised hierarchically within industry. Even in their primary role as organisers, entrepreneurs depend on the group. A company president can only issue general guidelines to his managers, who must inevitably organise and direct much of their departments on their own. The larger a company gets, the less personal and direct control an entrepreneur has over it. They must delegate out an increasing share of authority and responsibility, and is more dependent than ever on others to help him run things, investigate conditions, inform policy, and make recommendations. Moreover, the authority structures are from the “top-down” — indeed the firm is essentially a command economy, with all members part of a collective working on a common plan to achieve a common goal (i.e. it is essentially collectivist in nature — which means it is not too unsurprising that Lenin argued that state socialism could be considered as one big firm or office and why the system he built on that model was so horrific).


So the firm (the key component of the capitalist economy) is marked by a distinct {\bf lack} of individualism, a lack usually ignored by right libertarians (or, at best, considered as “unavoidable”). As these firms are hierarchical structures and workers are paid to obey, it does make {\bf some} sense — in a capitalist environment — to assume that the entrepreneur is the main actor, but as an individualistic model of activity it fails totally. Perhaps it would not be unfair to say that capitalist individualism celebrates the entrepreneur because this reflects a hierarchical system in which for the one to flourish, the many must obey? (Also see section 1.1).


Capitalist individualism does not recognise the power structures that exist within capitalism and how they affect individuals. In Brian Morris’ words, what they fail {\em “to recognise is that most productive relations under capitalism allow little scope for creativity and self-expression on the part of workers; that such relationships are not equitable; nor are they freely engaged in for the mutual benefit of both parties, for workers have no control over the production process or over the product of their labour. Rand [like other right-libertarians] misleadingly equates trade, artistic production and wage-slavery\unknown{} [but] wage-slavery \unknown{} is quite different from the trade principle”} as it is a form of {\em “exploitation.”} [{\bf Ecology \& Anarchism}, p. 190]


He further notes that {\em “[s]o called trade relations involving human labour are contrary to the egoist values Rand [and other capitalist individualists] espouses — they involve little in the way of independence, freedom, integrity or justice.”} [{\bf Ibid.}, p. 191]


Moreover, capitalist individualism actually {\bf supports} authority and hierarchy. As Joshua Chen and Joel Rogers point out, the {\em “achievement of short-run material satisfaction often makes it irrational [from an individualist perspective] to engage in more radical struggle, since that struggle is by definition against those institutions which provide one’s current gain.”} In other words, to rise up the company structure, to “better oneself,” (or even get a good reference) you cannot be a pain in the side of management — obedient workers do well, rebel workers do not.


Thus the hierarchical structures help develop an “individualistic” perspective which actually reinforces those authority structures. This, as Cohn and Rogers notes, means that {\em “the structure in which [workers] find themselves yields less than optimal social results from their isolated but economically rational decisions.”} [quoted by Alfie Kohn, {\bf No Contest}, p. 67, p. 260f]


Steve Biko, a black activist murdered by the South African police in the 1970s, argued that {\em “the most potent weapon of the oppressor is the mind of the oppressed.”} And this is something capitalists have long recognised. Their investment in “Public Relations” and “education” programmes for their employees shows this clearly, as does the hierarchical nature of the firm. By having a ladder to climb, the firm rewards obedience and penalises rebellion. This aims at creating a mind-set which views hierarchy as good and so helps produce servile people.


This is why anarchists would agree with Alfie Kohn when he argues that {\em “the individualist worldview is a profoundly conservative doctrine: it inherently stifles change.”} [{\bf Ibid.}, p. 67] So, what is the best way for a boss to maintain his or her power? Create a hierarchical workplace and encourage capitalist individualism (as capitalist individualism actually works {\bf against} attempts to increase freedom from hierarchy). Needless to say, such a technique cannot work forever — hierarchy also encourages revolt — but such divide and conquer can be {\bf very} effective.


And as anarchist author Michael Moorcock put it, {\em “Rugged individualism also goes hand in hand with a strong faith in paternalism — albeit a tolerant and somewhat distant paternalism — and many otherwise sharp-witted libertarians seem to see nothing in the morality of a John Wayne Western to conflict with their views. Heinlein’s paternalism is at heart the same as Wayne’s\unknown{} To be an anarchist, surely, is to reject authority but to accept self-discipline and community responsibility. To be a rugged individualist a la Heinlein and others is to be forever a child who must obey, charm and cajole to be tolerated by some benign, omniscient father: Rooster Coburn shuffling his feet in front of a judge he respects for his office (but not necessarily himself) in True Grit.”} [{\bf Starship Stormtroopers}]


One last thing, don’t be fooled into thinking that individualism or concern about individuality — not {\bf quite} the same thing — is restricted to the right, they are not. For example, the {\em “individualist theory of society \unknown{} might be advanced in a capitalist or in an anti-capitalist form \unknown{} the theory as developed by critics of capitalism such as Hodgskin and the anarchist Tucker saw ownership of capital by a few as an obstacle to genuine individualism, and the individualist ideal was realisable only through the free association of labourers (Hodgskin) or independent proprietorship (Tucker).”} [David Miller, {\bf Social Justice}, pp. 290–1]


And the reason why social anarchists oppose capitalism is that it creates a {\bf false} individualism, an abstract one which crushes the individuality of the many and justifies (and supports) hierarchical and authoritarian social relations. In Kropotkin’s words, {\em “what has been called ‘individualism’ up to now has been only a foolish egoism which belittles the individual. It did not led to what it was established as a goal: that is the complete, broad, and most perfectly attainable development of individuality.”} The new individualism desired by Kropotkin {\em “will not consist \unknown{} in the oppression of one’s neighbour \unknown{} [as this] reduced the [individualist] \unknown{}to the level of an animal in a herd.”} [{\bf Selected Writings}, p, 295, p. 296]


\subsection{4.4 How does private property affect relationships?
}

Obviously, capitalist private property affects relationships between people by creating structures of power. Property, as we have argued all through this FAQ, creates relationships based upon domination — and this cannot help but produce servile tendencies within those subject to them (it also produces rebellious tendencies as well, the actual ratio between the two tendencies dependent on the individual in question and the community they are in). As anarchists have long recognised, power corrupts — both those subjected to it and those who exercise it.


While few, if any, anarchists would fail to recognise the importance of possession — which creates the necessary space all individuals need to be themselves — they all agree that private property corrupts this liberatory aspect of “property” by allowing relationships of domination and oppression to be built up on top of it. Because of this recognition, all anarchists have tried to equalise property and turn it back into possession.


Also, capitalist individualism actively builds barriers between people. Under capitalism, money rules and individuality is expressed via consumption choices (i.e. money). But money does not encourage an empathy with others. As Frank Stronach (chair of Magna International, a Canadian auto-parts maker that shifted its production to Mexico) put it, {\em “[t]o be in business your first mandate is to make money, and money has no heart, no soul, conscience, homeland.”} [cited by Doug Henwood, {\bf Wall Street}, p. 113] And for those who study economics, it seems that this dehumanising effect also strikes them as well:



\startblockquote
{\em “Studying economics also seems to make you a nastier person. Psychological studies have shown that economics graduate students are more likely to ‘free ride’ — shirk contributions to an experimental ‘public goods’ account in the pursuit of higher private returns — than the general public. Economists also are less generous that other academics in charitable giving. Undergraduate economics majors are more likely to defect in the classic prisoner’s dilemma game that are other majors. And on other tests, students grow less honest — expressing less of a tendency, for example, to return found money — after studying economics, but not studying a control subject like astronomy.}


{\em “This is no surprise, really. Mainstream economics is built entirely on a notion of self-interested individuals, rational self-maximisers who can order their wants and spend accordingly. There’s little room for sentiment, uncertainty, selflessness, and social institutions. Whether this is an accurate picture of the average human is open to question, but there’s no question that capitalism as a system and economics as a discipline both reward people who conform to the model.”} [Doug Henwood, {\bf Op. Cit.}, p, 143]



\stopblockquote
Which, of course, highlights the problems within the “trader” model advocated by Ayn Rand. According to her, the trader is {\bf the} example of moral behaviour — you have something I want, I have something you want, we trade and we both benefit and so our activity is self-interested and no-one sacrifices themselves for another. While this has {\bf some} intuitive appeal it fails to note that in the real world it is a pure fantasy. The trader wants to get the best deal possible for themselves and if the bargaining positions are unequal then one person will gain at the expense of the other (if the “commodity” being traded is labour, the seller may not even have the option of not trading at all). The trader is only involved in economic exchange, and has no concern for the welfare of the person they are trading with. They are a bearer of things, {\bf not} an individual with a wide range of interests, concerns, hopes and dreams. These are irrelevant, unless you can make money out of them of course! Thus the trader is often a manipulator and outside novels it most definitely is a case of “buyer beware!”


If the trader model is taken as the basis of interpersonal relationships, economic gain replaces respect and empathy for others. It replaces human relationships with relationships based on things — and such a mentality does not encompass how interpersonal relationships affect both you and the society you life in. In the end, it impoverishes society and individuality. Yes, any relationship must be based upon self-interest (mutual aid is, after all, something we do because we benefit from it in some way) but the trader model presents such a {\bf narrow} self-interest that it is useless and actively impoverishes the very things it should be protecting — individuality and interpersonal relationships (see section I.7.4 on how capitalism does not protect individuality).


\subsection{4.5 Does private property co-ordinate without hierarchy?
}

It is usually to find right-libertarians maintain that private property (i.e. capitalism) allows economic activity to be co-ordinated by non-hierarchical means. In other words, they maintain that capitalism is a system of large scale co-ordination without hierarchy. These claims follow the argument of noted right-wing, “free market” economist Milton Friedman who contrasts {\em “central planning involving the use of coercion — the technique of the army or the modern totalitarian state”} with {\em “voluntary co-operation between individuals — the technique of the marketplace”} as two distinct ways of co-ordinating the economic activity of large groups ({\em “millions”}) of people. [{\bf Capitalism and Freedom}, p. 13].


However, this is just playing with words. As they themselves point out the internal structure of a corporation or capitalist company is {\bf not} a “market” (i.e. non-hierarchical) structure, it is a “non-market” (hierarchical) structure of a market participant (see section 2.2). However “market participants” are part of the market. In other words, capitalism is {\bf not} a system of co-ordination without hierarchy because it does contain hierarchical organisations which {\bf are an essential part of the system}!


Indeed, the capitalist company {\bf is} a form of central planning and shares the same “technique” as the army. As the pro-capitalist writer Peter Drucker noted in his history of General Motors, {\em “[t]here is a remarkably close parallel between General Motors’ scheme of organisation and those of the two institutions most renowned for administrative efficiency: that of the Catholic Church and that of the modern army \unknown{}”} [quoted by David Enger, {\bf Apostles of Greed}, p. 66]. And so capitalism is marked by a series of totalitarian organisations — and since when was totalitarianism liberty enhancing? Indeed, many “anarcho”-capitalists actually celebrate the command economy of the capitalist firm as being more “efficient” than self-managed firms (usually because democracy stops action with debate). The same argument is applied by the Fascists to the political sphere. It does not change much — nor does it become less fascistic — when applied to economic structures. To state the obvious, such glorification of workplace dictatorship seems somewhat at odds with an ideology calling itself “libertarian” or “anarchist”. Is dictatorship more liberty enhancing to those subject to it than democracy? Anarchists doubt it (see section A.2.11 for details).


In order to claim that capitalism co-ordinates individual activity without hierarchy right-libertarians have to abstract from individuals and how they interact {\bf within} companies and concentrate purely on relationships {\bf between} companies. This is pure sophistry. Like markets, companies require at least two or more people to work — both are forms of social co-operation. If co-ordination within companies is hierarchical, then the system they work within is based upon hierarchy. To claim that capitalism co-ordinates without hierarchy is simply false — its based on hierarchy and authoritarianism. Capitalist companies are based upon denying workers self-government (i.e. freedom) during work hours. The boss tells workers what to do, when to do, how to do and for how long. This denial of freedom is discussed in greater depth in sections B.1 and B.4.


Because of the relations of power it creates, opposition to capitalist private property (and so wage labour) and the desire to see it ended is an essential aspect of anarchist theory. Due to its ideological blind spot with regards to apparently “voluntary” relations of domination and oppression created by the force of circumstances (see section 2 for details), “anarcho”-capitalism considers wage labour as a form of freedom and ignore its fascistic aspects (when not celebrating those aspects). Thus “anarcho”-capitalism is not anarchist. By concentrating on the moment the contract is signed, they ignore that freedom is restricted during the contract itself. While denouncing (correctly) the totalitarianism of the army, they ignore it in the workplace. But factory fascism is just as freedom destroying as the army or political fascism.


Due to this basic lack of concern for freedom, “anarcho”-capitalists cannot be considered as anarchists. Their total lack of concern about factory fascism (i.e. wage labour) places them totally outside the anarchist tradition. Real anarchists have always been aware of that private property and wage labour restriction freedom and desired to create a society in which people would be able to avoid it. In other words, where {\bf all} relations are non-hierarchical and truly co-operative.


To conclude, to claim that private property eliminates hierarchy is false. Nor does capitalism co-ordinate economic activities without hierarchical structures. For this reason anarchists support co-operative forms of production rather than capitalistic forms.


\section{5 Will privatising “the commons” increase liberty?
}

“Anarcho”-capitalists aim for a situation in which {\em “no land areas, no square footage in the world shall remain ‘public,’”} in other words {\bf everything} will be {\em “privatised.”} [Murray Rothbard, {\bf Nations by Consent}, p. 84] They claim that privatising “the commons” (e.g. roads, parks, etc.) which are now freely available to all will increase liberty. Is this true? We have shown before why the claim that privatisation can protect the environment is highly implausible (see section E.2). Here we will concern ourselves with private ownership of commonly used “property” which we all take for granted and pay for with taxes.


Its clear from even a brief consideration of a hypothetical society based on “privatised” roads (as suggested by Murray Rothbard in {\bf For a New Liberty}, pp. 202–203 and David Friedman in {\bf The Machinery of Freedom}, pp. 98–101) that the only increase of liberty will be for the ruling elite. As “anarcho”-capitalism is based on paying for what one uses, privatisation of roads would require some method of tracking individuals to ensure that they pay for the roads they use. In the UK, for example, during the 1980s the British Tory government looked into the idea of toll-based motorways. Obviously having toll-booths on motorways would hinder their use and restrict “freedom,” and so they came up with the idea of tracking cars by satellite. Every vehicle would have a tracking device installed in it and a satellite would record where people went and which roads they used. They would then be sent a bill or have their bank balances debited based on this information (in the fascist city-state/company town of Singapore such a scheme {\bf has} been introduced).


If we extrapolate from this example to a system of {\bf fully} privatised “commons,” it would clearly require all individuals to have tracking devices on them so they could be properly billed for use of roads, pavements, etc. Obviously being tracked by private firms would be a serious threat to individual liberty. Another, less costly, option would be for private guards to randomly stop and question car-owners and individuals to make sure they had paid for the use of the road or pavement in question. “Parasites” would be arrested and fined or locked up. Again, however, being stopped and questioned by uniformed individuals has more in common with police states than liberty. Toll-boothing {\bf every} street would be highly unfeasible due to the costs involved and difficulties for use that it implies. Thus the idea of privatising roads and charging drivers to gain access seems impractical at best and distinctly freedom endangering if implemented at worse.


Of course, the option of owners letting users have free access to the roads and pavements they construct and run would be difficult for a profit-based company. No one could make a profit in that case. If companies paid to construct roads for their customers/employees to use, they would be financially hindered in competition with other companies that did not, and thus would be unlikely to do so. If they restricted use purely to their own customers, the tracking problem appears again.


Some may object that this picture of extensive surveillance of individuals would not occur or be impossible. However, Murray Rothbard (in a slightly different context) argued that technology would be available to collate information about individuals. He argued that {\em “[i]t should be pointed out that modern technology makes even more feasible the collection and dissemination of information about people’s credit ratings and records of keeping or violating their contracts or arbitration agreements. Presumably, an anarchist [sic!] society would see the expansion of this sort of dissemination of data.”} [{\em “Society Without A State”}, in {\bf Nomos XIX}, Pennock and Chapman (eds.), p. 199] So, perhaps, with the total privatisation of society we would also see the rise of private Big Brothers, collecting information about individuals for use by property owners. The example of the {\bf Economic League} (a British company who provided the “service” of tracking the political affiliations and activities of workers for employers) springs to mind.


And, of course, these privatisation suggestions ignore differences in income and market power. If, for example, variable pricing is used to discourage road use at times of peak demand (to eliminate traffic jams at rush-hour) as is suggested both by Murray Rothbard and David Friedman, then the rich will have far more “freedom” to travel than the rest of the population. And we may even see people having to go into debt just to get to work or move to look for work.


Which raises another problem with notion of total privatisation, the problem that it implies the end of freedom of travel. Unless you get permission or (and this seems more likely) pay for access, you will not be able to travel {\bf anywhere.} As Rothbard {\bf himself} makes clear, “anarcho”-capitalism means the end of the right to roam or even travel. He states that {\em “it became clear to me that a totally privatised country would not have open borders at all. If every piece of land in a country were owned \unknown{} no immigrant could enter there unless invited to enter and allowed to rent, or purchase, property.”} [{\bf Nations by Consent}, p. 84] What happens to those who cannot {\bf afford} to pay for access is not addressed (perhaps, being unable to exit a given capitalist’s land they will become bonded labourers? Or be imprisoned and used to undercut workers’ wages via prison labour? Perhaps they will just be shot as trespassers? Who can tell?). Nor is it addressed how this situation actually {\bf increases} freedom. For Rothbard, a {\em “totally privatised country would be as closed as the particular inhabitants and property owners [{\bf not} the same thing, we must point out] desire. It seems clear, then, that the regime of open borders that exists {\bf de facto} in the US really amounts to a compulsory opening by the central state\unknown{} and does not genuinely reflect the wishes of the proprietors.”} [{\bf Op. Cit.}, p. 85] Of course, the wishes of {\bf non}-proprietors (the vast majority) do not matter in the slightest. Thus, it is clear, that with the privatisation of “the commons” the right to roam, to travel, would become a privilege, subject to the laws and rules of the property owners. This can hardly be said to {\bf increase} freedom for anyone bar the capitalist class.


Rothbard acknowledges that {\em “in a fully privatised world, access rights would obviously be a crucial part of land ownership.”} [{\bf Nations by Consent}, p. 86] Given that there is no free lunch, we can imagine we would have to pay for such “rights.” The implications of this are obviously unappealing and an obvious danger to individual freedom. The problem of access associated with the idea of privatising the roads can only be avoided by having a “right of passage” encoded into the “general libertarian law code.” This would mean that road owners would be required, by law, to let anyone use them. But where are “absolute” property rights in this case? Are the owners of roads not to have the same rights as other owners? And if “right of passage” is enforced, what would this mean for road owners when people sue them for car-pollution related illnesses? (The right of those injured by pollution to sue polluters is the main way “anarcho”-capitalists propose to protect the environment. See sections E.2 and E.3). It is unlikely that those wishing to bring suit could find, never mind sue, the millions of individual car owners who could have potentially caused their illness. Hence the road-owners would be sued for letting polluting (or unsafe) cars onto “their” roads. The road-owners would therefore desire to restrict pollution levels by restricting the right to use their property, and so would resist the “right of passage” as an “attack” on their “absolute” property rights. If the road-owners got their way (which would be highly likely given the need for “absolute” property rights and is suggested by the variable pricing way to avoid traffic jams mentioned above) and were able to control who used their property, freedom to travel would be {\bf very} restricted and limited to those whom the owner considered “desirable.” Indeed, Murray Rothbard supports such a regime ({\em “In the free [sic!] society, they [travellers] would, in the first instance, have the right to travel only on those streets whose owners agree to have them there”} [{\bf The Ethics of Liberty}, p. 119]). The threat to liberty in such a system is obvious — to all but Rothbard and other right-libertarians, of course.


To take another example, let us consider the privatisation of parks, streets and other public areas. Currently, individuals can use these areas to hold political demonstrations, hand out leaflets, picket and so on. However, under “anarcho”-capitalism the owners of such property can restrict such liberties if they desire, calling such activities “initiation of force” (although they cannot explain how speaking your mind is an example of “force”). Therefore, freedom of speech, assembly and a host of other liberties we take for granted would be reduced (if not eliminated) under a right-“libertarian” regime. Or, taking the case of pickets and other forms of social struggle, its clear that privatising “the commons” would only benefit the bosses. Strikers or other activists picketing or handing out leaflets in shopping centre’s are quickly ejected by private security even today. Think about how much worse it would become under “anarcho”-capitalism when the whole world becomes a series of malls — it would be impossible to hold a picket when the owner of the pavement objects, for example (as Rothbard himself argues, {\bf Op. Cit.}, p. 132) and if the owner of the pavement also happens to be the boss being picketed, then workers’ rights would be zero. Perhaps we could also see capitalists suing working class organisations for littering their property if they do hand out leaflets (so placing even greater stress on limited resources).


The I.W.W. went down in history for its rigorous defence of freedom of speech because of its rightly famous “free speech” fights in numerous American cities and towns. Repression was inflicted upon wobblies who joined the struggle by “private citizens,” but in the end the I.W.W. won. Consider the case under “anarcho”-capitalism. The wobblies would have been “criminal aggressors” as the owners of the streets have refused to allow “undesirables” to use them to argue their case. If they refused to acknowledge the decree of the property owners, private cops would have taken them away. Given that those who controlled city government in the historical example were the wealthiest citizens in town, its likely that the same people would have been involved in the fictional (“anarcho”-capitalist) account. Is it a good thing that in the real account the wobblies are hailed as heroes of freedom but in the fictional one they are “criminal aggressors”? Does converting public spaces into private property {\bf really} stop restrictions on free speech being a bad thing?


Of course, Rothbard (and other right-libertarians) are aware that privatisation will not remove restrictions on freedom of speech, association and so on (while, at the same time, trying to portray themselves as supporters of such liberties!). However, for right-libertarians such restrictions are of no consequence. As Rothbard argues, any {\em “prohibitions would not be state imposed, but would simply be requirements for residence or for use of some person’s or community’s land area.”} [{\bf Nations by Consent}, p. 85] Thus we yet again see the blindness of right-libertarians to the commonality between private property and the state. The state also maintains that submitting to its authority is the requirement for taking up residence in its territory (see also section 2.3 for more on this). As Benjamin Tucker noted, the state can be defined as (in part) {\em “the assumption of sole authority over a given area and all within it.”} [{\bf The Individualist Anarchists}, p. 24] If the property owners can determine “prohibitions” (i.e. laws and rules) for those who use the property then they are the {\em “sole authority over a given area and all within it,”} i.e. a state. Thus privatising “the commons” means subjecting the non-property owners to the rules and laws of the property owners — in effect, privatising the state and turning the world into a series of Monarchies and oligarchies without the pretence of democracy and democratic rights.


These examples can hardly be said to be increasing liberty for society as a whole, although “anarcho” capitalists seem to think they would. So far from {\bf increasing} liberty for all, then, privatising the commons would only increase it for the ruling elite, by giving them yet another monopoly from which to collect income and exercise their power over. It would {\bf reduce} freedom for everyone else. As Peter Marshall notes, {\em “[i]n the name of freedom, the anarcho-capitalists would like to turn public spaces into private property, but freedom does not flourish behind high fences protected by private companies but expands in the open air when it is enjoyed by all”} [{\bf Demanding the Impossible}, p. 564].


Little wonder Proudhon argued that {\em “if the public highway is nothing but an accessory of private property; if the communal lands are converted into private property; if the public domain, in short, is guarded, exploited, leased, and sold like private property — what remains for the proletaire? Of what advantage is it to him that society has left the state of war to enter the regime of police?”} [{\bf System of Economic Contradictions}, p. 371]


\section{6 Is “anarcho”-capitalism against the state?
}

No. Due to its basis in private property, “anarcho”-capitalism implies a class division of society into bosses and workers. Any such division will require a state to maintain it. However, it need not be the same state as exists now. Regarding this point, “anarcho”-capitalism plainly advocates “defence associations” to protect property. For the “anarcho”-capitalist, however, these private companies are not states. For anarchists, they most definitely are.


According to Murray Rothbard [{\em “Society Without A State”}, in {\bf Nomos XIX}, Pennock and Chapman, eds., p. 192.], a state must have one or both of the following characteristics:



\startitemize[N]\relax
\item[] The ability to tax those who live within it.




 \item[] It asserts and usually obtains a coerced monopoly of the provision of defence over a given area.




 
\stopitemize
He makes the same point in {\bf The Ethics of Liberty} [p. 171].


Instead of this, the “anarcho”-capitalist thinks that people should be able to select their own “defence companies” (which would provide the needed police) and courts from the free market in “defence” which would spring up after the state monopoly has been eliminated. These companies {\em “all\unknown{} would have to abide by the basic law code”} [{\em “Society Without A State”}, p. 206]. Thus a {\em “general libertarian law code”} would govern the actions of these companies. This “law code” would prohibit coercive aggression at the very least, although to do so it would have to specify what counted as legitimate property, how said can be owned and what actually constitutes aggression. Thus the law code would be quite extensive.


How is this law code to be actually specified? Would these laws be democratically decided? Would they reflect common usage (i.e. custom)? “supply and demand”? “Natural law”? Given the strong dislike of democracy shown by “anarcho”-capitalists, we think we can safely say that some combination of the last two options would be used. Murray Rothbard, as noted in section 1.4, opposed the individualist anarchist principle that juries would judge both the facts and the law, suggesting instead that {\em “Libertarian lawyers and jurists”} would determine a {\em “rational and objective code of libertarian legal principles and procedures.”} The judges in his system would {\em “not [be] making the law but finding it on the basis of agreed-upon principles derived either from custom or reason.”} [{\em “Society without a State”}, {\bf Op. Cit.}, p. 206] David Friedman, on the other hand, argues that different defence firms would sell their own laws. [{\bf The Machinery of Freedom}, p. 116] It is sometimes acknowledged that non-libertarian laws may be demanded (and supplied) in such a market.


Around this system of “defence companies” is a free market in “arbitrators” and “appeal judges” to administer justice and the “basic law code.” Rothbard believes that such a system would see {\em “arbitrators with the best reputation for efficiency and probity\unknown{}[being] chosen by the various parties in the market\unknown{}[and] will come to be given an increasing amount of business.”} [Rothbard, {\bf Op. Cit.}, p.199] Judges {\em “will prosper on the market in proportion to their reputation for efficiency and impartiality.”} [{\bf Op. Cit.}, p. 204]


Therefore, like any other company, arbitrators would strive for profits and wealth, with the most successful ones becoming {\em “prosperous.”} Of course, such wealth would have no impact on the decisions of the judges, and if it did, the population (in theory) are free to select any other judge (although, of course, they would also {\em “strive for profits and wealth”} — which means the choice of character may be somewhat limited! — and the laws which they were using to guide their judgements would be enforcing capitalist rights).


Whether or not this system would work as desired is discussed in the following sections. We think that it will not. Moreover, we will argue that “anarcho”-capitalist “defence companies” meet not only the criteria of statehood we outlined in section B.2, but also Rothbard’s own criteria for the state, quoted above.


As regards the anarchist criterion, it is clear that “defence companies” exist to defend private property; that they are hierarchical (in that they are capitalist companies which defend the power of those who employ them); that they are professional coercive bodies; and that they exercise a monopoly of force over a given area (the area, initially, being the property of the person or company who is employing the “association”). If, as Ayn Rand noted (using a Weberian definition of the state) a government is an institution {\em “that holds the exclusive power to {\bf enforce} certain rules of conduct in a given geographical area”} [{\bf Capitalism: The Unknown Ideal}, p. 239] then these “defence companies” are the means by which the property owner (who exercises a monopoly to determine the rules governing their property) enforce their rules.


For this (and other reasons), we should call the “anarcho”-capitalist defence firms “private states” — that is what they are — and “anarcho”-capitalism “private state” capitalism.


Before discussing these points further, it is necessary to point out a relatively common fallacy of “anarcho”-capitalists. This is the idea that “defence” under the system they advocate means defending people, not territorial areas. This, for some, means that defence companies are not “states.” However, as people and their property and possessions do not exist merely in thought but on the Earth, it is obvious that these companies will be administering “justice” over a given area of the planet. It is also obvious, therefore, that these “defence associations” will operate over a (property-owner defined) area of land and enforce the property-owner’s laws, rules and regulations. The deeply anti-libertarian, indeed fascistic, aspects of this “arrangement” will be examined in the following sections.


\subsection{6.1 What’s wrong with this “free market” justice?
}

It does not take much imagination to figure out whose interests {\em “prosperous”} arbitrators, judges and defence companies would defend: their own, as well as those who pay their wages — which is to say, other members of the rich elite. As the law exists to defend property, then it (by definition) exists to defend the power of capitalists against their workers.


Rothbard argues that the {\em “judges”} would {\em “not [be] making the law but finding it on the basis of agreed-upon principles derived either from custom or reason”} [Rothbard, {\bf Op. Cit.}, p. 206]. However, this begs the question: {\bf whose} reason? {\bf whose} customs? Do individuals in different classes share the same customs? The same ideas of right and wrong? Would rich and poor desire the same from a {\em “basic law code”}? Obviously not. The rich would only support a code which defended their power over the poor.


Although only {\em “finding”} the law, the arbitrators and judges still exert an influence in the “justice” process, an influence not impartial or neutral. As the arbitrators themselves would be part of a profession, with specific companies developing within the market, it does not take a genius to realise that when “interpreting” the “basic law code,” such companies would hardly act against their own interests as companies. In addition, if the “justice” system was based on “one dollar, one vote,” the “law” would best defend those with the most “votes” (the question of market forces will be discussed in section 6.3). Moreover, even if “market forces” would ensure that “impartial” judges were dominant, all judges would be enforcing a {\bf very} partial law code (namely one that defended {\bf capitalist} property rights). Impartiality when enforcing partial laws hardly makes judgements less unfair.


Thus, due to these three pressures — the interests of arbitrators/judges, the influence of money and the nature of the law — the terms of “free agreements” under such a law system would be tilted in favour of lenders over debtors, landlords over tenants, employers over employees, and in general, the rich over the poor, just as we have today. This is what one would expect in a system based on “unrestricted” property rights and a (capitalist) free market. A similar tendency towards the standardisation of output in an industry in response to influences of wealth can be seen from the current media system (see section D.3 — How does wealth influence the mass media?)


Some “anarcho”-capitalists, however, claim that just as cheaper cars were developed to meet demand, so cheaper defence associations and “people’s arbitrators” would develop on the market for the working class. In this way impartiality will be ensured. This argument overlooks a few key points:


Firstly, the general “libertarian” law code would be applicable to {\bf all} associations, so they would have to operate within a system determined by the power of money and of capital. The law code would reflect, therefore, property {\bf not} labour and so “socialistic” law codes would be classed as “outlaw” ones. The options then facing working people is to select a firm which best enforced the {\bf capitalist} law in their favour. And as noted above, the impartial enforcement of a biased law code will hardly ensure freedom or justice for all.


Secondly, in a race between a Jaguar and a Volkswagen Beetle, who is more likely to win? The rich would have “the best justice money can buy,” as they do now. Members of the capitalist class would be able to select the firms with the best lawyers, best private cops and most resources. Those without the financial clout to purchase quality “justice” would simply be out of luck — such is the “magic” of the marketplace.


Thirdly, because of the tendency toward concentration, centralisation, and oligopoly under capitalism (due to increasing capital costs for new firms entering the market, as discussed in section C.4), a few companies would soon dominate the market — with obvious implications for “justice.”


Different firms will have different resources. In other words, in a conflict between a small firm and a larger one, the smaller one is at a disadvantage in terms of resources. They may not be in a position to fight the larger company if it rejects arbitration and so may give in simply because, as the “anarcho”-capitalists so rightly point out, conflict and violence will push up a company’s costs and so they would have to be avoided by smaller companies. It is ironic that the “anarcho”-capitalist implicitly assumes that every “defence company” is approximately of the same size, with the same resources behind it. In real life, this would clearly {\bf not} the case.


Fourthly, it is {\bf very} likely that many companies would make subscription to a specific “defence” firm or court a requirement of employment. Just as today many (most?) workers have to sign no-union contracts (and face being fired if they change their minds), it does not take much imagination to see that the same could apply to “defence” firms and courts. This was/is the case in company towns (indeed, you can consider unions as a form of “defence” firm and these companies refused to recognise them). As the labour market is almost always a buyer’s market, it is not enough to argue that workers can find a new job without this condition. They may not and so have to put up with this situation. And if (as seems likely) the laws and rules of the property-owner will take precedence in any conflict, then workers and tenants will be at a disadvantage no matter how “impartial” the judges.


Ironically, some “anarcho”-capitalists point to current day company/union negotiations as an example of how different defence firms would work out their differences peacefully. Sadly for this argument, union rights under “actually existing capitalism” were created and enforced by the state in direct opposition to capitalist “freedom of contract.” Before the law was changed, unions were often crushed by force — the companies were better armed, had more resources and had the law on their side. Today, with the “downsizing” of companies we can see what happens to “peaceful negotiation” and “co-operation” between unions and companies when it is no longer required (i.e. when the resources of both sides are unequal). The market power of companies far exceeds those of the unions and the law, by definition, favours the companies. As an example of how competing “protection agencies” will work in an “anarcho”-capitalist society, it is far more insightful than originally intended!


Now let us consider the {\em “basic law code”} itself. How the laws in the {\em “general libertarian law code”} would actually be selected is anyone’s guess, although many “anarcho”-capitalists support the myth of “natural law,” and this would suggest an unchangeable law code selected by those considered as “the voice of nature” (see section 11. for a discussion of its authoritarian implications). David Friedman argues that as well as a market in defence companies, there will also be a market in laws and rights. However, there will be extensive market pressure to unify these differing law codes into one standard one (imagine what would happen if ever CD manufacturer created a unique CD player, or every computer manufacturer different sized floppy-disk drivers — little wonder, then, that over time companies standardise their products). Friedman himself acknowledges that this process is likely (and uses the example of standard paper sizes to indicate such a process).


In any event, the laws would not be decided on the basis of “one person, one vote”; hence, as market forces worked their magic, the “general” law code would reflect vested interests and so be very hard to change. As rights and laws would be a commodity like everything else in capitalism, they would soon reflect the interests of the rich — particularly if those interpreting the law are wealthy professionals and companies with vested interests of their own. Little wonder that the individualist anarchists proposed “trial by jury” as the only basis for real justice in a free society. For, unlike professional “arbitrators,” juries are ad hoc, made up of ordinary people and do not reflect power, authority, or the influence of wealth. And by being able to judge the law as well as a conflict, they can ensure a populist revision of laws as society progresses.


Thus a system of “defence” on the market will continue to reflect the influence and power of property owners and wealth and not be subject to popular control beyond choosing between companies to enforce the capitalist laws.


\subsection{6.2 What are the social consequences of such a system?
}

The “anarcho” capitalist imagines that there will be police agencies, “defence associations,” courts, and appeals courts all organised on a free-market basis and available for hire. As David Weick points out, however, the major problem with such a system would not be the corruption of “private” courts and police forces (although, as suggested above, this could indeed be a problem):



\startblockquote
{\em “There is something more serious than the ‘Mafia danger’, and this other problem concerns the role of such ‘defence’ institutions in a given social and economic context.}


{\em “[The] context\unknown{} is one of a free-market economy with no restraints upon accumulation of property. Now, we had an American experience, roughly from the end of the Civil War to the 1930’s, in what were in effect private courts, private police, indeed private governments. We had the experience of the (private) Pinkerton police which, by its spies, by its {\bf agents provocateurs,} and by methods that included violence and kidnapping, was one of the most powerful tools of large corporations and an instrument of oppression of working people. We had the experience as well of the police forces established to the same end, within corporations, by numerous companies\unknown{} (The automobile companies drew upon additional covert instruments of a private nature, usually termed vigilante, such as the Black Legion). These were, in effect, private armies, and were sometimes described as such. The territories owned by coal companies, which frequently included entire towns and their environs, the stores the miners were obliged by economic coercion to patronise, the houses they lived in, were commonly policed by the private police of the United States Steel Corporation or whatever company owned the properties. The chief practical function of these police was, of course, to prevent labour organisation and preserve a certain balance of ‘bargaining.’}


{\em “These complexes were a law unto themselves, powerful enough to ignore, when they did not purchase, the governments of various jurisdictions of the American federal system. This industrial system was, at the time, often characterised as feudalism\unknown{}”} [{\em “Anarchist Justice”}, {\bf Op. Cit.}, pp. 223–224]



\stopblockquote
For a description of the weaponry and activities of these private armies, the economic historian Maurice Dobbs presents an excellent summary in {\bf Studies in Capitalist Development} [pp. 353–357]. According to a report on {\em “Private Police Systems”} cited by Dobbs, in a town dominated by Republican Steel, the {\em “civil liberties and the rights of labour were suppressed by company police. Union organisers were driven out of town.”} Company towns had their own (company-run) money, stores, houses and jails and many corporations had machine-guns and tear-gas along with the usual shot-guns, rifles and revolvers. The {\em “usurpation of police powers by privately paid ‘guards and ‘deputies’, often hired from detective agencies, many with criminal records”} was {\em “a general practice in many parts of the country.”}


The local (state-run) law enforcement agencies turned a blind-eye to what was going on (after all, the workers {\bf had} broken their contracts and so were “criminal aggressors” against the companies) even when union members and strikers were beaten and killed. The workers own defence organisations were the only ones willing to help them, and if the workers seemed to be winning then troops were called in to “restore the peace” (as happened in the Ludlow strike, when strikers originally cheered the troops as they thought they would defend their civil rights; needless to say, they were wrong).


Here we have a society which is claimed by many “anarcho”-capitalists as one of the closest examples to their “ideal,” with limited state intervention, free reign for property owners, etc. What happened? The rich reduced the working class to a serf-like existence, capitalist production undermined independent producers (much to the annoyance of individualist anarchists at the time), and the result was the emergence of the corporate America that “anarcho”-capitalists say they oppose.


Are we to expect that “anarcho”-capitalism will be different? That, unlike before, “defence” firms will intervene on behalf of strikers? Given that the “general libertarian law code” will be enforcing capitalist property rights, workers will be in exactly the same situation as they were then. Support of strikers violating property rights would be a violation of the “general libertarian law code” and be costly for profit making firms to do (if not dangerous as they could be “outlawed” by the rest). Thus “anarcho”-capitalism will extend extensive rights and powers to bosses, but few if any rights to rebellious workers. And this difference in power is enshrined within the fundamental institutions of the system.


In evaluating “anarcho”-capitalism’s claim to be a form of anarchism, Peter Marshall notes that {\em “private protection agencies would merely serve the interests of their paymasters.”} [{\bf Demanding the Impossible}, p. 653] With the increase of private “defence associations” under “really existing capitalism” today (associations that many “anarcho”-capitalists point to as examples of their ideas), we see a vindication of Marshall’s claim. There have been many documented experiences of protesters being badly beaten by private security guards. As far as market theory goes, the companies are only supplying what the buyer is demanding. The rights of others are {\bf not a factor} (yet more “externalities,” obviously). Even if the victims successfully sue the company, the message is clear — social activism can seriously damage your health. With a reversion to “a general libertarian law code” enforced by private companies, this form of “defence” of “absolute” property rights can only increase, perhaps to the levels previously attained in the heyday of US capitalism, as described above by Weick.


\subsection{6.3 But surely market forces will stop abuses by the rich?
}

Unlikely. The rise of corporations within America indicates exactly how a “general libertarian law code” would reflect the interests of the rich and powerful. The laws recognising corporations as “legal persons” were {\bf not} primarily a product of “the state” but of private lawyers hired by the rich — a result with which Rothbard would have no problem. As Howard Zinn notes:



\startblockquote
{\em “the American Bar Association, organised by lawyers accustomed to serving the wealthy, began a national campaign of education to reverse the [Supreme] Court decision [that companies could not be considered as a person]\unknown{} By 1886\unknown{} the Supreme Court had accepted the argument that corporations were ‘persons’ and their money was property protected by the process clause of the Fourteenth Amendment\unknown{} The justices of the Supreme Court were not simply interpreters of the Constitution. They were men of certain backgrounds, of certain [class] interests.”} [{\bf A People’s History of the United States}, p. 255]



\stopblockquote
Of course it will be argued that the Supreme Court is a monopoly and so our analysis is flawed. In “anarcho”-capitalism there is no monopoly. But the corporate laws came about because there was a demand for them. That demand would still have existed in “anarcho”-capitalism. Now, while there may be no Supreme Court, Rothbard does maintain that {\em “the basic Law Code \unknown{}would have to be agreed upon by all the judicial agencies”} but he maintains that this {\em “would imply no unified legal system”}! Even though {\em “[a]ny agencies that transgressed the basic libertarian law code would be open outlaws”} and soon crushed this is {\bf not}, apparently, a monopoly. [{\bf The Ethics of Liberty}, p. 234] So, you either agree to the law code or you go out of business. And that is {\bf not} a monopoly! Therefore, we think, our comments on the Supreme Court decision are valid.


If all the available defence firms enforce the same laws, then it can hardly be called “competitive”! And if this is the case (and it is) {\em “when private wealth is uncontrolled, then a police-judicial complex enjoying a clientele of wealthy corporations whose motto is self-interest is hardly an innocuous social force controllable by the possibility of forming or affiliating with competing ‘companies.’”} [Weick, {\bf Op. Cit.}, p. 225]


This is particularly true if these companies are themselves Big Business and so have a large impact on the laws they are enforcing. If the law code recognises and protects capitalist power, property and wealth as fundamental {\bf any} attempt to change this is “initiation of force” and so the power of the rich is written into the system from the start!


(And, we must add, if there is a general libertarian law code to which all must subscribe, where does that put customer demand? If people demand a non-libertarian law code, will defence firms refuse to supply it? If so, will not new firms, looking for profit, spring up that will supply what is being demanded? And will that not put them in direct conflict with the existing, pro-general law code ones? And will a market in law codes not just reflect economic power and wealth? David Friedman, who is for a market in law codes, argues that {\em “[i]f almost everyone believes strongly that heroin addiction is so horrible that it should not be permitted anywhere under any circumstances anarcho-capitalist institutions will produce laws against heroin. Laws are being produced on the market, and that is what the market wants.”} And he adds that {\em “market demands are in dollars, not votes. The legality of heroin will be determined, not by how many are for or against but how high a cost each side is willing to bear in order to get its way.”} [{\bf The Machinery of Freedom}, p. 127] And, as the market is less than equal in terms of income and wealth, such a position will mean that the capitalist class will have a higher effective demand than the working class, and more resources to pay for any conflicts that arise. Thus any law codes that develop will tend to reflect the interests of the wealthy.)


Which brings us nicely on to the next problem regarding market forces.


As well as the obvious influence of economic interests and differences in wealth, another problem faces the “free market” justice of “anarcho”-capitalism. This is the {\em “general libertarian law code”} itself. Even if we assume that the system actually works like it should in theory, the simple fact remains that these “defence companies” are enforcing laws which explicitly defend capitalist property (and so social relations). Capitalists own the means of production upon which they hire wage-labourers to work and this is an inequality established {\bf prior} to any specific transaction in the labour market. This inequality reflects itself in terms of differences in power within (and outside) the company and in the “law code” of “anarcho”-capitalism which protects that power against the dispossessed.


In other words, the law code within which the defence companies work assumes that capitalist property is legitimate and that force can legitimately be used to defend it. This means that, in effect, “anarcho”-capitalism is based on a monopoly of law, a monopoly which explicitly exists to defend the power and capital of the wealthy. The major difference is that the agencies used to protect that wealth will be in a weaker position to act independently of their pay-masters. Unlike the state, the “defence” firm is not remotely accountable to the general population and cannot be used to equalise even slightly the power relationships between worker and capitalist.


And, needless to say, it is very likely that the private police forces {\bf will} give preferential treatment to their wealthier customers (what business does not?) and that the law code will reflect the interests of the wealthier sectors of society (particularly if {\em “prosperous”} judges administer that code) in reality, even if not in theory. Since, in capitalist practice, “the customer is always right,” the best-paying customers will get their way in “anarcho”-capitalist society.


For example, in chapter 29 of {\bf The Machinery of Freedom}, David Friedman presents an example of how a clash of different law codes could be resolved by a bargaining process (the law in question is the death penalty). This process would involve one defence firm giving a sum of money to the other for them accepting the appropriate (anti/pro capital punishment) court. Friedman claims that {\em “[a]s in any good trade, everyone gains”} but this is obviously not true. Assuming the anti-capital punishment defence firm pays the pro one to accept an anti-capital punishment court, then, yes, both defence firms have made money and so are happy, so are the anti-capital punishment consumers but the pro-death penalty customers have only (perhaps) received a cut in their bills. Their desire to see criminals hanged (for whatever reason) has been ignored (if they were not in favour of the death penalty, they would not have subscribed to that company). Friedman claims that the deal, by allowing the anti-death penalty firm to cut its costs, will ensure that it {\em “keep its customers and even get more”} but this is just an assumption. It is just as likely to loose customers to a defence firm that refuses to compromise (and has the resources to back it up). Friedman’s assumption that lower costs will automatically win over people’s passions is unfounded. As is the assumption that both firms have equal resources and bargaining power. If the pro-capital punishment firm demands more than the anti can provide and has larger weaponry and troops, then the anti defence firm may have to agree to let the pro one have its way.


So, all in all, it is {\bf not} clear that {\em “everyone gains”} — there may be a sizeable percentage of those involved who do not “gain” as their desire for capital punishment is traded away by those who claimed they would enforce it.


In other words, a system of competing law codes and privatised rights does not ensure that {\bf all} consumers interests are meet. Given unequal resources within society, it is also clear that the “effective demand” of the parties involved to see their law codes enforced is drastically different. The wealthy head of a transnational corporation will have far more resources available to him to pay for {\bf his} laws to be enforced than one of his employees on the assembly line. Moreover, as we argue in sections 3.1 and 10.2, the labour market is usually skewed in favour of capitalists. This means that workers have to compromise to get work and such compromises may involve agreeing to join a specific “defence” firm or not join one at all (just as workers are often forced to sign non-union contracts today in order to get work). In other words, a privatised law system is very likely to skew the enforcement of laws in line with the skewing of income and wealth in society. At the very least, unlike every other market, the customer is {\bf not} guaranteed to get exactly what they demand simply because the product they “consume” is dependent on other within the same market to ensure its supply. The unique workings of the law/defence market are such as to deny customer choice (we will discuss other aspects of this unique market shortly).


Weick sums up by saying {\em “any judicial system is going to exist in the context of economic institutions. If there are gross inequalities of power in the economic and social domains, one has to imagine society as strangely compartmentalised in order to believe that those inequalities will fail to reflect themselves in the judicial and legal domain, and that the economically powerful will be unable to manipulate the legal and judicial system to their advantage. To abstract from such influences of context, and then consider the merits of an abstract judicial system\unknown{} is to follow a method that is not likely to take us far. This, by the way, is a criticism that applies\unknown{}to any theory that relies on a rule of law to override the tendencies inherent in a given social and economic system”} [Weick, {\bf Op. Cit.}, p. 225] (For a discussion of this problem as it would surface in attempts to protect the environment under “anarcho”-capitalism, see sections E.2 and E.3).


There is another reason why “market forces” will not stop abuse by the rich, or indeed stop the system from turning from private to public statism. This is due to the nature of the “defence” market (for a similar analysis of the “defence” market see Tyler Cowen’s {\em “Law as a Public Good: The Economics of Anarchy”} in {\bf Economics and Philosophy}, no. 8 (1992), pp. 249–267 and {\em “Rejoinder to David Friedman on the Economics of Anarchy”} in {\bf Economics and Philosophy}, no. 10 (1994), pp. 329–332). In “anarcho”-capitalist theory it is assumed that the competing “defence companies” have a vested interest in peacefully settling differences between themselves by means of arbitration. In order to be competitive on the market, companies will have to co-operate via contractual relations otherwise the higher price associated with conflict will make the company uncompetitive and it will go under. Those companies that ignore decisions made in arbitration would be outlawed by others, ostracised and their rulings ignored. By this process, it is argued, a system of competing “defence” companies will be stable and not turn into a civil war between agencies with each enforcing the interests of their clients against others by force.


However, there is a catch. Unlike every other market, the businesses in competition in the “defence” industry {\bf must} co-operate with its fellows in order to provide its services for its customers. They need to be able to agree to courts and judges, agree to abide by decisions and law codes and so forth. In economics there are other, more accurate, terms to describe co-operative activity between companies: collusion and cartels. Collusion and cartels is where companies in a specific market agree to work together to restrict competition and reap the benefits of monopoly power by working to achieve the same ends in partnership with each other. In other words this means that collusion is built into the system, with the necessary contractual relations between agencies in the “protection” market requiring that firms co-operate and, by so doing, to behave (effectively) as one large firm (and so, effectively, resemble the state even more than they already do). Quoting Adam Smith seems appropriate here: {\em “People of the same trade seldom meet together, even for merriment and diversion, but the conversation ends in a conspiracy against the public, or in some contrivance to raise prices.”} [{\bf The Wealth of Nations}, p. 117]


For example, when buying food it does not matter whether the supermarkets I visit have good relations with each other. The goods I buy are independent of the relationships that exist between competing companies. However, in the case of private states, this is {\bf not} the case. If a specific “defence” company has bad relationships with other companies in the market then it is against my self-interest to subscribe to it. Why join a private state if its judgements are ignored by the others and it has to resort to violence to be heard? This, as well as being potentially dangerous, will also push up the prices I have to pay. Arbitration is one of the most important services a defence firm can offer its customers and its market share is based upon being able to settle interagency disputes without risk of war or uncertainty that the final outcome will not be accepted by all parties.


Therefore, the market set-up within the “anarcho”-capitalist “defence” market is such that private states {\bf have to co-operate} with the others (or go out of business fast) and this means collusion can take place. In other words, a system of private states will have to agree to work together in order to provide the service of “law enforcement” to their customers and the result of such co-operation is to create a cartel. However, unlike cartels in other industries, the “defence” cartel will be a stable body simply because its members {\bf have} to work with their competitors in order to survive.


Let us look at what would happen after such a cartel is formed in a specific area and a new “defence company” desired to enter the market. This new company will have to work with the members of the cartel in order to provide its services to its customers (note that “anarcho”-capitalists already assume that they {\em “will have to”} subscribe to the same law code). If the new defence firm tries to under-cut the cartel’s monopoly prices, the other companies would refuse to work with it. Having to face constant conflict or the possibility of conflict, seeing its decisions being ignored by other agencies and being uncertain what the results of a dispute would be, few would patronise the new “defence company.” The new company’s prices would go up and so face either folding or joining the cartel. Unlike every other market, if a “defence company” does not have friendly, co-operative relations with other firms in the same industry then it will go out of business.


This means that the firms that are co-operating have but to agree not to deal with new firms which are attempting to undermine the cartel in order for them to fail. A “cartel busting” firm goes out of business in the same way an outlaw one does — the higher costs associated with having to solve all its conflicts by force, not arbitration, increases its production costs much higher than the competitors and the firm faces insurmountable difficulties selling its products at a profit (ignoring any drop of demand due to fears of conflict by actual and potential customers). Even if we assume that many people will happily join the new firm in spite of the dangers to protect themselves against the cartel and its taxation (i.e. monopoly profits), enough will remain members of the cartel (perhaps they will be fired if they change, perhaps they dislike change and think the extra money is worth peace, perhaps they fear that by joining the new company their peace will be disrupted or the outcomes of their problems with others too unsure to be worth it, perhaps they are shareholders and want to maintain their income) so that co-operation will still be needed and conflict unprofitable and dangerous (and as the cartel will have more resources than the new firm, it could usually hold out longer than the new firm could). In effect, breaking the cartel may take the form of an armed revolution — as it would with any state.


The forces that break up cartels and monopolies in other industries (such as free entry — although, of course the “defence” market will be subject to oligopolistic tendencies as any other and this will create barriers to entry, see section C.4) do not work here and so new firms have to co-operate or loose market share and/or profits. This means that “defence companies” will reap monopoly profits and, more importantly, have a monopoly of force over a given area.


Hence a monopoly of private states will develop in addition to the existing monopoly of law and this is a de facto monopoly of force over a given area (i.e. some kind of public state run by share holders). New companies attempting to enter the “defence” industry will have to work with the existing cartel in order to provide the services it offers to its customers. The cartel is in a dominant position and new entries into the market either become part of it or fail. This is exactly the position with the state, with “private agencies” free to operate as long as they work to the state’s guidelines. As with the monopolist “general libertarian law code”, if you do not toe the line, you go out of business fast.


It is also likely that a multitude of cartels would develop, with a given cartel operating in a given locality. This is because law enforcement would be localised in given areas as most crime occurs where the criminal lives. Few criminals would live in New York and commit crimes in Portland. However, as defence companies have to co-operate to provide their services, so would the cartels. Few people live all their lives in one area and so firms from different cartels would come into contact, so forming a cartel of cartels.


A cartel of cartels may (perhaps) be less powerful than a local cartel, but it would still be required and for exactly the same reasons a local one is. Therefore “anarcho”-capitalism would, like “actually existing capitalism,” be marked by a series of public states covering given areas, co-ordinated by larger states at higher levels. Such a set up would parallel the United States in many ways except it would be run directly by wealthy shareholders without the sham of “democratic” elections. Moreover, as in the USA and other states there will still be a monopoly of rules and laws (the “general libertarian law code”).


Some “anarcho”-capitalists claim that this will not occur, but that the co-operation needed to provide the service of law enforcement will somehow {\bf not} turn into collusion between companies. However, they are quick to argue that renegade “agencies” (for example, the so-called “Mafia problem” or those who reject judgements) will go out of business because of the higher costs associated with conflict and not arbitration. However, these higher costs are ensured because the firms in question do not co-operate with others. If other agencies boycott a firm but co-operate with all the others, then the boycotted firm will be at the same disadvantage — regardless of whether it is a cartel buster or a renegade.


The “anarcho”-capitalist is trying to have it both ways. If the punishment of non-conforming firms cannot occur, then “anarcho”-capitalism will turn into a war of all against all or, at the very least, the service of social peace and law enforcement cannot be provided. If firms cannot deter others from disrupting the social peace (one service the firm provides) then “anarcho”-capitalism is not stable and will not remain orderly as agencies develop which favour the interests of their own customers and enforce their own law codes at the expense of others. If collusion cannot occur (or is too costly) then neither can the punishment of non-conforming firms and “anarcho”-capitalism will prove to be unstable.


So, to sum up, the “defence” market of private states has powerful forces within it to turn it into a monopoly of force over a given area. From a privately chosen monopoly of force over a specific (privately owned) area, the market of private states will turn into a monopoly of force over a general area. This is due to the need for peaceful relations between companies, relations which are required for a firm to secure market share. The unique market forces that exist within this market ensure collusion and monopoly.


In other words, the system of private states will become a cartel and so a public state — unaccountable to all but its shareholders, a state of the wealthy, by the wealthy, for the wealthy. In other words, fascism.


\subsection{6.4 Why are these “defence associations” states?
}

It is clear that “anarcho”-capitalist defence associations meet the criteria of statehood outlined in section B.2 (“Why are anarchists against the state”). They defend property and preserve authority relationships, they practice coercion, and are hierarchical institutions which govern those under them on behalf of a “ruling elite,” i.e. those who employ both the governing forces and those they govern. Thus, from an anarchist perspective, these “defence associations” as most definitely states.


What is interesting, however, is that by their own definitions a very good case can be made that these “defence associations” as states in the “anarcho”-capitalist sense too. Capitalist apologists usually define a “government” (or state) as those who have a monopoly of force and coercion within a given area. Relative to the rest of the society, these defence associations would have a monopoly of force and coercion of a given piece of property; thus, by the “anarcho”-capitalists’ {\bf own definition} of statehood, these associations would qualify!


If we look at Rothbard’s definition of statehood, which requires (a) the power to tax and/or (b) a {\em “coerced monopoly of the provision of defence over a given area”}, “anarcho”-capitalism runs into trouble.


In the first place, the costs of hiring defence associations will be deducted from the wealth created by those who use, but do not own, the property of capitalists and landlords. Let not forget that a capitalist will only employ a worker or rent out land and housing if they make a profit from so doing. Without the labour of the worker, there would be nothing to sell and no wages to pay for rent. Thus a company’s or landlord’s “defence” firm will be paid from the revenue gathered from the capitalists power to extract a tribute from those who use, but do not own, a property. In other words, workers would pay for the agencies that enforce their employers’ authority over them via the wage system and rent — taxation in a more insidious form.


In the second, under capitalism most people spend a large part of their day on other people’s property — that is, they work for capitalists and/or live in rented accommodation. Hence if property owners select a “defence association” to protect their factories, farms, rental housing, etc., their employees and tenants will view it as a {\em “coerced monopoly of the provision of defence over a given area.”} For certainly the employees and tenants will not be able to hire their own defence companies to expropriate the capitalists and landlords. So, from the standpoint of the employees and tenants, the owners do have a monopoly of “defence” over the areas in question. Of course, the “anarcho”-capitalist will argue that the tenants and workers “consent” to {\bf all} the rules and conditions of a contract when they sign it and so the property owner’s monopoly is not “coerced.” However, the “consent” argument is so weak in conditions of inequality as to be useless (see sections 2.4 and 3.1, for example) and, moreover, it can and has been used to justify the state. In other words, “consent” in and of itself does not ensure that a given regime is not statist (see section 2.3 for more on this). So an argument along these lines is deeply flawed and can be used to justify regimes which are little better than “industrial feudalism” (such as, as indicated in section B.4, company towns, for example — an institution which right-libertarianism has no problem with). Even the {\em “general libertarian law code,”} could be considered a “monopoly of government over a particular area,” particularly if ordinary people have no real means of affecting the law code, either because it is market-driven and so is money-determined, or because it will be “natural” law and so unchangeable by mere mortals.


In other words, {\bf if} the state {\em “arrogates to itself a monopoly of force, of ultimate decision-making power, over a given area territorial area”} [Rothbard, {\bf The Ethics of Liberty}, p. 170] then its pretty clear that the property owner shares this power. The owner is, after all, the {\em “ultimate decision-making power”} in their workplace or on their land. If the boss takes a dislike to you (for example, you do not follow their orders) then you get fired. If you cannot get a job or rent the land without agreeing to certain conditions (such as not joining a union or subscribing to the “defence firm” approved by your employer) then you either sign the contract or look for something else. Of course Rothbard fails to note that bosses have this monopoly of power and is instead referring to {\em “prohibiting the voluntary purchase and sale of defence and judicial services.”} [{\bf Op. Cit.}, p. 171] But just as surely as the law of contract allows the banning of unions from a property, it can just as surely ban the sale and purchase of defence and judicial services (it could be argued that market forces will stop this happening, but this is unlikely as bosses usually have the advantage on the labour market and workers have to compromise to get a job — see section 10.2 on why this is the case). After all, in the company towns, only company money was legal tender and company police the only law enforcers.


Therefore, it is obvious that the “anarcho”-capitalist system meets the Weberian criteria of a monopoly to enforce certain rules in a given area of land. The {\em “general libertarian law code”} is a monopoly and property owners determine the rules that apply to their property. Moreover, if the rules that property owners enforce are subject to rules contained in the monopolistic {\em “general libertarian law code”} (for example, that they cannot ban the sale and purchase of certain products — such as defence — on their own territory) then “anarcho”-capitalism {\bf definitely} meets the Weberian definition of the state (as described by Ayn Rand as an institution {\em “that holds the exclusive power to {\bf enforce} certain rules of conduct in a given geographical area”} [{\bf Capitalism: The Unknown Ideal}, p. 239]) as its “law code” overrides the desires of property owners to do what they like on their own property.


Therefore, no matter how you look at it, “anarcho”-capitalism and its “defence” market promotes a {\em “monopoly of ultimate decision making power”} over a {\em “given territorial area”}. It is obvious that for anarchists, the “anarcho”-capitalist system is a state system. As, as we note, a reasonable case can be made for it also being a state in “anarcho”-capitalist theory as well.


So, in effect, “anarcho”-capitalism has a {\bf different} sort of state, one in which bosses hire and fire the policeman. As Peter Sabatini notes [in {\bf Libertarianism: Bogus Anarchy}], {\em “[w]ithin Libertarianism, Rothbard represents a minority perspective that actually argues for the total elimination of the state. However Rothbard’s claim as an anarchist is quickly voided when it is shown that he only wants an end to the public state. In its place he allows countless private states, with each person supplying their own police force, army, and law, or else purchasing these services from capitalist vendors\unknown{} Rothbard sees nothing at all wrong with the amassing of wealth, therefore those with more capital will inevitably have greater coercive force at their disposal, just as they do now.”}


Far from wanting to abolish the state, then, “anarcho”-capitalists only desire to privatise it — to make it solely accountable to capitalist wealth. Their “companies” perform the same services as the state, for the same people, in the same manner. However, there is one slight difference. Property owners would be able to select between competing companies for their “services.” Because such “companies” are employed by the boss, they would be used to reinforce the totalitarian nature of capitalist firms by ensuring that the police and the law they enforce are not even slightly accountable to ordinary people.


Looking beyond the “defence association” to the defence market itself (as we argued in the last section), this will become a cartel and so become some kind of public state. The very nature of the private state, its need to co-operate with others in the same industry, push it towards a monopoly network of firms and so a monopoly of force over a given area. Given the assumptions used to defend “anarcho”-capitalism, its system of private statism will develop into public statism — a state run by managers accountable only to the share-holding elite.


To quote Peter Marshall again, the “anarcho”-capitalists {\em “claim that all would benefit from a free exchange on the market, it is by no means certain; any unfettered market system would most likely sponsor a reversion to an unequal society with defence associations perpetuating exploitation and privilege.”} [{\bf Demanding the Impossible}, p. 565] History, and current practice, prove this point.


In short, “anarcho”-capitalists are not anarchists at all, they are just capitalists who desire to see private states develop — states which are strictly accountable to their paymasters without even the sham of democracy we have today. Hence a far better name for “anarcho”-capitalism would be “private-state” capitalism. At least that way we get a fairer idea of what they are trying to sell us. As Bob Black writes in {\bf The Libertarian as Conservative}, {\em “To my mind a right-wing anarchist is just a minarchist who’d abolish the state to his own satisfaction by calling it something else\unknown{} They don’t denounce what the state does, they just object to who’s doing it.”}


\subsection{6.5 What other effects would “free market” justice have?
}

Such a system would be dangerous simply because of the power it places in the hands of companies. As Michael Taylor notes, {\em “whether the [protection] market is competitive or not, it must be remembered that the product is a peculiar one: when we buy cars or shoes or telephone services we do not give the firm power based on force, but armed protection agencies, like the state, make customers (their own and others’) vulnerable, and having given them power we cannot be sure that they will use it only for our protection.”} [{\bf Community, Anarchy and Liberty}, p. 65]


As we argued above, there are many reasons to believe that a “protection” market will place most of society (bar the wealthy elite) in a “vulnerable” position. One such reason is the assumptions of the “anarcho”-capitalists themselves. As they note, capitalism is marked by an extreme division of labour. Instead of everyone having all the skills they need, these skills are distributed throughout society and all (so it is claimed) benefit.


This applies equally to the “defence” market. People subscribe to a “defence firm” because they either cannot or do not want the labour of having to protect their own property and person. The skills of defence, therefore, are concentrated in these companies and so these firms will have an advantage in terms of experience and mental state (they are trained to fight) as well as, as seems likely, weaponry. This means that most normal people will be somewhat at a disadvantage if a cartel of defence firms decides to act coercively. The division of labour society will discourage the spread of skills required for sustained warfare throughout society and so, perhaps, ensure that customers remain “vulnerable.” The price of liberty may be eternal vigilance, but are most people willing to include eternal preparation of war as well? For modern society, the answer seems to be no, they prefer to let others do that (namely the state and its armed forces). And, we should note, an armed society may be a polite one, but its politeness comes from fear, {\bf not} mutual respect and so totally phoney and soul destroying.


If we look at inequality within society, this may produce a ghettoisation effect within “anarcho”-capitalism. As David Friedman notes, conflict between defence firms is bad for business. Conflict costs money both in terms of weaponry used and increased (“danger money”) wages. For this reason he thinks that peaceful co-operation will exist between firms. However, if we look at poor areas with high crime rates then its clear that such an area will be a dangerous place. In other words, it is very likely to be high in conflict. But conflict increases costs, and so prices. Does this mean that those areas which need police most will also have the highest prices for law enforcement? That is the case with insurance now, so perhaps we will see whole areas turning into Hobbesian anarchy simply because the high costs associated with dangerous areas will make the effective demand for their services approach zero.


In a system based on “private statism,” police and justice would be determined by “free market” forces. As indicated in section B.4.1, right-libertarians maintain that one would have few rights on other peoples’ property, and so the owner’s will would be the law (possibly restricted somewhat by a “general libertarian law code”, perhaps not — see last section). In this situation, those who could not afford police protection would become victims of roving bandits and rampant crime, resulting in a society where the wealthy are securely protected in their bastions by their own armed forces, with a bunch of poor crowded around them for protection. This would be very similar to feudal Europe.


The competing police forces would also be attempting to execute the laws of their sponsors in areas that may not be theirs to begin with, which would lead to conflicts unless everyone agreed to follow a “general libertarian law code” (as Rothbard, for one, wants). If there were competing law codes, the problem of whose “laws” to select and enforce would arise, with each of the wealthy security sponsors desiring that their law control all of the land. And, as noted earlier, if there were {\bf one} “libertarian law code,” this would be a {\em “monopoly of government”} over a given area, and therefore statist.


In addition, it should be noted that the right-libertarian claim that under their system anarchistic associations would be allowed as long as they are formed voluntarily just reflects their usual vacuous concept of freedom. This is because such associations would exist within and be subject to the “general libertarian law code” of “anarcho”-capitalist society. These laws would reflect and protect the interests and power of those with capitalist property, meaning that unless these owners agree, trying to live an anarchist life would be nearly impossible (its all fine and well to say that those with property can do what they like, if you do not have property then experimentation could prove difficult — not to mention, of course, few areas are completely self-sufficient meaning that anarchistic associations will be subject to market forces, market forces which stress and reward the opposite of the values these communes were set up to create). Thus we must {\bf buy} the right to be free!


If, as anarchists desire, most people refuse to recognise or defend the rights of private property and freely associate accordingly to organise their own lives and ignore their bosses, this would still be classed as “initiation of force” under “anarcho”-capitalism, and thus repressed. In other words, like any authoritarian system, the “rules” within “anarcho”-capitalism do not evolve with society and its changing concepts (this can be seen from the popularity of “natural law” with right-libertarians, the authoritarian nature of which is discussed in section 11).


Therefore, in “anarcho”-capitalism you are free to follow the (capitalist) laws and to act within the limits of these laws. It is only within this context that you can experiment (if you can afford to). If you act outside these laws, then you will be subject to coercion. The amount of coercion required to prevent such actions depends on how willing people are to respect the laws. Hence it is not the case that an “anarcho”-capitalist society is particularly conducive to social experimentation and free evolution, as its advocates like to claim. Indeed, the opposite may be the case, as any capitalist system will have vast differences of wealth and power within it, thus ensuring that the ability to experiment is limited to those who can afford it. As Jonathan Wolff points out, the {\em “image of people freely moving from one utopia to another until they find their heaven, ignores the thought that certain choices may be irreversible\unknown{} This thought may lead to speculation about whether a law of evolution would apply to the plural utopias. Perhaps, in the long run, we may find the framework regulated by the law of survival of the economically most fit, and so we would expect to see a development not of diversity but of homogeneity. Those communities with great market power would eventually soak up all but the most resistant of those communities around them.”} [{\bf Robert Nozick: Property, Justice and the Minimal State}, p. 135]


And if the initial distribution of resources is similar to that already existing then the {\em “economically most fit”} will be capitalistic (as argued in section J.5.12, the capitalist market actively selects against co-operatives even though they are more productive). Given the head start provided by statism, it seems likely that explicitly capitalist utopia’s would remain the dominant type (particularly as the rights framework is such as to protect capitalist property rights). Moreover, we doubt that most “anarcho”-capitalists would embrace the ideology if it was more than likely that non-capitalist utopias would overcome the capitalist ones (after all, they {\bf are} self-proclaimed capitalists).


So, given that “anarcho”-capitalists who follow Murray Rothbard’s ideas and minimal-statist right-libertarians agree that {\bf all} must follow the basic {\em “general libertarian law code”} which defends capitalist property rights, we can safely say that the economically {\em “most fit”} would be capitalist ones. Hardly surprising if the law code reflects capitalist ideas of right and wrong. In addition, as George Reitzer has argued (see {\bf The McDonaldization of Society}), capitalism is driven towards standardisation and conformity by its own logic. This suggests that plurality of communities would soon be replaced by a series of “communities” which share the same features of hierarchy and ruling elites. (“Anarcho”-capitalists who follow David Friedman’s ideas consider it possible, perhaps likely, that a free market in laws will result in one standard law code and so this also applies to that school as well)


So, in the end, the “anarcho” capitalists argue that in their system you are free to follow the (capitalist) law and work in the (capitalist) economy, and if you are lucky, take part in a “commune” as a collective capitalist. How {\bf very} generous of them! Of course, any attempt to change said rules or economy are illegal and would be stopped by private states.


As well as indicating the falsity of “anarcho”-capitalist claims to support “experimentation,” this discussion has also indicated that coercion would not be absent from “anarcho”-capitalism. This would be the case only if everyone voluntarily respected private property rights and abided by the law (i.e. acted in a capitalist-approved way). As long as you follow the law, you will be fine — which is exactly the same as under public statism. Moreover, if the citizens of a society do not want a capitalist order, it may require a lot of coercion to impose it. This can be seen from the experiences of the Italian factory occupations in 1920 (see section A.5.5), in which workers refused to accept capitalist property or authority as valid and ignored it. In response to this change of thought within a large part of society, the capitalists backed fascism in order to stop the evolutionary process within society.


The socialist economic historian Maurice Dobbs, after reviewing the private armies in 1920s and 1930s America made much the same point:



\startblockquote
{\em “When business policy takes the step of financing and arming a mass political movement to capture the machinery of government, to outlaw opposing forms of organisation and suppress hostile opinions we have merely a further and more logical stage beyond [private armies]”} [{\bf Op, Cit.}, p. 357]



\stopblockquote
(Noted Austrian Economist Ludwig von Mises whose extreme free market liberal political and economic ideas inspired right-libertarianism in many ways had this to say about fascism: {\em “It cannot be denied that Fascism and similar movements aiming at the establishment of dictatorships are full of the best intentions and that their intervention has, for the moment, saved European civilisation. The merit that Fascism has thereby won for itself will live eternally in history.”} [{\bf Liberalism}, p. 51])


This example illustrates the fact that capitalism {\bf per se} is essentially authoritarian, because it is necessarily based on coercion and hierarchy, which explains why capitalists have resorted to the most extreme forms of authoritarianism — including totalitarian dictatorship — during crises that threatened the fundamental rules of the system itself. There is no reason to think that “anarcho”-capitalism would be any different.


Since “anarcho”-capitalism, with its private states, does not actually want to get rid of hierarchical forms of authority, the need for one government to unify the enforcement activities of the various defence companies becomes apparent. In the end, that is what “anarcho”-capitalism recognises with its “general libertarian law code” (based either on market forces or “natural law”). Thus it appears that one government/hierarchy over a given territory is inevitable under any form of capitalism. That being the case, it is obvious that a democratic form of statism, with its checks and balances, is preferable to a dictatorship that imposes “absolute” property rights and so “absolute” power.


Of course, we do have another option than either private or public statism. This is anarchism, the end of hierarchical authority and its replacement by the “natural” authority of communal and workplace self-management.


\section{7 How does the history of “anarcho”-capitalism show that it is not anarchist?
}

Of course, “anarcho”-capitalism does have historic precedents and “anarcho”-capitalists spend considerable time trying to co-opt various individuals into their self-proclaimed tradition of “anti-statist” liberalism. That, in itself, should be enough to show that anarchism and “anarcho”-capitalism have little in common as anarchism developed in opposition to liberalism and its defence of capitalism. Unsurprisingly, these “anti-state” liberals tended to, at best, refuse to call themselves anarchists or, at worse, explicitly deny they were anarchists.


One “anarcho”-capitalist overview of their tradition is presented by David M. Hart. His perspective on anarchism is typical of the school, noting that in his essay anarchism or anarchist {\em “are used in the sense of a political theory which advocates the maximum amount of individual liberty, a necessary condition of which is the elimination of governmental or other organised force.”} [David M. Hart, {\em “Gustave de Molinari and the Anti-statist Liberal Tradition: Part I”}, pp. 263–290, {\bf Journal of Libertarian Studies}, vol. V, no. 3, p. 284] Yet anarchism has {\bf never} been solely concerned with abolishing the state. Rather, anarchists have always raised economic and social demands and goals along with their opposition to the state. As such, anti-statism may be a necessary condition to be an anarchist, but not a sufficient one to count a specific individual or theory as anarchist.


Specifically, anarchists have turned their analysis onto private property noting that the hierarchical social relationships created by inequality of wealth (for example, wage labour) restricts individual freedom. This means that if we do seek {\em “the maximum of individual liberty”} then our analysis cannot be limited to just the state or government. Consequently, to limit anarchism as Hart does requires substantial rewriting of history, as can be seen from his account of William Godwin.


Hart tries to co-opt of William Godwin into the ranks of “anti-state” liberalism, arguing that he {\em “defended individualism and the right to property.”} [{\bf Op. Cit.}, p. 265] He, of course, quotes from Godwin to support his claim yet strangely truncates Godwin’s argument to exclude his conclusion that {\em “[w]hen the laws of morality shall be clearly understood, their excellence universally apprehended, and themselves seen to be coincident with each man’s private advantage, the idea of property in this sense will remain, but no man will have the least desire, for purposes of ostentation or luxury, to possess more than his neighbours.”} [{\bf An Enquiry into Political Justice}, p. 199] In other words, personal property (possession) would still exist but not private property in the sense of capital or inequality of wealth.


This analysis is confirmed in book 8 of Godwin’s classic work entitled {\em {\bf “On Property.”}} Needless to say, Hart fails to mention this analysis, unsurprising as it was later reprinted as a socialist pamphlet. Godwin thought that the {\em “subject of property is the key-stone that completes the fabric of political justice.”} Like Proudhon, Godwin subjects property as well as the state to an anarchist analysis. For Godwin, there were {\em “three degrees”} of property. The first is possession of things you need to live. The second is {\em “the empire to which every man is entitled over the produce of his own industry.”} The third is {\em “that which occupies the most vigilant attention in the civilised states of Europe. It is a system, in whatever manner established, by which one man enters into the faculty of disposing of the produce of another man’s industry.”} He notes that it is {\em “clear therefore that the third species of property is in direct contradiction to the second.”} [{\bf Op. Cit.}, p. 701 and p. 710–2]


Godwin, unlike classical liberals, saw the need to {\em “point out the evils of accumulated property,”} arguing that the the {\em “spirit of oppression, the spirit of servility, and the spirit of fraud \unknown{} are the immediate growth of the established administration of property. They are alike hostile to intellectual and moral improvement.”} Like the socialists he inspired, Godwin argued that {\em “it is to be considered that this injustice, the unequal distribution of property, the grasping and selfish spirit of individuals, is to be regarded as one of the original sources of government, and, as it rises in its excesses, is continually demanding and necessitating new injustice, new penalties and new slavery.”} He stressed, {\em “let it never be forgotten that accumulated property is usurpation.”} [{\bf Op. Cit.}, p. 732, pp. 717–8, and p. 718]


Godwin argued against the current system of property and in favour of {\em “the justice of an equal distribution of the good things of life.”} This would be based on {\em “[e]quality of conditions, or, in other words, an equal admission to the means of improvement and pleasure”} as this {\em “is a law rigorously enjoined upon mankind by the voice of justice.”} [{\bf Op. Cit.}, p. 725 and p. 736] Thus his anarchist ideas were applied to private property, noting like subsequent anarchists that economic inequality resulted in the loss of liberty for the many and, consequently, an anarchist society would see a radical change in property and property rights. As Kropotkin noted, Godwin {\em “stated in 1793 in a quite definite form the political and economic principle of Anarchism.”} Little wonder he, like so many others, argued that Godwin was {\em “the first theoriser of Socialism without government — that is to say, of Anarchism.”} [{\bf Environment and Evolution}, p. 62 and p. 26] For Kropotkin, anarchism was by definition not restricted to purely political issues but also attacked economic hierarchy, inequality and injustice. As Peter Marshall confirms, {\em “Godwin’s economics, like his politics, are an extension of his ethics.”} [{\bf Demanding the Impossible}, p. 210]


Godwin’s theory of property is significant because it reflected what was to become standard nineteenth century socialist thought on the matter. In Britain, his ideas influenced Robert Owen and, as a result, the early socialist movement in that country. His analysis of property, as noted, predated Proudhon’s classic anarchist analysis. As such, to state, as Hart did, that Godwin simply {\em “concluded that the state was an evil which had to be reduced in power if not eliminated completely”} while not noting his analysis of property gives a radically false presentation of his ideas. [Hart, {\bf Op. Cit.}, p. 265] However, it does fit into his flawed assertion that anarchism is purely concerned with the state. Any evidence to the contrary is simply ignored.


\subsection{7.1 Are competing governments anarchism?
}

No, of course not. Yet according to “anarcho”-capitalism, it is. This can be seen from the ideas of Gustave de Molinari.


Hart is on firmer ground when he argues that the 19\high{th} century French economist Gustave de Molinari is the true founder of “anarcho”-capitalism. With Molinari, he argues, {\em “the two different currents of anarchist thought converged: he combined the political anarchism of Burke and Godwin with the nascent economic anarchism of Adam Smith and Say to create a new forms of anarchism”} that has been called {\em “anarcho-capitalism, or free market anarchism.”} [{\bf Op. Cit.}, p. 269] Of course, Godwin (like other anarchists) did not limit his anarchism purely to “political” issues and so he discussed {\em “economic anarchism”} as well in his critique of private property (as Proudhon also did later). As such, to artificially split anarchism into political and economic spheres is both historically and logically flawed. While some dictionaries limit “anarchism” to opposition to the state, anarchists did and do not.


The key problem for Hart is that Molinari refused to call himself an anarchist. He did not even oppose government, as Hart himself notes Molinari proposed a system of insurance companies to provide defence of property and {\em “called these insurance companies ‘governments’ even though they did not have a monopoly within a given geographical area.”} As Hart notes, Molinari was the sole defender of such free-market justice at the time in France. [David M. Hart, {\em “Gustave de Molinari and the Anti-statist Liberal Tradition: Part II”}, pp. 399–434,{\bf Journal of Libertarian Studies}, vol. V, no. 4, p. 415 and p. 411] Molinari was clear that he wanted {\em “a regime of free government,”} counterpoising {\em “monopolist or communist governments”} to {\em “free governments.”} This would lead to {\em “freedom of government”} rather than its abolition (not freedom {\bf from} government). For Molinarie the future would not bring {\em “the suppression of the state which is the dream of the anarchists \unknown{} It will bring the diffusion of the state within society. That is \unknown{} ‘a free state in a free society.’”} [quoted by Hart, {\bf Op. Cit.}, p. 429, p. 411 and p. 422] As such, Molinari can hardly be considered an anarchist, even if “anarchist” is limited to purely being against government.


Moreover, in another sense Molinari was in favour of the state. As we discuss in section 6, these companies would have a monopoly within a given geographical area — they have to in order to enforce the property owner’s power over those who use, but do not own, the property in question. The key contradiction can be seen in Molinari’s advocating of company towns, privately owned communities (his term was a {\em “proprietary company”}). Instead of taxes, people would pay rent and the {\em “administration of the community would be either left in the hands of the company itself or handled special organisations set up for this purpose.”} Within such a regime {\em “those with the most property had proportionally the greater say in matters which affected the community.”} If the poor objected then they could simply leave. [{\bf Op. Cit.}, pp. 421–2 and p. 422]


Given this, the idea that Molinari was an anarchist in any form can be dismissed. His system was based on privatising government, not abolishing it (as he himself admitted). This would be different from the current system, of course, as landlords and capitalists would be hiring force directly to enforce their decisions rather than relying on a state which they control indirectly. This system, as we proved in section 6, would not be anarchist as can be seen from American history. There capitalists and landlords created their own private police forces and armies, which regularly attacked and murdered union organisers and strikers. As an example, there is Henry Ford’s Service Department (private police force):



\startblockquote
{\em “In 1932 a hunger march of the unemployed was planned to march up to the gates of the Ford plant at Dearborn\unknown{} The machine guns of the Dearborn police and the Ford Motor Company’s Service Department killed [four] and wounded over a score of others\unknown{} Ford was fundamentally and entirely opposed to trade unions. The idea of working men questioning his prerogatives as an owner was outrageous \unknown{} [T]he River Rouge plant\unknown{} was dominated by the autocratic regime of Bennett’s service men. Bennett . . organise[d] and train[ed] the three and a half thousand private policemen employed by Ford. His task was to maintain discipline amongst the work force, protect Ford’s property [and power], and prevent unionisation\unknown{} Frank Murphy, the mayor of Detroit, claimed that ‘Henry Ford employs some of the worst gangsters in our city.’ The claim was well based. Ford’s Service Department policed the gates of his plants, infiltrated emergent groups of union activists, posed as workers to spy on men on the line\unknown{} Under this tyranny the Ford worker had no security, no rights. So much so that any information about the state of things within the plant could only be freely obtained from ex-Ford workers.”} [Huw Beynon, {\bf Working for Ford}, pp. 29–30]



\stopblockquote
The private police attacked women workers handing out pro-union handbills and gave them {\em “a severe beating.”} At Kansas and Dallas {\em “similar beatings were handed out to the union men.”} This use of private police to control the work force was not unique. General Motors {\em “spent one million dollars on espionage, employing fourteen detective agencies and two hundred spies at one time [between 1933 and 1936]. The Pinkerton Detective Agency found anti-unionism its most lucrative activity.”} [{\bf Op. Cit.}, p. 34 and p. 32] We must also note that the Pinkerton’s had been selling their private police services for decades before the 1930s. For over 60 years the Pinkerton Detective Agency had {\em “specialised in providing spies, agent provocateurs, and private armed forces for employers combating labour organisations.”} By 1892 it {\em “had provided its services for management in seventy major labour disputes, and its 2 000 active agents and 30 000 reserves totalled more than the standing army of the nation.”} [Jeremy Brecher, {\bf Strike!}, p. 55] With this force available, little wonder unions found it so hard to survive in the USA.


Only an “anarcho”-capitalist would deny that this is a private government, employing private police to enforce private power. Given that unions could be considered as “defence” agencies for workers, this suggests a picture of how “anarcho”-capitalism may work in practice radically different from the pictures painted by its advocates. The reason is simple, it does not ignore inequality and subjects economics to an anarchist analysis. Little wonder, then, that Proudhon stressed that it {\em “becomes necessary for the workers to form themselves into democratic societies, with equal conditions for all members, on pain of a relapse into feudalism.”} Anarchism, in other words, would see {\em “[c]apitalistic and proprietary exploitation stopped everywhere, the wage system abolished”} and so {\em “the economic organisation [would] replac[e] the governmental and military system.”} [{\bf The General Idea of the Revolution}, p. 227 and p. 281] Clearly, the idea that Proudhon shared the same political goal as Molinari is a joke. He would have dismissed such a system as little more than an updated form of feudalism in which the property owner is sovereign and the workers subjects (see section B.4 for more details).


Unsurprisingly, Molinari (unlike the individualist anarchists) attacked the jury system, arguing that its obliged people to {\em “perform the duties of judges. This is pure communism.”} People would {\em “judge according to the colour of their opinions, than according to justice.”} [quoted by Hart, {\bf Op. Cit.}, p. 409] As the jury system used amateurs (i.e. ordinary people) rather than full-time professionals it could not be relied upon to defend the power and property rights of the rich. As we noted in section 1.4, Rothbard criticised the individualist anarchists for supporting juries for essentially the same reasons.


But, as is clear from Hart’s account, Molinari had little concern that working class people should have a say in their own lives beyond consuming goods. His perspective can be seen from his lament about those {\em “colonies where slavery has been abolished without the compulsory labour being replaced with an equivalent quantity of free [sic!] labour [i.e., wage labour], there has occurred the opposite of what happens everyday before our eyes. Simple workers have been seen to exploit in their turn the industrial {\bf entrepreneurs,} demanding from them wages which bear absolutely no relation to the legitimate share in the product which they ought to receive. The planters were unable to obtain for their sugar a sufficient price to cover the increase in wages, and were obliged to furnish the extra amount, at first out of their profits, and then out of their very capital. A considerable number of planters have been ruined as a result \unknown{} It is doubtless better that these accumulations of capital should be destroyed than that generations of men should perish [Marx: ‘how generous of M. Molinari’] but would it not be better if both survived?”} [quoted by Karl Marx, {\bf Capital}, vol. 1, p. 937f]


So workers exploiting capital is the {\em “opposite of what happens everyday before our eyes”}? In other words, it is normal that entrepreneurs {\em “exploit”} workers under capitalism? Similarly, what is a {\em “legitimate share”} which workers {\em “ought to receive”}? Surely that is determined by the eternal laws of supply and demand and not what the capitalists (or Molinari) thinks is right? And those poor former slave drivers, they really do deserve our sympathy. What horrors they face from the impositions subjected upon them by their ex-chattels — they had to reduce their profits! How dare their ex-slaves refuse to obey them in return for what their ex-owners think was their {\em “legitimate share in the produce”}! How {\em “simple”} these workers are, not understanding the sacrifices their former masters suffer nor appreciating how much more difficult it is for their ex-masters to create {\em “the product”} without the whip and the branding iron to aid them! As Marx so rightly comments: {\em “And what, if you please, is this ‘legitimate share’, which, according to [Molinari’s] own admission, the capitalist in Europe daily neglects to pay? Over yonder, in the colonies, where the workers are so ‘simple’ as to ‘exploit’ the capitalist, M. Molinari feels a powerful itch to use police methods to set on the right road that law of supply and demand which works automatically everywhere else.”} [{\bf Op. Cit.}, p. 937f]


An added difficulty in arguing that Molinari was an anarchist is that he was a contemporary of Proudhon, the first self-declared anarchist, and lived in a country with a vigorous anarchist movement. Surely if he was really an anarchist, he would have proclaimed his kinship with Proudhon and joined in the wider movement. He did not, as Hart notes as regards Proudhon:



\startblockquote
{\em “their differences in economic theory were considerable, and it is probably for this reason that Molinari refused to call himself an anarchist in spite of their many similarities in political theory. Molinari refused to accept the socialist economic ideas of Proudhon \unknown{} in Molinari’s mind, the term ‘anarchist’ was intimately linked with socialist and statist economic views.”} [{\bf Op. Cit.}, p. 415]



\stopblockquote
Yet Proudhon’s economic views, like Godwin’s, flowed from his anarchist analysis and principles. They cannot be arbitrarily separated as Hart suggests. So while arguing that {\em “Molinari was just as much an anarchist as Proudhon,”} Hart forgets the key issue. Proudhon was aware that private property ensured that the proletarian did not exercise {\em “self-government”} during working hours, i.e. was not a self-governing individual. As for Hart claiming that Proudhon had {\em “statist economic views”} it simply shows how far an “anarcho”-capitalist perspective is from genuine anarchism. Proudhon’s economic analysis, his critique of private property and capitalism, flowed from his anarchism and was an integral aspect of it.


To restrict anarchism purely to opposition to the state, Hart is impoverishing anarchist theory and denying its history. Given that anarchism was born from a critique of private property as well as government, this shows the false nature of Hart’s claim that {\em “Molinari was the first to develop a theory of free-market, proprietary anarchism that extended the laws of the market and a rigorous defence of property to its logical extreme.”} [{\bf Op. Cit.}, p. 415 and p. 416] Hart shows how far from anarchism Molinari was as Proudhon had turned his anarchist analysis to property, showing that {\em “defence of property”} lead to the oppression of the many by the few in social relationships identical to those which mark the state. Moreover, Proudhon, argued the state would always be required to defend such social relations. Privatising it would hardly be a step forward.


Unsurprisingly, Proudhon dismissed the idea that the laissez faire capitalists shared his goals. {\em “The school of Say,”} Proudhon argued, was {\em “the chief focus of counter-revolution next to the Jesuits”} and {\em “has for ten years past seemed to exist only to protect and applaud the execrable work of the monopolists of money and necessities, deepening more and more the obscurity of a science naturally difficult and full of complications.”} Much the same can be said of “anarcho”- capitalists, incidentally. For Proudhon, {\em “the disciples of Malthus and of Say, who oppose with all their might any intervention of the State in matters commercial or industrial, do not fail to avail themselves of this seemingly liberal attitude, and to show themselves more revolutionary than the Revolution. More than one honest searcher has been deceived thereby.”} However, this apparent “anti-statist” attitude of supporters of capitalism is false as pure free market capitalism cannot solve the social question, which arises because of capitalism itself. As such, it was impossible to abolish the state under capitalism. Thus {\em “this inaction of Power in economic matters was the foundation of government. What need should we have of a political organisation, if Power once permitted us to enjoy economic order?”} Instead of capitalism, Proudhon advocated the {\em “constitution of Value,”} the {\em “organisation of credit,”} the elimination of interest, the {\em “establishment of workingmen’s associations”} and {\em “the use of a just price.”} [{\bf The General Idea of the Revolution}, p. 225, p. 226 and p. 233]


Clearly, then, the claims that Molinari was an anarchist fail as he, unlike his followers, were aware of what anarchism actually stood for. Hart, in his own way, acknowledges this:



\startblockquote
{\em “In spite of his protestations to the contrary, Molinari should be considered an anarchist thinker. His attack on the state’s monopoly of defence must surely warrant the description of anarchism. His reluctance to accept this label stemmed from the fact that the socialists had used it first to describe a form of non-statist society which Molinari definitely opposed. Like many original thinkers, Molinari had to use the concepts developed by others to describe his theories. In his case, he had come to the same political conclusions as the communist anarchists although he had been working within the liberal tradition, and it is therefore not surprising that the terms used by the two schools were not compatible. It would not be until the latter half of the twentieth century that radical, free-trade liberals would use the word ‘anarchist’ to describe their beliefs.”} [{\bf Op. Cit.}, p. 416]



\stopblockquote
It should be noted that Proudhon was {\bf not} a communist-anarchist, but the point remains. The aims of anarchism were recognised by Molinari as being inconsistent with his ideology. Consequently, he (rightly) refused the label. If only his self-proclaimed followers in the {\em “latter half of the twentieth century”} did the same anarchists would not have to bother with them!


As such, it seems ironic that the founder of “anarcho”-capitalism should have come to the same conclusion as modern day anarchists on the subject of whether his ideas are a form of anarchism or not!


\subsection{7.2 Is government compatible with anarchism?
}

Of course not, but ironically this is the conclusion arrived at by Hart’s analyst of the British “voluntaryists,” particularly Auberon Herbert. Voluntaryism was a fringe part of the right-wing individualist movement inspired by Herbert Spencer, a spokesman for free market capitalism in the later half of the nineteenth century. As with Molinari, there is a problem with presenting this ideology as anarchist, namely that its leading light, Herbert, explicitly rejected the label “anarchist.”


Herbert was clearly aware of individualist anarchism and distanced himself from it. He argued that such a system would be {\em “pandemonium.”} He thought that people should {\em “not direct our attacks — as the anarchists do — {\bf against all government} , against government in itself”} but {\em “only against the overgrown, the exaggerated, the insolent, unreasonable and indefensible forms of government, which are found everywhere today.”} Government should be {\em “strictly limited to its legitimate duties in defence of self-ownership and individual rights.”} He stressed that {\em “we are governmentalists \unknown{} formally constituted by the nation, employing in this matter of force the majority method.”} Moreover, Herbert knew of, and rejected, individualist anarchism, considering it to be {\em “founded on a fatal mistake.”} [{\bf Essay X: The Principles Of Voluntaryism And Free Life}] As such, claims that he was an anarchist or “anarcho”-capitalist cannot be justified.


Hart is aware of this slight problem, quoting Herbert’s claim that he aimed for {\em “regularly constituted government, generally accepted by all citizens for the protection of the individual.”} [quoted by Hart, {\bf Op. Cit.}, p. 86] Like Molinari, Herbert was aware that anarchism was a form of socialism and that the political aims could not be artificially separated from its economic and social aims. As such, he was right {\bf not} to call his ideas anarchism as it would result in confusion (particularly as anarchism was a much larger movement than his). As Hart acknowledges, {\em “Herbert faced the same problems that Molinari had with labelling his philosophy. Like Molinari, he rejected the term ‘anarchism,’ which he associated with the socialism of Proudhon and \unknown{} terrorism.”} While {\em “quite tolerant”} of individualist anarchism, he thought they {\em “were mistaken in their rejections of ‘government.’”} However, Hart knows better than Herbert about his own ideas, arguing that his ideology {\em “is in fact a new form of anarchism, since the most important aspect of the modern state, the monopoly of the use of force in a given area, is rejected in no uncertain terms by both men.”} [{\bf Op. Cit.}, p. 86] He does mention that Benjamin Tucker called Herbert a {\em “true anarchist in everything but name,”} but Tucker denied that Kropotkin was an anarchist suggesting that he was hardly a reliable guide. [quoted by Hart, {\bf Op. Cit.}, p. 87] As it stands, it seems that Tucker was mistaken in his evaluation of Herbert’s politics.


Economically, Herbert was not an anarchist, arguing that the state should protect Lockean property rights. Of course, Hart may argue that these economic differences are not relevant to the issue of Herbert’s anarchism but that is simply to repeat the claim that anarchism is simply concerned with government, a claim which is hard to support. This position cannot be maintained, given that both Herbert and Molinari defended the right of capitalists and landlords to force their employees and tenants to follow their orders. Their “governments” existed to defend the capitalist from rebellious workers, to break unions, strikes and occupations. In other words, they were a monopoly of the use of force in a given area to enforce the monopoly of power in a given area (namely, the wishes of the property owner). While they may have argued that this was “defence of liberty,” in reality it is defence of power and authority.


What about if we just look at the political aspects of his ideas? Did Herbert actually advocate anarchism? No, far from it. He clearly demanded a minimal state based on voluntary taxation. The state would not use force of any kind, {\em “except for purposes of restraining force.”} He argued that in his system, while {\em “the state should compel no services and exact no payments by force,”} it {\em “should be free to conduct many useful undertakings \unknown{} in competition with all voluntary agencies \unknown{} in dependence on voluntary payments.”} [Herbert, {\bf Op. Cit.}] As such, {\em “the state”} would remain and unless he is using the term “state” in some highly unusual way, it is clear that he means a system where individuals live under a single elected government as their common law maker, judge and defender within a given territory.


This becomes clearer once we look at how the state would be organised. In his essay {\em “A Politician in Sight of Haven,”} Herbert does discuss the franchise, stating it would be limited to those who paid a voluntary {\em “income tax,”} anyone {\em “paying it would have the right to vote; those who did not pay it would be — as is just — without the franchise. There would be no other tax.”} The law would be strictly limited, of course, and the {\em “government \unknown{} must confine itself simply to the defense of life and property, whether as regards internal or external defense.”} In other words, Herbert was a minimal statist, with his government elected by a majority of those who choose to pay their income tax and funded by that (and by any other voluntary taxes they decided to pay). Whether individuals and companies could hire their own private police in such a regime is irrelevant in determining whether it is an anarchy.


This can be best seen by comparing Herbert with Ayn Rand. No one would ever claim Rand was an anarchist, yet her ideas were extremely similar to Herbert’s. Like Herbert, Rand supported laissez-faire capitalism and was against the “initiation of force.” Like Herbert, she extended this principle to favour a government funded by voluntary means [{\em “Government Financing in a Free Society,”} {\bf The Virtue of Selfishness}, pp. 116–20] Moreover, like Herbert, she explicitly denied being an anarchist and, again like Herbert, thought the idea of competing defence agencies (“governments”) would result in chaos. The similarities with Herbert are clear, yet no “anarcho”-capitalist would claim that Rand was an anarchist, yet they do claim that Herbert was.


This position is, of course, deeply illogical and flows from the non-anarchist nature of “anarcho”-capitalism. Perhaps unsurprisingly, when Rothbard discusses the ideas of the “voluntaryists” he fails to address the key issue of who determines the laws being enforced in society. For Rothbard, the key issue is {\bf who} is enforcing the law, not where that law comes from (as long, of course, as it is a law code he approves of). The implications of this is significant, as it implies that “anarchism” need not be opposed to either the state nor government! This can be clearly seen from Rothbard’s analysis of voluntary taxation.


Rothbard, correctly, notes that Herbert advocated voluntary taxation as the means of funding a state whose basic role was to enforce Lockean property rights. For Rothbard, the key issue was {\bf not} who determines the law but who enforces it. For Rothbard, it should be privatised police and courts and he suggests that the {\em “voluntary taxationists have never attempted to answer this problem; they have rather stubbornly assumed that no one would set up a competing defence agency within a State’s territorial limits.”} If the state {\bf did} bar such firms, then that system is not a genuine free market. However, {\em “if the government {\bf did} permit free competition in defence service, there would soon no longer be a central government over the territory. Defence agencies, police and judicial, would compete with one another in the same uncoerced manner as the producers of any other service on the market.”} [{\bf Power and Market}, p. 122 and p. 123]


However, this misses the point totally. The key issue that Rothbard ignores is who determines the laws which these private “defence” agencies would enforce. If the laws are determined by a central government, then the fact that citizen’s can hire private police and attend private courts does not stop the regime being statist. We can safely assume Rand, for example, would have had no problem with companies providing private security guards or the hiring of private detectives within the context of her minimal state. Ironically, Rothbard stresses the need for such a monopoly legal system:



\startblockquote
{\em “While ‘the government’ would cease to exist, the same cannot be said for a constitution or a rule of law, which, in fact, would take on in the free society a far more important function than at present. For the freely competing judicial agencies would have to be guided by a body of absolute law to enable them to distinguish objectively between defence and invasion. This law, embodying elaborations upon the basic injunction to defend person and property from acts of invasion, would be codified in the basic legal code. Failure to establish such a code of law would tend to break down the free market, for then defence against invasion could not be adequately achieved.”} [{\bf Op. Cit.}, p. 123–4]



\stopblockquote
So if you violate the {\em “absolute law”} defending (absolute) property rights then you would be in trouble. The problem now lies in determining who sets that law. Rothbard is silent on how his system of monopoly laws are determined or specified. The “voluntaryists” did propose a solution, namely a central government elected by the majority of those who voluntarily decided to pay an income tax. In the words of Herbert:



\startblockquote
{\em “We agree that there must be a central agency to deal with crime — an agency that defends the liberty of all men, and employs force against the uses of force; but my central agency rests upon voluntary support, whilst Mr. Levy’s central agency rests on compulsory support.”} [quoted by Carl Watner, {\em “The English Individualists As They Appear In Liberty,”} pp. 191–211, {\bf Benjamin R. Tucker and the Champions of Liberty}, p. 194]



\stopblockquote
And all Rothbard is concerned over private cops would exist or not! This lack of concern over the existence of the state and government flows from the strange fact that “anarcho”-capitalists commonly use the term “anarchism” to refer to any philosophy that opposes all forms of initiatory coercion. Notice that government does not play a part in this definition, thus Rothbard can analyse Herbert’s politics without commenting on who determines the law his private “defence” agencies enforce. For Rothbard, {\em “an anarchist society”} is defined {\em “as one where there is no legal possibility for coercive aggression against the person and property of any individual.”} He then moved onto the state, defining that as an {\em “institution which possesses one or both (almost always both) of the following properties: (1) it acquires its income by the physical coercion known as ‘taxation’; and (2) it acquires and usually obtains a coerced monopoly of the provision of defence service (police and courts) over a given territorial area.”} [{\em “Society without a State”}, in {\bf Nomos XIX}, Pennock and Chapman (eds.)., p. 192]


This is highly unusual definition of “anarchism,” given that it utterly fails to mention or define government. This, perhaps, is understandable as any attempt to define it in terms of {\em “monopoly of decision-making power”} results in showing that capitalism is statist (see section 1 for a summary). The key issue here is the term {\em “legal possibility.”} That suggestions a system of laws which determine what is {\em “coercive aggression”} and what constitutes what is and what is not legitimate “property.” Herbert is considered by “anarcho”-capitalists as one of them. Which brings us to a strange conclusion, that for “anarcho”-capitalists you can have a system of “anarchism” in which there is a government and state — as long as the state does not impose taxation nor stop private police forces from operating!


As Rothbard argues {\em “if a government based on voluntary taxation permits free competition, the result will be the purely free-market system \unknown{} The previous government would now simply be one competing defence agency among many on the market.”} [{\bf Power and Market}, p. 124] That the government is specifying what is and is not legal does not seem to bother him or even cross his mind. Why should it, when the existence of government is irrelevant to his definition of anarchism and the state? That private police are enforcing a monopoly law determined by the government seems hardly a step in the right direction nor can it be considered as anarchism. Perhaps this is unsurprising, for under his system there would be {\em “a basic, common Law Code”} which {\em “all would have to abide by”} as well as {\em “some way of resolving disputes that will gain a majority consensus in society \unknown{} whose decision will be accepted by the great majority of the public.”} [{\em “Society without a State,”} {\bf Op. Cit.}, p. 205]


At least Herbert is clear that this would be a government system, unlike Rothbard who assumes a monopoly law but seems to think that this is not a government or a state. As David Wieck argued, this is illogical for according to Rothbard {\em “all ‘would have to’ conform to the same legal code”} and this can only be achieved by means of {\em “the forceful action of adherents to the code against those who flout it”} and so {\em “in his system {\bf there would stand over against every individual the legal authority of all the others.} An individual who did not recognise private property as legitimate would surely perceive this as a tyranny of law, a tyranny of the majority or of the most powerful — in short, a hydra-headed state. If the law code is itself unitary, then this multiple state might be said to have properly a single head — the law \unknown{} But it looks as though one might still call this ‘a state,’ under Rothbard’s definition, by satisfying {\bf de facto} one of his pair of sufficient conditions: ‘It asserts and usually obtains a coerced monopoly of provision of defence service (police and courts) over a given territorial area’ \unknown{} Hobbes’s individual sovereign would seem to have become many sovereigns — with but one law, however, and in truth, therefore, a single sovereign in Hobbes’s more important sense of the latter term. One might better, and less confusingly, call this a libertarian state than an anarchy.”} [{\em “Anarchist Justice”}, in {\bf Nomos XIX}, Pennock and Chapman (eds.)., pp. 216–7]


The obvious recipients of the coercion of the new state would be those who rejected the authority of their bosses and landlords, those who reject the Lockean property rights Rothbard and Herbert hold dear. In such cases, the rebels and any “defence agency” (like, say, a union) which defended them would be driven out of business as it violated the law of the land. How this is different from a state banning competing agencies is hard to determine. This is a {\em “difficulty”} argues Wieck, which {\em “results from the attachment of a principle of private property, and of unrestricted accumulation of wealth, to the principle of individual liberty. This increases sharply the possibility that many reasonable people who respect their fellow men and women will find themselves outside the law because of dissent from a property interpretation of liberty.”} Similarly, there is the economic results of capitalism. {\em “One can imagine,”} Wieck continues, {\em “that those who lose out badly in the free competition of Rothbard’s economic system, perhaps a considerable number, might regard the legal authority as an alien power, state for them, based on violence, and might be quite unmoved by the fact that, just as under nineteenth century capitalism, a principle of liberty was the justification for it all.”} [{\bf Op. Cit.}, p. 217 and pp. 217–8]


\subsection{7.3 Can there be a “right-wing” anarchism?
}

Hart, of course, mentions the individualist anarchists, calling Tucker’s ideas {\em “{\bf laissez faire} liberalism.”} [{\bf Op. Cit.}, p. 87] However, Tucker called his ideas {\em “socialism”} and presented a left-wing critique of most aspects of liberalism, particularly its Lockean based private property rights. Tucker based much of his ideas on property on Proudhon, so if Hart dismisses the latter as a socialist then this must apply to the former. Given that he notes that there are {\em “two main kinds of anarchist thought,”} namely {\em “communist anarchism which denies the right of an individual to seek profit, charge rent or interest and to own property”} and a {\em “‘right-wing’ proprietary anarchism, which vigorously defends these rights”} then Tucker, like Godwin, would have to be placed in the {\em “left-wing”} camp. [{\em “Gustave de Molinari and the Anti-statist Liberal Tradition: Part II”}, {\bf Op. Cit.}, p. 427] Tucker, after all, argued that he aimed for the end of profit, interest and rent and attacked private property in land and housing beyond “occupancy and use.”


As can be seen, Hart’s account of the history of “anti-state” liberalism is flawed. Godwin is included only by ignoring his views on property, views which in many ways reflects the later “socialist” (i.e. anarchist) analysis of Proudhon. He then discusses a few individuals who were alone in their opinions even within extreme free market right and all of whom knew of anarchism and explicitly rejected the name for their respective ideologies. In fact, they preferred the term {\em “government”} to describe their systems which, on the face of it, would be hard to reconcile with the usual “anarcho”-capitalist definition of anarchism as being “no government.” Hart’s discussion of individualist anarchism is equally flawed, failing to discuss their economic views (just as well, as its links to “left-wing” anarchism would be obvious).


However, the similarities of Molinari’s views with what later became known as “anarcho”-capitalism are clear. Hart notes that with Molinari’s death in 1912, {\em “liberal anti-statism virtually disappeared until it was rediscovered by the economist Murray Rothbard in the late 1950’s”} [{\em “Gustave de Molinari and the Anti-statist Liberal Tradition: Part III”}, {\bf Op. Cit.}, p. 88] While this fringe is somewhat bigger than previously, the fact remains that the ideas expounded by Rothbard are just as alien to the anarchist tradition as Molinari’s. It is a shame that Rothbard, like his predecessors, did not call his ideology something other than anarchism. Not only would it have been more accurate, it would also have lead to much less confusion and no need to write this section of the FAQ! As it stands, the only reason why “anarcho”-capitalism is considered a form of “anarchism” by some is because one person (Rothbard) decided to steal the name of a well established and widespread political and social theory and movement and apply it to an ideology with little, if anything, in common with it.


As Hart inadvertently shows, it is not a firm base to build a claim. That anyone can consider “anarcho”-capitalism as anarchist simply flows from a lack of knowledge about anarchism. As numerous anarchists have argued. For example, {\em “Rothbard’s conjunction of anarchism with capitalism,”} according to David Wieck, {\em “results in a conception that is entirely outside the mainstream of anarchist theoretical writings or social movements \unknown{} this conjunction is a self-contradiction.”} He stressed that {\em “the main traditions of anarchism are entirely different. These traditions, and theoretical writings associated with them, express the perspectives and the aspirations, and also, sometimes, the rage, of the oppressed people in human society: not only those economically oppressed, although the major anarchist movements have been mainly movements of workers and peasants, but also those oppressed by power in all those social dimensions \unknown{} including of course that of political power expressed in the state.”} In other words, {\em “anarchism represents \unknown{} a moral commitment (Rothbard’s anarchism I take to be diametrically opposite).”} [{\em “Anarchist Justice”}, in {\bf Nomos XIX}, Pennock and Chapman (eds.), p. 215, p. 229 and p. 234]


It is a shame that some academics consider only the word Rothbard uses as relevant rather than the content and its relation to anarchist theory and history. If they did, they would soon realise that the expressed opposition of so many anarchists to “anarcho”-capitalism is something which cannot be ignored or dismissed. In other words, a “right-wing” anarchist cannot and does not exist, no matter how often they use that word to describe their ideology. As Bob Black put it, {\em “a right-wing anarchist is just a minarchist who’d abolish the state to his own satisfaction by calling it something else \unknown{} They don’t denounce what the state does, they just object to who’s doing it.”} [{\bf Libertarian as Conservative}]


The reason is simple. Anarchist economics and politics cannot be artificially separated, they are linked. Godwin and Proudhon did not stop their analysis at the state. They extended it the social relationships produced by inequality of wealth, i.e. economic power as well as political power. To see why, we need only consult Rothbard’s work. As noted in the last section, for Rothbard the key issue with the “voluntary taxationists” was not who determined the {\em “body of absolute law”} but rather who enforced it. In his discussion, he argued that a democratic “defence agency” is at a disadvantage in his “free market” system. As he put it:



\startblockquote
{\em “It would, in fact, be competing at a severe disadvantage, having been established on the principle of ‘democratic voting.’ Looked at as a market phenomenon, ‘democratic voting’ (one vote per person) is simply the method of the consumer ‘co-operative.’ Empirically, it has been demonstrated time and again that co-operatives cannot compete successfully against stock-owned companies, especially when both are equal before the law. There is no reason to believe that co-operatives for defence would be any more efficient. Hence, we may expect the old co-operative government to ‘wither away’ through loss of customers on the market, while joint-stock (i.e., corporate) defence agencies would become the prevailing market form.”}



\stopblockquote
Notice how he assumes that both a co-operative and corporation would be {\em “equal before the law.”} But who determines that law? Obviously {\bf not} a democratically elected government, as the idea of “one person, one vote” in determining the common law all are subject to is {\em “inefficient.”} Nor does he think, like the individualist anarchists, that the law would be judged by juries along with the facts. As we note in section 1.4, he rejects that in favour of it being determined by {\em “Libertarian lawyers and jurists.”} Thus the law is unchangeable by ordinary people and enforced by private defence agencies hired to protect the liberty and property of the owning class. In the case of a capitalist economy, this means defending the power of landlords and capitalists against rebel tenants and workers.


This means that Rothbard’s {\em “common Law Code”} will be determined, interpreted, enforced and amended by corporations based on the will of the majority of shareholders, i.e. the rich. That hardly seems likely to produce equality before the law. As he argues in a footnote:



\startblockquote
{\em “There is a strong {\bf a priori} reason for believing that corporations will be superior to co-operatives in any given situation. For if each owner receives only one vote regardless of how much money he has invested in a project (and earnings are divided in the same way), there is no incentive to invest more than the next man; in fact, every incentive is the other way. This hampering of investment militates strongly against the co-operative form.”}



\stopblockquote
So {\bf if} the law is determined by the defence agencies and courts then it will be determined by those who have invested most in these companies. As it is unlikely that the rich will invest in defence firms which do not support their property rights, power, profits and definition of property rights, it is clear that agencies which favour the wealthy will survive on the market. The idea that market demand will counter this class rule seems unlikely, given Rothbard’s own argument. After all, in order to compete successfully you need more than demand, you need source of investment. If co-operative defence agencies do form, they will be at a market disadvantage due to lack of investment. As argued in section J.5.12, even though co-operatives are more efficient than capitalist firms lack of investment (caused by the lack of control by capitalists Rothbard notes) stops them replacing wage slavery. Thus capitalist wealth and power inhibits the spread of freedom in production. If we apply his own argument to Rothbard’s system, we suggest that the market in “defence” will also stop the spread of more libertarian associations thanks to capitalist power and wealth. In other words, like any market, Rothbard’s “defence” market will simply reflect the interests of the elite, not the masses.


Moreover, we can expect any democratic defence agency (like a union) to support, say, striking workers or squatting tenants, to be crushed. This is because, as Rothbard stresses, {\bf all} “defence” firms would be expected to apply the {\em “common”} law, as written by {\em “Libertarian lawyers and jurists.”} If they did not they would quickly be labelled “outlaw” agencies and crushed by the others. Ironically, Tucker would join Bakunin and Kropotkin in an “anarchist” court accused to violating “anarchist” law by practising and advocating “occupancy and use” rather than the approved Rothbardian property rights. Even if these democratic “defence” agencies could survive and not be driven out of the market by a combination of lack of investment and violence due to their “outlaw” status, there is another problem. As we discussed in section 1, landlords and capitalists have a monopoly of decision making power over their property. As such, they can simply refuse to recognise any democratic agency as a legitimate defence association and use the same tactics perfected against unions to ensure that it does not gain a foothold in their domain (see section 6 for more details).


Clearly, then, a “right-wing” anarchism is impossible as any system based on capitalist property rights will simply be an oligarchy run by and for the wealthy. As Rothbard notes, any defence agency based on democratic principles will not survive in the “market” for defence simply because it does not allow the wealthy to control it and its decisions. Little wonder Proudhon argued that laissez-faire capitalism meant {\em “the victory of the strong over the weak, of those who own property over those who own nothing.”} [quoted by Peter Marshall, {\bf Demanding the Impossible}, p. 259]


\section{8 What role did the state take in the creation of capitalism?
}

If the “anarcho”-capitalist is to claim with any plausibility that “real” capitalism is non-statist or that it can exist without a state, it must be shown that capitalism evolved naturally, in opposition to state intervention. However, in reality, the opposite is the case. Capitalism was born from state intervention and, in the words of Kropotkin, {\em “the State \unknown{} and capitalism \unknown{} developed side by side, mutually supporting and re-enforcing each other.”} [{\bf Kropotkin’s Revolutionary Pamphlets}, p. 181]


Numerous writers have made this point. For example, in Karl Polanyi’s flawed masterpiece {\bf The Great Transformation} we read that {\em “the road to the free market was opened and kept open by an enormous increase in continuous, centrally organised and controlled interventionism”} by the state [p. 140]. This intervention took many forms — for example, state support during “mercantilism,” which allowed the “manufactures” (i.e. industry) to survive and develop, enclosures of common land, and so forth. In addition, the slave trade, the invasion and brutal conquest of the Americas and other “primitive” nations, and the looting of gold, slaves, and raw materials from abroad also enriched the European economy, giving the development of capitalism an added boost. Thus Kropotkin:



\startblockquote
{\em “The history of the genesis of capital has already been told by socialists many times. They have described how it was born of war and pillage, of slavery and serfdom, of modern fraud and exploitation. They have shown how it is nourished by the blood of the worker, and how little by little it has conquered the whole world.”} [{\bf Op. Cit.},p. 207]



\stopblockquote
Or, if Kropotkin seems too committed to be fair, we have John Stuart Mill’s statement that:



\startblockquote
{\em “The social arrangements of modern Europe commenced from a distribution of property which was the result, not of just partition, or acquisition by industry, but of conquest and violence\unknown{} “} [{\bf Principles of Political Economy}, p. 15]



\stopblockquote
Therefore, when supporters of “libertarian” capitalism say they are against the “initiation of force,” they mean only {\bf new} initiations of force; for the system they support was born from numerous initiations of force in the past. And, as can be seen from the history of the last 100 years, it also requires state intervention to keep it going (section D.1, “Why does state intervention occur?,” addresses this point in some detail). Indeed, many thinkers have argued that it was precisely this state support and coercion (particularly the separation of people from the land) that played the {\bf key} role in allowing capitalism to develop rather than the theory that {\em “previous savings”} did so. As the noted German thinker Franz Oppenheimer argued, {\em “the concept of a ‘primitive accumulation,’ or an original store of wealth, in land and in movable property, brought about by means of purely economic forces”} while {\em “seem[ing] quite plausible”} is in fact {\em “utterly mistaken; it is a ‘fairly tale,’ or it is a class theory used to justify the privileges of the upper classes.”} [{\bf The State}, pp. 5–6]


This thesis will be discussed in the following sections. It is, of course, ironic to hear right-wing libertarians sing the praises of a capitalism that never existed and urge its adoption by all nations, in spite of the historical evidence suggesting that only state intervention made capitalist economies viable — even in that Mecca of “free enterprise,” the United States. As Noam Chomsky argues, {\em “who but a lunatic could have opposed the development of a textile industry in New England in the early nineteenth century, when British textile production was so much more efficient that half the New England industrial sector would have gone bankrupt without very high protective tariffs, thus terminating industrial development in the United States? Or the high tariffs that radically undermined economic efficiency to allow the United States to develop steel and other manufacturing capacities? Or the gross distortions of the market that created modern electronics?”} [{\bf World Orders, Old and New}, p. 168]. To claim, therefore, that “mercantilism” is not capitalism makes little sense. Without mercantilism, “proper” capitalism would never have developed, and any attempt to divorce a social system from its roots is ahistoric and makes a mockery of critical thought.


Similarly, it is somewhat ironic when “anarcho”-capitalists and right libertarians claim that they support the freedom of individuals to choose how to live. After all, the working class was not given {\bf that} particular choice when capitalism was developing. Indeed, their right to choose their own way of life was constantly violated and denied. So to claim that {\bf now} (after capitalism has been created) we get the chance to try and live as we like is insulting in the extreme. The available options we have are not independent of the society we live in and are decisively shaped by the past. To claim we are “free” to live as we like (within the laws of capitalism) is basically to argue that we are able to “buy” the freedom that every individual is due from those who have stolen it from us in the first place!


Needless to say, some right-libertarians recognise that the state played a massive role in encouraging industrialisation (more correct to say “proletarianisation” as it created a working class which did not own the tools they used, although we stress that this process started on the land and not in industry). So they contrast “bad” business people (who took state aid) and “good” ones. Thus Rothbard’s comment that Marxists have {\em “made no particular distinction between ‘bourgeoisie’ who made use of the state, and bourgeoisie who acted on the free market.”} [{\bf The Ethics of Liberty}, p. 72]


But such an argument is nonsense as it ignores the fact that the “free market” is a network (and defined by the state by the property rights it enforces). For example, the owners of the American steel and other companies who grew rich and their companies big behind protectionist walls are obviously “bad” bourgeoisie. But are the bourgeoisie who supplied the steel companies with coal, machinery, food, “defence” and so on not also benefiting from state action? And the suppliers of the luxury goods to the wealthy steel company owners, did they not benefit from state action? Or the suppliers of commodities to the workers that laboured in the steel factories that the tariffs made possible, did they not benefit? And the suppliers to these suppliers? And the suppliers to these suppliers? Did not the users of technology first introduced into industry by companies protected by state orders also not benefit? Did not the capitalists who had a large and landless working class to select from benefit from the “land monopoly” even though they may not have, unlike other capitalists, directly advocated it? It increased the pool of wage labour for {\bf all} capitalists and increased their bargaining position/power in the labour market at the expense of the working class. In other words, such a policy helped maintain capitalist market power, irrespective of whether individual capitalists encouraged politicians to vote to create/maintain it. And, similarly, {\bf all} capitalists benefited from the changes in common law to recognise and protect capitalist private property and rights that the state enforced during the 19\high{th} century (see section B.2.5).


It appears that, for Rothbard, the collusion between state and business is the fault, not of capitalism, but of particular capitalists. The system is pure; only individuals are corrupt. But, for anarchists, the origins of the modern state-capitalist system lies not in the individual qualities of capitalists as such but in the dynamic and evolution of capitalism itself — a complex interaction of class interest, class struggle, social defence against the destructive actions of the market, individual qualities and so forth. In other words, Rothbard’s claims are flawed — they fail to understand capitalism as a {\bf system} and its dynamic nature.


Indeed, if we look at the role of the state in creating capitalism we could be tempted to rename “anarcho”-capitalism “marxian-capitalism”. This is because, given the historical evidence, a political theory can be developed by which the “dictatorship of the bourgeoisie” is created and that this capitalist state “withers away” into anarchy. That this means rejecting the economic and social ideas of Marxism and their replacement by their direct opposite should not mean that we should reject the idea (after all, that is what “anarcho”-capitalism has done to Individualist Anarchism!). But we doubt that many “anarcho”-capitalists will accept such a name change (even though this would reflect their politics far better; after all they do not object to past initiations of force, just current ones and many do seem to think that the modern state {\bf will} wither away due to market forces).


But this is beside the point. The fact remains that state action was required to create and maintain capitalism. Without state support it is doubtful that capitalism would have developed at all.


So, when the right suggests that “we” be “left alone,” what they mean by “we” comes into clear focus when we consider how capitalism developed. Artisans and peasants were only “left alone” to starve, and the working classes of industrial capitalism were only “left alone” outside work and for only as long as they respected the rules of their “betters.” As for the other side of the class divide, they desire to be “left alone” to exercise their power over others, as we will see. That modern “capitalism” is, in effect, a kind of “corporate mercantilism,” with states providing the conditions that allow corporations to flourish (e.g. tax breaks, subsidies, bailouts, anti-labour laws, etc.) says more about the statist roots of capitalism than the ideologically correct definition of capitalism used by its supporters.


\subsection{8.1 What social forces lay behind the rise of capitalism?
}

Capitalist society is a relatively recent development. As Murray Bookchin points out, for a {\em “long era, perhaps spanning more than five centuries,”} capitalism {\em “coexisted with feudal and simple commodity relationships”} in Europe. He argues that this period {\em “simply cannot be treated as ‘transitional’ without reading back the present into the past.”} [{\bf From Urbanisation to Cities}, p. 179] In other words, capitalism was not a inevitable outcome of “history” or social evolution.


He goes on to note that capitalism existed {\em “with growing significance in the mixed economy of the West from the fourteenth century up to the seventeenth”} but that it {\em “literally exploded into being in Europe, particularly England, during the eighteenth and especially nineteenth centuries.”} [{\bf Op. Cit.}, p. 181] The question arises, what lay behind this {\em “growing significance”}? Did capitalism {\em “explode”} due to its inherently more efficient nature or where there other, non-economic, forces at work? As we will show, it was most definitely the later one — capitalism was born not from economic forces but from the political actions of the social elites which its usury enriched. Unlike artisan (simple commodity) production, wage labour generates inequalities and wealth for the few and so will be selected, protected and encouraged by those who control the state in their own economic and social interests.


The development of capitalism in Europe was favoured by two social elites, the rising capitalist class within the degenerating medieval cities and the absolutist state. The medieval city was {\em “thoroughly changed by the gradual increase in the power of commercial capital, due primarily to foreign trade\unknown{} By this the inner unity of the commune was loosened, giving place to a growing caste system and leading necessarily to a progressive inequality of social interests. The privileged minorities pressed ever more definitely towards a centralisation of the political forces of the community\unknown{} Mercantilism in the perishing city republics led logically to a demand for larger economic units [i.e. to nationalise the market]; and by this the desire for stronger political forms was greatly strengthened\unknown{} Thus the city gradually became a small state, paving the way for the coming national state.”} [Rudolf Rocker, {\bf Nationalism and Culture}, p. 94]


The rising economic power of the proto-capitalists conflicted with that of the feudal lords, which meant that the former required help to consolidate their position. That aid came in the form of the monarchical state. With the force of absolutism behind it, capital could start the process of increasing its power and influence by expanding the “market” through state action.


As far as the absolutist state was concerned, it {\em “was dependent upon the help of these new economic forces, and vice versa\unknown{}” “The absolutist state,”} Rocker argues, {\em “whose coffers the expansion of commerce filled \unknown{}, at first furthered the plans of commercial capital. Its armies and fleets \unknown{} contributed to the expansion of industrial production because they demanded a number of things for whose large-scale production the shops of small tradesmen were no longer adapted. Thus gradually arose the so-called manufactures, the forerunners of the later large industries.”} [{\bf Op. Cit.}, p. 117–8]


Some of the most important state actions from the standpoint of early industry were the so-called Enclosure Acts, by which the “commons” — the free farmland shared communally by the peasants in most rural villages — was “enclosed” or incorporated into the estates of various landlords as private property (see section 8.3). This ensured a pool of landless workers who had no option but to sell their labour to capitalists. Indeed, the widespread independence caused by the possession of the majority of households of land caused the rising class of merchants to complain {\em “that men who should work as wage-labourers cling to the soil, and in the naughtiness of their hearts prefer independence as squatters to employment by a master.”} [R.H Tawney, cited by Allan Elgar in {\bf The Apostles of Greed}, p. 12]


In addition, other forms of state aid ensured that capitalist firms got a head start, so ensuring their dominance over other forms of work (such as co-operatives). A major way of creating a pool of resources that could be used for investment was the use of mercantilist policies which used protectionist measures to enrich capitalists and landlords at the expense of consumers and their workers. For example, one of most common complaints of early capitalists was that workers could not turn up to work regularly. Once they had worked a few days, they disappeared as they had earned enough money to live on. With higher prices for food, caused by protectionist measures, workers had to work longer and harder and so became accustomed to factory labour. In addition, mercantilism allowed native industry to develop by barring foreign competition and so allowed industrialists to reap excess profits which they could then use to increase their investments. In the words of Marian-socialist economic historian Maurice Dobbs:



\startblockquote
{\em “In short, the Mercantile System was a system of State-regulated exploitation through trade which played a highly important rule in the adolescence of capitalist industry: it was essentially the economic policy of an age of primitive accumulation.”} [{\bf Studies in Capitalism Development}, p. 209]



\stopblockquote
As Rocker summarises, {\em “when abolutism had victoriously overcome all opposition to national unification, but its furthering of mercantilism and economic monopoly it gave the whole social evolution a direction which could only lead to capitalism.”} [{\bf Op. Cit.}, pp. 116–7]


This process of state aid in capitalist development was also seen in the United States of America. As Edward Herman points out, the {\em “level of government involvement in business in the United States from the late eighteenth century to the present has followed a U-shaped pattern: There was extensive government intervention in the pre-Civil War period (major subsidies, joint ventures with active government participation and direct government production), then a quasi-laissez faire period between the Civil War and the end of the nineteenth century [a period marked by “the aggressive use of tariff protection” and state supported railway construction, a key factor in capitalist expansion in the USA], followed by a gradual upswing of government intervention in the twentieth century, which accelerated after 1930.”} [{\bf Corporate Control, Corporate Power}, p. 162]


Such intervention ensured that income was transferred from workers to capitalists. Under state protection, America industrialised by forcing the consumer to enrich the capitalists and increase their capital stock. {\em “According to one study, of the tariff had been removed in the 1830s ‘about half the industrial sector of New England would have been bankrupted’ \unknown{} the tariff became a near-permanent political institution representing government assistance to manufacturing. It kept price levels from being driven down by foreign competition and thereby shifted the distribution of income in favour of owners of industrial property to the disadvantage of workers and customers.”} [Richard B. Du Boff, {\bf Accumulation and Power}, p. 56]


This protection was essential, for as Du Boff notes, the {\em “end of the European wars in 1814 \unknown{} reopened the United States to a flood of British imports that drove many American competitors out of business. Large portions of the newly expanded manufacturing base were wiped out, bringing a decade of near-stagnation.”} Unsurprisingly, the {\em “era of protectionism began in 1816, with northern agitation for higher tariffs\unknown{} “} [{\bf Op. Cit.}, p. 14, p. 55]


Combined with ready repression of the labour movement and government “homesteading” acts (see section 8.5), tariffs were the American equivalent of mercantilism (which, after all, was above all else a policy of protectionism, i.e. the use of government to stimulate the growth of native industry). Only once America was at the top of the economic pile did it renounce state intervention (just as Britain did, we must note).


This is {\bf not} to suggest that government aid was limited to tariffs. The state played a key role in the development of industry and manufacturing. As John Zerzan notes, the {\em “role of the State is tellingly reflected by the fact that the ‘armoury system’ now rivals the older ‘American system of manufactures’ term as the more accurate to describe the new system of production methods”} developed in the early 1800s. [{\bf Elements of Refusal}, p. 100] Moreover, the {\em “lead in technological innovation [during the US Industrial Revolution] came in armaments where assured government orders justified high fixed-cost investments in special-pursue machinery and managerial personnel. Indeed, some of the pioneering effects occurred in government-owned armouries.”} [William Lazonick, {\bf Competitive Advantage on the Shop Floor}, p. 218] The government also {\em “actively furthered this process [of “commercial revolution”] with public works in transportation and communication.”} [Richard B. Du Boff, {\bf Op. Cit.}, p. 15]


In addition to this “physical” aid, {\em “state government provided critical help, with devices like the chartered corporation”} [{\bf Ibid.}] and, as we noted in section B.2.5, changes in the legal system which favoured capitalist interests over the rest of society.


Interestingly, there was increasing inequality between 1840 and 1860 in the USA This coincided with the victory of wage labour and industrial capitalism — the 1820s {\em “constituted a watershed in U.S. life. By the end of that decade \unknown{}industrialism assured its decisive American victory, by the end of the 1830s all of its cardinal features were definitely present.”} [John Zerzan, {\bf Op. Cit.}, p. 99] This is unsurprising, for as we have argued many times, the capitalist market tends to increase, not reduce, inequalities between individuals and classes. Little wonder the Individualist Anarchists at the time denounced the way that property had been transformed into {\em “a power [with which] to accumulate an income”} (to use the words of J.K. Ingalls).


Over all, as Paul Ormerod puts it, the {\em “advice to follow pure free-market polices seems \unknown{} to be contrary to the lessons of virtually the whole of economic history since the Industrial Revolution \unknown{} every country which has moved into \unknown{} strong sustained growth \unknown{} has done so in outright violation of pure, free-market principles.” “The model of entrepreneurial activity in the product market, with judicious state support plus repression in the labour market, seems to be a good model of economic development.”} [{\bf The Death of Economics}, p. 63]


Thus the social forces at work creating capitalism was a combination of capitalist activity and state action. But without the support of the state, it is doubtful that capitalist activity would have been enough to generate the initial accumulation required to start the economic ball rolling. Hence the necessity of Mercantilism in Europe and its modified cousin of state aid, tariffs and “homestead acts” in America.


\subsection{8.2 What was the social context of the statement “laissez-faire?”
}

The honeymoon of interests between the early capitalists and autocratic kings did not last long. {\em “This selfsame monarchy, which for weighty reasons sought to further the aims of commercial capital and was\unknown{} itself aided in its development by capital, grew at last into a crippling obstacle to any further development of European industry.”} [Rudolf Rocker, {\bf Nationalism and Culture}, p. 117]


This is the social context of the expression {\em “laissez-faire”} — a system which has outgrown the supports that protected it in its early stages of growth. Just as children eventually rebel against the protection and rules of their parents, so the capitalists rebelled against the over-bearing support of the absolutist state. Mercantilist policies favoured some industries and harmed the growth of industrial capitalism in others. The rules and regulations imposed upon those it did favour reduced the flexibility of capitalists to changing environments. As Rocker argues, {\em “no matter how the abolutist state strove, in its own interest, to meety the demands of commerce, it still put on industry countless fetters which became gradually more and more oppressive \unknown{} [it] became an unbearable burden \unknown{} which paralysed all economic and social life.”} [{\em Op. Cit.}, p. 119] All in all, mercantilism became more of a hindrance than a help and so had to be replaced. With the growth of economic and social power by the capitalist class, this replacement was made easier.


Errico Malatesta notes, {\em “[t]he development of production, the vast expansion of commerce, the immeasurable power assumed by money \unknown{} have guaranteed this supremacy [of economic power over the political power] to the capitalist class which, no longer content with enjoying the support of the government, demanded that government arise from its own ranks. A government which owed its origin to the right of conquest \unknown{} though subject by existing circumstances to the capitalist class, went on maintaining a proud and contemptuous attitude towards its now wealthy former slaves, and had pretensions to independence of domination. That government was indeed the defender, the property owners’ gendarme, but the kind of gendarmes who think they are somebody, and behave in an arrogant manner towards the people they have to escort and defend, when they don’t rob or kill them at the next street corner; and the capitalist class got rid of it \unknown{} [and replaced it] by a government [and state] \unknown{} at all times under its control and specifically organised to defend that class against any possible demands by the disinherited.”} [{\bf Anarchy}, pp. 19–20]


Malatesta here indicates the true meaning of {\em “leave us alone,”} or {\em “laissez-faire.”} The {\bf absolutist} state (not “the state” per se) began to interfere with capitalists’ profit-making activities and authority, so they determined that it had to go — as happened, for example, in the English, French and American revolutions. However, in other ways, state intervention in society was encouraged and applauded by capitalists. {\em “It is ironic that the main protagonists of the State, in its political and administrative authority, were the middle-class Utilitarians, on the other side of whose Statist banner were inscribed the doctrines of economic Laissez Faire”} [E.P. Thompson, {\bf The Making of the English Working Class}, p. 90]. Capitalists simply wanted {\bf capitalist} states to replace monarchical states, so that heads of government would follow state economic policies regarded by capitalists as beneficial to their class as a whole. And as development economist Lance Taylor argues:



\startblockquote
{\em “In the long run, there are no laissez-faire transitions to modern economic growth. The state has always intervened to create a capitalist class, and then it has to regulate the capitalist class, and then the state has to worry about being taken over by the capitalist class, but the state has always been there.”} [quoted by Noam Chomsky, {\bf Year 501}, p. 104]



\stopblockquote
In order to attack mercantilism, the early capitalists had to ignore the successful impact of its policies in developing industry and a “store of wealth” for future economic activity. As William Lazonick points out, {\em “the political purpose of [Adam Smith’s] the {\bf Wealth of Nations} was to attack the mercantilist institutions that the British economy had built up over the previous two hundred years\unknown{} In his attack on these institutions, Smith might have asked why the extent of the world market available to Britain in the late eighteenth century was {\bf so uniquely under British control.} If Smith had asked this ‘big question,’ he might have been forced to grant credit for [it] \unknown{} to the very mercantilist institutions he was attacking \unknown{}”} Moreover, he {\em “might have recognised the integral relation between economic and political power in the rise of Britain to international dominance.”} Overall, {\em “[w]hat the British advocates of laissez-faire neglected to talk about was the role that a system of national power had played in creating conditions for Britain to embark on its dynamic development path \unknown{} They did not bother to ask how Britain had attained th[e] position [of ‘workshop of the world’], while they conveniently ignored the on going system of national power — the British Empire — that \unknown{} continued to support Britain’s position.”} [{\bf Business Organisation and the Myth of the Market Economy}, p. 2, p. 3, p.5]


Similar comments are applicable to American supporters of laissez faire who fail to notice that the “traditional” American support for world-wide free trade is quite a recent phenomenon. It started only at the end of the Second World War (although, of course, {\bf within} America military Keynesian policies were utilised). While American industry was developing, the country had no time for laissez-faire. After it had grown strong, the United States began preaching laissez-faire to the rest of the world — and began to kid itself about its own history, believing its slogans about laissez-faire as the secret of its success. In addition to the tariff, nineteenth-century America went in heavily for industrial planning — occasionally under that name but more often in the name of national defence. The military was the excuse for what is today termed rebuilding infrastructure, picking winners, promoting research, and co-ordinating industrial growth (as it still is, we should add).


As Richard B. Du Boff points out, the “anti-state” backlash of the 1840s onwards in America was highly selective, as the general opinion was that {\em “[h]enceforth, if governments wished to subsidise private business operations, there would be no objection. But if public power were to be used to control business actions or if the public sector were to undertake economic initiatives on its own, it would run up against the determined opposition of private capital.”} [{\bf Accumulation and Power}, p. 26] In other words, the state could aid capitalists indirectly (via tariffs, land policy, repression of the labour movement, infrastructure subsidy and so on) and it would “leave them alone” to oppress and exploit workers, exploit consumers, build their industrial empires and so forth.


So, the expression “laissez-faire” dates from the period when capitalists were objecting to the restrictions that helped create them in the first place. It has little to do with freedom as such and far more to do with the needs of capitalist power and profits (as Murray Bookchin argues, it is an error to depict this {\em “revolutionary era and its democratic aspirations as ‘bourgeois,’ an imagery that makes capitalism a system more committed to freedom, or even ordinary civil liberties, than it was historically”} [{\bf From Urbanisation to Cities}, p. 180f]). Takis Fotopoules, in his essay {\em “The Nation-state and the Market”}, indicates that the social forces at work in “freeing” the market did not represent a “natural” evolution towards freedom:



\startblockquote
{\em “Contrary to what liberals and Marxists assert, marketisation of the economy was not just an evolutionary process, following the expansion of trade under mercantilism \unknown{} modern [i.e. capitalist] markets did not evolve out of local markets and/or markets for foreign goods \unknown{} the nation-state, which was just emerging at the end of the Middle Ages, played a crucial role creating the conditions for the ‘nationalisation’ of the market \unknown{} and \unknown{} by freeing the market [i.e. the rich and proto-capitalists] from effective social control.”} [{\bf Society and Nature}, Vol. 3, pp. 44–45]



\stopblockquote
The “freeing” of the market thus means freeing those who “own” most of the market (i.e. the wealthy elite) from {\em “effective social control,”} but the rest of society was not as lucky. Peter Kropotkin makes a similar point in {\bf Modern Science and Anarchism}, {\em “[w]hile giving the capitalist any degree of free scope to amass his wealth at the expense of the helpless labourers, the government has {\bf nowhere} and {\bf never}\unknown{}afforded the labourers the opportunity ‘to do as they pleased’.”} [{\bf Kropotkin’s Revolutionary Pamphlets}, p. 182]


The one essential form of support the “Libertarian” right wants the state (or “defence” firms) to provide capitalism is the enforcement of property rights — the right of property owners to “do as they like” on their own property, which can have obvious and extensive social impacts. What “libertarian” capitalists object to is attempts by others — workers, society as a whole, the state, etc. — to interfere with the authority of bosses. That this is just the defence of privilege and power (and {\bf not} freedom) has been discussed in section B and elsewhere in section F, so we will not repeat ourselves here.


Samuel Johnson once observed that {\em “we hear the loudest {\bf yelps} for liberty among the drivers of Negroes.”} Our modern “libertarian” capitalist drivers of wage-slaves are yelping for exactly the same kind of “liberty.” [Johnson quoted in Noam Chomsky, {\bf Year 501}, p. 141]


\subsection{8.3 What other forms did state intervention in creating capitalism take?
}

Beyond being a paymaster for new forms of production and social relations and defending the owners’ power, the state intervened economically in other ways as well. As we noted in section B.2.5, the state played a key role in transforming the law codes of society in a capitalistic fashion, ignoring custom and common law to do so. Similarly, the use of tariffs and the granting of monopolies to companies played an important role in accumulating capital at the expense of working people, as did the breaking of unions and strikes by force.


However, one of the most blatant of these acts was the enclosure of common land. In Britain, by means of the Enclosure Acts, land that had been freely used by poor peasants for farming their small family plots was claimed by large landlords as private property. As E.P. Thompson notes, {\em “Parliament and law imposed capitalist definitions to exclusive property in land”} [{\bf Customs in Common}, p. 163]. Property rights, which exclusively favoured the rich, replaced the use rights and free agreement that had governed peasant’s use of the commons. Unlike use rights, which rest in the individual, property rights require state intervention to create and maintain.


This stealing of the land should not be under estimated. Without land, you cannot live and have to sell your liberty to others. This places those with capital at an advantage, which will tend to increase, rather than decrease, the inequalities in society (and so place the landless workers at an increasing disadvantage over time). This process can be seen from early stages of capitalism. With the enclosure of the land, an agricultural workforce was created which had to travel where the work was. This influx of landless ex-peasants into the towns ensured that the traditional guild system crumbled and was transformed into capitalistic industry with bosses and wage slaves rather than master craftsmen and their journeymen. Hence the enclosure of land played a key role, for {\em “it is clear that economic inequalities are unlikely to create a division of society into an employing master class and a subject wage-earning class, unless access to the mans of production, including land, is by some means or another barred to a substantial section of the community.”} [Maurice Dobbs, {\bf Studies in Capitalist Development}, p. 253]


The importance of access to land is summarised by this limerick by the followers of Henry George (a 19\high{th} century writer who argued for a {\em “single tax”} and the nationalisation of land). The Georgites got their basic argument on the importance of land down these few, excellent lines:


{\em A college economist planned\crlf  To live without access to land\crlf  He would have succeeded\crlf  But found that he needed\crlf  Food, shelter and somewhere to stand.}


Thus the Individualist (and other) anarchists’ concern over the {\em “land monopoly”} of which the Enclosure Acts were but one part. The land monopoly, to use Tucker’s words, {\em “consists in the enforcement by government of land titles which do not rest upon personal occupancy and cultivation.”} [{\bf The Anarchist Reader}, p. 150] It is important to remember that wage labour first developed on the land and it was the protection of land titles of landlords and nobility, combined with enclosure, that meant people could not just work their own land.


In other words, the circumstances so created by enclosing the land and enforcing property rights to large estates ensured that capitalists did not have to point a gun at workers head to get them to work long hours in authoritarian, dehumanising conditions. In such circumstances, when the majority are dispossessed and face the threat of starvation, poverty, homelessness and so on, “initiation of force” is {\bf not required.} But guns {\bf were} required to enforce the system of private property that created the labour market in the first place, to enforce the enclosure of common land and protect the estates of the nobility and wealthy.


In addition to increasing the availability of land on the market, the enclosures also had the effect of destroying working-class independence. Through these Acts, innumerable peasants were excluded from access to their former means of livelihood, forcing them to migrate to the cities to seek work in the newly emerging factories of the budding capitalist class, who were thus provided with a ready source of cheap labour. The capitalists, of course, did not describe the results this way, but attempted to obfuscate the issue with their usual rhetoric about civilisation and progress. Thus John Bellers, a 17\high{th}-century supporter of enclosures, claimed that commons were {\em “a hindrance to Industry, and \unknown{} Nurseries of Idleness and Insolence.”} The {\em “forests and great Commons make the Poor that are upon them too much like the {\bf indians.}”} [quoted by Thompson, {\bf Op. Cit.}, p. 163] Elsewhere Thompson argues that the commons {\em “were now seen as a dangerous centre of indiscipline \unknown{} Ideology was added to self-interest. It became a matter of public-spirited policy for gentlemen to remove cottagers from the commons, reduce his labourers to dependence \unknown{}”} [{\bf The Making of the English Working Class}, pp. 242–3]


The commons gave working-class people a degree of independence which allowed them to be “insolent” to their betters. This had to be stopped, as it undermined to the very roots of authority relationships within society. The commons {\bf increased} freedom for ordinary people and made them less willing to follow orders and accept wage labour. The reference to “Indians” is important, as the independence and freedom of Native Americans is well documented. The common feature of both cultures was communal ownership of the means of production and free access to it (usufruct). This is discussed further in section I.7 (Won’t Libertarian Socialism destroy individuality?)


As the early American economist Edward Wakefield noted in 1833, {\em “where land is cheap and all are free, where every one who so pleases can easily obtain a piece of land for himself, not only is labour dear, as respects the labourer’s share of the product, but the difficulty is to obtain combined labour at any price.”} [{\bf England and America}, quoted by Jeremy Brecher and Tim Costello, {\bf Commonsense for Hard Times}, p. 24]


The enclosure of the commons (in whatever form it took — see section 8.5 for the US equivalent) solved both problems — the high cost of labour, and the freedom and dignity of the worker. The enclosures perfectly illustrate the principle that capitalism requires a state to ensure that the majority of people do not have free access to any means of livelihood and so must sell themselves to capitalists in order to survive. There is no doubt that if the state had “left alone” the European peasantry, allowing them to continue their collective farming practices (“collective farming” because, as Kropotkin shows in {\bf Mutual Aid}, the peasants not only shared the land but much of the farm labour as well), capitalism could not have taken hold (see {\bf Mutual Aid}, pp. 184–189, for more on the European enclosures). As Kropotkin notes, {\em “[i]nstances of commoners themselves dividing their lands were rare, everywhere the State coerced them to enforce the division, or simply favoured the private appropriation of their lands”} by the nobles and wealthy. [{\bf Mutual Aid}, p. 188]


Thus Kropotkin’s statement that {\em “to speak of the natural death of the village community [or the commons] in virtue of economical law is as grim a joke as to speak of the natural death of soldiers slaughtered on a battlefield.”} [{\bf Op. Cit.}, p. 189]


Like the more recent case of fascist Chile, “free market” capitalism was imposed on the majority of society by an elite using the authoritarian state. This was recognised by Adam Smith when he opposed state intervention in {\bf The Wealth of Nations}. In Smith’s day, the government was openly and unashamedly an instrument of wealth owners. Less than 10 per cent of British men (and no women) had the right to vote. When Smith opposed state interference, he was opposing the imposition of wealth owners’ interests on everybody else (and, of course, how “liberal”, nevermind “libertarian”, is a political system in which the many follow the rules and laws set-down in the so-called interests of all by the few? As history shows, any minority given, or who take, such power {\bf will} abuse it in their own interests). Today, the situation is reversed, with neo-liberals and right libertarians opposing state interference in the economy (e.g. regulation of Big Business) so as to prevent the public from having even a minor impact on the power or interests of the elite.


The fact that “free market” capitalism always requires introduction by an authoritarian state should make all honest “Libertarians” ask: How “free” is the “free market”? And why, when it is introduced, do the rich get richer and the poor poorer? This was the case in Chile (see Section C.11). For the poverty associated with the rise of capitalism in England 200 years ago, E.P. Thompson’s {\bf The Making of the English Working Class} provides a detailed discussion. Howard Zinn’s {\bf A People’s History of the United States} describes the poverty associated with 19\high{th}-century US capitalism.


\subsection{8.4 Aren’t the enclosures a socialist myth?
}

The short answer is no, they are not. While a lot of historical analysis has been spent in trying to deny the extent and impact of the enclosures, the simple fact is (in the words of noted historian E.P. Thompson) enclosure {\em “was a plain enough case of class robbery, played according to the fair rules of property and law laid down by a parliament of property-owners and lawyers.”} [{\bf The Making of the English Working Class}, pp. 237–8]


The enclosures were one of the ways that the {\em “land monopoly”} was created. The land monopoly was used to refer to capitalist property rights and ownership of land by (among others) the Individualist Anarchists. Instead of an {\em “occupancy and use”} regime advocated by anarchists, the land monopoly allowed a few to bar the many from the land — so creating a class of people with nothing to sell but their labour. While this monopoly is less important these days in developed nations (few people know how to farm) it was essential as a means of consolidating capitalism. Given the choice, most people preferred to become independent farmers rather than wage workers (see next section).


However, the importance of the enclosure movement is downplayed by supporters of capitalism. Little wonder, for it is something of an embarrassment for them to acknowledge that the creation of capitalism was somewhat less than “immaculate” — after all, capitalism is portrayed as an almost ideal society of freedom. To find out that an idol has feet of clay and that we are still living with the impact of its origins is something pro-capitalists must deny. So {\bf is} the enclosures a socialist myth? Most claims that it is flow from the work of the historian J.D. Chambers’ famous essay {\em “Enclosures and the Labour Supply in the Industrial Revolution.”} [{\bf Economic History Review}, 2\high{nd} series, no. 5, August 1953] In this essay, Chambers attempts to refute Karl Marx’s account of the enclosures and the role it played in what Marx called {\em “primitive accumulation.”}


We cannot be expected to provide an extensive account of the debate that has raged over this issue. All we can do is provide a summary of the work of William Lazonick who presented an excellent reply to those who claim that the enclosures were an unimportant historical event. We are drawing upon his summary of his excellent essay {\em “Karl Marx and Enclosures in England”} [{\bf Review of Radical Political Economy}, no. 6, Summer, 1974] which can be found in his books {\bf Competitive Advantage on the Shop Floor} and {\bf Business Organisation and the Myth of the Market Economy}. There are three main claims against the socialist account of the enclosures. We will cover each in turn.


Firstly, it is often claimed that the enclosures drove the uprooted cottager and small peasant into industry. However, this was never claimed. It is correct that the agricultural revolution associated with the enclosures {\bf increased} the demand for farm labour as claimed by Chambers and others. And this is the whole point — enclosures created a pool of dispossessed labourers who had to sell their time/liberty to survive. The {\em “critical transformation was not the level of agricultural employment before and after enclosure but the changes in employment relations caused by the reorganisation of landholdings and the reallocation of access to land.”} [{\bf Competitive Advantage on the Shop Floor}, p. 30] Thus the key feature of the enclosures was that it created a supply for farm labour, a supply that had no choice but to work for another. This would drive down wages and increase demand. Moreover, freed from the land, these workers could later move to the towns in search for better work.


Secondly, it is argued that the number of small farm owners increased, or at least did not greatly decline, and so the enclosure movement was unimportant. Again, this misses the point. Small farm owners can still employ wage workers (i.e. become capitalist farmers as opposed to “yeomen” — independent peasant proprietor). As Lazonick notes, {\em “[i]t is true that after 1750 some petty proprietors continued to occupy and work their own land. But in a world of capitalist agriculture, the yeomanry no longer played an important role in determining the course of capitalist agriculture. As a social class that could influence the evolution of British economy society, the yeomanry had disappeared.”} [{\bf Op. Cit.}, p. 32]


Thirdly, it is often claimed that it was population growth, rather than enclosures, that caused the supply of wage workers. So was population growth more important that enclosures? Maurice Dobbs argues that {\em “the centuries in which a proletariat was most rapidly recruited were apt to be those of slow rather than of rapid natural increase of population, and the paucity or plenitude of a labour reserve in different countries was not correlated with comparable difference in their rates of population-growth.”} [{\bf Studies in Capitalist Development}, p. 223] Moreover, the population argument ignores the question of whether the changes in society caused by enclosures and the rise of capitalism have an impact on the observed trends towards earlier marriage and larger families after 1750. Lazonick argues that {\em “[t]here is reason to believe that they did.”} [{\bf Op. Cit.}, p. 33] Also, of course, the use of child labour in the factories created an economic incentive to have more children, an incentive created by the developing capitalist system. Overall, Lazonick notes that {\em “[t]o argue that population growth created the industrial labour supply is to ignore these momentous social transformations”} associated with the rise of capitalism [{\bf Business Organisation and the Myth of the Market Economy}, p. 273].


In other words, there is good reason to think that the enclosures, far from being some kind of socialist myth, in fact played a key role in the development of capitalism. As Lazonick himself notes, {\em “Chambers misunderstood” “the argument concerning the ‘institutional creation’ of a proletarianised (i.e. landless) workforce. Indeed, Chamber’s own evidence and logic tend to support the Marxian [and anarchist!] argument, when it is properly understood.”} [{\bf Op. Cit.}, p. 273]


\subsection{8.5 What about the lack of enclosures in the Americas?
}

The enclosure movement was but one way of creating the {\em “land monopoly”} which ensured the creation of a working class. The circumstances facing the ruling class in the Americas were distinctly different than in the Old World and so the “land monopoly” took a different form there. In the Americas, enclosures were unimportant as customary land rights did not really exist. Here the problem was that (after the original users of the land were eliminated, of course) there were vast tracts of land available for people to use.


Unsurprisingly, there was a movement towards independent farming and this pushed up the price of labour, by reducing the supply. Capitalists found it difficult to find workers willing to work for them at wages low enough to provide them with sufficient profits. It was due the difficulty in finding cheap enough labour that capitalists in America turned to slavery. All things being equal, wage labour {\bf is} more productive than slavery. But in early America all things were {\bf not} equal. Having access to cheap (indeed, free) land meant that working people had a choice, and few desired to become wage slaves. Because of this, capitalists turned to slavery in the South and the “land monopoly” in the North and West.


This was because, in the words of Maurice Dobbs, it {\em “became clear to those who wished to reproduce capitalist relations of production in the new country that the foundation-stone of their endeavour must be the restriction of land-ownership to a minority and the exclusion of the majority from any share in [productive] property.”} [{\bf Studies in Capitalist Development}, pp. 221–2] As one radical historian puts it, {\em “[w]hen land is ‘free’ or ‘cheap’. as it was in different regions of the United States before the 1830s, there was no compulsion for farmers to introduce labour-saving technology. As a result, ‘independent household production’ \unknown{} hindered the development of capitalism \unknown{} [by] allowing large portions of the population to escape wage labour.”} [Charlie Post, {\em “The ‘Agricultural Revolution’ in the United States”}, pp. 216–228, {\bf Science and Society}, vol. 61, no. 2, p. 221]


It was precisely this option (i.e. of independent production) that had to be destroyed in order for capitalist industry to develop. The state had to violate the holy laws of “supply and demand” by controlling the access to land in order to ensure the normal workings of “supply and demand” in the labour market (i.e. that the bargaining position on the labour market favoured employer over employee). Once this situation became the typical one (i.e. when the option of self-employment was effectively eliminated) a (protectionist based) “laissez-faire” approach could be adopted and state action used only to protect private property from the actions of the dispossessed.


So how was this transformation of land ownership achieved?


Instead of allowing settlers to appropriate their own farms as was the case before the 1830s, the state stepped in once the army had cleared out the original users. Its first major role was to enforce legal rights of property on unused land. Land stolen from the Native Americans was sold at auction to the highest bidders, namely speculators, who then sold it on to farmers. This process started right {\em “after the revolution, [when] huge sections of land were bought up by rich speculators”} and their claims supported by the law [Howard Zinn, {\bf A People’s History of the United States}, p. 125] Thus land which should have been free was sold to land-hungry farmers and the few enriched themselves at the expense of the many. Not only did this increase inequality within society, it also encouraged the development of wage labour — having to pay for land would have ensured that many immigrants remained on the East Coast until they had enough money. Thus a pool of people with little option but to sell their labour was increased due to state protection of unoccupied land. That the land usually ended up in the hands of farmers did not (could not) countermand the shift in class forces that this policy created.


This was also the essential role of the various “Homesteading Acts” and, in general, the {\em “Federal land law in the 19\high{th} century provided for the sale of most of the public domain at public auction to the higher bidder \unknown{} Actual settlers were forced to buy land from speculators, at prices considerably above the federal minimal price”} (which few people could afford anyway) [Charlie Post, {\bf Op. Cit.}, p. 222]. Little wonder the Individualist Anarchists supported an {\em “occupancy and use”} system of land ownership as a key way of stopping capitalist and landlord usury as well as the development of capitalism itself.


This change in the appropriation of land had significant effects on agriculture and the desirability of taking up farming for immigrants. As Post notes, {\em “[w]hen the social conditions for obtaining and maintaining possession of land change, as they did in the midwest between 1830 and 1840, pursuing the goal of preserving [family ownership and control] \unknown{} produced very different results. In order to pay growing mortgages, debts and taxes, family farmers were compelled to specialise production toward cash crops and to market more and more of their output.”} [{\bf Op. Cit.}, p. 221–2]


So, in order to pay for land which was formerly free, farmers got themselves into debt and increasingly turned to the market to pay it off. Thus, the {\em “Federal land system, by transforming land into a commodity and stimulating land speculation, made the midwestern farmers dependent upon markets for the continual possession of their farms.”} [Charlie Post, {\bf Op. Cit.}, p. 223] Once on the market, farmers had to invest in new machinery and this also got them into debt. In the face of a bad harvest or market glut, they could not repay their loans and their farms had to be sold to so do so. By 1880, 25\% of all farms were rented by tenants, and the numbers kept rising.


This means that Murray Rothbard’s comment that {\em “once the land was purchased by the settler, the injustice disappeared”} is nonsense — the injustice was transmitted to other parts of society and this, along with the legacy of the original injustice, lived on and helped transform society towards capitalism [{\bf The Ethics of Liberty}, p. 73]. In addition, his comments about {\em “the establishment in North America of a truly libertarian land system”} would be one the Individualist Anarchists would have seriously disagreed with! [{\bf Ibid.}]


Thus state action, in restricting free access to the land, ensured that workers were dependent on wage labour. In addition, the {\em “transformation of social property relations in northern agriculture set the stage for the ‘agricultural revolution’ of the 1840s and 1850s \unknown{} [R]ising debts and taxes forced midwestern family farmers to compete as commodity producers in order to maintain their land-holding \unknown{} The transformation \unknown{} was the central precondition for the development of industrial capitalism in the United States.”} [Charlie Post, {\bf Ibid.}, p. 226]


In addition to seizing the land and distributing it in such a way as to benefit capitalist industry, the {\em “government played its part in helping the bankers and hurting the farmers; it kept the amount of money — based in the gold supply — steady while the population rose, so there was less and less money in circulation. The farmer had to pay off his debts in dollars that were harder to get. The bankers, getting loans back, were getting dollars worth more than when they loaned them out — a kind of interest on top of interest. That was why \unknown{} farmers’ movements [like the Individualist Anarchists, we must add] \unknown{} [talked about] putting more money in circulation.”} [Howard Zinn, {\bf Op. Cit.}, p. 278]


Overall, therefore, state action ensured the transformation of America from a society of independent workers to a capitalist one. By creating and enforcing the “land monopoly” (of which state ownership of unoccupied land and its enforcement of landlord rights were the most important) the state ensured that the balance of class forces tipped in favour of the capitalist class. By removing the option of farming your own land, the US government created its own form of enclosure and the creation of a landless workforce with little option but to sell its liberty on the “free market”. This, combined with protectionism, ensured the transformation of American society from a pre-capitalist one into a capitalist one. They was nothing “natural” about it.


Little wonder the Individualist Anarchist J.K. Ingalls attacked the “land monopoly” in the following words:



\startblockquote
{\em “The earth, with its vast resources of mineral wealth, its spontaneous productions and its fertile soil, the free gift of God and the common patrimony of mankind, has for long centuries been held in the grasp of one set of oppressors by right of conquest or right of discovery; and it is now held by another, through the right of purchase from them. All of man’s natural possessions \unknown{} have been claimed as property; nor has man himself escaped the insatiate jaws of greed. The invasion of his rights and possessions has resulted \unknown{} in clothing property with a power to accumulate an income.”} [quoted by James Martin, {\bf Men Against the State}, p. 142]



\stopblockquote
\subsection{8.6 How did working people view the rise of capitalism?
}

The best example of how hated capitalism was can be seen by the rise and spread of the socialist movement, in all its many forms, across the world. It is no coincidence that the development of capitalism also saw the rise of socialist theories. However, in order to fully understand how different capitalism was from previous economic systems, we will consider early capitalism in the US, which for many “Libertarians” is {\bf the} example of the “capitalism-equals-freedom” argument.


Early America was pervaded by artisan production — individual ownership of the means of production. Unlike capitalism, this system is {\bf not} marked by the separation of the worker from the means of life. Most people did not have to work for another, and so did not. As Jeremy Brecher notes, in 1831 the {\em “great majority of Americans were farmers working their own land, primarily for their own needs. Most of the rest were self-employed artisans, merchants, traders, and professionals. Other classes — employees and industrialists in the North, slaves and planters in the South — were relatively small. The great majority of Americans were independent and free from anybody’s command.”} [{\bf Strike!}, p. xxi] These conditions created the high cost of combined (wage) labour which ensured the practice of slavery existed.


However, toward the middle of the 19\high{th} century the economy began to change. Capitalism began to be imported into American society as the infrastructure was improved, which allowed markets for manufactured goods to grow. Soon, due to (state-supported) capitalist competition, artisan production was replaced by wage labour. Thus “evolved” modern capitalism. Many workers understood, resented, and opposed their increasing subjugation to their employers ({\em “the masters”}, to use Adam Smith’s expression), which could not be reconciled with the principles of freedom and economic independence that had marked American life and sunk deeply into mass consciousness during the days of the early economy. In 1854, for example, a group of skilled piano makers wrote that {\em “the day is far distant when they [wage earners] will so far forget what is due to manhood as to glory in a system forced upon them by their necessity and in opposition to their feelings of independence and self-respect. May the piano trade be spared such exhibitions of the degrading power of the day [wage] system.”} [quoted by Brecher and Costello, {\bf Common Sense for Hard Times}, p. 26]


Clearly the working class did not consider working for a daily wage, in contrast to working for themselves and selling their own product, to be a step forward for liberty or individual dignity. The difference between selling the product of one’s labour and selling one’s labour (i.e. oneself) was seen and condemned ({\em “[w]hen the producer \unknown{} sold his product, he retained himself. But when he came to sell his labour, he sold himself \unknown{} the extension [of wage labour] to the skilled worker was regarded by him as a symbol of a deeper change”} [Norman Ware, {\bf The Industrial Worker, 1840–1860}, p. xiv]). Indeed, one group of workers argued that they were {\em “slaves in the strictest sense of the word”} as they had {\em “to toil from the rising of the sun to the going down of the same for our masters — aye, masters, and for our daily bread”} [Quoted by Ware, {\bf Op. Cit.}, p. 42] and another argued that {\em “the factory system contains in itself the elements of slavery, we think no sound reasoning can deny, and everyday continues to add power to its incorporate sovereignty, while the sovereignty of the working people decreases in the same degree.”} [quoted by Brecher and Costello, {\bf Op. Cit.}, p. 29]


Almost as soon as there were wage workers, there were strikes, machine breaking, riots, unions and many other forms of resistance. John Zerzan’s argument that there was a {\em “relentless assault on the worker’s historical rights to free time, self-education, craftsmanship, and play was at the heart of the rise of the factory system”} is extremely accurate [{\bf Elements of Refusal}, p. 105]. And it was an assault that workers resisted with all their might. In response to being subjected to the “law of value,” workers rebelled and tried to organise themselves to fight the powers that be and to replace the system with a co-operative one. As the printer’s union argued, {\em “[we] regard such an organisation [a union] not only as an agent of immediate relief, but also as an essential to the ultimate destruction of those unnatural relations at present subsisting between the interests of the employing and the employed classes\unknown{}[W]hen labour determines to sell itself no longer to speculators, but to become its own employer, to own and enjoy itself and the fruit thereof, the necessity for scales of prices will have passed away and labour will be forever rescued from the control of the capitalist.”} [quoted by Brecher and Costello, {\bf Op. Cit.}, pp. 27–28]


Little wonder, then, why wage labourers considered capitalism as a form of {\em “slavery”} and why the term {\em “wage slavery”} became so popular in the anarchist movement. It was just reflecting the feelings of those who experienced the wages system at first hand and joined the socialist and anarchist movements. As labour historian Norman Ware notes, the {\em “term ‘wage slave’ had a much better standing in the forties [of the 19\high{th} century] than it has today. It was not then regarded as an empty shibboleth of the soap-box orator. This would suggest that it has suffered only the normal degradation of language, has become a {\bf cliche}, not that it is a grossly misleading characterisation.”} [{\bf Op. Cit.}, p. xvf]


These responses of workers to the experience of wage labour is important to show that capitalism is by no means “natural.” The fact is the first generation of workers tried to avoid wage labour is at all possible as they hated the restrictions of freedom it imposed upon them. They were perfectly aware that wage labour was wage slavery — that they were decidedly {\bf unfree} during working hours and subjected to the will of another. While many working people now are accustomed to wage labour (while often hating their job) the actual process of resistance to the development of capitalism indicates well its inherently authoritarian nature. Only once other options were closed off and capitalists given an edge in the “free” market by state action did people accept and become accustomed to wage labour.


Opposition to wage labour and factory fascism was/is widespread and seems to occur wherever it is encountered. {\em “Research has shown”}, summarises William Lazonick, {\em “that the ‘free-born Englishman’ of the eighteenth century — even those who, by force of circumstance, had to submit to agricultural wage labour — tenaciously resisted entry into the capitalist workshop.”} [{\bf Business Organisation and the Myth of the Market Economy}, p. 37] British workers shared the dislike of wage labour of their American cousins. A {\em “Member of the Builders’ Union”} in the 1830s argued that the trade unions {\em “will not only strike for less work, and more wages, but will ultimately {\bf abolish wages}, become their own masters and work for each other; labour and capital will no longer be separate but will be indissolubly joined together in the hands of workmen and work-women.”} [quoted by Geoffrey Ostergaard, {\bf The Tradition of Workers’ Control}, p. 133] This is unsurprising, for as Ostergaard notes, {\em “the workers then, who had not been swallowed up whole by the industrial revolution, could make critical comparisons between the factory system and what preceded it.”} [{\bf Op. Cit.}, p. 134] While wage slavery may seem “natural” today, the first generation of wage labourers saw the transformation of the social relationships they experienced in work, from a situation in which they controlled their own work (and so themselves) to one in which {\bf others} controlled them, and they did not like it. However, while many modern workers instinctively hate wage labour and having bosses, without the awareness of some other method of working, many put up with it as “inevitable.” The first generation of wage labourers had the awareness of something else (although, of course, a flawed something else) and this gave then a deep insight into the nature of capitalism and produced a deeply radical response to it and its authoritarian structures.


Far from being a “natural” development, then, capitalism was imposed on a society of free and independent people by state action. Those workers alive at the time viewed it as {\em “unnatural relations”} and organised to overcome it. These feelings and hopes still exist, and will continue to exist until such time as we organise and {\em “abolish the wage system”} (to quote the IWW preamble) and the state that supports it.


\subsection{8.7 Why is the history of capitalism important?
}

Simply because it provides us with an understanding of whether that system is “natural” and whether it can be considered as just and free. If the system was created by violence, state action and other unjust means then the apparent “freedom” which we currently face within it is a fraud, a fraud masking unnecessary and harmful relations of domination, oppression and exploitation. Moreover, by seeing how capitalist relationships were viewed by the first generation of wage slaves reminds us that just because many people have adjusted to this regime and consider it as normal (or even natural) it is nothing of the kind.


Murray Rothbard is well aware of the importance of history. He considered the {\em “moral indignation”} of socialism arises from the argument {\em “that the capitalists have stolen the rightful property of the workers, and therefore that existing titles to accumulated capital are unjust.”} He argues that given {\em “this hypothesis, the remainder of the impetus for both Marxism and anarchosyndicalism follow quote logically.”} [{\bf The Ethics of Liberty}, p. 52]


So some right-libertarians are aware that the current property owners have benefited extensively from violence and state action in the past. Murray Rothbard argues (in {\bf The Ethics of Liberty}, p. 57) that if the just owners cannot be found for a property, then the property simply becomes again unowned and will belong to the first person to appropriate and utilise it. If the current owners are not the actual criminals then there is no reason at all to dispossess them of their property; if the just owners cannot be found then they may keep the property as the first people to use it (of course, those who own capital and those who use it are usually different people, but we will ignore this obvious point).


Thus, since all original owners and the originally dispossessed are long dead nearly all current title owners are in just possession of their property except for recently stolen property. The principle is simple, dispossess the criminals, restore property to the dispossessed if they can be found otherwise leave titles where they are (as Native American tribes owned the land {\bf collectively} this could have an interesting effect on such a policy in the USA. Obviously tribes that were wiped out need not apply, but would such right-libertarian policy recognise such collective, non-capitalist ownership claims? We doubt it, but we could be wrong — the Libertarian Party Manifesto states that their “just” property rights will be restored. And who defines “just”? And given that unclaimed federal land will be given to Native Americans, its seems pretty likely that the {\bf original} land will be left alone).


Of course, that this instantly gives an advantage to the wealthy on the new “pure” market is not mentioned. The large corporations that, via state protection and support, built their empires and industrial base will still be in an excellent position to continue to dominate the market. Wealthy land owners, benefiting from the effects of state taxation and rents caused by the “land monopoly” on farmstead failures, will keep their property. The rich will have a great initial advantage and this may be more than enough to maintain them in there place. After all, exchanges between worker and owner tend to reinforce existing inequalities, {\bf not} reduce them (and as the owners can move their capital elsewhere or import new, lower waged, workers from across the world, its likely to stay that way).


So Rothbard’s “solution” to the problem of past force seems to be (essentially) a justification of existing property titles and not a serious attempt to understand or correct past initiations of force that have shaped society into a capitalist one and still shape it today. The end result of his theory is to leave things pretty much as they are, for the past criminals are dead and so are their victims.


However, what Rothbard fails to note is that the {\bf results} of this state action and coercion are still with us. He totally fails to consider that the theft of productive wealth has a greater impact on society than the theft itself. The theft of {\bf productive} wealth shapes society in so many ways that {\bf all} suffer from it (including current generations). This (the externalities generated by theft) cannot be easily undone by individualistic “solutions”.


Let us take an example somewhat more useful that the one Rothbard uses (namely, a stolen watch). A watch cannot really be used to generate wealth (although if I steal a watch, sell it and buy a winning lottery ticket, does that mean I can keep the prize after returning the money value of your watch to you? Without the initial theft, I would not have won the prize but obviously the prize money far exceeds the amount stolen. Is the prize money mine?). Let us take a tool of production rather than a watch.


Let assume a ship sinks and 50 people get washed ashore on an island. One woman has foresight to take a knife from the ship and falls unconscious on the beach. A man comes along and steals her knife. When the woman awakes she cannot remember if she had managed to bring the knife ashore with her or not. The man maintains that he brought it with him and no one else saw anything. The survivors decide to split the island equally between them and work it separately, exchanging goods via barter.


However, the man with the knife has the advantage and soon carves himself a house and fields from the wilderness. Seeing that they need the knife and the tools created by the knife to go beyond mere existing, some of the other survivors hire themselves to the knife owner. Soon he is running a surplus of goods, including houses and equipment which he decides to hire out to others. This surplus is then used to tempt more and more of the other islanders to work for him, exchanging their land in return for the goods he provides. Soon he owns the whole island and never has to work again. His hut is well stocked and extremely luxurious. His workers face the option of following his orders or being fired (i.e. expelled from the island and so back into the water and certain death). Later, he dies and leaves his knife to his son. The woman whose knife it originally was had died long before, childless.


Note that the theft did not involve taking any land. All had equal access to it. It was the initial theft of the knife which provided the man with market power, an edge which allowed him to offer the others a choice between working by themselves or working for him. By working for him they did “benefit” in terms of increased material wealth (and also made the thief better off) but the accumulate impact of unequal exchanges turned them into the effective slaves of the thief.


Now, would it {\bf really} be enough to turn the knife over to the whoever happened to be using it once the theft was discovered (perhaps the thief made a death-bed confession). Even if the woman who had originally taken it from the ship been alive, would the return of the knife {\bf really} make up for the years of work the survivors had put in enriching the the thief or the “voluntary exchanges” which had resulted in the thief owning all the island? The equipment people use, the houses they life in and the food they eat are all the product of many hours of collective work. Does this mean that the transformation of nature which the knife allowed remain in the hands of the descendants of the thief or become the collective property of all? Would dividing it equally between all be fair? Not everyone worked equally hard to produce it. So we have a problem — the result of the initial theft is far greater than the theft considered in isolation due to the productive nature of what was stolen.


In other words, what Rothbard ignores in his attempt to undermine anarchist use of history is that when the property stolen is of a productive nature, the accumulative effect of its use is such as to affect all of society. Productive assets produce {\bf new} property, {\bf new} values, create a {\bf new} balance of class forces, {\bf new} income and wealth inequalities and so on. This is because of the {\bf dynamic} nature of production and human life. When the theft is such that it creates accumulative effects after the initial act, it is hardly enough to say that it does not really matter any more. If a nobleman invests in a capitalist firm with the tribute he extracted from his peasants, then (once the firm starts doing well) sells the land to the peasants and uses that money to expand his capitalist holdings, does that {\bf really} make everything all right? Does not the crime transmit with the cash? After all, the factory would not exist without the prior exploitation of the peasants.


In the case of actually existing capitalism, born as it was of extensive coercive acts, the resultant of these acts have come to shape the {\bf whole} society. For example, the theft of common land (plus the enforcement of property rights — the land monopoly — to vast estates owned by the aristocracy) ensured that working people had no option to sell their labour to the capitalists (rural or urban). The terms of these contracts reflected the weak position of the workers and so capitalists extracted surplus value from workers and used it to consolidate their market position and economic power. Similarly, the effect of mercantilist policies (and protectionism) was to enrich the capitalists at the expense of workers and allow them to build industrial empires.


The accumulative effect of these acts of violation of a “free” market was to create a class society wherein most people “consent” to be wage slaves and enrich the few. While those who suffered the impositions are long gone and the results of the specific acts have multiplied and magnified well beyond their initial form. And we are still living with them. In other words, the initial acts of coercion have been transmitted and transformed by collective activity (wage labour) into society-wide affects.


Rothbard argues in the situation where the descendants (or others) of those who initially tilled the soil and their aggressors ({\em “or those who purchased their claims”}) still extract {\em “tribute from the modern tillers”} that this is a case of {\em “{\bf continuing} aggression against the true owners”}. This means that {\em “the land titles should be transferred to the peasants, without compensation to the monopoly landlords.”} [{\bf Op. Cit.}, p. 65] But what he fails to note is that the extracted “tribute” could have been used to invest in industry and transform society. Why ignore what the “tribute” has been used for? Does stolen property not remain stolen property after it has been transferred to another? And if the stolen property is used to create a society in which one class has to sell their liberty to another, then surely any surplus coming from those exchanges are also stolen (as it was generated directly and indirectly by the theft).


Yes, anarchist agree with Rothbard — peasants should take the land they use but which is owned by another. But this logic can equally be applied to capitalism. Workers are still living with the effects of past initiations of force and capitalists still extract “tribute” from workers due to the unequal bargaining powers within the labour market that this has created. The labour market, after all, was created by state action (directly or indirectly) and is maintained by state action (to protect property rights and new initiations of force by working people). The accumulative effects of stealing productive resources as been to increase the economic power of one class compared to another. As the victims of these past abuses are long gone and attempts to find their descendants meaningless (because of the generalised effects the thefts in question), anarchists feel we are justified in demanding the {\bf {\em “expropriation of the expropriators”}}.


Due to Rothbard’s failure to understand the dynamic and generalising effects that result from the theft of productive resources (i.e. externalities that occur from coercion of one person against a specific set of others) and the creation of a labour market, his attempt to refute anarchist analysis of the history of “actually existing capitalism” also fails. Society is the product of collective activity and should belong to us all (although whether and how we divide it up is another question).


\section{9 Is Medieval Iceland an example of “anarcho”-capitalism working in practice?
}

Ironically, medieval Iceland is a good example of why “anarcho”-capitalism will {\bf not} work, degenerating into de facto rule by the rich. It should be pointed out first that Iceland, nearly 1,000 years ago, was not a capitalistic system. In fact, like most cultures claimed by “anarcho”-capitalists as examples of their “utopia,” it was a communal, not individualistic, society, based on artisan production, with extensive communal institutions as well as individual “ownership” (i.e. use) and a form of social self-administration, the {\bf thing} — both local and Iceland-wide — which can be considered a “primitive” form of the anarchist communal assembly.


As William Ian Miller points out {\em “[p]eople of a communitarian nature\unknown{} have reason to be attracted [to Medieval Iceland]\unknown{} the limited role of lordship, the active participation of large numbers of free people \unknown{} in decision making within and without the homestead. The economy barely knew the existence of markets. Social relations preceded economic relations. The nexus of household, kin, Thing, even enmity, more than the nexus of cash, bound people to each other. The lack of extensive economic differentiation supported a weakly differentiated class system \unknown{} [and material] deprivations were more evenly distributed than they would be once state institutions also had to be maintained.”} [{\bf Bloodtaking and Peacemaking: Feud, Law and Society in Saga Iceland}, p. 306]


At this time Iceland {\em “remained entirely rural. There were no towns, not even villages, and early Iceland participated only marginally in the active trade of Viking Age Scandinavia.”} There was a {\em “diminished level of stratification, which emerged from the first phase of social and economic development, lent an appearance of egalitarianism — social stratification was restrained and political hierarchy limited.”} [Jesse Byock, {\bf Viking Age Iceland}, p. 2] That such a society could be classed as “capitalist” or even considered a model for an advanced industrial society is staggering.


Kropotkin in {\bf Mutual Aid} indicates that Norse society, from which the settlers in Iceland came, had various “mutual aid” institutions, including communal land ownership (based around what he called the {\em “village community”}) and the {\bf thing} (see also Kropotkin’s {\bf The State: Its Historic Role} for a discussion of the “village community”). It is reasonable to think that the first settlers in Iceland would have brought such institutions with them and Iceland did indeed have its equivalent of the commune or “village community,” the {\bf Hreppar}, which developed early in the country’s history. Like the early local assemblies, it is not much discussed in the Sagas but is mentioned in the law book, the Grágás, and was composed of a minimum of twenty farms and had a five member commission. The Hreppar was self-governing and, among other things, was responsible for seeing that orphans and the poor within the area were fed and housed. The Hreppar also served as a property insurance agency and assisted in case of fire and losses due to diseased livestock.


In addition, as in most pre-capitalist societies, there were “commons”, common land available for use by all. During the summer, {\em “common lands and pastures in the highlands, often called {\bf almenning}, were used by the region’s farmers for grazing.”} This increased the independence of the population from the wealthy as these {\em “public lands offered opportunities for enterprising individuals to increase their store of provisions and to find saleable merchandise.”} [Jesse Byock, {\bf Op. Cit.}, p. 47 and p. 48]


Thus Icelandic society had a network of solidarity, based upon communal life:



\startblockquote
{\em  “The status of farmers as free agents was reinforced by the presence of communal units called {\bf hreppar} (sing. {\bf hreppr}) \unknown{} these [were] geographically defined associations of landowners\unknown{} the {\bf hreppr} were self-governing \unknown{}[and] guided by a five-member steering committee \unknown{} As early as the 900s, the whole country seems to have been divided into {\bf hreppar} \unknown{} {\bf Hreppar} provided a blanket of local security, allowing the landowning farmers a measure of independence to participate in the choices of political life \unknown{}}


{\em “Through copoperation among their members, {\bf hreppar} organised and controlled summer grazing lands, organised communal labour, and provided an immediate local forum for settling disputes. Crucially, they provided fire and livestock insurance for local farmers\unknown{} [They also] saw to the feeding and housing of local orphans, and administered poor relief to people who were recognised as inhabitants of their area. People who could not provide for themselves were assigned to member farms, which took turns in providing for them.”} [Byock, {\bf Op. Cit.}, pp. 137–8]



\stopblockquote
In practice this meant that {\em “each commune was a mutual insurance company, or a miniature welfare state. And membership in the commune was not voluntary. Each farmer had to belong to the commune in which his farm was located and to contribute to its needs.”} [Gissurarson quoted by Birgit T. Runolfsson Solvason, {\bf Ordered Anarchy, State and Rent-Seeking: The Icelandic Commonwealth, 930–1262}] The Icelandic Commonwealth did not allow farmers {\bf not} to join its communes and {\em “[o]nce attached to the local {\bf hreppr}, a farm’s affliation could not be changed.”} However, they did play a key role in keeping the society free as the {\bf hreppr} {\em “was essentially non-political and addressed subsistence and economic security needs. Its presence freed farmers from depending on an overclass to provide comparable services or corresponding security measures.”} [Byock, {\bf Op. Cit.}, p. 138]


Therefore, the Icelandic Commonwealth can hardly be claimed in any significant way as an example of “anarcho”-capitalism in practice. This can also be seen from the early economy, where prices were subject to popular judgement at the {\bf skuldaping} ({\em “payment-thing”}) {\bf not} supply and demand. [Kirsten Hastrup, {\bf Culture and History in Medieval Iceland}, p. 125] Indeed, with its communal price setting system in local assemblies, the early Icelandic commonwealth was more similar to Guild Socialism (which was based upon guild’s negotiating “just prices” for goods and services) than capitalism. Therefore Miller correctly argues that it would be wrong to impose capitalist ideas and assumptions onto Icelandic society:



\startblockquote
{\em “Inevitably the attempt was made to add early Iceland to the number of regions that socialised people in nuclear families within simple households\unknown{} what the sources tell us about the shape of Icelandic householding must compel a different conclusion.”} [{\bf Op. Cit.}, p. 112]



\stopblockquote
In other words, Kropotkin’s analysis of communal society is far closer to the reality of Medieval Iceland than “anarcho”-capitalist attempts to turn it into a some kind of capitalist utopia.


However, the communal nature of Icelandic society also co-existed (as in most such cultures) with hierarchical institutions, including some with capitalistic elements, namely private property and “private states” around the local {\bf godar.} The godar were local chiefs who also took the role of religious leaders. As the {\bf Encyclopaedia Britannica} explains, {\em “a kind of local government was evolved [in Iceland] by which the people of a district who had most dealings together formed groups under the leadership of the most important or influential man in the district”} (the godi). The godi {\em “acted as judge and mediator”} and {\em “took a lead in communal activities”} such as building places of worship. These {\em “local assemblies\unknown{} are heard of before the establishment of the althing”} (the national thing). This althing led to co-operation between the local assemblies.


Thus Icelandic society had different elements, one based on the local chiefs and communal organisations. Society was marked by inequalities as {\em “[a]mong the landed there were differences in wealth and prominence. Distinct cleavages existed between landowners and landless people and between free men and slaves.”} This meant it was {\em “marked by aspects of statelessness and egalitarianism as well as elements of social hierarchy \unknown{} Although Iceland was not a democratic system, proto-democratic tendencies existed.”} [Byock, {\bf Op. Cit.}, p. 64 and p. 65] The Icelandic social system was designed to reduce the power of the wealthy by enhancing communal institutions:



\startblockquote
{\em “The society \unknown{} was based on a system of decentralised self-government \unknown{} The Viking Age settlers began by establishing local things, or assemblies, which had been the major forum for meetings of freemen and aristocrats in the old Scandinavian and Germanic social order\unknown{} They [the Icelanders] excluded overlords with coercive power and expended the mandate of the assembly to fill the full spectrum of the interests of the landed free farmers. The changes transformed a Scandinavian decision-making body that mediated between freemen and overlords into an Icelandic self-contained governmental system without overlords. At the core of Icelandic government was the Althing, a national assembly of freemen.”} [Byock, {\bf Op. Cit.}, p. 75]



\stopblockquote
Therefore we see communal self-management in a basic form, {\bf plus} co-operation between communities as well. These communistic, mutual-aid features exist in many non-capitalist cultures and are often essential for ensuring the people’s continued freedom within those cultures ( section B.2.5 on why the wealthy undermine these popular {\em “folk-motes”} in favour of centralisation). Usually, the existence of private property (and so inequality) soon led to the destruction of communal forms of self-management (with participation by all male members of the community as in Iceland), which are replaced by the rule of the rich.


While such developments are a commonplace in most “primitive” cultures, the Icelandic case has an unusual feature which explains the interest it provokes in “anarcho”-capitalist circles. This feature was that individuals could seek protection from any godi. As the {\bf Encyclopaedia Britannica} puts it, {\em “the extent of the godord [chieftancy] was not fixed by territorial boundaries. Those who were dissatisfied with their chief could attach themselves to another godi\unknown{} As a result rivalry arose between the godar [chiefs]; as may be seen from the Icelandic Sagas.”} This was because, while there were {\em “a central legislature and uniform, country-wide judicial and legal systems,”} people would seek the protection of any godi, providing payment in return. [Byock, {\bf Op. Cit.}, p. 2] These godi, in effect, would be subject to “market forces,” as dissatisfied individuals could affiliate themselves to other godi. This system, however, had an obvious (and fatal) flaw. As the {\bf Encyclopaedia Britannica} points out:



\startblockquote
{\em “The position of the godi could be bought and sold, as well as inherited; consequently, with the passing of time, the godord for large areas of the country became concentrated in the hands of one man or a few men. This was the principal weakness of the old form of government: it led to a struggle of power and was the chief reason for the ending of the commonwealth and for the country’s submission to the King of Norway.”}



\stopblockquote
It was the existence of these hierarchical elements in Icelandic society that explain its fall from anarchistic to statist society. As Kropotkin argued {\em “from chieftainship sprang on the one hand the State and on the other {\bf private} property.”} [{\bf Act for Yourselves}, p. 85] Kropotkin’s insight that chieftainship is a transitional system has been confirmed by anthropologists studying “primitive” societies. They have come to the conclusion that societies made up of chieftainships or chiefdoms are not states: {\em “Chiefdoms are neither stateless nor state societies in the fullest sense of either term: they are on the borderline between the two. Having emerged out of stateless systems, they give the impression of being on their way to centralised states and exhibit characteristics of both.”} [Y. Cohen quoted by Birgit T. Runolfsson Solvason, {\bf Op. Cit.}] Since the Commonwealth was made up of chiefdoms, this explains the contradictory nature of the society — it was in the process of transition, from anarchy to statism, from a communal economy to one based on private property.


The {\bf political} transition within Icelandic society went hand in hand with an {\bf economic} transition (both tendencies being mutually reinforcing). Initially, when Iceland was settled, large-scale farming based on extended households with kinsmen was the dominant economic mode. This semi-communal mode of production changed as the land was divided up (mostly through inheritance claims) between the 10\high{th} and 11\high{th} centuries. This new economic system based upon individual {\bf possession} and artisan production was then slowly displaced by tenant farming, in which the farmer worked for a landlord, starting in the late 11\high{th} century. This economic system (based on tenant farming, i.e. capitalistic production) ensured that {\em “great variants of property and power emerged.”} [Kirsten Hastrup, {\bf Culture and History in Medieval Iceland}, pp. 172–173]


So significant changes in society started to occur in the eleventh century, as {\em “slavery all but ceased. Tenant farming \unknown{} took [its] place.”} Iceland was moving from an economy based on {\bf possession} to one based on {\bf private property} and so {\em “the renting of land was a widely established practice by the late eleventh century \unknown{} the status of the {\bf godar} must have been connected with landownership and rents.”} This lead to increasing oligarchy and so the mid- to late-twelfth century was {\em “characterised by the appearance of a new elite, the big chieftains who are called storgodar \unknown{} [who] struggled from the 1220s to the 1260s to win what had earlier been unobtainable for Icelandic leaders, the prize of overlordship or centralised executive authority.”} [Byock, {\bf Op. Cit.}, p. 269 and pp. 3–4]


During this evolution in ownership patterns and the concentration of wealth and power into the hands of a few, we should note that the godi’s and wealthy landowners’ attitude to profit making also changed, with market values starting to replace those associated with honour, kin, and so on. Social relations became replaced by economic relations and the nexus of household, kin and Thing was replaced by the nexus of cash and profit. The rise of capitalistic social relationships in production and values within society was also reflected in exchange, with the local marketplace, with its pricing {\em “subject to popular judgement”} being {\em “subsumed under central markets.”} [Hastrup, {\bf Op. Cit.}, p. 225]


With a form of wage labour (tenant farming) being dominant within society, it is not surprising that great differences in wealth started to appear. Also, as protection did not come free, it is not surprising that a godi tended to become rich also (in Kropotkin’s words, {\em “the individual accumulation of wealth and power”}). Powerful godi would be useful for wealthy landowners when disputes over land and rent appeared, and wealthy landowners would be useful for a godi looking for income. Concentrations of wealth, in other words, produce concentrations of social and political power (and vice versa) — {\em “power always follows wealth.”} [Kropotkin, {\bf Mutual Aid}, p. 131]


The transformation of {\bf possession} into {\bf property} and the resulting rise of hired labour was a {\bf key} element in the accumulation of wealth and power, and the corresponding decline in liberty among the farmers. Moreover, with hired labour springs dependency — the worker is now dependent on good relations with their landlord in order to have access to the land they need. With such reductions in the independence of part of Icelandic society, the undermining of self-management in the various Things was also likely as labourers could not vote freely as they could be subject to sanctions from their landlord for voting the “wrong” way ({\em “The courts were less likely to base judgements on the evidence than to adjust decisions to satisfy the honour and resources of powerful individuals.”} [Byock, {\bf Op. Cit.}, p. 185]).. Thus hierarchy within the economy would spread into the rest of society, and in particular its social institutions, reinforcing the effects of the accumulation of wealth and power.


The resulting classification of Icelandic society played a key role in its move from relative equality and anarchy to a class society and statism. As Millar points out:



\startblockquote
{\em “as long as the social organisation of the economy did not allow for people to maintain retinues, the basic egalitarian assumptions of the honour system\unknown{} were reflected reasonably well in reality\unknown{} the mentality of hierarchy never fully extricated itself from the egalitarian ethos of a frontier society created and recreated by juridically equal farmers. Much of the egalitarian ethic maintained itself even though it accorded less and less with economic realities\unknown{} by the end of the commonwealth period certain assumptions about class privilege and expectations of deference were already well enough established to have become part of the lexicon of self-congratulation and self-justification.”} [{\bf Op. Cit.}, pp. 33–4]



\stopblockquote
This process in turn accelerated the destruction of communal life and the emergence of statism, focused around the godord. In effect, the godi and wealthy farmers became rulers of the country. Political changes simply reflected economic changes from a communalistic, anarchistic society to a statist, propertarian one. Ironically, this process was a natural aspect of the system of competing chiefs recommended by “anarcho”-capitalists:



\startblockquote
{\em  “In the twelfth and thirteenth centuries Icelandic society experienced changes in the balance of power. As part of the evolution to a more stratified social order, the number of chieftains diminished and the power of the remaining leaders grew. By the thirteenth century six large families had come to monopolise the control and ownership of many of the original chieftaincies.”} [Byock, {\bf Op. Cit.}, p. 341]



\stopblockquote
These families were called {\bf storgodar} and they {\em “gained control over whole regions.”} This process was not imposed, as {\em “the rise in social complexity was evolutionary rather than revolutionary \unknown{} they simply moved up the ladder.”} This political change reflected economic processes, for {\em “[a]t the same time other social transformations were at work. In conjunction with the development of the {\bf storgadar} elite, the most successful among the {\bf baendr} [farmers] also moved up a rung on the social ladder, being ‘big farmers’ or {\bf Storbaendr}”} [{\bf Op. Cit.}, p. 342] Unsurprisingly, it was the rich farmers who initiated the final step towards normal statism and by the 1250s the {\bf storbaendr} and their followers had grown weary of the {\bf storgodar} and their quarrels. In the end they accepted the King of Norway’s offer to become part if his kingdom.


The obvious conclusion is that as long as Iceland was not capitalistic, it was anarchic and as it became more capitalistic, it became more statist.


This process, wherein the concentration of wealth leads to the destruction of communal life and so the anarchistic aspects of a given society, can be seen elsewhere, for example, in the history of the United States after the Revolution or in the degeneration of the free cities of Medieval Europe. Peter Kropotkin, in his classic work {\bf Mutual Aid}, documents this process in some detail, in many cultures and time periods. However, that this process occurred in a society which is used by “anarcho”-capitalists as an example of their system in action reinforces the anarchist analysis of the statist nature of “anarcho”-capitalism and the deep flaws in its theory, as discussed in section 6.


As Miller argues, {\em “[i]t is not the have-nots, after all, who invented the state. The first steps toward state formation in Iceland were made by churchmen\unknown{} and by the big men content with imitating Norwegian royal style. Early state formation, I would guess, tended to involve redistributions, not from rich to poor, but from poor to rich, from weak to strong.”} [{\bf Op. Cit.}, p. 306]


The “anarcho”-capitalist argument that Iceland was an example of their ideology working in practice is derived from the work of David Friedman. Friedman is less gun-ho than many of his followers, arguing in {\bf The Machinery of Freedom}, that Iceland only had some features of an “anarcho”-capitalist society and these provide some evidence in support of his ideology. How a pre-capitalist society can provide any evidence to support an ideology aimed at an advanced industrial and urban economy is hard to say as the institutions of that society cannot be artificially separated from its social base. Ironically, though, it does present some evidence against “anarcho”-capitalism precisely because of the rise of capitalistic elements within it.


Friedman is aware of how the Icelandic Republic degenerated and its causes. He states in a footnote in his 1979 essay {\em “Private Creation and Enforcement of Law: A Historical Case”} that the {\em “question of why the system eventually broke down is both interesting and difficult. I believe that two of the proximate causes were increased concentration of wealth, and hence power, and the introduction into Iceland of a foreign ideology — kingship. The former meant that in many areas all or most of the godord were held by one family and the latter that by the end of the Sturlung period the chieftains were no longer fighting over the traditional quarrels of who owed what to whom, but over who should eventually rule Iceland. The ultimate reasons for those changes are beyond the scope of this paper.”}


However, from an anarchist point of view, the “foreign” ideology of kingship would be the {\bf product} of changing socio-economic conditions that were expressed in the increasing concentration of wealth and not its cause. After all, the settlers of Iceland were well aware of the “ideology” of kingship for the 300 years during which the Republic existed. As Byock notes, Iceland {\em “inherited the tradition and the vocabulary of statehood from its European origins \unknown{} On the mainland, kings were enlarging their authority at the expense of the traditional rights of free farmers. The emigrants to Iceland were well aware of this process \unknown{} available evidence does suggest that the early Icelanders knew quite well what they did not want. In particular they were collectively opposed to the centralising aspects of a state.”} [{\bf Op. Cit.}, p. 64–6] Unless some kind of collective and cultural amnesia occurred, the notion of a “foreign ideology” causing the degeneration is hard to accept. Moreover, only the concentration of wealth allowed would-be Kings the opportunity to develop and act and the creation of boss-worker social relationships on the land made the poor subject to, and familiar with, the concept of authority. Such familiarity would spread into all aspects of life and, combined with the existence of “prosperous” (and so powerful) godi to enforce the appropriate servile responses, ensured the end of the relative equality that fostered Iceland’s anarchistic tendencies in the first place.


In addition, as private property is a monopoly of rulership over a given area, the conflict between chieftains for power was, at its most basic, a conflict of who would {\bf own} Iceland, and so rule it. The attempt to ignore the facts that private property creates rulership (i.e. a monopoly of government over a given area) and that monarchies are privately owned states does Friedman’s case no good. In other words, the system of private property has a built in tendency to produce both the ideology and fact of Kingship — the power structures implied by Kingship are reflected in the social relations which are produced by private property.


Friedman is also aware that an {\em “objection [to his system] is that the rich (or powerful) could commit crimes with impunity, since nobody would be able to enforce judgement against them. Where power is sufficiently concentrated this might be true; this was one of the problems which led to the eventual breakdown of the Icelandic legal system in the thirteenth century. But so long as power was reasonably dispersed, as it seem to have been for the first two centuries after the system was established, this was a less serious problem.”} [{\bf Op. Cit.}]


Which is quite ironic. Firstly, because the first two centuries of Icelandic society was marked by {\bf non-capitalist} economic relations (communal pricing and family/individual possession of land). Only when capitalistic social relationships developed (hired labour and property replacing possession and market values replacing social ones) in the 12\high{th} century did power become concentrated, leading to the breakdown of the system in the 13\high{th} century. Secondly, because Friedman is claiming that “anarcho”-capitalism will only work if there is an approximate equality within society! But this state of affairs is one most “anarcho”-capitalists claim is impossible and undesirable!


They claim there will {\bf always} be rich and poor. But inequality in wealth will also become inequality of power. When “actually existing” capitalism has become more free market the rich have got richer and the poor poorer. Apparently, according to the “anarcho”-capitalists, in an even “purer” capitalism this process will be reversed! It is ironic that an ideology that denounces egalitarianism as a revolt against nature implicitly requires an egalitarian society in order to work.


In reality, wealth concentration is a fact of life in {\bf any} system based upon hierarchy and private property. Friedman is aware of the reasons why “anarcho”-capitalism will become rule by the rich but prefers to believe that “pure” capitalism will produce an egalitarian society! In the case of the commonwealth of Iceland this did not happen — the rise in private property was accompanied by a rise in inequality and this lead to the breakdown of the Republic into statism.


In short, Medieval Iceland nicely illustrates David Weick’s comments (as quoted in section 6.3) that {\em “when private wealth is uncontrolled, then a police-judicial complex enjoying a clientele of wealthy corporations whose motto is self-interest is hardly an innocuous social force controllable by the possibility of forming or affiliating with competing ‘companies.’”} This is to say that “free market” justice soon results in rule by the rich, and being able to affiliate with “competing” “defence companies” is insufficient to stop or change that process.


This is simply because any defence-judicial system does not exist in a social vacuum. The concentration of wealth — a natural process under the “free market” (particularly one marked by private property and wage labour) — has an impact on the surrounding society. Private property, i.e. monopolisation of the means of production, allows the monopolists to become a ruling elite by exploiting, and so accumulating vastly more wealth than, the workers. This elite then uses its wealth to control the coercive mechanisms of society (military, police, “private security forces,” etc.), which it employs to protect its monopoly and thus its ability to accumulate ever more wealth and power. Thus, private property, far from increasing the freedom of the individual, has always been the necessary precondition for the rise of the state and rule by the rich. Medieval Iceland is a classic example of this process at work.


\section{10 Would laissez-faire capitalism be stable?
}

Unsurprisingly, right-libertarians combine their support for “absolute property rights” with a whole-hearted support for laissez-faire capitalism. In such a system (which they maintain, to quote Ayn Rand, is an {\em “unknown ideal”}) everything would be private property and there would be few (if any) restrictions on “voluntary exchanges.” “Anarcho”-capitalists are the most extreme of defenders of pure capitalism, urging that the state itself be privatised and no voluntary exchange made illegal (for example, children would be considered the property of their parents and it would be morally right to turn them into child prostitutes — the child has the option of leaving home if they object).


As there have been no example of “pure” capitalism it is difficult to say whether their claims about are true (for a discussion of a close approximation see the section 10.3). This section of the FAQ is an attempt to discover whether such a system would be stable or whether it would be subject to the usual booms and slumps. Before starting we should note that there is some disagreement within the right-libertarian camp itself on this subject (although instead of stability they usually refer to “equilibrium” — which is an economics term meaning that all of a societies resources are fully utilised).


In general terms, most right-Libertarians’ reject the concept of equilibrium as such and instead stress that the economy is inherently a dynamic (this is a key aspect of the Austrian school of economics). Such a position is correct, of course, as such noted socialists as Karl Marx and Michal Kalecki and capitalist economists as Keynes recognised long ago. There seems to be two main schools of thought on the nature of disequilibrium. One, inspired by von Mises, maintains that the actions of the entrepreneur/capitalist results in the market co-ordinating supply and demand and another, inspired by Joseph Schumpeter, who question whether markets co-ordinate because entrepreneurs are constantly innovating and creating new markets, products and techniques.


Of course both actions happen and we suspect that the differences in the two approaches are not important. The important thing to remember is that “anarcho”-capitalists and right-libertarians in general reject the notion of equilibrium — but when discussing their utopia they do not actually indicate this! For example, most “anarcho”-capitalists will maintain that the existence of government (and/or unions) causes unemployment by either stopping capitalists investing in new lines of industry or forcing up the price of labour above its market clearing level (by, perhaps, restricting immigration, minimum wages, taxing profits). Thus, we are assured, the worker will be better off in “pure” capitalism because of the unprecedented demand for labour it will create. However, full employment of labour is an equilibrium in economic terms and that, remember, is impossible due to the dynamic nature of the system. When pressed, they will usually admit there will be periods of unemployment as the market adjusts or that full unemployment actually means under a certain percentage of unemployment. Thus, if you (rightly) reject the notion of equilibrium you also reject the idea of full employment and so the labour market becomes a buyers market and labour is at a massive disadvantage.


The right-libertarian case is based upon logical deduction, and the premises required to show that laissez-faire will be stable are somewhat incredible. If banks do not set the wrong interest rate, if companies do not extend too much trade credit, if workers are willing to accept (real wage related) pay cuts, if workers altruistically do not abuse their market power in a fully employed society, if interest rates provide the correct information, if capitalists predict the future relatively well, if banks and companies do not suffer from isolation paradoxes, then, perhaps, laissez-faire will be stable.


So, will laissez-faire capitalism be stable? Let us see by analysing the assumptions of right-libertarianism — namely that there will be full employment and that a system of private banks will stop the business cycle. We will start on the banking system first (in section 10.1) followed by the effects of the labour market on economic stability (in section 10.2). Then we will indicate, using the example of 19\high{th} century America, that actually existing (“impure”) laissez-faire was very unstable.


Explaining booms and busts by state action plays an ideological convenience as it exonerates market processes as the source of instability within capitalism. We hope to indicate in the next two sections why the business cycle is inherent in the system (see also sections C.7, C.8 and C.9).


\subsection{10.1 Would privatising banking make capitalism stable?
}

It is claimed that the existence of the state (or, for minimal statists, government policy) is the cause of the business cycle (recurring economic booms and slumps). This is because the government either sets interest rates too low or expands the money supply (usually by easing credit restrictions and lending rates, sometimes by just printing fiat money). This artificially increases investment as capitalists take advantage of the artificially low interest rates. The real balance between savings and investment is broken, leading to over-investment, a drop in the rate of profit and so a slump (which is quite socialist in a way, as many socialists also see over-investment as the key to understanding the business cycle, although they obviously attribute the slump to different causes — namely the nature of capitalist production, not that the credit system does not play its part — see section C.7).


In the words of Austrian Economist W. Duncan Reekie, {\em “[t]he business cycle is generated by monetary expansion and contraction \unknown{} When new money is printed it appears as if the supply of savings has increased. Interest rates fall and businessmen are misled into borrowing additional founds to finance extra investment activity \unknown{} This would be of no consequence if it had been the outcome of [genuine saving] \unknown{} -but the change was government induced. The new money reaches factor owners in the form of wages, rent and interest \unknown{} the factor owners will then spend the higher money incomes in their existing consumption:investment proportions \unknown{} Capital goods industries will find their expansion has been in error and malinvestments have been inoccured.”} [{\bf Markets, Entrepreneurs and Liberty}, pp. 68–9]


In other words, there has been {\em “wasteful mis-investment due to government interference with the market.”} [{\bf Op. Cit.}, p. 69] In response to this (negative) influence in the workings of the market, it is suggested by right-libertarians that a system of private banks should be used and that interest rates are set by them, via market forces. In this way an interest rate that matches the demand and supply for savings will be reached and the business cycle will be no more. By truly privatising the credit market, it is hoped by the business cycle will finally stop.


Unsurprisingly, this particular argument has its weak points and in this section of the FAQ we will try to show exactly why this theory is wrong.


Let us start with Reckie’s starting point. He states that the {\em “main problem”} of the slump is {\em “why is there suddenly a ’{\bf cluster}’ of business errors? Businessmen and entrepreneurs are market experts (otherwise they would not survive) and why should they all make mistakes simultaneously?”} [{\bf Op. Cit.}, p. 68] It is this {\em “cluster”} of mistakes that the Austrians’ take as evidence that the business cycle comes from outside the workings of the market (i.e. is exogenous in nature). Reekie argues that an {\em “error cluster only occurs when all entrepreneurs have received the wrong signals on potential profitability, and all have received the signals simultaneously through government interference with the money supply.”} [{\bf Op. Cit.}, p. 74] But is this {\bf really} the case?


The simple fact is that groups of (rational) individuals can act in the same way based on the same information and this can lead to a collective problem. For example, we do not consider it irrational that everyone in a building leaves it when the fire alarm goes off and that the flow of people can cause hold-ups at exits. Neither do we think that its unusual that traffic jams occur, after all those involved are all trying to get to work (i.e. they are reacting to the same desire). Now, is it so strange to think that capitalists who all see the same opportunity for profit in a specific market decide to invest in it? Or that the aggregate outcome of these individually rational decisions may be irrational (i.e. cause a glut in the market)?


In other words, a “cluster” of business failures may come about because a group of capitalists, acting in isolation, over-invest in a given market. They react to the same information (namely super profits in market X), arrange loans, invest and produce commodities to meet demand in that market. However, the aggregate result of these individually rational actions is that the aggregate supply far exceeds demand, causing a slump in that market and, perhaps, business failures. The slump in this market (and the potential failure of some firms) has an impact on the companies that supplied them, the companies that are dependent on their employees wages/demand, the banks that supplied the credit and so forth. The accumulative impact of this slump (or failures) on the chain of financial commitments of which they are but one link can be large and, perhaps, push an economy into general depression. Thus the claim that it is something external to the system that causes depression is flawed.


It could be claimed the interest rate is the problem, that it does not accurately reflect the demand for investment or relate it to the supply of savings. But, as we argued in section C.8, it is not at all clear that the interest rate provides the necessary information to capitalists. They need investment information for their specific industry, but the interest rate is cross-industry. Thus capitalists in market X do not know if the investment in market X is increasing and so this lack of information can easily cause “mal-investment” as over-investment (and so over-production) occurs. As they have no way of knowing what the investment decisions of their competitors are or now these decisions will affect an already unknown future, capitalists may over-invest in certain markets and the net effects of this aggregate mistake can expand throughout the whole economy and cause a general slump. In other words, a cluster of business failures can be accounted for by the workings of the market itself and {\bf not} the (existence of) government.


This is {\bf one} possible reason for an internally generated business cycle but that is not the only one. Another is the role of class struggle which we discuss in the next section and yet another is the endogenous nature of the money supply itself. This account of money (proposed strongly by, among others, the post-Keynesian school) argues that the money supply is a function of the demand for credit, which itself is a function of the level of economic activity. In other words, the banking system creates as much money as people need and any attempt to control that creation will cause economic problems and, perhaps, crisis (interestingly, this analysis has strong parallels with mutualist and individualist anarchist theories on the causes of capitalist exploitation and the business cycle). Money, in other words, emerges from {\bf within} the system and so the right-libertarian attempt to “blame the state” is simply wrong.


Thus what is termed “credit money” (created by banks) is an essential part of capitalism and would exist without a system of central banks. This is because money is created from within the system, in response to the needs of capitalists. In a word, money is endogenous and credit money an essential part of capitalism.


Right-libertarians do not agree. Reekie argues that {\em “[o]nce fractional reserve banking is introduced, however, the supply of money substitutes will include fiduciary media. The ingenuity of bankers, other financial intermediaries and the endorsement and {\bf guaranteeing of their activities by governments and central banks} has ensured that the quantity of fiat money is immense.”} [{\bf Op. Cit.}, p. 73]


Therefore, what “anarcho”-capitalists and other right-libertarians seem to be actually complaining about when they argue that “state action” creates the business cycle by creating excess money is that the state {\bf allows} bankers to meet the demand for credit by creating it. This makes sense, for the first fallacy of this sort of claim is how could the state {\bf force} bankers to expand credit by loaning more money than they have savings. And this seems to be the normal case within capitalism — the central banks accommodate bankers activity, they do not force them to do it. Alan Holmes, a senior vice president at the New York Federal Reserve, stated that:



\startblockquote
{\em “In the real world, banks extend credit, creating deposits in the process, and look for the reserves later. The question then becomes one of whether and how the Federal Reserve will accommodate the demand for reserves. In the very short run, the Federal Reserve has little or no choice about accommodating that demand, over time, its influence can obviously be felt.”} [quoted by Doug Henwood, {\bf Wall Street}, p. 220]



\stopblockquote
(Although we must stress that central banks are {\bf not} passive and do have many tools for affecting the supply of money. For example, central banks can operate “tight” money policies which can have significant impact on an economy and, via creating high enough interest rates, the demand for money.)


It could be argued that because central banks exist, the state creates an “environment” which bankers take advantage off. By not being subject to “free market” pressures, bankers could be tempted to make more loans than they would otherwise in a “pure” capitalist system (i.e. create credit money). The question arises, would “pure” capitalism generate sufficient market controls to stop banks loaning in excess of available savings (i.e. eliminate the creation of credit money/fiduciary media).


It is to this question we now turn.


As noted above, the demand for credit is generated from {\bf within} the system and the comments by Holmes reinforce this. Capitalists seek credit in order to make money and banks create it precisely because they are also seeking profit. What right-libertarians actually object to is the government (via the central bank) {\bf accommodating} this creation of credit. If only the banks could be forced to maintain a savings to loans ration of one, then the business cycle would stop. But is this likely? Could market forces ensure that bankers pursue such a policy? We think not — simply because the banks are profit making institutions. As post-Keynesianist Hyman Minsky argues, {\em “[b]ecause bankers live in the same expectational climate as businessmen, profit-seeking bankers will find ways of accommodating their customers\unknown{} Banks and bankers are not passive managers of money to lend or to invest; they are in business to maximise profits\unknown{}”} [quoted by L. Randall Wray, {\bf Money and Credit in Capitalist Economies}, p. 85]


This is recognised by Reekie, in passing at least (he notes that {\em “fiduciary media could still exist if bankers offered them and clients accepted them”} [{\bf Op. Cit.}, p. 73]). Bankers will tend to try and accommodate their customers and earn as much money as possible. Thus Charles P. Kindleberger comments that monetary expansion {\em “is systematic and endogenous rather than random and exogenous”} seem to fit far better the reality of capitalism that the Austrian and right-libertarian viewpoint [{\bf Manias, Panics, and Crashes}, p. 59] and post-Keynesian L. Randall Wray argues that {\em “the money supply \unknown{} is more obviously endogenous in the monetary systems which predate the development of a central bank.”} [{\bf Op. Cit.}, p. 150]


In other words, the money supply cannot be directly controlled by the central bank since it is determined by private decisions to enter into debt commitments to finance spending. Given that money is generated from {\bf within} the system, can market forces ensure the non-expansion of credit (i.e. that the demand for loans equals the supply of savings)? To begin to answer this question we must note that investment is {\em “essentially determined by expected profitability.”} [Philip Arestis, {\bf The Post-Keynesian Approach to Economics}, p. 103] This means that the actions of the banks cannot be taken in isolation from the rest of the economy. Money, credit and banks are an essential part of the capitalist system and they cannot be artificially isolated from the expectations, pressures and influences of that system.


Let us assume that the banks desire to maintain a loans to savings ratio of one and try to adjust their interest rates accordingly. Firstly, changes in the rate of interest {\em “produce only a very small, if any, movement in business investment”} according to empirical evidence [{\bf Op. Cit.}, pp. 82–83] and that {\em “the demand for credit is extremely inelastic with respect to interest rates.”} [L. Randall Wray, {\bf Op. Cit.}, p. 245] Thus, to keep the supply of savings in line with the demand for loans, interest rates would have to increase greatly (indeed, trying to control the money supply by controlling the monetary bases in this way will only lead to very big fluctuations in interest rates). And increasing interest rates has a couple of paradoxical effects.


According to economists Joseph Stiglitz and Andrew Weiss (in {\em “Credit Rationing in Markets with Imperfect Knowledge”}, {\bf American Economic Review}, no. 71, pp. 393–410) interest rates are subject to what is called the {\em “lemons problem”} (asymmetrical information between buyer and seller). Stiglitz and Weiss applied the “lemons problem” to the credit market and argued (and unknowingly repeated Adam Smith) that at a given interest rate, lenders will earn lower return by lending to bad borrowers (because of defaults) than to good ones. If lenders try to increase interest rates to compensate for this risk, they may chase away good borrowers, who are unwilling to pay a higher rate, while perversely not chasing away incompetent, criminal, or malignantly optimistic borrowers. This means that an increase in interest rates may actually increase the possibilities of crisis, as more loans may end up in the hands of defaulters.


This gives banks a strong incentive to keep interest rates lower than they otherwise could be. Moreover, {\em “increases in interest rates make it more difficult for economic agents to meet their debt repayments”} [Philip Arestis, {\bf Op. Cit.}, pp. 237–8] which means when interest rates {\bf are} raised, defaults will increase and place pressures on the banking system. At high enough short-term interest rates, firms find it hard to pay their interest bills, which cause/increase cash flow problems and so {\em “[s]harp increases in short term interest rates \unknown{}leads to a fall in the present value of gross profits after taxes (quasi-rents) that capital assets are expected to earn.”} [Hyman Minsky, {\bf Post-Keynesian Economic Theory}, p. 45]


In addition, {\em “production of most investment goods is undertaken on order and requires time for completion. A rise in interest rates is not likely to cause firms to abandon projects in the process of production \unknown{} This does not mean \unknown{} that investment is completely unresponsive to interest rates. A large increase in interest rates causes a ‘present value reversal’, forcing the marginal efficiency of capital to fall below the interest rate. If the long term interest rate is also pushed above the marginal efficiency of capital, the project may be abandoned.”} [Wray, {\bf Op. Cit.}, pp. 172–3] In other words, investment takes {\bf time} and there is a lag between investment decisions and actual fixed capital investment. So if interest rates vary during this lag period, initially profitable investments may become white elephants.


As Michal Kalecki argued, the rate of interest must be lower than the rate of profit otherwise investment becomes pointless. The incentive for a firm to own and operate capital is dependent on the prospective rate of profit on that capital relative to the rate of interest at which the firm can borrow at. The higher the interest rate, the less promising investment becomes.


If investment is unresponsive to all but very high interest rates (as we indicated above), then a privatised banking system will be under intense pressure to keep rates low enough to maintain a boom (by, perhaps, creating credit above the amount available as savings). And if it does this, over-investment and crisis is the eventual outcome. If it does not do this and increases interest rates then consumption and investment will dry up as interest rates rise and the defaulters (honest and dishonest) increase and a crisis will eventually occur.


This is because increasing interest rates may increase savings {\bf but} it also reduce consumption ({\em “high interest rates also deter both consumers and companies from spending, so that the domestic economy is weakened and unemployment rises”} [Paul Ormerod, {\bf The Death of Economics}, p. 70]). This means that firms can face a drop off in demand, causing them problems and (perhaps) leading to a lack of profits, debt repayment problems and failure. An increase in interest rates also reduces demand for investment goods, which also can cause firms problems, increase unemployment and so on. So an increase in interest rates (particularly a sharp rise) could reduce consumption and investment (i.e. reduce aggregate demand) and have a ripple effect throughout the economy which could cause a slump to occur.


In other words, interest rates and the supply and demand of savings/loans they are meant to reflect may not necessarily move an economy towards equilibrium (if such a concept is useful). Indeed, the workings of a “pure” banking system without credit money may increase unemployment as demand falls in both investment and consumption in response to high interest rates and a general shortage of money due to lack of (credit) money resulting from the “tight” money regime implied by such a regime (i.e. the business cycle would still exist). This was the case of the failed Monetarist experiments on the early 1980s when central banks in America and Britain tried to pursue a “tight” money policy. The “tight” money policy did not, in fact, control the money supply. All it did do was increase interest rates and lead to a serious financial crisis and a deep recession (as Wray notes, {\em “the central bank uses tight money polices to raise interest rates”} [{\bf Op. Cit.}, p. 262]). This recession, we must note, also broke the backbone of working class resistance and the unions in both countries due to the high levels of unemployment it generated. As intended, we are sure.


Such an outcome would not surprise anarchists, as this was a key feature of the Individualist and Mutualist Anarchists’ arguments against the “money monopoly” associated with specie money. They argued that the “money monopoly” created a “tight” money regime which reduced the demand for labour by restricting money and credit and so allowed the exploitation of labour (i.e. encouraged wage labour) and stopped the development of non-capitalist forms of production. Thus Lysander Spooner’s comments that workers need {\em “{\bf money capital} to enable them to buy the raw materials upon which to bestow their labour, the implements and machinery with which to labour \unknown{} Unless they get this capital, they must all either work at a disadvantage, or not work at all. A very large portion of them, to save themselves from starvation, have no alternative but to sell their labour to others \unknown{}”} [{\bf A Letter to Grover Cleveland}, p. 39] It is interesting to note that workers {\bf did} do well during the 1950s and 1960s under a “liberal” money regime than they did under the “tighter” regimes of the 1980s and 1990s.


We should also note that an extended period of boom will encourage banks to make loans more freely. According to Minsky’s {\em “financial instability model”} crisis (see {\em “The Financial Instability Hypothesis”} in {\bf Post-Keynesian Economic Theory} for example) is essentially caused by risky financial practices during periods of financial tranquillity. In other words, {\em “stability is destabilising.”} In a period of boom, banks are happy and the increased profits from companies are flowing into their vaults. Over time, bankers note that they can use a reserve system to increase their income and, due to the general upward swing of the economy, consider it safe to do so (and given that they are in competition with other banks, they may provide loans simply because they are afraid of losing customers to more flexible competitors). This increases the instability within the system (as firms increase their debts due to the flexibility of the banks) and produces the possibility of crisis if interest rates are increased (because the ability of business to fulfil their financial commitments embedded in debts deteriorates).


Even if we assume that interest rates {\bf do} work as predicted in theory, it is false to maintain that there is one interest rate. This is not the case. {\em “Concentration of capital leads to unequal access to investment funds, which obstructs further the possibility of smooth transitions in industrial activity. Because of their past record of profitability, large enterprises have higher credit ratings and easier access to credit facilities, and they are able to put up larger collateral for a loan.”} [Michael A. Bernstein, {\bf The Great Depression}, p. 106] As we noted in section C.5.1, the larger the firm, the lower the interest rate they have to pay. Thus banks routinely lower their interest rates to their best clients even though the future is uncertain and past performance cannot and does not indicate future returns. Therefore it seems a bit strange to maintain that the interest rate will bring savings and loans into line if there are different rates being offered.


And, of course, private banks cannot affect the underlying fundamentals that drive the economy — like productivity, working class power and political stability — any more than central banks (although central banks can influence the speed and gentleness of adjustment to a crisis).


Indeed, given a period of full employment a system of private banks may actually speed up the coming of a slump. As we argue in the next section, full employment results in a profits squeeze as firms face a tight labour market (which drives up costs) and, therefore, increased workers’ power at the point of production and in their power of exit. In a central bank system, capitalists can pass on these increasing costs to consumers and so maintain their profit margins for longer. This option is restricted in a private banking system as banks would be less inclined to devalue their money. This means that firms will face a profits squeeze sooner rather than later, which will cause a slump as firms cannot make ends meet. As Reekie notes, inflation {\em “can temporarily reduce employment by postponing the time when misdirected labour will be laid off”} but as Austrian’s (like Monetarists) think {\em “inflation is a monetary phenomenon”} he does not understand the real causes of inflation and what they imply for a “pure” capitalist system [{\bf Op. Cit.}, p. 67, p. 74]. As Paul Ormerod points out {\em “the claim that inflation is always and everywhere purely caused by increases in the money supply, and that there the rate of inflation bears a stable, predictable relationship to increases in the money supply is ridiculous.”} And he notes that {\em “[i]ncreases in the rate of inflation tend to be linked to falls in unemployment, and vice versa”} which indicates its {\bf real} causes — namely in the balance of class power and in the class struggle. [{\bf The Death of Economics}, p. 96, p. 131]


Moreover, if we do take the Austrian theory of the business cycle at face value we are drawn to conclusion that in order to finance investment savings must be increased. But to maintain or increase the stock of loanable savings, inequality must be increased. This is because, unsurprisingly, rich people save a larger proportion of their income than poor people and the proportion of profits saved are higher than the proportion of wages. But increasing inequality (as we argued in section 3.1) makes a mockery of right-libertarian claims that their system is based on freedom or justice.


This means that the preferred banking system of “anarcho”-capitalism implies increasing, not decreasing, inequality within society. Moreover, most firms (as we indicated in section C.5.1) fund their investments with their own savings which would make it hard for banks to loan these savings out as they could be withdrawn at any time. This could have serious implications for the economy, as banks refuse to fund new investment simply because of the uncertainty they face when accessing if their available savings can be loaned to others (after all, they can hardly loan out the savings of a customer who is likely to demand them at any time). And by refusing to fund new investment, a boom could falter and turn to slump as firms do not find the necessary orders to keep going.


So, would market forces create “sound banking”? The answer is probably not. The pressures on banks to make profits come into conflict with the need to maintain their savings to loans ration (and so the confidence of their customers). As Wray argues, {\em “as banks are profit seeking firms, they find ways to increase their liabilities which don’t entail increases in reserve requirements”} and {\em “[i]f banks share the profit expectations of prospective borrowers, they can create credit to allow [projects/investments] to proceed.”} [{\bf Op. Cit.}, p. 295, p. 283] This can be seen from the historical record. As Kindleberger notes, {\em “the market will create new forms of money in periods of boom to get around the limit”} imposed on the money supply [{\bf Op. Cit.}, p. 63]. Trade credit is one way, for example. Under the Monetarist experiments of 1980s, there was {\em “deregulation and central bank constraints raised interest rates and created a moral hazard — banks made increasingly risky loans to cover rising costs of issuing liabilities. Rising competition from nonbanks and tight money policy forced banks to lower standards and increase rates of growth in an attempt to ‘grow their way to profitability’”} [{\bf Op. Cit.}, p. 293]


Thus credit money (“fiduciary media”) is an attempt to overcome the scarcity of money within capitalism, particularly the scarcity of specie money. The pressures that banks face within “actually existing” capitalism would still be faced under “pure” capitalism. It is likely (as Reekie acknowledges) that credit money would still be created in response to the demands of business people (although not at the same level as is currently the case, we imagine). The banks, seeking profits themselves and in competition for customers, would be caught between maintaining the value of their business (i.e. their money) and the needs to maximise profits. As a boom develops, banks would be tempted to introduce credit money to maintain it as increasing the interest rate would be difficult and potentially dangerous (for reasons we noted above). Thus, if credit money is not forth coming (i.e. the banks stick to the Austrian claims that loans must equal savings) then the rise in interest rates required will generate a slump. If it is forthcoming, then the danger of over-investment becomes increasingly likely. All in all, the business cycle is part of capitalism and {\bf not} caused by “external” factors like the existence of government.


As Reekie notes, to Austrians {\em “ignorance of the future is endemic”} [{\bf Op. Cit.}, p. 117] but you would be forgiven for thinking that this is not the case when it comes to investment. An individual firm cannot know whether its investment project will generate the stream of returns necessary to meet the stream of payment commitments undertaken to finance the project. And neither can the banks who fund those projects. Even {\bf if} a bank does not get tempted into providing credit money in excess of savings, it cannot predict whether other banks will do the same or whether the projects it funds will be successful. Firms, looking for credit, may turn to more flexible competitors (who practice reserve banking to some degree) and the inflexible bank may see its market share and profits decrease. After all, commercial banks {\em “typically establish relations with customers to reduce the uncertainty involved in making loans. Once a bank has entered into a relationship with a customer, it has strong incentives to meet the demands of that customer.”} [Wray, {\bf Op. Cit.}, p. 85]


There are example of fully privatised banks. For example, in the United States ({\em “which was without a central bank after 1837”}) {\em “the major banks in New York were in a bind between their roles as profit seekers, which made them contributors to the instability of credit, and as possessors of country deposits against whose instability they had to guard.”} [Kindleberger, {\bf Op. Cit.}, p. 85]


In Scotland, the banks were unregulated between 1772 and 1845 but {\em “the leading commercial banks accumulated the notes of lessor ones, as the Second Bank of the United States did contemporaneously in [the USA], ready to convert them to specie if they thought they were getting out of line. They served, that is, as an informal controller of the money supply. For the rest, as so often, historical evidence runs against strong theory, as demonstrated by the country banks in England from 1745 to 1835, wildcat banking in Michigan in the 1830s, and the latest experience with bank deregulation in Latin America.”} [{\bf Op. Cit.}, p. 82] And we should note there were a few banking “wars” during the period of deregulation in Scotland which forced a few of the smaller banks to fail as the bigger ones refused their money and that there was a major bank failure in the Ayr Bank.


Kendleberger argues that central banking {\em “arose to impose control on the instability of credit”} and did not cause the instability which right-libertarians maintain it does. And as we note in section 10.3, the USA suffered massive economic instability during its period without central banking. Thus, {\bf if} credit money {\bf is} the cause of the business cycle, it is likely that a “pure” capitalism will still suffer from it just as much as “actually existing” capitalism (either due to high interest rates or over-investment).


In general, as the failed Monetarist experiments of the 1980s prove, trying to control the money supply is impossible. The demand for money is dependent on the needs of the economy and any attempt to control it will fail (and cause a deep depression, usually via high interest rates). The business cycle, therefore, is an endogenous phenomenon caused by the normal functioning of the capitalist economic system. Austrian and right-libertarian claims that {\em “slump flows boom, but for a totally unnecessary reason: government inspired mal-investment”} [Reekie, {\bf Op. Cit.}, p. 74] are simply wrong. Over-investment {\bf does} occur, but it is {\bf not} {\em “inspired”} by the government. It is {\em “inspired”} by the banks need to make profits from loans and from businesses need for investment funds which the banks accommodate. In other words, by the nature of the capitalist system.


\subsection{10.2 How does the labour market effect capitalism?
}

In many ways, the labour market is the one that affects capitalism the most. The right-libertarian assumption (like that of mainstream economics) is that markets clear and, therefore, the labour market will also clear. As this assumption has rarely been proven to be true in actuality (i.e. periods of full employment within capitalism are few and far between), this leaves its supporters with a problem — reality contradicts the theory.


The theory predicts full employment but reality shows that this is not the case. Since we are dealing with logical deductions from assumptions, obviously the theory cannot be wrong and so we must identify external factors which cause the business cycle (and so unemployment). In this way attention is diverted away from the market and its workings — after all, it is assumed that the capitalist market works — and onto something else. This “something else” has been quite a few different things (most ridiculously, sun spots in the case of one of the founders of marginalist economics, William Stanley Jevons). However, these days most pro-free market capitalist economists and right-libertarians have now decided it is the state.


In this section of the FAQ we will present a case that maintains that the assumption that markets clear is false at least for one, unique, market — namely, the market for labour. As the fundamental assumption underlying “free market” capitalism is false, the logically consistent superstructure built upon comes crashing down. Part of the reason why capitalism is unstable is due to the commodification of labour (i.e. people) and the problems this creates. The state itself can have positive and negative impacts on the economy, but removing it or its influence will not solve the business cycle.


Why is this? Simply due to the nature of the labour market.


Anarchists have long realised that the capitalist market is based upon inequalities and changes in power. Proudhon argued that {\em “[t]he manufacturer says to the labourer, ‘You are as free to go elsewhere with your services as I am to receive them. I offer you so much.’ The merchant says to the customer, ‘Take it or leave it; you are master of your money, as I am of my goods. I want so much.’ Who will yield? The weaker.”} He, like all anarchists, saw that domination, oppression and exploitation flow from inequalities of market/economic power and that the {\em “power of invasion lies in superior strength.”} [{\bf What is Property?}, p. 216, p. 215]


This applies with greatest force to the labour market. While mainstream economics and right-libertarian variations of it refuse to acknowledge that the capitalist market is a based upon hierarchy and power, anarchists (and other socialists) do not share this opinion. And because they do not share this understanding with anarchists, right-libertarians will never be able to understand capitalism or its dynamics and development. Thus, when it comes to the labour market, it is essential to remember that the balance of power within it is the key to understanding the business cycle. Thus the economy must be understood as a system of power.


So how does the labour market effect capitalism? Let us consider a growing economy, on that is coming out of a recession. Such a growing economy stimulates demand for employment and as unemployment falls, the costs of finding workers increase and wage and condition demands of existing workers intensify. As the economy is growing and labour is scare, the threat associated with the hardship of unemployment is weakened. The share of profits is squeezed and in reaction to this companies begin to cut costs (by reducing inventories, postponing investment plans and laying off workers). As a result, the economy moves into a downturn. Unemployment rises and wage demands are moderated. Eventually, this enables the share of profits first of all to stabilise, and then rise. Such an {\em “interplay between profits and unemployment as the key determinant of business cycles”} is {\em “observed in the empirical data.”} [Paul Ormerod, {\bf The Death of Economics}, p. 188]


Thus, as an economy approaches full employment the balance of power on the labour market changes. The sack is no longer that great a threat as people see that they can get a job elsewhere easily. Thus wages and working conditions increase as companies try to get new (and keep) existing employees and output is harder to maintain. In the words of economist William Lazonick, labour {\em “that is able to command a higher price than previously because of the appearance of tighter labour markets is, by definition, labour that is highly mobile via the market. And labour that is highly mobile via the market is labour whose supply of effort is difficult for managers to control in the production process. Hence, the advent of tight labour markets generally results in more rapidly rising average costs \unknown{}as well as upward shifts in the average cost curve\unknown{}”} [{\bf Business Organisation and the Myth of the Market Economy}, p. 106]


In other words, under conditions of full-employment {\em “employers are in danger of losing the upper hand.”} [Juliet B. Schor, {\bf The Overworked American}, p. 75] Schor argues that {\em “employers have a structural advantage in the labour market, because there are typically more candidates ready and willing to endure this work marathon [of long hours] than jobs for them to fill.”} [p. 71] Thus the labour market is usually a buyers market, and so the sellers have to compromise. In the end, workers adapt to this inequality of power and instead of getting what they want, they want what they get.


But under full employment this changes. As we argued in section B.4.4 and section C.7, in such a situation it is the bosses who have to start compromising. And they do not like it. As Schor notes, America {\em “has never experienced a sustained period of full employment. The closest we have gotten is the late 1960s, when the overall unemployment rate was under 4 percent for four years. But that experience does more to prove the point than any other example. The trauma caused to business by those years of a tight labour market was considerable. Since then, there has been a powerful consensus that the nation cannot withstand such a low rate of unemployment.”} [{\bf Op. Cit.}, pp. 75–76]


So, in other words, full employment is not good for the capitalist system due to the power full employment provides workers. Thus unemployment is a necessary requirement for a successful capitalist economy and not some kind of aberration in an otherwise healthy system. Thus “anarcho”-capitalist claims that “pure” capitalism will soon result in permanent full employment are false. Any moves towards full employment will result in a slump as capitalists see their profits squeezed from below by either collective class struggle or by individual mobility in the labour market.


This was recognised by Individualist Anarchists like Benjamin Tucker, who argued that mutual banking would {\em “give an unheard of impetus to business, and consequently create an unprecedented demand for labour, — a demand which would always be in excess of the supply, directly contrary of the present condition of the labour market.”} [{\bf The Anarchist Reader}, pp. 149–150] In other words, full employment would end capitalist exploitation, drive non-labour income to zero and ensure the worker the full value of her labour — in other words, end capitalism. Thus, for most (if not all) anarchists the exploitation of labour is only possible when unemployment exists and the supply of labour exceeds the demand for it. Any move towards unemployment will result in a profits squeeze and either the end of capitalism or an economic slump.


Indeed, as we argued in the last section, the extended periods of (approximately) full employment until the 1960s had the advantage that any profit squeeze could (in the short run anyway) be passed onto working class people in the shape of inflation. As prices rise, labour is made cheaper and profits margins supported. This option is restricted under a “pure” capitalism (for reasons we discussed in the last section) and so “pure” capitalism will be affected by full employment faster than “impure” capitalism.


As an economy approaches full employment, {\em “hiring new workers suddenly becomes much more difficult. They are harder to find, cost more, and are less experiences. Such shortages are extremely costly for a firm.”} [Schor, {\bf Op. Cit.}, p. 75] This encourages a firm to pass on these rises to society in the form of price rises, so creating inflation. Workers, in turn, try to maintain their standard of living. {\em “Every general increase in labour costs in recent years,”} note J. Brecher and J. Costello in the late 1970s, {\em “has followed, rather than preceded, an increase in consumer prices. Wage increases have been the result of workers’ efforts to catch up after their incomes have already been eroded by inflation. Nor could it easily be otherwise. All a businessman has to do to raise a price \unknown{} [is to] make an announcement\unknown{} Wage rates \unknown{} are primarily determined by contracts”} and so cannot be easily adjusted in the short term. [{\bf Common Sense for Bad Times}, p, 120]


These full employment pressures will still exist with “pure” capitalism (and due to the nature of the banking system will not have the safety value of inflation). This means that periodic profit squeezes will occur, due to the nature of a tight labour market and the increased power of workers this generates. This in turn means that a “pure” capitalism will be subject to periods of unemployment (as we argued in section C.9) and so still have a business cycle. This is usually acknowledged by right-libertarians in passing, although they seem to think that this is purely a “short-term” problem (it seems a strange “short-term” problem that continually occurs).


But such an analysis is denied by right-libertarians. For them government action, combined with the habit of many labour unions to obtain higher than market wage rates for their members, creates and exacerbates mass unemployment. This flows from the deductive logic of much capitalist economics. The basic assumption of capitalism is that markets clear. So if unemployment exists then it can only be because the price of labour (wages) is too high (Austrian Economist W. Duncan Reekie argues that unemployment will {\em “disappear provided real wages are not artificially high”} [{\bf Markets, Entrepreneurs and Liberty}, p. 72]).


Thus the assumption provokes the conclusion — unemployment is caused by an unclearing market as markets always clear. And the cause for this is either the state or unions. But what if the labour market {\bf cannot} clear without seriously damaging the power and profits of capitalists? What if unemployment is required to maximise profits by weakening labours’ bargaining position on the market and so maximising the capitalists power? In that case unemployment is caused by capitalism, not by forces external to it.


However, let us assume that the right-libertarian theory is correct. Let us assume that unemployment is all the fault of the selfish unions and that a job-seeker {\em “who does not want to wait will always get a job in the unhampered market economy.”} [von Mises, {\bf Human Action}, p. 595]


Would crushing the unions reduce unemployment? Let us assume that the unions have been crushed and government has been abolished (or, at the very least, become a minimum state). The aim of the capitalist class is to maximise their profits and to do this they invest in labour saving machinery and otherwise attempt to increase productivity. But increasing productivity means that the prices of goods fall and falling prices mean increasing real wages. It is high real wages that, according to right-libertarians, that cause unemployment. So as a reward for increasing productivity, workers will have to have their money wages cut in order to stop unemployment occurring! For this reason some employers might refrain from cutting wages in order to avoid damage to morale — potentially an important concern.


Moreover, wage contracts involve {\bf time} — a contract will usually agree a certain wage for a certain period. This builds in rigidity into the market, wages cannot be adjusted as quickly as other commodity prices. Of course, it could be argued that reducing the period of the contract and/or allowing the wage to be adjusted could overcome this problem. However, if we reduce the period of the contract then workers are at a suffer disadvantage as they will not know if they have a job tomorrow and so they will not be able to easily plan their future (an evil situation for anyone to be in). Moreover, even without formal contracts, wage renegotiation can be expensive. After all, it takes time to bargain (and time is money under capitalism) and wage cutting can involve the risk of the loss of mutual good will between employer and employee. And would {\bf you} give your boss the power to “adjust” your wages as he/she thought was necessary? To do so would imply an altruistic trust in others not to abuse their power.


Thus a “pure” capitalism would be constantly seeing employment increase and decrease as productivity levels change. There exist important reasons why the labour market need not clear which revolve around the avoidance/delaying of wage cuts by the actions of capitalists themselves. Thus, given a choice between cutting wages for all workers and laying off some workers without cutting the wages of the remaining employees, it is unsurprising that capitalists usually go for the later. After all, the sack is an important disciplining device and firing workers can make the remaining employees more inclined to work harder and be more obedient.


And, of course, many employers are not inclined to hire over-qualified workers. This is because, once the economy picks up again, their worker has a tendency to move elsewhere and so it can cost them time and money finding a replacement and training them. This means that involuntary unemployment can easily occur, so reducing tendencies towards full employment even more. In addition, one of the assumptions of the standard marginalist economic model is one of decreasing returns to scale. This means that as employment increases, costs rise and so prices also rise (and so real wages fall). But in reality many industries have {\bf increasing} returns to scale, which means that as production increases unit costs fall, prices fall and so real wages rise. Thus in such an economy unemployment would increase simply because of the nature of the production process!


Moreover, as we argued in-depth in section C.9, a cut in money wages is not a neutral act. A cut in money wages means a reduction in demand for certain industries, which may have to reduce the wages of its employees (or fire them) to make ends meet. This could produce a accumulative effect and actually {\bf increase} unemployment rather than reduce it.


In addition, there are no “self-correcting” forces at work in the labour market which will quickly bring employment back to full levels. This is for a few reasons. Firstly, the supply of labour cannot be reduced by cutting back production as in other markets. All we can do is move to other areas and hope to find work there. Secondly, the supply of labour can sometimes adjust to wage decreases in the wrong direction. Low wages might drive workers to offer a greater amount of labour (i.e. longer hours) to make up for any short fall (or to keep their job). This is usually termed the {\em “efficiency wage”} effect. Similarly, another family member may seek employment in order to maintain a given standard of living. Falling wages may cause the number of workers seeking employment to {\bf increase}, causing a full further fall in wages and so on (and this is ignoring the effects of lowering wages on demand discussed in section C.9).


The paradox of piece work is an important example of this effect. As Schor argues, {\em “piece-rate workers were caught in a viscous downward spiral of poverty and overwork\unknown{} When rates were low, they found themselves compelled to make up in extra output what they were losing on each piece. But the extra output produced glutted the market and drove rates down further.”} [Juliet C. Schor, {\bf The Overworked American}, p, 58]


Thus, in the face of reducing wages, the labour market may see an accumulative move away from (rather than towards) full employment, The right-libertarian argument is that unemployment is caused by real wages being too high which in turn flows from the assumption that markets clear. If there is unemployment, then the price of the commodity labour is too high — otherwise supply and demand would meet and the market clear. But if, as we argued above, unemployment is essential to discipline workers then the labour market {\bf cannot} clear except for short periods. If the labour market clears, profits are squeezed. Thus the claim that unemployment is caused by “too high” real wages is false (and as we argue in section C.9, cutting these wages will result in deepening any slump and making recovery longer to come about).


In other words, the assumption that the labour market must clear is false, as is any assumption that reducing wages will tend to push the economy quickly back to full employment. The nature of wage labour and the “commodity” being sold (i.e. human labour/time/liberty) ensure that it can never be the same as others. This has important implications for economic theory and the claims of right-libertarians, implications that they fail to see due to their vision of labour as a commodity like any other.


The question arises, of course, of whether, during periods of full employment, workers could not take advantage of their market power and gain increased workers’ control, create co-operatives and so reform away capitalism. This was the argument of the Mutualist and Individualist anarchists and it does have its merits. However, it is clear (see section J.5.12) that bosses hate to have their authority reduced and so combat workers’ control whenever they can. The logic is simple, if workers increase their control within the workplace the manager and bosses may soon be out of a job and (more importantly) they may start to control the allocation of profits. Any increase in working class militancy may provoke capitalists to stop/reduce investment and credit and so create the economic environment (i.e. increasing unemployment) necessary to undercut working class power.


In other words, a period of full unemployment is not sufficient to reform capitalism away. Full employment (nevermind any struggle over workers’ control) will reduce profits and if profits are reduced then firms find it hard to repay debts, fund investment and provide profits for shareholders. This profits squeeze would be enough to force capitalism into a slump and any attempts at gaining workers’ self-management in periods of high employment will help push it over the edge (after all, workers’ control without control over the allocation of any surplus is distinctly phoney). Moreover, even if we ignore the effects of full employment may not last due to problems associated with over-investment (see section C.7.2), credit and interest rate problems (see section 10.1) and realisation/aggregate demand disjoints. Full employment adds to the problems associated with the capitalist business cycle and so, if class struggle and workers power did not exist or cost problem, capitalism would still not be stable.


If equilibrium is a myth, then so is full employment. It seems somewhat ironic that “anarcho”-capitalists and other right-libertarians maintain that there will be equilibrium (full employment) in the one market within capitalism it can never actually exist in! This is usually quietly acknowledged by most right-libertarians, who mention in passing that some “temporary” unemployment {\bf will} exist in their system — but “temporary” unemployment is not full employment. Of course, you could maintain that all unemployment is “voluntary” and get round the problem by denying it, but that will not get us very far.


So it is all fine and well saying that “libertarian” capitalism would be based upon the maxim {\em “From each as they choose, to each as they are chosen.”} [Robert Nozick, {\bf Anarchy, State, and Utopia}, p. 160] But if the labour market is such that workers have little option about what they “choose” to give and fear that they will {\bf not} be chosen, then they are at a disadvantage when compared to their bosses and so “consent” to being treated as a resource from the capitalist can make a profit from. And so this will result in any “free” contract on the labour market favouring one party at the expense of the other — as can be seen from “actually existing capitalism”.


Thus any “free exchange” on the labour market will usually {\bf not} reflect the true desires of working people (and who will make all the “adjusting” and end up wanting what they get). Only when the economy is approaching full employment will the labour market start to reflect the true desires of working people and their wage start to approach its full product. And when this happens, profits are squeezed and capitalism goes into slump and the resulting unemployment disciplines the working class and restores profit margins. Thus full employment will be the exception rather than the rule within capitalism (and that is a conclusion which the historical record indicates).


In other words, in a normally working capitalist economy any labour contracts will not create relationships based upon freedom due to the inequalities in power between workers and capitalists. Instead, any contracts will be based upon domination, {\bf not} freedom. Which prompts the question, how is libertarian capitalism {\bf libertarian} if it erodes the liberty of a large class of people?


\subsection{10.3 Was laissez-faire capitalism stable?
}

Firstly, we must state that a pure laissez-faire capitalist system has not existed. This means that any evidence we present in this section can be dismissed by right-libertarians for precisely this fact — it was not “pure” enough. Of course, if they were consistent, you would expect them to shun all historical and current examples of capitalism or activity within capitalism, but this they do not. The logic is simple — if X is good, then it is permissible to use it. If X is bad, the system is not pure enough.


However, as right-libertarians {\bf do} use historical examples so shall we. According to Murray Rothbard, there was {\em “quasi-laissez-faire industrialisation [in] the nineteenth century”} [{\bf The Ethics of Liberty}, p. 264] and so we will use the example of nineteenth century America — as this is usually taken as being the closest to pure laissez-faire — in order to see if laissez-faire is stable or not.


Yes, we are well aware that 19\high{th} century USA was far from laissez-faire — there was a state, protectionism, government economic activity and so on — but as this example has been often used by right-Libertarians’ themselves (for example, Ayn Rand) we think that we can gain a lot from looking at this imperfect approximation of “pure” capitalism (and as we argued in section 8, it is the “quasi” aspects of the system that counted in industrialisation, {\bf not} the laissez-faire ones).


So, was 19\high{th} century America stable? No, it most definitely was not.


Firstly, throughout that century there were a continual economic booms and slumps. The last third of the 19\high{th} century (often considered as a heyday of private enterprise) was a period of profound instability and anxiety. Between 1867 and 1900 there were 8 complete business cycles. Over these 396 months, the economy expanded during 199 months and contracted during 197. Hardly a sign of great stability (since the end of world war II, only about a fifth of the time has spent in periods of recession or depression, by way of comparison). Overall, the economy went into a slump, panic or crisis in 1807, 1817, 1828, 1834, 1837, 1854, 1857, 1873, 1882, and 1893 (in addition, 1903 and 1907 were also crisis years).


Part of this instability came from the eras banking system. {\em “Lack of a central banking system,”} writes Richard Du Boff, {\em “until the Federal Reserve act of 1913 made financial panics worse and business cycle swings more severe”} [{\bf Accumulation and Power}, p. 177] It was in response to this instability that the Federal Reserve system was created; and as Doug Henwood notes {\em “the campaign for a more rational system of money and credit was not a movement of Wall Street vs. industry or regional finance, but a broad movement of elite bankers and the managers of the new corporations as well as academics and business journalists. The emergence of the Fed was the culmination of attempts to define a standard of value that began in the 1890s with the emergence of the modern professionally managed corporation owned not by its managers but dispersed public shareholders.”} [{\bf Wall Street}, p. 93] Indeed, the Bank of England was often forced to act as lender of last resort to the US, which had no central bank.


In the decentralised banking system of the 19\high{th} century, during panics thousands of banks would hoard resources, so starving the system for liquidity precisely at the moment it was most badly needed. The creation of trusts was one way in which capitalists tried to manage the system’s instabilities (at the expense of consumers) and the corporation was a response to the outlawing of trusts. {\em “By internalising lots of the competitive system’s gaps — by bring more transactions within the same institutional walls — corporations greatly stabilised the economy.”} [Henwood, {\bf Op. Cit.}, p. 94]


All during the hey-day of laissez faire we also find popular protests against the money system used, namely specie (in particular gold), which was considered as a hindrance to economic activity and expansion (as well as being a tool for the rich). The Individualist Anarchists, for example, considered the money monopoly (which included the use of specie as money) as the means by which capitalists ensured that {\em “the labourers \unknown{} [are] kept in the condition of wage labourers,”} and reduced {\em “to the conditions of servants; and subject to all such extortions as their employers \unknown{} may choose to practice upon them”}, indeed they became the {\em “mere tools and machines in the hands of their employers”}. With the end of this monopoly, {\em “[t]he amount of money, capable of being furnished \unknown{} [would assure that all would] be under no necessity to act as a servant, or sell his or her labour to others.”} [Lysander Spooner, {\bf A Letter to Grover Cleveland}, p. 47, p. 39, p. 50, p. 41] In other words, a specie based system (as desired by many “anarcho”-capitalists) was considered a key way of maintaining wage labour and exploitation.


Interestingly, since the end of the era of the Gold Standard (and so commodity money) popular debate, protest and concern about money has disappeared. The debate and protest was in response to the {\bf effects} of commodity money on the economy — with many people correctly viewing the seriously restrictive monetary regime of the time responsible for economic problems and crisis as well as increasing inequalities. Instead radicals across the political spectrum urged a more flexible regime, one that did not cause wage slavery and crisis by reducing the amount of money in circulation when it could be used to expand production and reduce the impact of slumps. Needless to say, the Federal Reserve system in the USA was far from the institution these populists wanted (after all, it is run by and for the elite interests who desired its creation).


That the laissez-faire system was so volatile and panic-ridden suggests that “anarcho”-capitalist dreams of privatising everything, including banking, and everything will be fine are very optimistic at best (and, ironically, it was members of the capitalist class who lead the movement towards state-managed capitalism in the name of “sound money”).


\section{11 What is the myth of “Natural Law”?
}

Natural Law, and the related concept of Natural Rights, play an important part in Libertarian and “anarcho”-capitalist ideology. Right-libertarians are not alone in claiming that their particular ideology is based on the “law of nature”. Hitler, for one, claimed the same thing for Nazi ideology. So do numerous other demagogues, religious fanatics, and political philosophers. However, each likes to claim that only {\bf their} “natural law” is the “real” one, all the others being subjective impositions. We will ignore these assertions (they are not arguments) and concentrate on explaining why natural law, in all its forms, is a myth. In addition, we will indicate its authoritarian implications.


Instead of such myths anarchists urge people to “work it out for themselves” and realise that any ethical code is subjective and not a law of nature. If its a good “code”, then others will become convinced of it by your arguments and their intellect. There is no need to claim its a function of “man’s nature”!


The following books discuss the subject of “Natural Law” in greater depth and are recommended for a fuller discussion of the issues raised in this section:


Robert Anton Wilson, {\bf Natural Law} and L.A. Rollins, {\bf The Myth of Natural Law}.


We should note that these books are written by people associated, to some degree, with right-libertarianism and, of course, we should point out that not all right-libertarians subscribe to “natural law” theories (David Friedman, for example, does not). However, such a position seems to be the minority in right-Libertarianism (Ayn Rand, Robert Nozick and Murray Rothbard, among others, did subscribe to it). We should also point out that the Individualist Anarchist Lysander Spooner also subscribed to “natural laws” (which shows that, as we noted above, the concept is not limited to one particular theory or ideology). We present a short critique of Spooner’s ideas on this subject in section G.7.


Lastly, it could be maintained that it is a common “straw man” to maintain that supporters of Natural Law argue that their Laws are like the laws of physics (and so are capable of stopping people’s actions just as the law of gravity automatically stops people flying from the Earth). But that is the whole point — using the term “Natural Law” implies that the moral rights and laws that its supporters argue for are to be considered just like the law of gravity (although they acknowledge, of course, that unlike gravity, {\bf their} {\em “natural laws”} {\bf can be violated in nature}). Far from saying that the rights they support are just that (i.e. rights {\bf they} think are good) they try to associate them with universal facts. For example, Lysander Spooner (who, we must stress, used the concept of “Natural law” to {\bf oppose} the transformation of America into a capitalist society, unlike Rand, Nozick and Rothbard who use it to defend capitalism) stated that:



\startblockquote
{\em “the true definition of law is, that it is a fixed, immutable, natural principle; and not anything that man ever made, or can make, unmake, or alter. Thus we speak of the laws of matter, and the laws of mind; of the laws of gravitation, the laws of light, heat, and electricity\unknown{}etc., etc\unknown{} The law of justice is just as supreme and universal in the moral world, as these others are in the mental or physical world; and is as unalterable as are these by any human power. And it is just as false and absurd to talk of anybody’s having the power to abolish the law of justice, and set up their own in its stead, as it would be to talk of their having the power to abolish the law of gravitation, or any other natural laws of the universe, and set up their own will in the place of them.”} [{\bf A Letter to Grover Cleveland}, p. 88]



\stopblockquote
Rothbard and other capitalist supporters of “Natural Law” make the same sort of claims (as we will see). Now, why, if they are aware of the fact that unlike gravity their “Natural Laws” can be violated, do they use the term at all? Benjamin Tucker said that “Natural Law” was a {\em “religious”} concept — and this provides a clue. To say “Do not violate these rights, otherwise I will get cross” does not have {\bf quite} the same power as “Do not violate these rights, they are facts of natural and you are violating nature” (compare to “Do not violate these laws, or you will go to hell”). So to point out that “Natural Law” is {\bf not} the same as the law of gravity (because it has to be enforced by humans) is not attacking some kind of “straw man” — it is exposing the fact that these “Natural Laws” are just the personal prejudices of those who hold them. If they do not want then to be exposed as such then they should call their laws what they are — personal ethical laws — rather than compare them to the facts of nature.


\subsection{11.1 Why the term “Natural Law” in the first place?
}

Murray Rothbard claims that {\em “Natural Law theory rests on the insight\unknown{} that each entity has distinct and specific properties, a distinct ‘nature,’ which can be investigated by man’s reason”} [{\bf For a New Liberty}, p. 25] and that {\em “man has rights because they are {\bf natural} rights. They are grounded in the nature of man.”} [{\bf The Ethics of Liberty}, p. 155]


To put it bluntly, this form of “analysis” was originated by Aristotle and has not been used by science for centuries. Science investigates by proposing theories and hypotheses to explain empirical observations, testing and refining them by experiment. In stark contrast, Rothbard {\bf invents} definitions ({\em “distinct” “natures”}) and then draws conclusions from them. Such a method was last used by the medieval Church and is devoid of any scientific method. It is, of course, a fiction. It attempts to deduce the nature of a “natural” society from {\em a priori} considerations of the “innate” nature of human beings, which just means that the assumptions necessary to reach the desired conclusions have been built into the definition of “human nature.” In other words, Rothbard defines humans as having the “distinct and specific properties” that, given his assumptions, will allow his dogma (private state capitalism) to be inferred as the “natural” society for humans.


Rothbard claims that {\em “if A, B, C, etc., have differing attributes, it follows that they have different {\bf natures.}”} [{\bf The Ethics of Liberty}, p. 9] Does this means that as every individual is unique (have different attributes), they have different natures? Skin and hair colour are different attributes, does this mean that red haired people have different natures than blondes? That black people have different natures than white (and such a “theory” of “natural law” was used to justify slavery — yes, slaves {\bf are} human but they have “different natures” than their masters and so slavery is okay). Of course Rothbard aggregates “attributes” to species level, but why not higher? Humans are primates, does that mean we have the same natures are monkeys or gorillas? We are also mammals as well, we share many of the same attributes as whales and dogs. Do we have similar natures?


But this is by the way. To continue we find that after defining certain “natures,” Rothbard attempts to derive {\em “Natural Rights and Laws”} from them. However, these {\em “Natural Laws”} are quite strange, as they can be violated in nature! Real natural laws (like the law of gravity) {\bf cannot} be violated and therefore do not need to be enforced. The “Natural Laws” the “Libertarian” desires to foist upon us are not like this. They need to be enforced by humans and the institutions they create. Hence, Libertarian “Natural Laws” are more akin to moral prescriptions or juridical laws. However, this does not stop Rothbard explicitly {\em “plac[ing]”} his {\em “Natural Laws” “alongside physical or ‘scientific’ natural laws.”} [{\bf The Ethics of Liberty}, p. 42]


So why do so many Libertarians use the term “Natural Law?” Simply, it gives them the means by which to elevate their opinions, dogmas, and prejudices to a metaphysical level where nobody will dare to criticise or even think about them. The term smacks of religion, where “Natural Law” has replaced “God’s Law.” The latter fiction gave the priest power over believers. “Natural Law” is designed to give the Libertarian ideologist power over the people that he or she wants to rule.


How can one be against a “Natural Law” or a “Natural Right”? It is impossible. How can one argue against gravity? If private property, for example, is elevated to such a level, who would dare argue against it? Ayn Rand listed having landlords and employers along with {\em “the laws of nature.”} They are {\bf not} similar: the first two are social relationships which have to be imposed by the state; the {\em “laws of nature”} (like gravity, needing food, etc.) are {\bf facts} which do not need to be imposed. Rothbard claims that {\em “the natural fact is that labour service {\bf is} indeed a commodity.”} [{\bf Op. Cit.}, p. 40] However, this is complete nonsense — labour service as a commodity is a {\bf social} fact, dependent on the distribution of property within society, its social customs and so forth. It is only “natural” in the sense that it exists within a given society (the state is also “natural” as it also exists within nature at a given time). But neither wage slavery or the state is “natural” in the sense that gravity is natural or a human having two arms is. Indeed, workers at the dawn of capitalism, faced with selling their labour services to another, considered it as decidedly “unnatural” and used the term “wage slavery” to describe it!


Thus, where and when a “fact” appears is essential. For example, Rothbard claims that {\em “[a]n apple, let fall, will drop to the ground; this we all observe and acknowledge to be {\bf in the nature} of the apple.”} [{\bf The Ethics of Liberty}, p. 9] Actually, we do not “acknowledge” anything of the kind. We acknowledge that the apple was subject to the force of gravity and that is why it fell. The same apple, “let fall” in a space ship would {\bf not} drop to the floor. Has the “nature” of the apple changed? No, but the situation it is in has. Thus any attempt to generate abstract “natures” requires you to ignore reality in favour of ideals.


Because of the confusion its usage creates, we are tempted to think that the use of “Natural Law” dogma is an attempt to {\bf stop} thinking, to restrict analysis, to force certain aspects of society off the political agenda by giving them a divine, everlasting quality.


Moreover, such an “individualist” account of the origins of rights will always turn on a muddled distinction between individual rationality and some vague notion of rationality associated with membership of the human species. How are we to determine what is rational for an individual {\bf as and individual} and what is rational for that same individual {\bf as a human being}? It is hard to see that we can make such a distinction for {\em “[i]f I violently interfere with Murray Rothbard’s freedom, this may violate the ‘natural law’ of Murray Rothbard’s needs, but it doesn’t violate the ‘natural law’ of {\bf my} needs.”} [L.A. Rollins, {\bf The Myth of Natural Rights}, p. 28] Both parties, after all, are human and if such interference is, as Rothbard claims, {\em “antihuman”} then why? {\em “If it helps me, a human, to advance my life, then how can it be unequivocally ‘antihuman’?”} [L. A. Rollins, {\bf Op. Cit.}, p. 27] Thus “natural law” is contradictory as it is well within the bounds of human nature to violate it.


This means that in order to support the dogma of “Natural Law,” the cultists {\bf must} ignore reality. Ayn Rand claims that {\em “the source of man’s rights is\unknown{}the law of identity. A is A — and Man is Man.”} But Rand (like Rothbard) {\bf defines} {\em “Man”} as an {\em “entity of a specific kind — a rational being”} [{\bf The Virtue of Selfishness}, pp. 94–95]. Therefore she cannot account for {\bf irrational} human behaviours (such as those that violate “Natural Laws”), which are also products of our “nature.” To assert that such behaviours are not human is to assert that A can be not-A, thus contradicting the law of identity. Her ideology cannot even meet its own test.


\subsection{11.2 But “Natural Law” provides protection for individual rights from violation by the State. Those who are against Natural Law desire total rule by the state.
}

The second statement represents a common “Libertarian” tactic. Instead of addressing the issues, they accuse an opponent of being a “totalitarian” (or the less sinister “statist”). In this way, they hope to distract attention from, and so avoid discussing, the issue at hand (while at the same time smearing their opponent). We can therefore ignore the second statement.


Regarding the first, “Natural Law” has {\bf never} stopped the rights of individuals from being violated by the state. Such “laws” are as much use as a chocolate fire-guard. If “Natural Rights” could protect one from the power of the state, the Nazis would not have been able to murder six million Jews. The only thing that stops the state from attacking people’s rights is individual (and social) power — the ability and desire to protect oneself and what one considers to be right and fair. As the anarchist Rudolf Rocker pointed out:



\startblockquote
{\em “Political [or individual] rights do not exist because they have been legally set down on a piece of paper, but only when they have become the ingrown habit of a people, and when any attempt to impair them will be meet with the violent resistance of the populace\unknown{}One compels respect from others when he knows how to defend his dignity as a human being\unknown{}The people owe all the political rights and privileges which we enjoy today, in greater or lesser measure, not to the good will of their governments, but to their own strength.”} [{\bf Anarcho-Syndicalism}, p. 64]



\stopblockquote
Of course, if is there are no “Natural Rights,” then the state has no “right” to murder you or otherwise take away what are commonly regarded as human rights. One can object to state power without believing in “Natural Law.”


\subsection{11.3 Why is “Natural Law” authoritarian?
}

Rights, far from being fixed, are the product of social evolution and human action, thought and emotions. What is acceptable now may become unacceptable in the future. Slavery, for example, was long considered “natural.” In fact, John Locke, the “father” of “Natural Rights,” was heavily involved in the slave trade. He made a fortune in violating what is today regarded as a basic human right: not to be enslaved. Many in Locke’s day claimed that slavery was a “Natural Law.” Few would say so now.


Thomas Jefferson indicates exactly why “Natural Law” is authoritarian when he wrote {\em “[s]ome men look at constitutions with sanctimonious reverence, and deem them like the ark of the Covenant, too sacred to be touched. They ascribe to the men of the preceding age a wisdom more than human, and suppose what they did to be beyond amendment\unknown{}laws and institutions must go hand in hand with the progress of the human mind\unknown{} as that becomes more developed, more enlightened, as new discoveries are made, institutions must advance also, to keep pace with the times\unknown{} We might as well require a man to wear still the coat which fitted him when a boy as civilised society to remain forever under the regimen of their barbarous ancestors.”}


The “Natural Law” cult desires to stop the evolutionary process by which new rights are recognised. Instead they wish to fix social life into what {\bf they} think is good and right, using a form of argument that tries to raise their ideology above critique or thought. Such a wish is opposed to the fundamental feature of liberty: the ability to think for oneself. Michael Bakunin writes {\em “the liberty of man consists solely in this: that he obeys natural laws because he has {\bf himself} recognised them as such, and not because they have been externally imposed upon him by any extrinsic will whatever, divine or human, collective or individual.”} [{\bf Bakunin on Anarchism}, p. 227]


Thus anarchism, in contrast to the “natural law” cult, recognises that “natural laws” (like society) are the product of individual evaluation of reality and social life and are, therefore, subject to change in the light of new information and ideas (Society {\em “progresses slowly through the moving power of individual initiative”} [Bakunin, {\bf The Political Philosophy of Bakunin}, p. 166] and so, obviously, do social rights and customs). Ethical or moral “laws” (which is what the “Natural Law” cult is actually about) is not a product of “human nature” or abstract individuals. Rather, it is a {\bf social} fact, a creation of society and human interaction. In Bakunin’s words, {\em “moral law is not an individual but a social fact, a creation of society”} and any {\em “natural laws”} are {\em “inherent in the social body”} (and so, we must add, not floating abstractions existing in “man’s nature”). [{\bf Ibid.}, p. 125, p. 166]


The case for liberty and a free society is based on the argument that, since every individual is unique, everyone can contribute something that no one else has noticed or thought about. It is the free interaction of individuals which allows them, along with society and its customs and rights, to evolve, change and develop. “Natural Law,” like the state, tries to arrest this evolution. It replaces creative inquiry with dogma, making people subject to yet another god, destroying critical thought with a new rule book.


In addition, if these “Natural Laws” are really what they are claimed to be, they are necessarily applicable to {\bf all} of humanity (Rothbard explicitly acknowledges this when he wrote that {\em “one of the notable attributes of natural law”} is {\em “its applicability to all men, regardless of time or place”} [{\bf The Ethics of Liberty}, p. 42]). In other words, every other law code {\bf must} (by definition) be “against nature” and there exists {\bf one} way of life (the “natural” one). The authoritarian implications of such arrogance is clear. That the Dogma of Natural Law was only invented a few hundred years ago, in one part of the planet, does not seem to bother its advocates. Nor does the fact that for the vast majority of human existence, people have lived in societies which violated almost {\bf all} of their so-called “Natural Laws” To take one example, before the late Neolithic, most societies were based on usufruct, or free access to communally held land and other resources [see Murray Bookchin, {\bf The Ecology of Freedom}]. Thus for millennia, all human beings lived in violation of the supposed “Natural Law” of private property — perhaps the chief “law” in the “Libertarian” universe.


If “Natural Law” did exist, then all people would have discovered these “true” laws years ago. To the contrary, however, the debate is still going on, with (for example) fascists and “Libertarians” each claiming “the laws of nature” (and socio-biology) as their own.


\subsection{11.4 Does “Natural Law” actually provides protection for individual liberty?
}

But, it seems fair to ask, does “natural law” actually respect individuals and their rights (i.e. liberty)? We think not. Why?


According to Rothbard, {\em “the natural law ethic states that for man, goodness or badness can be determined by what fulfils or thwarts what is best for man’s nature.”} [{\bf The Ethics of Liberty}, p. 10] But, of course, what may be “good” for “man” may be decidedly {\bf bad} for men (and women). If we take the example of the sole oasis in a desert (see section 4.2) then, according to Rothbard, the property owner having the power of life and death over others is “good” while, if the dispossessed revolt and refuse to recognise his “property”, this is “bad”! In other words, Rothbard’s “natural law” is good for {\bf some} people (namely property owners) while it can be bad for others (namely the working class). In more general terms, this means that a system which results in extensive hierarchy (i.e. {\bf archy}, power) is “good” (even though it restricts liberty for the many) while attempts to {\bf remove} power (such as revolution and the democratisation of property rights) is “bad”. Somewhat strange logic, we feel.


However such a position fails to understand {\bf why} we consider coercion to be wrong/unethical. Coercion is wrong because it subjects an individual to the will of another. It is clear that the victim of coercion is lacking the freedom that the philosopher Isaiah Berlin describes in the following terms:



\startblockquote
{\em “I wish my life and decisions to depend on myself, not on external forces of whatever kind. I wish to be an instrument of my own, not of other men’s, acts of will. I wish to be a subject, not an object; to be moved by reasons, by conscious purposes, which are my own, not by causes which affect me, as it were, from outside. I wish to be somebody, not nobody; a doer — deciding, not being decided for, self-directed and not acted upon by external nature or by other mean as if I were a thing, or an animal, or a slave incapable of playing a human role, that is, of conceiving goals and policies of my own and realising them.”} [{\bf Four Essays on Liberty}, p. 131]



\stopblockquote
Or, as Alan Haworth points out, {\em “we have to view coercion as a violation of what Berlin calls {\bf positive} freedom.”} [{\bf Anti-Libertarianism}, p. 48]


Thus, if a system results in the violation of (positive) liberty by its very nature — namely, subject a class of people to the will of another class (the worker is subject to the will of their boss and is turned into an order-taker) — then it is justified to end that system. Yes, it is “coercion” is dispossess the property owner — but “coercion” exists only for as long as they desire to exercise power over others. In other words, it is not domination to remove domination! And remember it is the domination that exists in coercion which fuels our hatred of it, thus “coercion” to free ourselves from domination is a necessary evil in order to stop far greater evils occurring (as, for example, in the clear-cut case of the oasis monopoliser).


Perhaps it will be argued that domination is only bad when it is involuntary, which means that it is only the involuntary nature of coercion that makes it bad, not the domination it involves. By this argument wage slavery is not domination as workers voluntarily agree to work for a capitalist (after all, no one puts a gun to their heads) and any attempt to overthrow capitalist domination is coercion and so wrong. However, this argument ignores that fact that {\bf circumstances} force workers to sell their liberty and so violence on behalf of property owners is not (usually) required — market forces ensure that physical force is purely “defensive” in nature. And as we argued in section 2.2, even Rothbard recognised that the economic power associated with one class of people being dispossessed and another empowered by this fact results in relations of domination which cannot be considered “voluntary” by any stretch of the imagination (although, of course, Rothbard refuses to see the economic power associated with capitalism — when its capitalism, he cannot see the wood for the trees — and we are ignoring the fact that capitalism was created by extensive use of coercion and violence — see section 8).


Thus, “Natural law” and attempts to protect individuals rights/liberty and see a world in which people are free to shape their own lives are fatally flawed if they do not recognise that private property is incompatible with these goals. This is because the existence of capitalist property smuggles in power and so domination (the restriction of liberty, the conversion of some into order-givers and the many into order-takers) and so Natural Law does not fulfil its promise that each person is free to pursue their own goals. The unqualified right of property will lead to the domination and degradation of large numbers of people (as the oasis monopoliser so graphically illustrates).


And we stress that anarchists have no desire to harm individuals, only to change institutions. If a workplace is taken over by its workers, the owners are not harmed physically. If the oasis is taken from the monopoliser, the ex-monopoliser becomes like other users of the oasis (although probably {\bf disliked} by others). Thus anarchists desire to treat people as fairly as possible and not replace one form of coercion and domination with another — individuals must {\bf never} be treated as abstractions (if they have power over you, destroy what creates the relation of domination, {\bf not} the individual, in other words! And if this power can be removed without resorting to force, so much the better — a point which social and individualist anarchists disagree on, namely whether capitalism can be reformed away or not comes directly from this. As the Individualists think it can, they oppose the use of force. Most social anarchists think it cannot, and so support revolution).


This argument may be considered as “utilitarian” (the greatest good for the greatest number) and so treats people not as “ends in themselves” but as “means to an end”. Thus, it could be argued, “natural law” is required to ensure that {\bf all} (as opposed to some, or many, or the majority of) individuals are free and have their rights protected.


However, it is clear that “natural law” can easily result in a minority having their freedom and rights respected, while the majority are forced by circumstances (created by the rights/laws produced by applying “natural law” we must note) to sell their liberty and rights in order to survive. If it is wrong to treat anyone as a “means to an end”, then it is equally wrong to support a theory or economic system that results in people having to negate themselves in order to live. A respect for persons — to treat them as ends and never as means — is not compatible with private property.


The simple fact is that {\bf there are no easy answers} — we need to weight up our options and act on what we think is best. Yes, such subjectivism lacks the “elegance” and simplicity of “natural law” but it reflects real life and freedom far better. All in all, we must always remember that what is “good” for man need not be good for people. “Natural law” fails to do this and stands condemned.


\subsection{11.5 But Natural Law was discovered, not invented!
}

This statement truly shows the religious nature of the Natural Law cult. To see why its notion of “discovery” is confused, let us consider the Law of Gravity. Newton did not “discover” the law of gravity, he invented a theory which explained certain observed phenomena in the physical world. Later Einstein updated Newton’s theories in ways that allowed for a better explanation of physical reality. Thus, unlike “Natural Law,” scientific laws can be updated and changed as our knowledge changes and grows. As we have already noted, however, “Natural Laws” cannot be updated because they are derived from fixed definitions (Rothbard is pretty clear on this, he states that it is {\em “[v]ery true”} that natural law is {\em “universal, fixed and immutable”} and so are {\em “‘absolute’ principles of justice”} and that they are {\em “independent of time and place”} [{\bf The Ethics of Liberty}, p. 19]). However, what he fails to understand is that what the “Natural Law” cultists are “discovering” are simply the implications of their own definitions, which in turn simply reflect their own prejudices and preferences.


Since “Natural Laws” are thus “unchanging” and are said to have been “discovered” centuries ago, it’s no wonder that many of its followers look for support in socio-biology, claiming that their “laws” are part of the genetic structure of humanity. But socio-biology has dubious scientific credentials for many of its claims. Also, it has authoritarian implications {\bf exactly} like Natural Law. Murray Bookchin rightly characterises socio-biology as {\em “suffocatingly rigid; it not only impedes action with the autocracy of a genetic tyrant but it closes the door to any action that is not biochemically defined by its own configuration. When freedom is nothing more than the recognition of necessity\unknown{}we discover the gene’s tyranny over the greater totality of life\unknown{}when knowledge becomes dogma (and few movements are more dogmatic than socio-biology) freedom is ultimately denied.”} [{\em “Socio-biology or Social Ecology”}, in {\bf Which way for the Ecology Movement?} pp. 49 — 75, p. 60]


In conclusion the doctrine of Natural Law, far from supporting individual freedom, is one of its greatest enemies. By locating individual rights within “Man’s Nature,” it becomes an unchanging set of dogmas. Do we really know enough about humanity to say what are “Natural” and universal Laws, applicable forever? Is it not a rejection of critical thinking and thus individual freedom to do so?


\subsection{11.6 Why is the notion of “discovery” contradictory?
}

Ayn Rand indicates the illogical and contradictory nature of the concepts of “discovering” “natural law” and the “natural rights” this “discovery” argument creates when she stated that her theory was {\em “objective.”} Her “Objectivist” political theory {\em “holds that good is neither an attribute of ‘things in themselves’ nor man’s emotional state, but {\bf an evaluation} of the facts of reality by man’s consciousness according to a rational standard of value\unknown{} The objective theory holds that {\bf the good is an aspect of reality in relation to man} — and that it must be discovered, not invented, by man.”} [{\bf Capitalism: The Unknown Ideal}, p. 22]


However, this is playing with words. If something is “discovered” then it has always been there and so is an intrinsic part of it. If “good” {\bf is} “discovered” by “man” then “good” exists independently of people — it is waiting to be “discovered.” In other words, “good” is an attribute of {\em “man as man,”} of {\em “things in themselves”} (in addition, such a theory also implies that there is just {\bf one} possible interpretation of what is “good” for all humanity). This can be seen when Rand talks about her system of “objective” values and rights.


When discussing the difference between {\em “subjective,”} {\em “intrinsic”} and {\em “objective”} values Rand noted that {\em “intrinsic”} and {\em “subjective”} theories {\em “make it possible for a man to believe what is good is independent of man’s mind and can be achieved by physical force.”} [{\bf Op. Cit.}, p. 22] In other words, intrinsic and subjective values justify tyranny. However, her {\em “objective”} values are placed squarely in {\em “Man’s Nature”} — she states that {\em “[i]ndividual rights are the means of subordinating society to moral law”} and that {\em “the source of man’s rights is man’s nature.”} [{\bf Op. Cit.}, p. 320, p. 322]


She argues that the {\em “{\bf intrinsic} theory holds that the good is inherent in certain things or actions, as such, regardless of their context and consequences, regardless of any benefit or injury they may cause to the actors and subjects involved.”} [{\bf Op. Cit.}, p. 21] According to the {\bf Concise Oxford Dictionary}, {\em “intrinsic”} is defined as {\em “inherent,”} {\em “essential,”} {\em “belonging naturally”} and defines {\em “nature”} as {\em “a thing’s, or person’s, innate or essential qualities or character.”} In other words, if, as Rand maintains, man’s rights {\bf are} the product of {\em “man’s nature”} then such rights are {\bf intrinsic}! And if, as Rand maintains, such rights are the {\em “extension of morality into the social system”} then morality itself is also intrinsic.


Again, her ideology fails to meet its own tests — and opens the way for tyranny. This can be seen by her whole hearted support for wage slavery and her total lack of concern how it, and concentrations of wealth and power, affect the individuals subjected to them. For, after all, what is “good” is “inherent” in capitalism, regardless of the context, consequences, benefits or injuries it may cause to the actors and subjects involved.


The key to understanding her contradictory and illogical ideology lies in her contradictory use of the word “man.” Sometimes she uses it to describe individuals but usually it is used to describe the human race collectively ({\em “man’s nature,” “man’s consciousness”}). But “Man” does not have a consciousness, only individuals do. Man is an abstraction, it is individuals who live and think, not “Man.” Such “Man worship” — like Natural Law — has all the markings of a religion.


As Max Stirner argues {\em “liberalism is a religion because it separates my essence from me and sets it above me, because it exalts ‘Man’ to the same extent as any other religion does to God\unknown{} it sets me beneath Man.”} [{\bf The Ego and Its Own}, p. 176] Indeed, he {\em “who is infatuated with {\bf Man} leaves persons out of account so far as that infatuation extends, and floats in an ideal, sacred interest. {\bf Man}, you see, is not a person, but an ideal, a spook.”} [{\bf Op. Cit.}, p.79]


Rand argues that we must evaluate {\em “the facts of reality by man’s consciousness according to a rational standard of value”} but who determines that value? She states that {\em “[v]alues are not determined by fiat nor by majority vote”} [p. 24] but, however, neither can they be determined by “man” or “man’s consciousness” because “man” does not exist. Individuals exist and have consciousness and because they are unique have different values (but as we argued in section A.2.19, being social creatures these values are generalised across individuals into social, i.e. objective, values). So, the abstraction “man” does not exist and because of this we see the healthy sight of different individuals convincing others of their ideas and theories by discussion, presenting facts and rational debate. This can be best seen in scientific debate.


The aim of the scientific method is to invent theories that explain facts, the theories are not part of the facts but created by the individual’s mind in order to explain those facts. Such scientific “laws” can and do change in light of new information and new thought. In other words, the scientific method is the creation of subjective theories that explain the objective facts. Rand’s method is the opposite — she assumes “man’s nature,” “discovers” what is “good” from those assumptions and draws her theories by deduction from that. This is the {\bf exact} opposite of the scientific method and, as we noted above, comes to us straight from the Roman Catholic church.


It is the subjective revolt by individuals against what is considered “objective” fact or “common sense” which creates progress and develops ethics (what is considered “good” and “right”) and society. This, in turn, becomes “accepted fact” until the next free thinker comes along and changes how we view the world by presenting {\bf new} evidence, re-evaluating old ideas and facts or exposing the evil effects associated with certain ideas (and the social relationships they reflect) by argument, fact and passion. Attempts to impose {\em “an evaluation of the facts of reality by man’s consciousness”} would be a death blow to this process of critical thought, development and evaluation of the facts of reality by individual’s consciousness. Human thought would be subsumed by dogma.









\page[yes]

%%%% backcover

\startmode[a4imposed,a4imposedbc,letterimposed,letterimposedbc,a5imposed,%
  a5imposedbc,halfletterimposed,halfletterimposedbc,quickimpose]
\alibraryflushpages
\stopmode

\page[blank]

\startalignment[middle]
{\tfa The Anarchist Library
\blank[small]
Anti-Copyright}
\blank[small]
\currentdate
\stopalignment

\blank[big]
\framed[frame=off,location=middle,width=\textwidth]
       {\externalfigure[logo][width=0.25\textwidth]}



\vfill
\setupindenting[no]
\setsmallbodyfont

\startalignment[middle,nothyphenated,nothanging,stretch]

\blank[line]
% \framed[frame=off,location=middle,width=\textwidth]
%       {\externalfigure[logo][width=0.25\textwidth]}


The Anarchist FAQ Editorial Collective



An Anarchist FAQ (12/17)






June 18, 2009. Version 13.1


\stopalignment
\blank[line]

\startalignment[hyphenated,middle]


Copyright (C) 1995–2009 The Anarchist FAQ Editorial Collective: Iain McKay, Gary Elkin, Dave Neal, Ed Boraas\crlf  Permission is granted to copy, distribute and/or modify this document under the terms of the GNU Free Documentation License, Version 1.1 or any later version published by the Free Software Foundation, and/or the terms of the GNU General Public License, Version 2.0 or any later version published by the Free Software Foundation.\crlf  See the Licenses page at \goto{www.gnu.org}[url(http://www.gnu.org/)] for more details.




\stopalignment

\stoptext


