% -*- mode: tex -*-
%%%%%%%%%%%%%%%%%%%%%%%%%%%%%%%%%%%%%%%%%%%%%%%%%%%%%%%%%%%%%%%%%%%%%%%%%%%%%%%%
%                                STANDARD                                      %
%%%%%%%%%%%%%%%%%%%%%%%%%%%%%%%%%%%%%%%%%%%%%%%%%%%%%%%%%%%%%%%%%%%%%%%%%%%%%%%%
\enabletrackers[fonts.missing]
\definefontfeature[default][default]
                  [protrusion=quality,
                    expansion=quality,
                    script=latn]
\setupalign[hz,hanging]
\setuptolerance[tolerant]
\setbreakpoints[compound]
\setupindenting[yes,1em]
\setupfootnotes[way=bychapter,align={hz,hanging}]
\setupbodyfont[modern] % this is a stinky workaround to load lmodern
\setupbodyfont[libertine,11pt]

\setuppagenumbering[alternative=singlesided,location={footer,middle}]
\setupcaptions[width=fit,align={hz,hanging},number=no]

\startmode[a4imposed,a4imposedbc,letterimposed,letterimposedbc,a5imposed,%
  a5imposedbc,halfletterimposed,halfletterimposedbc]
  \setuppagenumbering[alternative=doublesided]
\stopmode

\setupbodyfontenvironment[default][em=italic]


\setupheads[%
  sectionnumber=no,number=no,
  align=flushleft,
  align={flushleft,nothyphenated,verytolerant,stretch},
  indentnext=yes,
  tolerance=verytolerant]

\definehead[awikipart][chapter]

\setuphead[awikipart]
          [%
            number=no,
            footer=empty,
            style=\bfd,
            before={\blank[force,2*big]},
            align={middle,nothyphenated,verytolerant,stretch},
            after={\page[yes]}
          ]

% h3
\setuphead[chapter]
          [style=\bfc]

\setuphead[title]
          [style=\bfc]


% h4
\setuphead[section]
          [style=\bfb]

% h5
\setuphead[subsection]
          [style=\bfa]

% h6
\setuphead[subsubsection]
          [style=bold]


\setuplist[awikipart]
          [alternative=b,
            interaction=all,
            width=0mm,
            distance=0mm,
            before={\blank[medium]},
            after={\blank[small]},
            style=\bfa,
            criterium=all]
\setuplist[chapter]
          [alternative=c,
            interaction=all,
            width=1mm,
            before={\blank[small]},
            style=bold,
            criterium=all]
\setuplist[section]
          [alternative=c,
            interaction=all,
            width=1mm,
            style=\tf,
            criterium=all]
\setuplist[subsection]
          [alternative=c,
            interaction=all,
            width=8mm,
            distance=0mm,
            style=\tf,
            criterium=all]
\setuplist[subsubsection]
          [alternative=c,
            interaction=all,
            width=15mm,
            style=\tf,
            criterium=all]


% center

\definestartstop
  [awikicenter]
  [before={\blank[line]\startalignment[middle]},
   after={\stopalignment\blank[line]}]

% right

\definestartstop
  [awikiright]
  [before={\blank[line]\startalignment[flushright]},
   after={\stopalignment\blank[line]}]


% blockquote

\definestartstop
  [blockquote]
  [before={\blank[big]
    \setupnarrower[middle=1em]
    \startnarrower
    \setupindenting[no]
    \setupwhitespace[medium]},
  after={\stopnarrower
    \blank[big]}]

% verse

\definestartstop
  [awikiverse]
  [before={\blank[big]
      \setupnarrower[middle=2em]
      \startnarrower
      \startlines},
    after={\stoplines
      \stopnarrower
      \blank[big]}]

\definestartstop
  [awikibiblio]
  [before={%
      \blank[big]
      \setupnarrower[left=1em]
      \startnarrower[left]
        \setupindenting[yes,-1em,first]},
    after={\stopnarrower
      \blank[big]}]
                
% same as above, but with no spacing around
\definestartstop
  [awikiplay]
  [before={%
      \setupnarrower[left=1em]
      \startnarrower[left]
        \setupindenting[yes,-1em,first]},
    after={\stopnarrower}]



% interaction
% we start the interaction only if it's not an imposed format.
\startnotmode[a4imposed,a4imposedbc,letterimposed,letterimposedbc,a5imposed,%
  a5imposedbc,halfletterimposed,halfletterimposedbc]
  \setupinteraction[state=start,color=black,contrastcolor=black,style=bold]
  \placebookmarks[awikipart,chapter,section,subsection,subsubsection][force=yes]
  \setupinteractionscreen[option=bookmark]
\stopnotmode



\setupexternalfigures[%
  maxwidth=\textwidth,
  maxheight=\textheight,
  factor=fit]

\setupitemgroup[itemize][each][packed][indenting=no]

\definemakeup[titlepage][pagestate=start,doublesided=no]

%%%%%%%%%%%%%%%%%%%%%%%%%%%%%%%%%%%%%%%%%%%%%%%%%%%%%%%%%%%%%%%%%%%%%%%%%%%%%%%%
%                                IMPOSER                                       %
%%%%%%%%%%%%%%%%%%%%%%%%%%%%%%%%%%%%%%%%%%%%%%%%%%%%%%%%%%%%%%%%%%%%%%%%%%%%%%%%

\startusercode

function optimize_signature(pages,min,max)
   local minsignature = min or 40
   local maxsignature = max or 80
   local originalpages = pages

   -- here we want to be sure that the max and min are actual *4
   if (minsignature%4) ~= 0 then
      global.texio.write_nl('term and log', "The minsig you provided is not a multiple of 4, rounding up")
      minsignature = minsignature + (4 - (minsignature % 4))
   end
   if (maxsignature%4) ~= 0 then
      global.texio.write_nl('term and log', "The maxsig you provided is not a multiple of 4, rounding up")
      maxsignature = maxsignature + (4 - (maxsignature % 4))
   end
   global.assert((minsignature % 4) == 0, "I suppose something is wrong, not a n*4")
   global.assert((maxsignature % 4) == 0, "I suppose something is wrong, not a n*4")

   --set needed pages to and and signature to 0
   local neededpages, signature = 0,0

   -- this means that we have to work with n*4, if not, add them to
   -- needed pages 
   local modulo = pages % 4
   if modulo==0 then
      signature=pages
   else
      neededpages = 4 - modulo
   end

   -- add the needed pages to pages
   pages = pages + neededpages
   
   if ((minsignature == 0) or (maxsignature == 0)) then 
      signature = pages -- the whole text
   else
      -- give a try with the signature
      signature = find_signature(pages, maxsignature)
      
      -- if the pages, are more than the max signature, find the right one
      if pages>maxsignature then
	 while signature<minsignature do
	    pages = pages + 4
	    neededpages = 4 + neededpages
	    signature = find_signature(pages, maxsignature)
	    --         global.texio.write_nl('term and log', "Trying signature of " .. signature)
	 end
      end
      global.texio.write_nl('term and log', "Parameters:: maxsignature=" .. maxsignature ..
		   " minsignature=" .. minsignature)

   end
   global.texio.write_nl('term and log', "ImposerMessage:: Original pages: " .. originalpages .. "; " .. 
	 "Signature is " .. signature .. ", " ..
	 neededpages .. " pages are needed, " .. 
	 pages ..  " of output")
   -- let's do it
   tex.print("\\dorecurse{" .. neededpages .. "}{\\page[empty]}")

end

function find_signature(number, maxsignature)
   global.assert(number>3, "I can't find the signature for" .. number .. "pages")
   global.assert((number % 4) == 0, "I suppose something is wrong, not a n*4")
   local i = maxsignature
   while i>0 do
      -- global.texio.write_nl('term and log', "Trying " .. i  .. "for max of " .. maxsignature)
      if (number % i) == 0 then
	 return i
      end
      i = i - 4
   end
end

\stopusercode

\define[1]\fillthesignature{
  \usercode{optimize_signature(#1, 40, 80)}}


\define\alibraryflushpages{
  \page[yes] % reset the page
  \fillthesignature{\the\realpageno}
}


% various papers 
\definepapersize[halfletter][width=5.5in,height=8.5in]
\definepapersize[halfafour][width=148.5mm,height=210mm]
\definepapersize[quarterletter][width=4.25in,height=5.5in]
\definepapersize[halfafive][width=105mm,height=148mm]
\definepapersize[generic][width=210mm,height=279.4mm]

%% this is the default ``paper'' which should work with both letter and a4

\setuppapersize[generic][generic]
\setuplayout[%
  backspace=42mm,
  topspace=31mm,% 176 / 15
  height=195mm,%130mm,
  footer=9mm, %
  header=0pt, % no header
  width=126mm] % 10.5 x 11

\startmode[libertine]
  \usetypescript[libertine]
  \setupbodyfont[libertine,11pt]
\stopmode

\startmode[pagella]
  \setupbodyfont[pagella,11pt]
\stopmode

\startmode[antykwa]
  \setupbodyfont[antykwa-poltawskiego,11pt]
\stopmode

\startmode[iwona]
  \setupbodyfont[iwona-medium,11pt]
\stopmode

\startmode[helvetica]
  \setupbodyfont[heros,11pt]
\stopmode

\startmode[century]
  \setupbodyfont[schola,11pt]
\stopmode

\startmode[modern]
  \setupbodyfont[modern,11pt]
\stopmode

\startmode[charis]
  \setupbodyfont[charis,11pt]
\stopmode        

\startmode[mini]
  \setuppapersize[S33][S33] % 176 × 176 mm
  \setuplayout[%
    backspace=20pt,
    topspace=15pt,% 176 / 15
    height=280pt,%130mm,
    footer=20pt, %
    header=0pt, % no header
    width=260pt] % 10.5 x 11
\stopmode

% for the plain A4 and letter, we use the classic LaTeX dimensions
% from the article class
\startmode[a4]
  \setuppapersize[A4][A4]
  \setuplayout[%
    backspace=42mm,
    topspace=45mm,
    height=218mm,
    footer=10mm,
    header=0pt, % no header
    width=126mm]
\stopmode

\startmode[letter]
  \setuppapersize[letter][letter]
  \setuplayout[%
    backspace=44mm,
    topspace=46mm,
    height=199mm,
    footer=10mm,
    header=0pt, % no header
    width=126mm]
\stopmode


% A4 imposed (A5), with no bc

\startmode[a4imposed]
% DIV=15 148 × 210: these are meant not to have binding correction,
  % but just to play safe, let's say 1mm => 147x210
  \setuppapersize[halfafour][halfafour]
  \setuplayout[%
    backspace=10.8mm, % 146/15 = 9.8 + 1
    topspace=14mm, % 210/15 =  14
    height=182mm, % 14 x 12 + 14 of the footer
    footer=14mm, %
    header=0pt, % no header
    width=117.6mm] % 9.8 x 12
\stopmode

% A4 imposed (A5), with bc
\startmode[a4imposedbc]
  \setuppapersize[halfafour][halfafour]
  \setuplayout[% 14 mm was a bit too near to the spine, using the glue binding
    backspace=17.3mm,  % 140/15 + 8 =
    topspace=14mm, % 210/15 =  14
    height=182mm, % 14 x 12 + 14 of the footer
    footer=14mm, %
    header=0pt, % no header
    width=112mm] % 9.333 x 12
\stopmode


\startmode[letterimposedbc] % 139.7mm x 215.9 mm
  \setuppapersize[halfletter][halfletter]
  % DIV=15 8mm binding corr, => 132 x 216
  \setuplayout[%
    backspace=16.8mm, % 8.8 + 8
    topspace=14.4mm, % 216/15 =  14.4
    height=187.2mm, % 15.4 x 11 + 15 of the footer
    footer=14.4mm, %
    header=0pt, % no header
    width=105.6mm] % 8.8 x 12
\stopmode

\startmode[letterimposed] % 139.7mm x 215.9 mm
  \setuppapersize[halfletter][halfletter]
  % DIV=15, 1mm binding correction. => 138.7x215.9
  \setuplayout[%
    backspace=10.3mm, % 9.24 + 1
    topspace=14.4mm, % 216/15 =  14.4
    height=187.2mm, % 15.4 x 11 + 15 of the footer
    footer=14.4mm, %
    header=0pt, % no header
    width=111mm] % 9.24 x 12
\stopmode

%%% new formats for mini books
%%% \definepapersize[halfafive][width=105mm,height=148mm]

\startmode[a5imposed]
% DIV=12 105x148 : these are meant not to have binding correction,
  % but just to play safe, let's say 1mm => 104x148
  \setuppapersize[halfafive][halfafive]
  \setuplayout[%
    backspace=9.6mm,
    topspace=12.3mm,
    height=123.5mm, % 14 x 12 + 14 of the footer
    footer=12.3mm, %
    header=0pt, % no header
    width=78.8mm] % 9.8 x 12
\stopmode

% A5 imposed (A6), with bc
\startmode[a5imposedbc]
% DIV=12 105x148 : with binding correction,
  % let's say 8mm => 96x148
  \setuppapersize[halfafive][halfafive]
  \setuplayout[%
    backspace=16mm,
    topspace=12.3mm,
    height=123.5mm, % 14 x 12 + 14 of the footer
    footer=12.3mm, %
    header=0pt, % no header
    width=72mm] % 9.8 x 12
\stopmode

%%% \definepapersize[quarterletter][width=4.25in,height=5.5in]

% DIV=12 width=4.25in (108mm),height=5.5in (140mm) 
\startmode[halfletterimposed] % 107x140
  \setuppapersize[quarterletter][quarterletter]
  \setuplayout[%
    backspace=10mm,
    topspace=11.6mm,
    height=116mm,
    footer=11.6mm,
    header=0pt, % no header
    width=80mm] % 9.24 x 12
\stopmode

\startmode[halfletterimposedbc]
  \setuppapersize[quarterletter][quarterletter]
  \setuplayout[%
    backspace=15.4mm,
    topspace=11.6mm,
    height=116mm,
    footer=11.6mm,
    header=0pt, % no header
    width=76mm] % 9.24 x 12
\stopmode

\startmode[quickimpose]
  \setuppapersize[A5][A4,landscape]
  \setuparranging[2UP]
  \setuppagenumbering[alternative=doublesided]
  \setuplayout[% 14 mm was a bit too near to the spine, using the glue binding
    backspace=17.3mm,  % 140/15 + 8 =
    topspace=14mm, % 210/15 =  14
    height=182mm, % 14 x 12 + 14 of the footer
    footer=14mm, %
    header=0pt, % no header
    width=112mm] % 9.333 x 12
\stopmode

\startmode[tenpt]
  \setupbodyfont[10pt]
\stopmode

\startmode[twelvept]
  \setupbodyfont[12pt]
\stopmode

%%%%%%%%%%%%%%%%%%%%%%%%%%%%%%%%%%%%%%%%%%%%%%%%%%%%%%%%%%%%%%%%%%%%%%%%%%%%%%%%
%                            DOCUMENT BEGINS                                   %
%%%%%%%%%%%%%%%%%%%%%%%%%%%%%%%%%%%%%%%%%%%%%%%%%%%%%%%%%%%%%%%%%%%%%%%%%%%%%%%%


\mainlanguage[en]


\starttext

\starttitlepagemakeup
  \startalignment[middle,nothanging,nothyphenated,stretch]


  \switchtobodyfont[18pt] % author
  {\bf \em

Anonymous  \par}
  \blank[2*big]
  \switchtobodyfont[24pt] % title
  {\bf

Sheep in Wolves’ Clothing

\par}
  \blank[big]
  \switchtobodyfont[20pt] % subtitle
  {\bf 

the end of activism and other related thoughts

\par}
  \vfill
  \stopalignment
  \startalignment[middle,bottom,nothyphenated,stretch,nothanging]
  \switchtobodyfont[global]

July 1014

  \stopalignment
\stoptitlepagemakeup



\page[yes,right]

This piece of writing has developed from a recent interaction I had with the local activist scene\footnote{I use the term “scene” because it does not represent an organization or milieu since there is no set group of people, but is mostly random with a small handful of regulars.}, as well as from prior experiences of my brief involvement within this group. I recently attempted to criticize the actions of these specific people but my ideas were swept under the rug as elitist, and I an inactive “armchair revolutionary”\footnote{Throughout this writing I will put quotations around certain terms, indicating that while they are popular words that are thrown around casually, there’s not always a consensus on what these terms actually mean, and thus I’m reluctant to use them but to do out of simplicity or lack of better word at the time.}. In other words, my critique was swept aside as irrelevant because of my lack of activist street-cred. I do not deplore these descriptions aimed at me nor am I offended by their statements as I see that it emerges from their inability to receive criticism. In fact, the situation has provided me with an opportunity to elucidate some ideas that I’ve previously found difficult to articulate, however, this interaction helped me put them in context and for that I am thankful. Normally I wouldn’t consider this small disagreement a worthwhile discussion, but I believe that within it there are some necessary points to be made and some false illusions to shatter. I also assume that this discussion can be useful to others if they so wish to engage.


My original critique was that of symbolic protests and their ineffectuality and inherent moralism. By symbolic protest, I mean an action that wishes to show distaste towards a particular issue without having any material effect on the status quo of capital accumulation. Appealing to emotion and obtaining the moral high ground are common tactics of such an action. The local “activist scene” is well versed in these kinds of actions; holding signs, ambiguously attempting to change public opinion, and trying their darn hardest to get enough people to attend\footnote{The emphasis on “getting numbers” is a large part of activist culture, mostly at the expense of quality of actions. Quality is often ignored and one can then easily blame the failure of an action on not enough people showing up, instead of perhaps looking at the real causes. The “next time we’ll just have to get more numbers” is a clever deterrence from self-analysis, especially because no one is quite sure what that magical number could be, hence no amount of numbers is ever enough.}. In this particular situation it was a counter-demonstration towards a local group of people who are protesting against a social service center holding refugee peoples from foreign countries who were trying to cross the U.S./Mexican border for various reasons. I’ll not go into the politics of this situation, nor will I discuss the actual protest in length but only the ideas contained within it. According to the hosts of the event, there existed obvious racism and nationalism in the initial protest, so the local IWW chapter\footnote{Industrial Workers of the World} organized a counter-demonstration.


In an attempt for the IWW to connect their ideology to that of this particular demonstration, they tried to make themselves relevant in two ways. The first was by calling the refugee children, “working class”.\footnote{They have since taken this statement off of the online event page the day before the event for reasons unknown to me.}  This sounds nice but is ultimately untrue by definition. To be part of the working class, one must be employed and therefore in direct contact with the means of production owned by the ruling class. By indiscriminately labeling someone as working class, it distorts the class struggle and undermines anyone who is indeed in this economic position and hence their primacy in the overthrow of capital. The second attempt was to connect anti-racism with the working class struggle. Again, this sounds acceptable but if we look closer, it is another attempt at the same distortion. The class struggle is the result of the economic structures of capitalism, whereas racism, while upheld by these structures, is only a result of them. Racism, like sexism, are social constructs that are {\em exploited} by capital in order to provide cheap or free labor\footnote{This is applicable to sexism as well as racism. Similarly to how the slave trade provided free labor to capitalists which the economy of this country was largely founded upon; sexism, the division of labor between the social constructed division between men and women, has in history and continues to provide cheap and free labor to the capitalist economy mostly in the form of housework, childcare, etc. This also relates to the so-called “immigration question”, where a race of people are demonized through public opinion and the media as a moral justification for paying them extremely low wages, but are in fact a large part of cheap labor in the U.S. economy. Without this demonization, there would be a demand that these “immigrants” get a normal wage like everyone else. The anti-immigrant people can spout nationalism all day but don’t mind purchasing the cheap fruits and vegetables that are only affordable through this immigrant cheap labor forcefully imposed by capital. This is also a good example of how systematic structures generate public opinion, not the (commonly thought but fundamentally false) other way around, because it is seen how capital benefits from its manufacturing of opinion, and therefore less about racism than it is about the accumulation of capital through the means of this cheap labor.}, but it is not the goal of the working class to fight against the {\em symptoms} of capital, but instead the roots, that of class division and wage labor.\footnote{Some might interpret this as anti-antiracism. Besides the fact that it is untrue, I simply state these ideas because I believe a distinction is necessary. I acknowledge the totality that encompasses race and capital but believe it to be important that we have an understanding of how they interact, mostly in order to recognize that totality more clearly in order to better sharpen our daggers for its attack.}  So the fact that a union organization is hosting an anti-racist demonstration perhaps shows that they have veered off the path of the class struggle and have now ventured into something else entirely.


The counter-demonstration was primarily promoted as an anti-fascist event. Through the tactic of standing on the opposite side of the street with cardboard signs attempting to shame the initial protesters (while graciously letting the police mediate this interaction, for the “safety of the protesters), they claim they are fighting fascism in their “communities”. Let’s look at this a little closer (besides the obvious reality that they are “fighting fascism” but have no fucking problem about the police being in their presence). Although the ‘movement’ of anti-fascism has been held up throughout the years as a necessary struggle within capitalism, I’m inclined to say that it’s actually destructive, seeing that it fools people into confusionism by serving to blur the lines between the subjectivity of opinion and the objectivity of material reality. The main point that is brought against anti-fascism is that it attempts to fight with ideas, with opinions. For instance, promoting the idea that racism resides within individual mindsets, and that it can be fought by confronting these specific people, is a false notion and shows a underdeveloped understanding of reality and not only undermines but hides the fact that such ideas as racism have a systematic foundation, and that this foundation is upheld because attention is diverted into the realm of individual confrontation. Anti-fascism assumes that opinions, ideas or social opinion generate structural systems, when in fact the opposite is true. This truth is hidden and capital, with its primary role in upholding and materially benefiting from this diversion, walks away smiling. In this way, when ‘anti-fascists’ attempt to confront fascism within individual people, their behavior is essentially pro-capitalist.


But to move back into the realm of micro-drama, I’d like to consider the points that were made against my initial critique, not to defend myself in any way but to extract from them some concepts that I believe are worthy of discussion.


For some reason, there seems to be this idea that if you’re going to offer critique, then you are theoretically obligated to provide an alternative or redirection for the sake of constructiveness. This mentality is a result of bourgeois morality, where productivity takes center stage, and where destruction, whether of private property or ideological illusions, is morally wrong and quickly condemned in and of itself. This is nothing but a tactic to sidestep the critique by putting the pressure back onto the critic instead of {\em looking at the actions that brought about this critique in the first place}. By avoiding the initial critique and immediately demanding an alternative, it seems that one is trying to build a castle on top of water, not to mention being overly dependent on others for their own theoretical growth. It is not the job of the critic to give suggestions, but to offer critique. I will destroy falsities but I will not tell you what to do, just as I don’t expect someone to tell me what to do in order for me to do it. However, if you look closely enough in between the lines, I’m sure you can extract the main ideas from this critique and evaluate them for yourself if you find them useful.


The other charge that my critique is less valid because of my “lack of activity” is almost funny. Activism is a social identity that is based on a dichotomy between activism and it’s opposite, non-activism, or instead, political inactivity.  Within this dichotomy, activists justify themselves solely on the basis of their opposites, ie “doing something is better than doing nothing”\footnote{I’m not sure where this idea (read: moral ploy) originates from but it seems to be the foundation of activism.}. As a result, following the activist logic, the actions one participates in are entirely justified the moment they point to someone who is “inactive”.  But unfortunately for the identity of the activist, this other of inactivity does not exist, since within the apparatus of capitalism that dominates every aspect of daily life, we have no choice whether or not we a part of it. Sidestepping critique because “at least I’m doing {\em something”} has no legitimate foundation, is irrelevant to the conversation and often serves as nothing but self-assurance.\footnote{The fact that I’ve heard people descriptively list the things that they have accomplished, without me asking, in order to show me my utterly contemptible inactivity, seems to clarify this point quite accurately.  Your text here\unknown{}}


Despite any subjectivities or moral claims, there exists an objective economic reality that we all take part in, and since it creates the conditions for the daily maintenance of social relations, we currently have no significant effect on it other than its perpetuation. I coin this term {\em economic realism} and I believe it to be significant that this idea receives more attention. This concept is naturally {\em amoral} because it acknowledges capital accumulation and the resulting class struggle not as a set of opinions or ideas but as a material reality, and sees these aspects of idealism as an obstacle to seeing this reality clearly, thus affecting our goal of actualizing the end of capital. I am not suggesting that we must all become {\em realists} in the conventional sense of recognizing that we can have no impact on this world, but suggesting the opposite and find that putting this concept to use theoretically can provide us with a lens for looking at things in a more honest manner, where we can begin to look for ways to move us closer to that goal. For instance, once we begin to utilize this concept, we can see more clearly and honestly that most of the actions are not much more than feel-good activities, drenched in restrictive moralism and change-the-world illusions that are ultimately irrelevant to the class struggle that some claim to represent, and have little, if at all, effect on the perpetuation of capital. This is no concrete set of rules to be followed, but concepts to be played around with, to be added onto; I can merely provide some creative tools, but I will not force it upon someone to build with them.


After all, I am not attempting to get anyone to “change their ways”.  I am simply developing my own theories by seeing through the misconceptions of activist identity and ideology. This identity, with its herd mentality and puritan morality, is propped up on false assumptions and if anything it is my intention to expose this.  I’m not interested in groupthink, socially-conditioned morality or false unity, and see these things as theoretical laziness. I could go on and on about these things, but to be honest I’m kinda over it. I’ll continue to meditate on these concepts and ideas and perhaps elaborate on them more deeply another time, but for now, I think I’ll go sit comfortably on my armchair, preparing to criticize any word or action that I believe rightfully deserves it, because any revolution that deters criticism is not a revolution that I want to be a part of.


And it’s as simple as that.









\page[yes]

%%%% backcover

\startmode[a4imposed,a4imposedbc,letterimposed,letterimposedbc,a5imposed,%
  a5imposedbc,halfletterimposed,halfletterimposedbc,quickimpose]
\alibraryflushpages
\stopmode

\page[blank]

\startalignment[middle]
{\tfa The Anarchist Library
\blank[small]
Anti-Copyright}
\blank[small]
\currentdate
\stopalignment

\blank[big]
\framed[frame=off,location=middle,width=\textwidth]
       {\externalfigure[logo][width=0.25\textwidth]}



\vfill
\setupindenting[no]
\setsmallbodyfont

\startalignment[middle,nothyphenated,nothanging,stretch]

\blank[line]
% \framed[frame=off,location=middle,width=\textwidth]
%       {\externalfigure[logo][width=0.25\textwidth]}


Anonymous



Sheep in Wolves’ Clothing



the end of activism and other related thoughts




July 1014


\stopalignment
\blank[line]

\startalignment[hyphenated,middle]




scanned from original


\stopalignment

\stoptext


