% -*- mode: tex -*-
%%%%%%%%%%%%%%%%%%%%%%%%%%%%%%%%%%%%%%%%%%%%%%%%%%%%%%%%%%%%%%%%%%%%%%%%%%%%%%%%
%                                STANDARD                                      %
%%%%%%%%%%%%%%%%%%%%%%%%%%%%%%%%%%%%%%%%%%%%%%%%%%%%%%%%%%%%%%%%%%%%%%%%%%%%%%%%
\enabletrackers[fonts.missing]
\definefontfeature[default][default]
                  [protrusion=quality,
                    expansion=quality,
                    script=latn]
\setupalign[hz,hanging]
\setuptolerance[tolerant]
\setbreakpoints[compound]
\setupindenting[yes,1em]
\setupfootnotes[way=bychapter,align={hz,hanging}]
\setupbodyfont[modern] % this is a stinky workaround to load lmodern
\setupbodyfont[libertine,11pt]

\setuppagenumbering[alternative=singlesided,location={footer,middle}]
\setupcaptions[width=fit,align={hz,hanging},number=no]

\startmode[a4imposed,a4imposedbc,letterimposed,letterimposedbc,a5imposed,%
  a5imposedbc,halfletterimposed,halfletterimposedbc]
  \setuppagenumbering[alternative=doublesided]
\stopmode

\setupbodyfontenvironment[default][em=italic]


\setupheads[%
  sectionnumber=no,number=no,
  align=flushleft,
  align={flushleft,nothyphenated,verytolerant,stretch},
  indentnext=yes,
  tolerance=verytolerant]

\definehead[awikipart][chapter]

\setuphead[awikipart]
          [%
            number=no,
            footer=empty,
            style=\bfd,
            before={\blank[force,2*big]},
            align={middle,nothyphenated,verytolerant,stretch},
            after={\page[yes]}
          ]

% h3
\setuphead[chapter]
          [style=\bfc]

\setuphead[title]
          [style=\bfc]


% h4
\setuphead[section]
          [style=\bfb]

% h5
\setuphead[subsection]
          [style=\bfa]

% h6
\setuphead[subsubsection]
          [style=bold]


\setuplist[awikipart]
          [alternative=b,
            interaction=all,
            width=0mm,
            distance=0mm,
            before={\blank[medium]},
            after={\blank[small]},
            style=\bfa,
            criterium=all]
\setuplist[chapter]
          [alternative=c,
            interaction=all,
            width=1mm,
            before={\blank[small]},
            style=bold,
            criterium=all]
\setuplist[section]
          [alternative=c,
            interaction=all,
            width=1mm,
            style=\tf,
            criterium=all]
\setuplist[subsection]
          [alternative=c,
            interaction=all,
            width=8mm,
            distance=0mm,
            style=\tf,
            criterium=all]
\setuplist[subsubsection]
          [alternative=c,
            interaction=all,
            width=15mm,
            style=\tf,
            criterium=all]


% center

\definestartstop
  [awikicenter]
  [before={\blank[line]\startalignment[middle]},
   after={\stopalignment\blank[line]}]

% right

\definestartstop
  [awikiright]
  [before={\blank[line]\startalignment[flushright]},
   after={\stopalignment\blank[line]}]


% blockquote

\definestartstop
  [blockquote]
  [before={\blank[big]
    \setupnarrower[middle=1em]
    \startnarrower
    \setupindenting[no]
    \setupwhitespace[medium]},
  after={\stopnarrower
    \blank[big]}]

% verse

\definestartstop
  [awikiverse]
  [before={\blank[big]
      \setupnarrower[middle=2em]
      \startnarrower
      \startlines},
    after={\stoplines
      \stopnarrower
      \blank[big]}]

\definestartstop
  [awikibiblio]
  [before={%
      \blank[big]
      \setupnarrower[left=1em]
      \startnarrower[left]
        \setupindenting[yes,-1em,first]},
    after={\stopnarrower
      \blank[big]}]
                
% same as above, but with no spacing around
\definestartstop
  [awikiplay]
  [before={%
      \setupnarrower[left=1em]
      \startnarrower[left]
        \setupindenting[yes,-1em,first]},
    after={\stopnarrower}]



% interaction
% we start the interaction only if it's not an imposed format.
\startnotmode[a4imposed,a4imposedbc,letterimposed,letterimposedbc,a5imposed,%
  a5imposedbc,halfletterimposed,halfletterimposedbc]
  \setupinteraction[state=start,color=black,contrastcolor=black,style=bold]
  \placebookmarks[awikipart,chapter,section,subsection,subsubsection][force=yes]
  \setupinteractionscreen[option=bookmark]
\stopnotmode



\setupexternalfigures[%
  maxwidth=\textwidth,
  maxheight=\textheight,
  factor=fit]

\setupitemgroup[itemize][each][packed][indenting=no]

\definemakeup[titlepage][pagestate=start,doublesided=no]

%%%%%%%%%%%%%%%%%%%%%%%%%%%%%%%%%%%%%%%%%%%%%%%%%%%%%%%%%%%%%%%%%%%%%%%%%%%%%%%%
%                                IMPOSER                                       %
%%%%%%%%%%%%%%%%%%%%%%%%%%%%%%%%%%%%%%%%%%%%%%%%%%%%%%%%%%%%%%%%%%%%%%%%%%%%%%%%

\startusercode

function optimize_signature(pages,min,max)
   local minsignature = min or 40
   local maxsignature = max or 80
   local originalpages = pages

   -- here we want to be sure that the max and min are actual *4
   if (minsignature%4) ~= 0 then
      global.texio.write_nl('term and log', "The minsig you provided is not a multiple of 4, rounding up")
      minsignature = minsignature + (4 - (minsignature % 4))
   end
   if (maxsignature%4) ~= 0 then
      global.texio.write_nl('term and log', "The maxsig you provided is not a multiple of 4, rounding up")
      maxsignature = maxsignature + (4 - (maxsignature % 4))
   end
   global.assert((minsignature % 4) == 0, "I suppose something is wrong, not a n*4")
   global.assert((maxsignature % 4) == 0, "I suppose something is wrong, not a n*4")

   --set needed pages to and and signature to 0
   local neededpages, signature = 0,0

   -- this means that we have to work with n*4, if not, add them to
   -- needed pages 
   local modulo = pages % 4
   if modulo==0 then
      signature=pages
   else
      neededpages = 4 - modulo
   end

   -- add the needed pages to pages
   pages = pages + neededpages
   
   if ((minsignature == 0) or (maxsignature == 0)) then 
      signature = pages -- the whole text
   else
      -- give a try with the signature
      signature = find_signature(pages, maxsignature)
      
      -- if the pages, are more than the max signature, find the right one
      if pages>maxsignature then
	 while signature<minsignature do
	    pages = pages + 4
	    neededpages = 4 + neededpages
	    signature = find_signature(pages, maxsignature)
	    --         global.texio.write_nl('term and log', "Trying signature of " .. signature)
	 end
      end
      global.texio.write_nl('term and log', "Parameters:: maxsignature=" .. maxsignature ..
		   " minsignature=" .. minsignature)

   end
   global.texio.write_nl('term and log', "ImposerMessage:: Original pages: " .. originalpages .. "; " .. 
	 "Signature is " .. signature .. ", " ..
	 neededpages .. " pages are needed, " .. 
	 pages ..  " of output")
   -- let's do it
   tex.print("\\dorecurse{" .. neededpages .. "}{\\page[empty]}")

end

function find_signature(number, maxsignature)
   global.assert(number>3, "I can't find the signature for" .. number .. "pages")
   global.assert((number % 4) == 0, "I suppose something is wrong, not a n*4")
   local i = maxsignature
   while i>0 do
      -- global.texio.write_nl('term and log', "Trying " .. i  .. "for max of " .. maxsignature)
      if (number % i) == 0 then
	 return i
      end
      i = i - 4
   end
end

\stopusercode

\define[1]\fillthesignature{
  \usercode{optimize_signature(#1, 40, 80)}}


\define\alibraryflushpages{
  \page[yes] % reset the page
  \fillthesignature{\the\realpageno}
}


% various papers 
\definepapersize[halfletter][width=5.5in,height=8.5in]
\definepapersize[halfafour][width=148.5mm,height=210mm]
\definepapersize[quarterletter][width=4.25in,height=5.5in]
\definepapersize[halfafive][width=105mm,height=148mm]
\definepapersize[generic][width=210mm,height=279.4mm]

%% this is the default ``paper'' which should work with both letter and a4

\setuppapersize[generic][generic]
\setuplayout[%
  backspace=42mm,
  topspace=31mm,% 176 / 15
  height=195mm,%130mm,
  footer=9mm, %
  header=0pt, % no header
  width=126mm] % 10.5 x 11

\startmode[libertine]
  \usetypescript[libertine]
  \setupbodyfont[libertine,11pt]
\stopmode

\startmode[pagella]
  \setupbodyfont[pagella,11pt]
\stopmode

\startmode[antykwa]
  \setupbodyfont[antykwa-poltawskiego,11pt]
\stopmode

\startmode[iwona]
  \setupbodyfont[iwona-medium,11pt]
\stopmode

\startmode[helvetica]
  \setupbodyfont[heros,11pt]
\stopmode

\startmode[century]
  \setupbodyfont[schola,11pt]
\stopmode

\startmode[modern]
  \setupbodyfont[modern,11pt]
\stopmode

\startmode[charis]
  \setupbodyfont[charis,11pt]
\stopmode        

\startmode[mini]
  \setuppapersize[S33][S33] % 176 × 176 mm
  \setuplayout[%
    backspace=20pt,
    topspace=15pt,% 176 / 15
    height=280pt,%130mm,
    footer=20pt, %
    header=0pt, % no header
    width=260pt] % 10.5 x 11
\stopmode

% for the plain A4 and letter, we use the classic LaTeX dimensions
% from the article class
\startmode[a4]
  \setuppapersize[A4][A4]
  \setuplayout[%
    backspace=42mm,
    topspace=45mm,
    height=218mm,
    footer=10mm,
    header=0pt, % no header
    width=126mm]
\stopmode

\startmode[letter]
  \setuppapersize[letter][letter]
  \setuplayout[%
    backspace=44mm,
    topspace=46mm,
    height=199mm,
    footer=10mm,
    header=0pt, % no header
    width=126mm]
\stopmode


% A4 imposed (A5), with no bc

\startmode[a4imposed]
% DIV=15 148 × 210: these are meant not to have binding correction,
  % but just to play safe, let's say 1mm => 147x210
  \setuppapersize[halfafour][halfafour]
  \setuplayout[%
    backspace=10.8mm, % 146/15 = 9.8 + 1
    topspace=14mm, % 210/15 =  14
    height=182mm, % 14 x 12 + 14 of the footer
    footer=14mm, %
    header=0pt, % no header
    width=117.6mm] % 9.8 x 12
\stopmode

% A4 imposed (A5), with bc
\startmode[a4imposedbc]
  \setuppapersize[halfafour][halfafour]
  \setuplayout[% 14 mm was a bit too near to the spine, using the glue binding
    backspace=17.3mm,  % 140/15 + 8 =
    topspace=14mm, % 210/15 =  14
    height=182mm, % 14 x 12 + 14 of the footer
    footer=14mm, %
    header=0pt, % no header
    width=112mm] % 9.333 x 12
\stopmode


\startmode[letterimposedbc] % 139.7mm x 215.9 mm
  \setuppapersize[halfletter][halfletter]
  % DIV=15 8mm binding corr, => 132 x 216
  \setuplayout[%
    backspace=16.8mm, % 8.8 + 8
    topspace=14.4mm, % 216/15 =  14.4
    height=187.2mm, % 15.4 x 11 + 15 of the footer
    footer=14.4mm, %
    header=0pt, % no header
    width=105.6mm] % 8.8 x 12
\stopmode

\startmode[letterimposed] % 139.7mm x 215.9 mm
  \setuppapersize[halfletter][halfletter]
  % DIV=15, 1mm binding correction. => 138.7x215.9
  \setuplayout[%
    backspace=10.3mm, % 9.24 + 1
    topspace=14.4mm, % 216/15 =  14.4
    height=187.2mm, % 15.4 x 11 + 15 of the footer
    footer=14.4mm, %
    header=0pt, % no header
    width=111mm] % 9.24 x 12
\stopmode

%%% new formats for mini books
%%% \definepapersize[halfafive][width=105mm,height=148mm]

\startmode[a5imposed]
% DIV=12 105x148 : these are meant not to have binding correction,
  % but just to play safe, let's say 1mm => 104x148
  \setuppapersize[halfafive][halfafive]
  \setuplayout[%
    backspace=9.6mm,
    topspace=12.3mm,
    height=123.5mm, % 14 x 12 + 14 of the footer
    footer=12.3mm, %
    header=0pt, % no header
    width=78.8mm] % 9.8 x 12
\stopmode

% A5 imposed (A6), with bc
\startmode[a5imposedbc]
% DIV=12 105x148 : with binding correction,
  % let's say 8mm => 96x148
  \setuppapersize[halfafive][halfafive]
  \setuplayout[%
    backspace=16mm,
    topspace=12.3mm,
    height=123.5mm, % 14 x 12 + 14 of the footer
    footer=12.3mm, %
    header=0pt, % no header
    width=72mm] % 9.8 x 12
\stopmode

%%% \definepapersize[quarterletter][width=4.25in,height=5.5in]

% DIV=12 width=4.25in (108mm),height=5.5in (140mm) 
\startmode[halfletterimposed] % 107x140
  \setuppapersize[quarterletter][quarterletter]
  \setuplayout[%
    backspace=10mm,
    topspace=11.6mm,
    height=116mm,
    footer=11.6mm,
    header=0pt, % no header
    width=80mm] % 9.24 x 12
\stopmode

\startmode[halfletterimposedbc]
  \setuppapersize[quarterletter][quarterletter]
  \setuplayout[%
    backspace=15.4mm,
    topspace=11.6mm,
    height=116mm,
    footer=11.6mm,
    header=0pt, % no header
    width=76mm] % 9.24 x 12
\stopmode

\startmode[quickimpose]
  \setuppapersize[A5][A4,landscape]
  \setuparranging[2UP]
  \setuppagenumbering[alternative=doublesided]
  \setuplayout[% 14 mm was a bit too near to the spine, using the glue binding
    backspace=17.3mm,  % 140/15 + 8 =
    topspace=14mm, % 210/15 =  14
    height=182mm, % 14 x 12 + 14 of the footer
    footer=14mm, %
    header=0pt, % no header
    width=112mm] % 9.333 x 12
\stopmode

\startmode[tenpt]
  \setupbodyfont[10pt]
\stopmode

\startmode[twelvept]
  \setupbodyfont[12pt]
\stopmode

%%%%%%%%%%%%%%%%%%%%%%%%%%%%%%%%%%%%%%%%%%%%%%%%%%%%%%%%%%%%%%%%%%%%%%%%%%%%%%%%
%                            DOCUMENT BEGINS                                   %
%%%%%%%%%%%%%%%%%%%%%%%%%%%%%%%%%%%%%%%%%%%%%%%%%%%%%%%%%%%%%%%%%%%%%%%%%%%%%%%%


\mainlanguage[en]


\starttext

\starttitlepagemakeup
  \startalignment[middle,nothanging,nothyphenated,stretch]


  \switchtobodyfont[18pt] % author
  {\bf \em

Wayne Price  \par}
  \blank[2*big]
  \switchtobodyfont[24pt] % title
  {\bf

Is It a State?

\par}
  \blank[big]
  \switchtobodyfont[20pt] % subtitle
  {\bf 

Anarchists Are Really Against the State: A Response to Marxists

\par}
  \vfill
  \stopalignment
  \startalignment[middle,bottom,nothyphenated,stretch,nothanging]
  \switchtobodyfont[global]

April, 2013

  \stopalignment
\stoptitlepagemakeup



\title{Contents}

\placelist[awikipart,chapter,section,subsection]



\page[yes,right]


\startblockquote
Marxists argue that anarchists really do advocate a state, or something indistinguishable from one, but do not admit it. But what anarchists advocate is the overturning of the existing state and the creation of a new, nonstate, association of councils, assemblies, and a popular mililtia. There is no such thing as a “workers’ state.”



\stopblockquote
Most people believe that a society without a state, as advocated by anarchists, would be chaos (“anarchy”). Many think that anarchists want a society essentially as it is, but without police (which is, in fact, advocated by pro-capitalist anti-statists who miscall themselves “libertarians”). This would indeed result in chaos, until either the Mafia or the security guards hired by the rich (or both) become the new state.


A more sophisticated criticism is to say that anarchists really do advocate a state, they just do not call it by that name. As Hal Draper, a Marxist, wrote, “\unknown{}The state has been a societal necessity\unknown{}.As soon as antistatism\unknown{}even raises the question of what is to replace the state\unknown{}then it has always been obvious that the state, abolished in fancy, gets reintroduced in some other form\unknown{}.In anarchistic utopias\unknown{}the pointed ears of a very undemocratic state poke out\unknown{}” (Draper, 1990; p. 109).


Leninists argue that what anarchists argue for, is, at best, indistinguishable from the Marxist idea of a “workers’ state” (the “dictatorship of the proletariat”). To them, this would be “transitional” between the overturned capitalist state and an eventual stateless society. They refer anarchists to Marx’s Civil War in France (on the 1871 Paris Commune) and to Lenin’s State and Revolution, the most libertarian thing he wrote.


But what revolutionary, class-struggle, anarchists propose is not a state. It is a realistic alternative to the state.


\chapter{After the Revolution
}

After a revolutionary transformation from capitalism to socialist-anarchism, there will be a need to coordinate various aspects of society, particularly self-managed industries and communes. There will need to be a way to settle disputes among different sectors of society as well as between individuals. There will be a need to develop an economic plan, democratically, from the bottom up. This will be especially true during and immediately after the revolution, given the inherent conflicts and difficulties of the period.


There will be a need to oppose counter-revolutionary armed forces, sent by still-existing imperialist states or, in a civil war, by internal reactionary armies. Anti-social individuals, created by the loveless society of previous capitalism, will still need to be dealt with. Anarchists do not believe in punishment or revenge, but we do believe in protecting the people from conscienceless and emotionally wounded persons.


Anarchists have long advocated federations of workplace councils and neighborhood assemblies to carry out these tasks (detailed in price, 2007). In revolution after revolution, workers and oppressed have created self-governing councils, committees, and assemblies, in workplaces and neighborhoods. During revolutions anarchists call on the people to form such associations and bring them together to coordinate the struggle. The concept of federated councils was raised by Bakunin and Kropotkin, and especially by the Friends of Durruti Group in Spain, 1938. Implicitly this includes the right of working people to freely organize themselves to fight for their ideas among the rest of the population (a pluralistic “multi-party” democracy—which is not the same as allowing any parties to take over and rule).


There should be no more specialized bodies of armed people, such as the military or police. Instead there would be an organized, armed, population, a militia of working people and the formerly oppressed, under the direction of the council federation. These would exist until considered unnecessary. Popular armed forces (including guerilla and partisan armies) have worked quite well in the past and even now in parts of the world. Methods of public safety would be worked out mostly on a local level, in a society of freedom and plenty for all.


To this approach, Leninists and some others respond, “You anarchists are really advocating a state.” They point to the experience of the Paris Commune and the original Russian soviets (councils), and say that this is what they want too—but that they are being honest about calling it a state. They note that, in his State and Revolution, Lenin had interpreted Marx to say that this working class state would “immediately” begin to “wither away” or “die out”—immediately, from the first day. Working people would more and more become involved in directly managing society themselves, while pro-capitalist resistance would die down. A state—a specialized, centralized, and repressive institution--would be established but then the need for it would decrease and finally vanish. Is this really different from what anarchists want, they ask?


\chapter{What is the State?
}

To deal with this question, we have to define what we mean by “the state.” Frederick Engels, Marx’s closest comrade, described societies before states, such as hunter-gatherer societies or early agriculturalists. There was a certain amount of community coercion and even “wars.” But this was carried out by an armed population, or at least the armed men of the community. When society became divided into classes, rulers and ruled, this was no longer possible. The state is distinguished by “the institution of a public force which is no longer immediately identical with the people’s own organization of themselves as an armed power\unknown{}.This public force exists in every state; it consists not merely of armed men but also of material appendages, prisons and coercive institutions of all kinds\unknown{}.Officials now present themselves as organs of society standing above society\unknown{} representatives of a power which estranges them from society\unknown{}.” (1972; pp. 229—230). I think that anarchists would accept this description.


Like the anarchists of the time, Marx and Engels were very impressed by the ultra-democratic workers’ self-organization of the Paris Commune. Among other things, it replaced the standing permanent army by a popular militia, the National Guard. For such reasons, in 1875, Engels wrote a letter proposing changes in the party program: “The whole talk about the state should be dropped, especially since the Commune, which was no longer a state in the proper sense of the word\unknown{}. We would therefore propose replacing ‘state’ everywhere by ‘Gemeinwesen’ [community], a good old German word which can very well take the place of the French word ‘commune’ “ (quoted in Lenin, 1970; p. 333).


I do not intend to get into a fuller discussion of the Marxist concept of the state, the “dictatorship of the proletariat,” or related subjects (again, see my book, price 2007). My point is only that, even by Marxist description, the state is a socially-alienated, bureaucratic, military-police machine above the rest of society. By this description, it is not something which the working class can use, neither to transform society into a classless, nonoppressive, system, nor to manage society after its transformation. There can be no such thing as a “workers’ state.”


I am not quibbling about words. People may call things whatever they want; it’s a semi-free country. But we need to recognize that the council system is qualitatively different from all the states in history. All these states—even those set up by popular revolutions, such as the bourgeois-democratic French revolution or U.S. revolution—established the rule of a minority over an exploited majority. They had to be separate from the people, distinct institutions, no matter how democratic in form. But the federated councils of the workers’ commune, backed by the armed people, is the self-organized people itself, not a distinct institution. It may carry out certain tasks which states have done in the past, but it is not useful to describe it as a state. When everyone governs, there is no “government.”


\chapter{Leninism and the State
}

Lenin argued that it was necessary to overturn the existing, capitalist, state, and to build a new state, a workers’ state—temporarily, transitionally--which would eventually “wither away.” What the revolutionaries will be doing, what they will be working at, is building the new state. The “withering away” of the state will be left to take care of itself. With such an approach, it should not be surprising that what the Leninists produced is\unknown{}.a state.


“The very revolutionaries who claim that they are against the state, and for eliminating the state\unknown{}see as their central task after a revolution to build up a state that is more solid, more centralized and more all-embracing than the old one. \unknown{}The point is not that the workers and other oppressed people should not build up a strong set of organizations during and after a revolution to manage the economy and society, defend their gains and suppress the exploiters, etc. But they also need to take steps o prevent a new state from arising and oppressing them. That is, they need to figure out how they are going to build a stateless society” (Taber, 1988; pp. 56 \& 58). In other words, the centralized and repressive aspects of political organization should actively “be withered” by the working population.


\chapter{Trotskyism and the State
}

Trotskyists often say to anarchists that they want what we want, an association of councils tied to a workers’ militia. This is, they say, what they mean by a “workers’ state.” So far, so good.


But they also use “workers’ state” to described the Russian regime of Lenin and Trotsky up to about 1923. This was a one-party police state dictatorship, and not at all a radically democratic council system. At the time of the 1917 revolution there had been democratic soviets (councils), factory committees, independent unions, a range of socialist parties and anarchist groups (parties and groups which supported the revolution and fought on the side of the Bolsheviks during the Civil War), and dissenting caucuses inside the Bolshevik party. Between 1918 and 1921, this lively working class democracy was destroyed. I am not arguing why this happened (Trotskyists claim it was entirely due to objective conditions; anarchists claim that Lenin and Trotsky’s authoritarian politics had much to do with it). But it did happen. So the Trotskyists are left calling a state in which the workers had no power, a “workers’ state.” Given the chance, how do we know that they would not create the same kind of “workers’ state” again (if the “objective conditions” existed)?


It gets worse. One wing of the Trotskyist movement is called “orthodox Trotskyism” or “Soviet defensists.” They follow Trotsky’s stated view that the Soviet Union under Stalin was a totalitarian mass-murdering regime, but was also a “workers’ state” (a “degenerated workers’ state”). This was because it expanded nationalized property and for no other reason. Similarly, the regimes of Eastern Europe, China, and Cuba were also “workers’ states” without any worker control (“deformed workers’ states,” except Cuba which most regarded as a pretty good “workers’ state”).


There is a more democratic wing of Trotskyism, which rejected Trotsky’s view of Stalin’s USSR. They believe (with most anarchists) that the bureaucracy became a new ruling class and the economy became “state capitalist” or some new type of exploitative system.


But they still believe that Lenin and Trotsky’s regime was a “workers’ state.” And they believe that Stalin’s rule remained a “workers’ state” up to some turning point (1929, when the industrialization drive began, or the late 1930s, in the time of the great purge trials when the party was remade).


My point is that, for Trotskyists, the concept of a “workers’ state” is not only a label for a council system, slightly different from that of the anarchists. It is a concept they use to cover for drastically undemocratic institutions.


Other Leninists exist, such as Communists in the tradition of the old pro-Moscow parties, Maoists, and some others. They rarely refer to Marx’s goal of a stateless society. They support the monstrous one-party tyrannies of Stalin or Mao. But they often follow a reformist approach, that is, try to change society through the existing state rather than by seeking to overturn it and create something new. The Communist Parties are notorious for this approach. But even Maoists may follow it, as is exemplified by the Maoists in Nepal who are trying to take over a bourgeois state through parliamentary maneuvering. Even the Trotskyists have, in practice, abandoned their Leninist position of needing to overthrow the bourgeois state. This is seen by their support for Hugo Chavez’ effort to establish “socialism” through the Venezuelan capitalist state or their support for pro-capitalist politicians running for election, such as Ralph Nadar.


Another view was expressed by Paul Mattick, Sr., a council communist (libertarian Marxist). (I am not discussing who has the “correct” interpretation of Marx on the state. Nor am I discussing the issue raised earlier by Draper about authoritarian tendencies within anarchism). For “Marx and Engels\unknown{}the victorious working class would neither institute a new state nor seize control of the existing state\unknown{}. It is not through the state that socialism can be realized, as this would exclude the self-determination of the working class, which is the essence of socialism” (1983; pp160—161).


Revolutionary anarchists and other revolutionary libertarian socialists aim for the workers and all oppressed to break up the existing states and replace them with radically democratic, self-managed, societies.


{\bf wayne price}


References



\startitemize[1]\relax
\item[] Draper, Hal (1990). Karl Marx’s Theory of Revolution; Vol. IV: Critique of Other Socialisms. NY: Monthly Review.




 \item[] Engels, Frederick (1972). The Origin of the Family, Private Property, and the State. NY: International Publishers.




 \item[] Lenin, V.I. (1970). Selected Works; vol. 2. Moscow: Progress Publishers.




 \item[] Mattick, Paul, Sr. (1983). Marxism: Last Refuge of the Bourgeoisie? Armonk NY: M.E. Sharpe.




 \item[] price, wayne (2007). The Abolition of the State: Anarchist and Marxist Perspectives. Bloomington IN: AuthorHouse.




 \item[] Taber, Ron (1988). A Look at Leninism. NY: Aspect Foundation.




 
\stopitemize







\page[yes]

%%%% backcover

\startmode[a4imposed,a4imposedbc,letterimposed,letterimposedbc,a5imposed,%
  a5imposedbc,halfletterimposed,halfletterimposedbc,quickimpose]
\alibraryflushpages
\stopmode

\page[blank]

\startalignment[middle]
{\tfa The Anarchist Library
\blank[small]
Anti-Copyright}
\blank[small]
\currentdate
\stopalignment

\blank[big]
\framed[frame=off,location=middle,width=\textwidth]
       {\externalfigure[logo][width=0.25\textwidth]}



\vfill
\setupindenting[no]
\setsmallbodyfont

\startalignment[middle,nothyphenated,nothanging,stretch]

\blank[line]
% \framed[frame=off,location=middle,width=\textwidth]
%       {\externalfigure[logo][width=0.25\textwidth]}


Wayne Price



Is It a State?



Anarchists Are Really Against the State: A Response to Marxists




April, 2013


\stopalignment
\blank[line]

\startalignment[hyphenated,middle]




Retrieved on July 2, 2014 from http://anarkismo.net/article/25430?search\_text=wayne\%20price\&print\_page=true


\stopalignment

\stoptext


