% -*- mode: tex -*-
%%%%%%%%%%%%%%%%%%%%%%%%%%%%%%%%%%%%%%%%%%%%%%%%%%%%%%%%%%%%%%%%%%%%%%%%%%%%%%%%
%                                STANDARD                                      %
%%%%%%%%%%%%%%%%%%%%%%%%%%%%%%%%%%%%%%%%%%%%%%%%%%%%%%%%%%%%%%%%%%%%%%%%%%%%%%%%
\enabletrackers[fonts.missing]
\definefontfeature[default][default]
                  [protrusion=quality,
                    expansion=quality,
                    script=latn]
\setupalign[hz,hanging]
\setuptolerance[tolerant]
\setbreakpoints[compound]
\setupindenting[yes,1em]
\setupfootnotes[way=bychapter,align={hz,hanging}]
\setupbodyfont[modern] % this is a stinky workaround to load lmodern
\setupbodyfont[libertine,11pt]

\setuppagenumbering[alternative=singlesided,location={footer,middle}]
\setupcaptions[width=fit,align={hz,hanging},number=no]

\startmode[a4imposed,a4imposedbc,letterimposed,letterimposedbc,a5imposed,%
  a5imposedbc,halfletterimposed,halfletterimposedbc]
  \setuppagenumbering[alternative=doublesided]
\stopmode

\setupbodyfontenvironment[default][em=italic]


\setupheads[%
  sectionnumber=no,number=no,
  align=flushleft,
  align={flushleft,nothyphenated,verytolerant,stretch},
  indentnext=yes,
  tolerance=verytolerant]

\definehead[awikipart][chapter]

\setuphead[awikipart]
          [%
            number=no,
            footer=empty,
            style=\bfd,
            before={\blank[force,2*big]},
            align={middle,nothyphenated,verytolerant,stretch},
            after={\page[yes]}
          ]

% h3
\setuphead[chapter]
          [style=\bfc]

\setuphead[title]
          [style=\bfc]


% h4
\setuphead[section]
          [style=\bfb]

% h5
\setuphead[subsection]
          [style=\bfa]

% h6
\setuphead[subsubsection]
          [style=bold]


\setuplist[awikipart]
          [alternative=b,
            interaction=all,
            width=0mm,
            distance=0mm,
            before={\blank[medium]},
            after={\blank[small]},
            style=\bfa,
            criterium=all]
\setuplist[chapter]
          [alternative=c,
            interaction=all,
            width=1mm,
            before={\blank[small]},
            style=bold,
            criterium=all]
\setuplist[section]
          [alternative=c,
            interaction=all,
            width=1mm,
            style=\tf,
            criterium=all]
\setuplist[subsection]
          [alternative=c,
            interaction=all,
            width=8mm,
            distance=0mm,
            style=\tf,
            criterium=all]
\setuplist[subsubsection]
          [alternative=c,
            interaction=all,
            width=15mm,
            style=\tf,
            criterium=all]


% center

\definestartstop
  [awikicenter]
  [before={\blank[line]\startalignment[middle]},
   after={\stopalignment\blank[line]}]

% right

\definestartstop
  [awikiright]
  [before={\blank[line]\startalignment[flushright]},
   after={\stopalignment\blank[line]}]


% blockquote

\definestartstop
  [blockquote]
  [before={\blank[big]
    \setupnarrower[middle=1em]
    \startnarrower
    \setupindenting[no]
    \setupwhitespace[medium]},
  after={\stopnarrower
    \blank[big]}]

% verse

\definestartstop
  [awikiverse]
  [before={\blank[big]
      \setupnarrower[middle=2em]
      \startnarrower
      \startlines},
    after={\stoplines
      \stopnarrower
      \blank[big]}]

\definestartstop
  [awikibiblio]
  [before={%
      \blank[big]
      \setupnarrower[left=1em]
      \startnarrower[left]
        \setupindenting[yes,-1em,first]},
    after={\stopnarrower
      \blank[big]}]
                
% same as above, but with no spacing around
\definestartstop
  [awikiplay]
  [before={%
      \setupnarrower[left=1em]
      \startnarrower[left]
        \setupindenting[yes,-1em,first]},
    after={\stopnarrower}]



% interaction
% we start the interaction only if it's not an imposed format.
\startnotmode[a4imposed,a4imposedbc,letterimposed,letterimposedbc,a5imposed,%
  a5imposedbc,halfletterimposed,halfletterimposedbc]
  \setupinteraction[state=start,color=black,contrastcolor=black,style=bold]
  \placebookmarks[awikipart,chapter,section,subsection,subsubsection][force=yes]
  \setupinteractionscreen[option=bookmark]
\stopnotmode



\setupexternalfigures[%
  maxwidth=\textwidth,
  maxheight=\textheight,
  factor=fit]

\setupitemgroup[itemize][each][packed][indenting=no]

\definemakeup[titlepage][pagestate=start,doublesided=no]

%%%%%%%%%%%%%%%%%%%%%%%%%%%%%%%%%%%%%%%%%%%%%%%%%%%%%%%%%%%%%%%%%%%%%%%%%%%%%%%%
%                                IMPOSER                                       %
%%%%%%%%%%%%%%%%%%%%%%%%%%%%%%%%%%%%%%%%%%%%%%%%%%%%%%%%%%%%%%%%%%%%%%%%%%%%%%%%

\startusercode

function optimize_signature(pages,min,max)
   local minsignature = min or 40
   local maxsignature = max or 80
   local originalpages = pages

   -- here we want to be sure that the max and min are actual *4
   if (minsignature%4) ~= 0 then
      global.texio.write_nl('term and log', "The minsig you provided is not a multiple of 4, rounding up")
      minsignature = minsignature + (4 - (minsignature % 4))
   end
   if (maxsignature%4) ~= 0 then
      global.texio.write_nl('term and log', "The maxsig you provided is not a multiple of 4, rounding up")
      maxsignature = maxsignature + (4 - (maxsignature % 4))
   end
   global.assert((minsignature % 4) == 0, "I suppose something is wrong, not a n*4")
   global.assert((maxsignature % 4) == 0, "I suppose something is wrong, not a n*4")

   --set needed pages to and and signature to 0
   local neededpages, signature = 0,0

   -- this means that we have to work with n*4, if not, add them to
   -- needed pages 
   local modulo = pages % 4
   if modulo==0 then
      signature=pages
   else
      neededpages = 4 - modulo
   end

   -- add the needed pages to pages
   pages = pages + neededpages
   
   if ((minsignature == 0) or (maxsignature == 0)) then 
      signature = pages -- the whole text
   else
      -- give a try with the signature
      signature = find_signature(pages, maxsignature)
      
      -- if the pages, are more than the max signature, find the right one
      if pages>maxsignature then
	 while signature<minsignature do
	    pages = pages + 4
	    neededpages = 4 + neededpages
	    signature = find_signature(pages, maxsignature)
	    --         global.texio.write_nl('term and log', "Trying signature of " .. signature)
	 end
      end
      global.texio.write_nl('term and log', "Parameters:: maxsignature=" .. maxsignature ..
		   " minsignature=" .. minsignature)

   end
   global.texio.write_nl('term and log', "ImposerMessage:: Original pages: " .. originalpages .. "; " .. 
	 "Signature is " .. signature .. ", " ..
	 neededpages .. " pages are needed, " .. 
	 pages ..  " of output")
   -- let's do it
   tex.print("\\dorecurse{" .. neededpages .. "}{\\page[empty]}")

end

function find_signature(number, maxsignature)
   global.assert(number>3, "I can't find the signature for" .. number .. "pages")
   global.assert((number % 4) == 0, "I suppose something is wrong, not a n*4")
   local i = maxsignature
   while i>0 do
      -- global.texio.write_nl('term and log', "Trying " .. i  .. "for max of " .. maxsignature)
      if (number % i) == 0 then
	 return i
      end
      i = i - 4
   end
end

\stopusercode

\define[1]\fillthesignature{
  \usercode{optimize_signature(#1, 40, 80)}}


\define\alibraryflushpages{
  \page[yes] % reset the page
  \fillthesignature{\the\realpageno}
}


% various papers 
\definepapersize[halfletter][width=5.5in,height=8.5in]
\definepapersize[halfafour][width=148.5mm,height=210mm]
\definepapersize[quarterletter][width=4.25in,height=5.5in]
\definepapersize[halfafive][width=105mm,height=148mm]
\definepapersize[generic][width=210mm,height=279.4mm]

%% this is the default ``paper'' which should work with both letter and a4

\setuppapersize[generic][generic]
\setuplayout[%
  backspace=42mm,
  topspace=31mm,% 176 / 15
  height=195mm,%130mm,
  footer=9mm, %
  header=0pt, % no header
  width=126mm] % 10.5 x 11

\startmode[libertine]
  \usetypescript[libertine]
  \setupbodyfont[libertine,11pt]
\stopmode

\startmode[pagella]
  \setupbodyfont[pagella,11pt]
\stopmode

\startmode[antykwa]
  \setupbodyfont[antykwa-poltawskiego,11pt]
\stopmode

\startmode[iwona]
  \setupbodyfont[iwona-medium,11pt]
\stopmode

\startmode[helvetica]
  \setupbodyfont[heros,11pt]
\stopmode

\startmode[century]
  \setupbodyfont[schola,11pt]
\stopmode

\startmode[modern]
  \setupbodyfont[modern,11pt]
\stopmode

\startmode[charis]
  \setupbodyfont[charis,11pt]
\stopmode        

\startmode[mini]
  \setuppapersize[S33][S33] % 176 × 176 mm
  \setuplayout[%
    backspace=20pt,
    topspace=15pt,% 176 / 15
    height=280pt,%130mm,
    footer=20pt, %
    header=0pt, % no header
    width=260pt] % 10.5 x 11
\stopmode

% for the plain A4 and letter, we use the classic LaTeX dimensions
% from the article class
\startmode[a4]
  \setuppapersize[A4][A4]
  \setuplayout[%
    backspace=42mm,
    topspace=45mm,
    height=218mm,
    footer=10mm,
    header=0pt, % no header
    width=126mm]
\stopmode

\startmode[letter]
  \setuppapersize[letter][letter]
  \setuplayout[%
    backspace=44mm,
    topspace=46mm,
    height=199mm,
    footer=10mm,
    header=0pt, % no header
    width=126mm]
\stopmode


% A4 imposed (A5), with no bc

\startmode[a4imposed]
% DIV=15 148 × 210: these are meant not to have binding correction,
  % but just to play safe, let's say 1mm => 147x210
  \setuppapersize[halfafour][halfafour]
  \setuplayout[%
    backspace=10.8mm, % 146/15 = 9.8 + 1
    topspace=14mm, % 210/15 =  14
    height=182mm, % 14 x 12 + 14 of the footer
    footer=14mm, %
    header=0pt, % no header
    width=117.6mm] % 9.8 x 12
\stopmode

% A4 imposed (A5), with bc
\startmode[a4imposedbc]
  \setuppapersize[halfafour][halfafour]
  \setuplayout[% 14 mm was a bit too near to the spine, using the glue binding
    backspace=17.3mm,  % 140/15 + 8 =
    topspace=14mm, % 210/15 =  14
    height=182mm, % 14 x 12 + 14 of the footer
    footer=14mm, %
    header=0pt, % no header
    width=112mm] % 9.333 x 12
\stopmode


\startmode[letterimposedbc] % 139.7mm x 215.9 mm
  \setuppapersize[halfletter][halfletter]
  % DIV=15 8mm binding corr, => 132 x 216
  \setuplayout[%
    backspace=16.8mm, % 8.8 + 8
    topspace=14.4mm, % 216/15 =  14.4
    height=187.2mm, % 15.4 x 11 + 15 of the footer
    footer=14.4mm, %
    header=0pt, % no header
    width=105.6mm] % 8.8 x 12
\stopmode

\startmode[letterimposed] % 139.7mm x 215.9 mm
  \setuppapersize[halfletter][halfletter]
  % DIV=15, 1mm binding correction. => 138.7x215.9
  \setuplayout[%
    backspace=10.3mm, % 9.24 + 1
    topspace=14.4mm, % 216/15 =  14.4
    height=187.2mm, % 15.4 x 11 + 15 of the footer
    footer=14.4mm, %
    header=0pt, % no header
    width=111mm] % 9.24 x 12
\stopmode

%%% new formats for mini books
%%% \definepapersize[halfafive][width=105mm,height=148mm]

\startmode[a5imposed]
% DIV=12 105x148 : these are meant not to have binding correction,
  % but just to play safe, let's say 1mm => 104x148
  \setuppapersize[halfafive][halfafive]
  \setuplayout[%
    backspace=9.6mm,
    topspace=12.3mm,
    height=123.5mm, % 14 x 12 + 14 of the footer
    footer=12.3mm, %
    header=0pt, % no header
    width=78.8mm] % 9.8 x 12
\stopmode

% A5 imposed (A6), with bc
\startmode[a5imposedbc]
% DIV=12 105x148 : with binding correction,
  % let's say 8mm => 96x148
  \setuppapersize[halfafive][halfafive]
  \setuplayout[%
    backspace=16mm,
    topspace=12.3mm,
    height=123.5mm, % 14 x 12 + 14 of the footer
    footer=12.3mm, %
    header=0pt, % no header
    width=72mm] % 9.8 x 12
\stopmode

%%% \definepapersize[quarterletter][width=4.25in,height=5.5in]

% DIV=12 width=4.25in (108mm),height=5.5in (140mm) 
\startmode[halfletterimposed] % 107x140
  \setuppapersize[quarterletter][quarterletter]
  \setuplayout[%
    backspace=10mm,
    topspace=11.6mm,
    height=116mm,
    footer=11.6mm,
    header=0pt, % no header
    width=80mm] % 9.24 x 12
\stopmode

\startmode[halfletterimposedbc]
  \setuppapersize[quarterletter][quarterletter]
  \setuplayout[%
    backspace=15.4mm,
    topspace=11.6mm,
    height=116mm,
    footer=11.6mm,
    header=0pt, % no header
    width=76mm] % 9.24 x 12
\stopmode

\startmode[quickimpose]
  \setuppapersize[A5][A4,landscape]
  \setuparranging[2UP]
  \setuppagenumbering[alternative=doublesided]
  \setuplayout[% 14 mm was a bit too near to the spine, using the glue binding
    backspace=17.3mm,  % 140/15 + 8 =
    topspace=14mm, % 210/15 =  14
    height=182mm, % 14 x 12 + 14 of the footer
    footer=14mm, %
    header=0pt, % no header
    width=112mm] % 9.333 x 12
\stopmode

\startmode[tenpt]
  \setupbodyfont[10pt]
\stopmode

\startmode[twelvept]
  \setupbodyfont[12pt]
\stopmode

%%%%%%%%%%%%%%%%%%%%%%%%%%%%%%%%%%%%%%%%%%%%%%%%%%%%%%%%%%%%%%%%%%%%%%%%%%%%%%%%
%                            DOCUMENT BEGINS                                   %
%%%%%%%%%%%%%%%%%%%%%%%%%%%%%%%%%%%%%%%%%%%%%%%%%%%%%%%%%%%%%%%%%%%%%%%%%%%%%%%%


\mainlanguage[en]


\starttext

\starttitlepagemakeup
  \startalignment[middle,nothanging,nothyphenated,stretch]


  \switchtobodyfont[18pt] % author
  {\bf \em

Fredy Perlman  \par}
  \blank[2*big]
  \switchtobodyfont[24pt] % title
  {\bf

The Incoherence of the Intellectual

\par}
  \blank[big]
  \switchtobodyfont[20pt] % subtitle
  {\bf 

C. Wright Mills’ Struggle to Unite Knowledge and Action

\par}
  \vfill
  \stopalignment
  \startalignment[middle,bottom,nothyphenated,stretch,nothanging]
  \switchtobodyfont[global]

April 1969

  \stopalignment
\stoptitlepagemakeup



\title{Contents}

\placelist[awikipart,chapter,section,subsection]



\page[yes,right]

\chapter{INTRODUCTION
}

\section{Man as History‑Maker
}


\startblockquote
The philosophers have only {\em interpreted} the world in various ways; the point, however, is to {\em change} it. — Karl Marx



\stopblockquote
For C. Wright Mills, {\em the most important issue of political reflection—and of political action—in our time} is {\em the problem of the historical agency of change, of the social and institutional means of structural change.}\footnote{C. Wright Mills, “Letter to the New Left,” {\em New Left Review}, No. 5 (September‑October, 1960), pp, 18‑23; republished in {\em Power, Politics and People: The Collected Essays of C. Wright Mills} (edited by Irving L. Horowitz), New York: Oxford University Press, 1963, p. 254.} The problem of social change, of revolutionary practice, occupies a central place in Mills’ writings, which stretch over a period of two decades. For Mills, this is not a speculative problem; it is not a subject for contemplation. It is an intensely practical and personal problem. It raises questions about the relation of the individual to history, about the relevance of intellectual activity to the making of history, about the unity of thought and action, theory and practice. It raises questions about the difference or lack of difference an individual’s life makes, and questions about man’s choice of himself as practical or meditative, active or passive, whole or fragmented. It raises questions about the professor’s relations to his job and to his contemporaries, and questions about the insurgent’s relations to those to whom he tries to communicate a revolutionary strategy. Mills did not answer these questions; he posed them, and for posing them he was left standing alone in a United States which contained no revolutionaries during a period he called {\em the mindless years.} Alone, he could not always defend his positions, and was frequently pushed back. He died a short time before he would have been joined by a new American left {\em prepared to act boldly and win over the less bold by their success.} \footnote{Mills, {\em The New Men of Power: America’s Labor Leaders}, New York: Harcourt Brace \& Co., 1948, p. 274.} He did not leave the new insurgents clear answers; he left them lucidly posed questions. And he left the world revolutionary movement the model of a rebel who continued to struggle in complete isolation, and the task of finding answers to the questions he posed.


\chapter{One: The Search for Radical Strategy 1939‑1948
}

\section{Political Commitment and a Definition of Strategy
}

Mills committed himself to political struggle publicly in 1942, in a review of Franz Neumann’s analysis of Nazi Germany.\footnote{Mills, “Locating the Enemy: The Nazi Behemoth Dissected.” (Review of Franz Neumann’s {\em Behemoth: The Structure and Practice of National Socialism}.) Vol{\em .} 4, {\em Partisan Review} (September‑October, 1942), pp. 432‑437; in {\em Power, Politics and People,} pp. 170‑178.} Mills did not read Neumann’s dissection of the Nazi {\em Behemoth} as a description of a distant enemy: {\em The analysis of Behemoth casts light upon capitalism in democracies\unknown{} If you read his book thoroughly, you see the harsh outlines of possible futures close around you. With leftwing thought confused and split and dribbling trivialities, he locates} {\em the enemy with a 500 watt glare. And Nazi is only one of his names.}\footnote{{\em Ibid.}, p{\em .} 177.} The enemy is not located as a spectacle, as an object for passive contemplation and academic dissection. Locating the enemy is the first step toward locating oneself in the face of the enemy, it is the first step toward political struggle: Neumann’s {\em book will move all of us into deeper levels of analysis and strategy. It had better. Behemoth is everywhere united.} \footnote{{\em Ibid.}, p. 178.}


Mills’ choice of the words {\em analysis} and {\em strategy} is significant: it is an early statement of a problem that becomes central in later works: the link between thought and action, between consciousness and existence, between theory and practice. This choice of words is also significant as a political application of words he had used and defined earlier in purely academic contexts.


In an article published two years before the review of {\em Behemoth,} Mills had defined {\em strategies of action} as motives which appeal to others.\footnote{“Situated Actions and Vocabularies of Motive,” {\em American Sociological Review}, Vol. V, No. 6 (December, 1940), in {\em Power, Politics and People,} p. 443.} Motives are defined as {\em named consequences of action.}\footnote{{\em Ibid}., p. 441.} (In later works, Mills called such motives {\em ideals} or {\em goals}.) The motives do not originate in the individual’s biology; they are provided by his culture, through his interactions\footnote{“Language, Logic and Culture,” {\em American Sociological Review}, Vol. IV, No. 5 (October, 1939), in {\em Power, Politics and People}, pp. 423‑438.} with others. (This is the main point of Mills’ first published article, an article which illustrates that at twenty‑three Mills was already master of the dull and bureaucratic writing style of professional academics, a style which, he later observed, {\em has little or nothing to do with the complexity of subject matter, and nothing at all with profundity of thought. It has to do almost entirely with certain confusions of the academic writer about his own status.}\footnote{{\em The Sociological Imagination}, New York: Oxford University Press, 1959, p. 218.} As soon as he illustrated how easily one can master the academic style, Mills abandoned it and wrote his works in a clear and straightforward language.)


Since a {\em strategy} consists of the named consequences of an action undertaken with others in a particular situation, the situation has to be defined in such a way that the consequences of action can be named. This is the task of {\em analysis.} If the strategy {\em had better} cope with the enemy described by Neumann, then the situation to be analyzed is {\em capitalism in democracy.} It has to be shown that certain kinds of action can change the social situation; if this cannot be shown, then the consequences of action cannot be named, and there can be neither motives nor strategies.


Elements for such analysis of the social situation (Mills later called such analysis a {\em definition of reality)} can be found in Mills’ doctoral dissertation.\footnote{A Sociological Account of Pragmatism, 1942. Published as {\em Sociology and Pragmatism: The Higher Learning in America,} New York: Oxford University Press, 1966.} Here Mills follows John Dewey’s rejection of an unchanging “human nature,” and of a psychology of “instincts,” as explanations of the continuity of social institutions. The continuity is explained in terms of socially acquired “customs” and “habits.” Furthermore, “habit means will,” so that the repeated daily activities of people are voluntary acts.\footnote{{\em Sociology and Pragmatism}, pp. 452‑453.} This means that “institutions” can be changed through human activity, that collective actions can have social consequences, and therefore that “strategies of action” can be formulated. Mills points out that Dewey applies his concept of action only to independent craftsmen and farmers: {\em His concept of action is} {\bf of an individual}; it {\em is} {\em not political action.}\footnote{{\em Ibid}., pp. 392‑393.} However, political action, namely collective practice based on named consequences (or on {\em theory}) is also possible, since {\em It is obvious that Marxism as a doctrine and movement has linked practice and theory.}\footnote{{\em Ibid}., p. 428.}


Thus a political strategy is based on a definition of a social situation which can be changed by collective activity, and it consists of the named consequences of action which appeal to others. The others, in Dewey’s language, are the “Public,” a community of self-directed individuals.\footnote{{\em Ibid.,} p. 437.} The activity which links the individual to a public is communication, and for Dewey communication takes the specific form of education, since {\em this psychology’s stress on the modifiability of human nature opens wide the possibility of improvement by means of the educational enterprise; it is slanted specifically to} {\bf educational} {\em endeavors.}\footnote{{\em Ibid.,} p. 455.} Dewey’s strategy was social reform through educational reform; one of the named consequences of this activity was to be “that of building up an intelligent and capable civil‑service.”\footnote{John Dewey, {\em Freedom and Culture,} quoted in {\em Ibid}., p. 434.} In Dewey’s view, the only alternative to social reformism “seems to be a concentration of power that points toward ultimate dictatorship\unknown{} “\footnote{{\em Ibid.}}


Although Mills rejected {\em Dewey’s style of liberalism}\footnote{{\em Ibid.,} p. 461.} and educational reformism, he seems, at least partially, to have shared Dewey’s conception of “publics” composed of self‑directed individuals, since Dewey’s conception of “The Eclipse of the Public”\footnote{{\em Ibid.,} p. 436.} reappears as Mills’ own conception in works he is to write more than a decade later; even Dewey’s reformist program of installing a civil service reappears in works where Mills exposes and rejects all shades of liberalism. The conception of others as potentially self‑directed individuals implies a non­-manipulative view of communication and seems to exclude the cynical and manipulative conception which crept into Mills’ thought from other influences.


However, the specific public, the community to whom Mills is to communicate a political strategy, the historical agency which can potentially transform the social situation, is not yet mentioned, and the strategy itself has not yet been formulated.


\section{Elements for a Retreat from Political Commitment
}

According to Professor Irving Louis Horowitz, “Mills benefited from his contact with European trained scholars at the University of Wisconsin—especially Hans H. Gerth.”\footnote{Irving Louis Horowitz, “Introduction: The Intellectual Genesis of C, Wright Mills,” {\em Ibid}., p. 23.} In 1942, the same year he published his review of Neumann’s {\em Behemoth,} Mills published another book review, written with Gerth.\footnote{Mills, “A Marx for the Managers” (with H. H. Gerth, {\em Ethics:} An {\em International Journal of Legal, Political and Social Thought}, Vol. 52, No. 2 (January, 1942), in {\em Power, Politics and People}, pp. 53‑71.} Some of the questions raised by Mills in his earlier writings are treated very differently in this article.


Mills benefited from his contact with Gerth to sharpen, yet also to blunt, his definitions of the social context of human activity. In the place of Dewey’s ideas about “custom” and “habit” as voluntary activities which account for the continuity of institutions, the Mills‑Gerth article puts “historical drift.”\footnote{{\em Ibid}., p. 53.} The article develops Max Weber’s thesis that the “historical drift” of industrial societies is bureaucratization.


{\em It is this} form {\em of organization which is taken to be the substance of history.}\footnote{{\em Ibid}.} The two authors mention the fact that this drift is not a force of nature which imposes itself over human beings. {\em It is the men who nurse the big machines, the industrial population, who implement that which makes history.} This distinction between those {\em who implement} history and {\em that which makes history is} not a grammatical ambiguity: the following sentence says, {\em For Weber, impersonal rationality stands as a polar opposite to personal charisma, the extra­ ordinary gift of leaders.}\footnote{{\em Ibid.,} pp. 53‑54.}


The bureaucratization and routinization of life takes place within three dominant {\em structures of power, military, industrial and governmental} and it is the leaders of these structures who make the {\em ultimate decisions}.\footnote{{\em Ibid.,} p. 67.} The view of history which emerges is one where active leaders decide and passive followers implement. It is not pointed out that if the followers did not repeatedly decide to continue following {\em (habit means will),} the leaders would not have the power to make any ultimate decisions.


With this definition of social reality, historical change is still possible; furthermore, the historic agencies who transform social reality, the {\em revolutionary masses,} can be defined. However, these “masses” are not active subjects; they are not the self‑determined individuals mentioned earlier. The masses are objects, they are followers, they “implement” history, {\em it is they who make revolutionary leaders successful,} and it is the leaders who make ultimate decisions. {\em In modern history always behind the elites and parties there are revolutionary masses}.\footnote{{\em Ibid}., p. 71.}


This conception of elites and masses drives a wedge into the heart of the community mentioned by Mills earlier. The elite and the mass are two separate communities, only one of which consists of self‑determined individuals. The dominant activities of these separate communities are different: one decides and the other implements. The separation between these two sets of people and activities is similar to the separation between the “academic community” and the “world outside.”


In this context, strategy cannot take the form of motives of action which are shared by people in a common situation, since the {\em elite} and the {\em mass} are not in the same situation. Furthermore, the link between the leader and the masses does not consist of communication within a community of individuals, but of that kind of manipulation of the masses that makes the leader successful.


This conception of historical change in fact excludes the possibility of significant change. If bureaucratization is the {\em historical drift} and the {\em substance of history,} if {\em Behemoth is everywhere united,} and if revolutionary strategy is to lead to a struggle against the enemy located by Neumann {\em with a 500 watt glare,} then the Mills‑Gerth article does not move {\em into deeper levels of analysis and strategy.} In fact, it is hard to see just how “Mills benefited from his contact with European trained scholars at the University of Wisconsin—especially Hans H. Gerth.” The historical drift cannot be stopped; the {\em masses} who are fragments of bureaucratic {\em structures of power} cannot destroy these structures to become self‑determined human beings. The {\em masses} can, at best, implement a {\em revolution,} which in this article means that they can be manipulated into pushing new leaders and {\em elites,} like the Nazi Party, into the dominant bureaucracies; the most that radical strategy can accomplish in the face of Behemoth is: {\em radical shifts in the distribution of power and in the composition of personnel}.\footnote{{\em Ibid.}}


\section{The Powerless Intellectual
}

Two years after his excursion with Gerth, in an article titled “The Powerless People: The Role of the Intellectual in Society,”\footnote{‘The Powerless People: The Role of the Intellectual in Society,” {\em Politics}, Vol. l; No. 3 (April, 1944), in {\em Power, Politics and People}, where it is published under the title “The Social Role of the Intellectual,” pp. 292‑304.} Mills tried to find his way out of the maze where the excursion had left him in order to move into {\em deeper levels of analysis and strategy.}


In “The Powerless People,” Mills tries to break out of the world of leaders and followers, elites and masses, since his own existence is denied by this type of analysis. He tries to locate himself, and on this basis to define his social situation.


{\em If he is to think politically in a realistic way, the intellectual must constantly know his own social position.}\footnote{{\em Ibid}., p. 299.} Mills, the intellectual, is clearly not one of the {\em elite,} since he is powerless. {\em We continue to know more and more about modern society, but we find the centers of political initiative less and less accessible. This generates a personal malady that is particularly acute in the intellectual who has labored under the illusion that his thinking makes a difference. In the world of today the more his knowledge of affairs grows, the less effective the impact of his thinking seems to become. Since he grows more frustrated as his knowledge increases, it seems that knowledge leads to powerlessness. He feels helpless in the fundamental sense that he cannot control what he is able to foresee.}\footnote{{\em Ibid}., p. 293.} This powerlessness and helplessness are not attributes of the intellectual as a member of a manipulated and dependent mass; they are due to a {\em failure of nerve}\footnote{{\em Ibid.}} (since {\em habit means will.}) Neither a leader nor a follower, the intellectual is also not an academic spectator who observes human history from outside. {\em The “detached spectator” does not know his helplessness because he never tries to surmount it. But the political man is always aware that while events are not in his hands he must bear their consequences.}\footnote{{\em Ibid}., p. 294.}


The intellectual has been reduced to an instrument for manipulation and to a manipulated object. He wants his thought to make a difference, but he is in fact politically irrelevant. His power to make a difference, to have consequences, is separate from him and strange to him. This separation of the individual from his own power, this gap between a person’s decisions and their social consequences, this incoherence or lack of unity between thought and action, characterize not only the situation of the intellectual, but also that of the wage‑worker, the salaried clerk, the student. However, Mills does not analyze the situation which is common to all these people, a situation in which they alienate their power to shape their environment, to {\em make a difference} in the world. Mills limits his analysis to the intellectual, and does not develop a conception of alienation; for Mills, alienation means disaffection; it is not a fact about people’s situation, but a feeling about their situation (people are {\em alienated} if they {\em don’t believe in the work they’re doing}\footnote{{\em Ibid.}, p. 300.}.


Once he is conscious of his own incoherence, of the separation between his thought and his activity, the intellectual struggles to break out of this powerlessness, to get to its roots, {\em to unmask and to smash the stereotypes of vision and intellect with which modern communications swamp us.}\footnote{{\em Ibid}., p. 299.} To get to the roots of his situation, the intellectual must not only smash the stereotypes which veil the situation, but also the spectacles of the “future” which divert his attention from the real situation. {\em The more the antagonisms of the actual present must be suffered, the more the future is drawn upon as a source of pseudo‑unity and synthetic morale\unknown{} Most of these commodities are not plans with any real chance to be realized. They are baits for various strata, and sometimes for quite vested groups, to support contemporary irresponsibilities\unknown{} Discussions of the future which accept the present basis for it serve either as diversions from immediate realities or as tacit i ntellectual sanctions of future disasters}.\footnote{{\em Ibid}., pp. 302‑303.}


Among the veil makers and obfuscators, Mills singles out professors for his sharpest critiques in “The Powerless People” and also in a critique of textbooks published a year earlier. Since professors and textbooks are important sources of the stereotypes which clutter people’s minds, some of the explanations of the intellectual’s failure of nerve may fruitfully be sought there. What Mills found in a sample of textbooks on social psychology, all of it perpetrated as some kind of science, included an {\em emphasis upon the ‘processual’ and ‘organic’ character of society \unknown{} From the standpoint of political action, such a view may mean a reformism dealing with masses of detail and furthers a tendency to be apolitical. There can be no bases or points of entry for larger social action in a structureless flux\unknown{} The liberal ‘multiple‑factor’ view does not lead to a conception which would permit \unknown{} political action\unknown{} If one fragmentalizes society into ‘factors,’ into elemental bits, naturally one will then need quite a few of them to account for something, and one can never be sure they are all in\unknown{} The ‘organic’ orientation of liberalism has stressed all those social factors which tend to a harmonious balance of elements\unknown{} In seeing everything social as a continuous process, changes in pace and}


{\em revolutionary dislocations are missed or are taken as signs of the ’pathological’ , \unknown{} The ideally adjusted man of the social pathologists is “socialized.’ This term seems to operate ethically as the opposite of ‘selfish;’ it implies that the adjusted man conforms to middle‑class morality and motives and ‘participates’ in the gradual progress of respectable institutions. If he is not a ‘joiner,’ he certainly gets around and into many community organizations. If he is socialized, the individual thinks of others and is kindly toward them. He does not brood or mope about but is somewhat extrovert, eagerly participating in his community’s institutions. His mother and father were not divorced, nor was his home ever broken\unknown{} The less abstract the traits and fulfilled ‘needs’ of ‘the adjusted man’ are, the more they gravitate toward the norms of independent middle‑class persons verbally living out Protestant ideals in the small towns of America}.\footnote{“The Professional Ideology of Social Pathologists,” {\em American Journal of Sociology}, Vol. XLIX, No. 2 (September, 1943), in {\em Power, Politics and People}, pp. 536‑537 and pp. 551‑552.}


The professor who rejects the ideology of the politically impotent clerk, who refuses to trivialize himself and others, but who does not struggle against his impotence, may seek to escape by becoming a passive spectator whose goal is understanding. However, {\em Simply understanding is an ideal of the man who has a capacity to know truth but not the chance, the skill, or the guts, as the case may be, to communicate them} [sic] {\em with political effectiveness.}\footnote{“The Powerless People,” {\em loc. cit}., p. 300.}


In this context, when Mills writes that, {\em in general, the larger universities are still the freest places in which to work},\footnote{{\em Ibid}., pp. 296‑297.} he seems to be apologizing for his own choice of career. It is clear that, to Mills, being {\em free} did not mean that professors could publish books about their own powerlessness. Furthermore, he pointed out that professors were not even too free to do that, since {\em the deepest problem of freedom for teachers is not the occasional ousting of a professor, but a vague general fear—sometimes politely known as ‘discretion,’ ‘good taste, ’ or ‘balanced judgment.’ It is a fear which leads to self‑intimidation and finally becomes so habitual that the scholar is unaware of it. The real restraints are not so much external prohibitions as control of the insurgent by the agreements of academic gentlemen}.\footnote{{\em Ibid}., p. 297.} {\em Since ‘the job’ is a pervasive political sanction and censorship of most middle‑class intellectuals, the political psychology of the scared employee becomes relevant}.\footnote{{\em Ibid}., p. 300.} If the professor works in {\em the freest} place in which to work, then the situation of other sections of the population is, by implication, even more cramped than that of this {\em scared employee}. In that case, the community of {\em powerless people is} much larger than the academic community, and there is at least a possibility that the more powerless will be more interested in political action than the {\em freest}. If Mills’ statement about the freedom of the intellectual is taken seriously, then the basis on which the intellectual is to engage in political action is not clear: is he to struggle because he’s one of the powerless people, or because he’s already the freest member of American society?


Mills’ analysis of the situation of the professor is consistent with the title of his article, not with the justification of his chosen career. {\em The professor after all is legally an employee, subject to all that this fact involves}.\footnote{{\em Ibid}., p. 297.} And what this involves is not different for the professor than for the worker who sells his labor or for the clerk who sells his time; the only difference is what is sold. {\em When you sell the lies of others you are also selling yourself. To sell your self is to turn your self into a commodity.} \footnote{{\em Ibid.}, p. 300.}


In order to smash the official stereotypes of thought, to go beyond the various forms of academic escape, Mills abandons the world of charismatic leaders and manipulated masses and returns to Dewey’s community of self‑directed individuals; he {\em would like to stand for a politics of truth in a democratically responsible society}.\footnote{{\em Ibid}., p. 304.} This means that the individual does more than make moral evaluations which may help him {\em enrich his experience, expand his sensitivities, and perhaps adjust to his own suffering. But he will not solve the problems he is up against. He is not confronting them at their deeper sources}.\footnote{{\em Ibid}., pp. 298‑299.} And it means doing more than making detached, “objective” analyses of a spectacle in which the observer is not engaged, since this {\em is more like a specialized form of retreat than the intellectual orientation of a man.} What is involved is a location of oneself and a definition of reality which make coherent action possible. {\em If the thinker does not relate himself to the value of truth in political struggle, he cannot responsibly cope with the whole of live experience}.\footnote{{\em Ibid}., p. 299.}


The individual is able to formulate a political strategy only after he has located himself within his social situation. This is {\em necessary in order that he may be aware of the sphere of strategy that is really open to his influence. If he forgets this, his thinking may exceed his sphere of strategy so far as to make impossible any translation of his thought into action, his own or that of others. His thought may thus become fantastic. If he remembers his powerlessness too well, assumes that his sphere of strategy is restricted to the point of impotence, then his thought may easily become politically trivial,} And once he has formulated a strategy, he must communicate it with political effectiveness. {\em Knowledge that is not communicated has a way of turning the mind sour, of being obscured, and finally of being forgotten. For the sake of the integrity of the discoverer, his discovery must be effectively communicated}.\footnote{{\em Ibid}., p. 300.}


Mills does not formulate a specific strategy in this article, though he does refer to {\em discussion of world affairs} that proceeds {\em in terms of the struggle for power.}\footnote{{\em Ibid.}, p. 303.} The agents engaged in this struggle for power are not defined. Mills clearly does not refer to a struggle between intellectuals and the corporate‑military {\em elite.} However, it is not clear if he is referring to a struggle in which intellectual leaders manipulate dependent masses into {\em radical shifts in the distribution of power and in the composition of personnel,} or a struggle in which all the powerless people, intellectuals as well as workers, peasants, clerks, and students move to appropriate their alienated power.


\section{A Radical Strategy and a Liberating Agency of Change
}

In Mills’ next two major works,\footnote{{\em From Max Weber Essays in Sociology} (Translated and edited from the German by Mills with H.H. Gerth), New York: Oxford University Press; London: Routledge \& Kegan Paul Ltd., 1946, and {\em The New Men of Power.}} the rift between the academic spectator who takes the dependence of the “mass” and his own impotence for granted, and the radical intellectual committed to politically relevant action, becomes so wide that “C. Wright Mills” seems to become the name of two different authors.


Mills once again collaborated with Professor Hans H. Gerth, this time on a book of essays by Max Weber published in 1946. Whether he “benefited” primarily from his renewed contact with Weber or with Gerth, the Introduction to the book, written by Mills with Gerth, provides a frame of reference from which Mills would never again completely break loose.


Unlike the highly critical introductions to Veblen and Marx written by Mills in later years, the introduction to Weber is reverent, “objective,” and uncritical. Weber is introduced as {\em a political man and a political intellectual},\footnote{Introduction to {\em From Max Weber}, p. 32.} namely as a model of something which {\em the powerless people} are not. As a young man, Weber was a National Liberal; {\em in the middle ‘nineties, Weber was an imperialist, defending the power‑interest of the national state as the ultimate value and using the vocabulary of social Darwinism.}\footnote{{\em Ibid}., p. 35.} During World War 1, {\em He clamored for ‘military bases’ as far flung as Warsaw and to the north of there. And he wished the German army to occupy Liege and Namur for twenty years}.\footnote{{\em Ibid}., p. 39.} When he moved to a “democratic” position, it was not because he saw {\em democracy as an intrinsically valuable body of ideas\unknown{} He saw democratic institutions and ideas pragmatically: not in terms of their ‘inner worth’ but in terms of their consequences in the selection of efficient political leaders. And he felt that in modem society such leaders must be able to build up and control a large, well-disciplined machine, in the American sense}.\footnote{{\em Ibid}., p. 38.} And finally, {\em It is, of course, quite vain to speculate whether Weber with his Machiavellian attitude might ever have turned Nazi. To be sure, his philosophy of charisma—his skepticism and his pragmatic view of democratic sentiment—might have given him such affinities. But his humanism, his love for the underdog, his hatred of sham and lies, and his unceasing campaign against racism and anti‑Semitic demagoguery would have made him at least as sharp a ‘critic, if not a sharper one, of Hitler than his brother Alfred has been}.\footnote{{\em Ibid}., p. 43.}


Weber’s definition of reality is one in which {\em the politics of truth in a democratically responsible society} would have no meaning, because revolutionary political strategies cannot be formulated. The {\em comprehensive underlying trend} of modern society is bureaucratization, a {\em process of rationalization} identified with {\em mechanism, depersonalization, and oppressive routine}.\footnote{{\em Ibid}., p. 50.} This trend does not consist of voluntary collective activities, but of processes which take place behind men’s backs and over which they have no control. Even a revolutionary movement is, at best, only an instrument of these processes. {\em Socialist class struggles are merely a vehicle implementing this trend.} \footnote{{\em Ibid}.} In short, the comprehensive trend of history, like the law of gravity, is beyond man’s reach, and the political intellectual, like the physicist of nineteenth century European science, is merely a member of an audience who observes a spectacle. In this context Mills does not say that {\em simply understanding is an ideal of the man who has the capacity to know truth but not the chance, the skill, or the guts, as the case may be, to communicate\unknown{} with political effectiveness.}


Weber does provide some elements which could lead out of this passive observation of underlying trends. The bureaucratization takes place in a context where the wage worker is separated from the means of production, where {\em The modern soldier is equally ‘separated’ from the means of violence, the scientist from the means of enquiry, and the civil servant from the means of administration.} Mills does not, however, follow this lead into a study of alienation as an activity. Instead, he merely says that Weber {\em relativizes} Marx’s conclusions about the alienation of the wage worker.\footnote{{\em Ibid.}}


In this world where men are reduced to fragments of bureaucracies whose aims they neither understand nor control, there can be no publics of self‑determined individuals whose collective action has consequences on the underlying trend of history. In the place of such publics, Weber offered an alternative which appealed to large numbers of impotent, fragmented men in the twentieth century. Weber {\em places great emphasis upon the rise of charismatic leaders. Their movements are enthusiastic, and in such extraordinary enthusiasms class and status barriers sometimes give way to fraternization and exuberant community sentiments. Charismatic heroes and prophets are thus viewed as truly revolutionary forces in history. Bureaucracy and other institutions, especially those of the household, are seen as routines of workaday life; charisma is opposed to all institutional routines, those of tradition and those subject to rational management. This holds for the economic order: Weber characterizes conquistadores and robber barons as charismatic figures\unknown{} they have in common the fact that people obey them because of faith in their personally extraordinary qualities\unknown{} the monumentalized individual becomes the sovereign of history}.\footnote{{\em Ibid}., pp. 52‑53.} With detachment and even with reverence, Mills and Gerth observe, in 1946, that in Weber’s view men cannot collectively make their own history; even revolutionary movements can merely implement what are already the underlying trends of history; and that {\em Weber introduces a balancing conception for bureaucracy: the concept of ‘charisma,}\footnote{{\em Ibid}., p. 52.} according to which man nevertheless makes history, but only one man, the charismatic leader, Superman.


\section{---
}

As if to dissociate himself from Gerth, Weber and the Charismatic Leader, Mills opens his next major work with the following frontispiece:\footnote{{\em The New Men of Power}, Frontispiece.}



\startblockquote
{\em When that boatload of wobblies come} 


{\em Up to Everett, the sheriff says}  


{\em Don’t you come no further} 


{\em Who the hell’s yer leader anyhow?} 


{\bf Who’s yer leader?}  


{\em And them wobblies yelled right back} 


{\bf We ain’t got no leader} 


{\bf We’re all leaders} 


{\em And they kept right on comin’} 



\stopblockquote

\startblockquote
—From an interview with an unknown worker Sutfliffe, Nevada June, 1947.



\stopblockquote
Mills’ first published book, completed when he was 32, is neither a contribution to academic sociology nor a detached and apolitical accomplishment along the journey of a successful professional career. It is {\em a politically motivated task},\footnote{{\em The Sociological Imagination}, p. 200.} and as such it takes up projects which had been left unfinished before the second excursion with Professor Gerth.


Rejecting the impotence of the academic intellectual, this politically motivated task aims to {\em be politically relevant,}\footnote{{\em The New Men of Power}, p. 10.} to go beyond those independent leftists for whom {\em political alertness is becoming a contemplative state rather than a spring of action: they are frequently overwhelmed by visions, but they have no organized will\unknown{} they see bureaucracy everywhere and they are afraid}.\footnote{{\em Ibid}., p. 18.} The book aims to expose the labor leader who is {\em walking backwards into the future envisioned by the sophisticated conservatives. By his long‑term pursuit of the short end, he is helping move the society of the United States into a corporate form of garrison state}.\footnote{{\em Ibid}., p. 233.} In this book events are not explained in terms of underlying trends or inevitable historical processes, but in terms of decision and indecision, action and inaction, radical will and failure of nerve. In this context, political thinking becomes a practical activity, and strategy once again has meaning, because consequential collective action is once again defined as possible.


To regain his bearings, to locate himself through a {\em fresh perception} of his context, Mills again undertakes to {\em unmask and to smash the stereotypes of vision and intellect} which hide the consequences of people’s activity from their view. A chapter is devoted to {\em The Liberal Rhetoric,} which {\em has become the medium of exchange among political, scholarly, business, and labor spokesmen.}\footnote{{\em Ibid.,} p. 111.} The formulas of this ritualized substitute for communication do not clarify the social situation but obscure it. {\em The rhetoric of liberalism is related neither to the specific stands taken nor to what might be happening outside the range of the spokesman’s voice. As applied to business‑labor relations, the liberal rhetoric is not so much a point of view as a social phenomenon\unknown{} The liberal rhetoric personalizes and moralizes business‑labor relations. It does not talk of any contradiction of interests but of highly placed persons\unknown{}}\footnote{{\em Ibid}., p. 111 and 113.} Within the framework of this social phenomenon, pious wishes about the personal morals of the highly placed persons replace political theory and practice: {\em “If only the spokesmen for both sides were uniformly men of good will and if only they were intelligent, then there would be no breach between the interests of the working people and those of the managers of property”}\footnote{{\em Ibid.}, p. 114.} In sharp contrast to the liberal rhetoric, {\em the program of the far left \unknown{} attempts to get to the root of what is happening and what might be done about it} and consequently, {\em by the public relations‑minded standards of sophisticated conservatives, it is naively outspoken and stupidly rational.}\footnote{{\em Ibid}., p. 240.}


To get to the root of what is happening, and to define it in ways that make clear what might be done about it, Mills has no use for an underlying trend or a substance of history which runs its course like an incurable disease whatever men decide and do. Instead of the unrelenting and inevitable march of the nearly cosmic process of bureaucratization, Mills now sees big businessmen installing the dominant bureaucracies of post‑war America under the very noses of the labor leaders to whom workers had delegated their struggle (and thus alienated their power). {\em The sophisticated conservatives see the world, rather than some sector of it, as an object of profit. They have planned a series of next steps which amount to a New Deal on a world scale operated by big businessmen.}\footnote{{\em Ibid}., pp. 240‑241.} This is no natural law; it is {\em The Program of the Right,} a program which consists of nothing less than the establishment of the Power Elite and the Permanent War Economy described in Mills’ later works as the socio‑econornic system of the United States. {\em From the union of the military, the scientific, and the monopoly business elite, ‘a combined chief of staff for America’s free private enterprise is to be drawn. If the sophisticated conservatives have their way, the next New Deal will be a war economy rather than a welfare economy, although the conservative’s liberal rhetoric might put the first in the guise of the second.}\footnote{{\em Ibid}., pp. 248‑249.}


There was nothing natural or inevitable about this process to Mills in 1948; it was both unnatural and avoidable. Lacking a concept of alienation, he does not go to the root of the {\em political apathy of the American worker}\footnote{{\em Ibid}., p. 269.} who unmanned himself by allowing labor leaders to speak and act for him, but Mills does narrate what the labor leaders did with the worker’s alienated power: \unknown{} the {\em labor leader often assumes the liberal tactics and rhetoric of big business co‑operation; he asks for the program of the sophisticated conservative; he asks for a place in the new society\unknown{}}\footnote{{\em Ibid}., p. 249 and p. 233.} and thus, {\em he is helping move the society of the United States into a corporate form of garrison state.} Watching the labor leader bow and crawl, the political intellectual chooses his own course of action—the intellectual who, as Mills knew well in 1948, was not made powerless by underlying historical trends but by his own decisions; whose situation as a scared employee was not imposed on him from above or below, but was deliberately chosen. {\em The two greatest blinders of the intellectual who today might fight against the main drift are new and fascinating career chances, which often involve opportunities to practice his skill rather freely, and the ideology of liberalism, which tends to expropriate his chance to think straight. The two go together, for the liberal ideology, as now used by intellectuals, acts as a device whereby he can take advantage of the new career chances but retain the illusion that his soul remains his own. As the labor leader moves from ideas to politics, so the intellectual moves from ideas to career.}\footnote{{\em Ibid}., p. 281.} As a result of the choices made by those to whom workers had given up their power to act and think, {\em the main and constant function of a union is to contract labor to an employer and to have a voice in the terms of that contract\unknown{} the labor leader is a business entrepreneur in the important and specialized business of contracting a supply of trained labor\unknown{} The labor leader organizes and sells wage workers to the highest bidder on the best terms available. He is a jobber of labor power. He accepts the general conditions of labor under capitalism and then, as a contracting agent operating within that system, he higgles and bargains over wages, hours and working conditions for the members of his union. The labor leader is the worker’s entrepreneur in a way sometimes similar to the way the corporation manager is the stockholder’s entrepreneur.}\footnote{{\em Ibid.}, p. 6.}


In later works, Mills is going to write about the {\em collapse} of historical agencies of change, about a {\em Labor Metaphysic} which holds that workers are going to arise spontaneously, about a {\em promise of labor} which was not fulfilled;\footnote{See {\em Power, Politics and People}, pages 187, 105‑108, 232, 255‑259.} he is going to describe these false hopes as if they were traps into which he had once fallen, as if he had once believed that American workers were about to initiate a vast anti-capitalist struggle, which Mills would join as soon as the workers began it. But, whatever traps the straw men of the {\em labor movement} may have fallen into, Mills was never in such traps (at least not in his published works). He had not even mentioned the American worker as a revolutionary force before {\em The New Men of Power,} and in this book he considers the American worker politically apathetic. He does say that the U.S. worker may, under certain circumstances, be willing to take steps toward his own humanization, but by saying this he merely gives the U.S. worker attributes which, after all, this person shares with all normal human beings. In the light of the analysis he makes in {\em The New Men of Power,} Mills’ later pronouncements about the automatic agency of change which collapsed, his disclaimers of any {\em Labor Metaphysic,} his “disappointments” with the {\em promise of labor,} can only be interpreted as excuses for his own movement from ideas to career, as liberal ideological devices which he used to take advantage of new career chances while retaining the illusion that his soul remained his own.


In 1948 Mills does not seem to have been waiting for the {\em politically apathetic} workers to “arise.” He was concerned, rather, with defining the circumstances in which workers might be willing to move. And the first condition for such movement was to cope with the apathy, the dependence, the lack of initiative and self‑determination which largely account for the worker’s powerlessness and dehumanization: \unknown{} {\em the power of democratic initiation must be allowed and fostered in the rank and file\unknown{} During their struggle, the people involved would become humanly and politically alert.}\footnote{{\em The New Men of Power}, pp. 252‑253.}


Only then can the left be {\em linked securely with large forces of rebellion.}\footnote{{\em Ibid}., p. 250.} However, forces of rebellion do not “arise” any more automatically than individuals who strive to communicate radical goals and strategies, and workers do not become apathetic any more automatically than the professors or labor leaders who abandon these political tasks in order to enjoy academic or political privileges with the explanation that the historical agency “collapsed.” {\em Yet it is somehow easier to excuse in the others; they are not leaders of a protest of such proportion; they follow the main drift with a certain fitness and pleasure, feeling there is something to gain from it, which there often is. But the labor leader represents the only potentially liberating mass force; and as he becomes a man in politics, like the rest, he forgets about political ideas\unknown{} Programs take time; of the long meantime, the labor leader is afraid; he crawls again into politics‑as‑usual.}\footnote{{\em Ibid}., pp. 169‑170.}


On the basis of a definition of reality which clarifies the activities responsible for people’s powerlessness, and the location of potential historical subjects who may be willing to struggle for their lost power, Mills is able, for the first time, to link thought with projected action, to formulate a general political strategy. In its broadest form, the strategy is {\em To have an American labor movement capable of carrying out the program of the left, making allies among the middle class, and moving upstream against the main drift}\unknown{}\footnote{{\em Ibid}., p. 291.} Before the program of the left can be carried out, it must be communicated —and this communication is precisely one of the tasks of Mills’ book. {\em We shall attempt to do only one thing: to make the collective dream of the left manifest.}\footnote{{\em Ibid}., p. 251.} Only after the strategy has been communicated with political effectiveness will it be possible to speak of workers as a potential agency of change, and then only because the strategy consists of a commonly undertaken project. {\em The  American worker has a high potential militancy when he is pushed, and if he knows what the issue is. Such a man, identified with unions as communities and given a chance to build them, will not respond apathetically when outside political forces attempt to molest what is his.}\footnote{{\em Ibid}., pp. 269‑270.} Whatever {\em promise} there is in this perspective, it is not based on the expectation that a Savior in the shape of a class conscious proletariat will descend from heaven to pull mankind out of the main drift, but rather on one’s own determination to fight and on one’s ability to define a field of strategy within which the struggle can be effective. Consequently, one cannot later be “disappointed” by the fact that the Savior did not arrive, but only by one’s own timidity, indecision and failure to choose. {\em The American labor unions and a new American left can release political energies, develop real hopefulness, and open matters up for counter‑symbols only if they are prepared to act boldly and win over the less bold by their success. The labor leaders and the U.S. workers are not alone if they choose to fight. They have potential allies of pivotal importance. All those who suffer the results of irresponsible social decisions and who hold a disproportionately small share of the values available to man in modern society are potential members of the left. The U.S. public is by no means a compact reactionary mass. If labor and the left are not to lose the fight against the main drift by default and out of timidity, they will have to choose with whom they will stand up and against whom they will stand.}\footnote{{\em Ibid}., p. 274.}


In spite of the lucidity with which Mills exposes the choice confronting the political intellectual, he is frequently at pains to build himself an escape hatch, and he closes the book with it: {\em It is the task of the labor leaders to allow and to initiate a union of the power and the intellect. They are the only ones who can do it, ­that is why they are now the strategic elite in American society. Never has so much depended upon men who are so ill‑prepared and so little inclined to assume the responsibility}.\footnote{{\em Ibid}., last paragraph of the book, p. 291.} This last paragraph of the book flatly contradicts much of the book’s content, and particularly the frontispiece in which the wobblies yell {\bf We’re all leaders}—{\em and they kept right on comin!} The last paragraph is not written by the same man who inserted the frontispiece, nor by the member of {\em a new American left} who is determined {\em to act boldly and win over the less bold}; it is written by a more passive type of man, a sociology professor who benefitted from his contact with Max Weber and Hans Gerth. The rift between the frontispiece and the last paragraph was never bridged by Mills; it seems that the Weberian leaders and the leaderless Wobblies occupied separate compartments in Mills’ mind, and since either one, or the other, emerged from a compartment at any given time, the two never directly confronted each other.


\chapter{Two: The Mindless Years 1950‑1956
}

\section{The Cheerful Robot and the Rift between Thought and Action
}

Abandoning workers to labor leaders who have crawled again into politics‑as‑usual, Mills took up new and fascinating career chances which involved opportunities to practice his skill rather freely. Between 1950 and 1956, he wrote two major books and numerous articles, and took his third, largest and last excursion with Professor Gerth. In all these works, the influence of Weber and Gerth is dominant; the independent political radical is pushed to the margins, and in the work with Gerth is altogether absent. Yet this framework cannot hold the man who once committed himself to deeper levels of analysis and strategy, and at the end of this period the margins expand and once again become the central concerns. However, like the earlier interruptions of Mills’ search for political coherence, the new excursions and retreats are not overcome, and as a result they leave large scars.


To professors of sociology, the period which Mills later called {\em the mindless years} is Mills’ most “creative” period: he wrote a sociology textbook with Gerth, plus two original contributions to the “profession.” Although the well documented observations of the original works are somewhat marred by marginal observations which are cryptic and controversial, the textbook clearly lives up to the expectations of the head scholars of the profession: “Whether use of the book precedes, accompanies, or follows intensive study of the short‑run present in the laboratory, field and clinic, it should broaden the horizon of the student who generally comes into social psychology either through the gateway of psychology or of sociology.”\footnote{Robert K. Merton’s Foreword to C. Wright Mills and H. H. Gerth, {\em Character and Social Structure: The Psychology of Social Institutions}, New York: Harcourt Brace \& Company, 1953, pp. vii‑viii.}


In this essay, I will focus attention to the controversial margins, because this is where Mills analyzed himself, his fellow academics, and his dehumanizing experience in the white collar hierarchy.


The introduction to {\em White Collar} contains the most comprehensive analysis of alienation that Mills ever made. {\em In the case of the white‑collar man, the alienation of the wage‑worker from the products of his work is carried one step nearer to its Kafka‑like completion. The salaried employee does not make anything, although he may handle much that he greatly desires but cannot have. No product of craftsmanship can be his to contemplate with pleasure as it is being created and after it is made. Being alienated from any product of his labor, and going year after year through the same paper routine, he turns his leisure all the more frenziedly to the} {\bf ersatz} {\em diversion that is sold him, and partakes of the synthetic excitement that neither eases nor releases. He is bored at work and restless at play, and this terrible alternation wears him out \unknown{} When white‑collar people get jobs, they sell not only their time and energy, but their personalities as well. They sell by the week or month their smiles and their kindly gestures, and they must practice the prompt repression of resentment and aggression\unknown{} Self‑alienation is thus an accompaniment of his alienated labor}.\footnote{Mills, {\em White Collar: The American Middle Classes}, New York: Oxford University Press, pp. xvi‑xvii.} The separation of the individual from his own activity and even from his gestures, the individual’s lack of power over his own self, is accompanied by a feeling of general powerlessness, by political indifference. {\em To be politically indifferent is to see no political meaning in one’s life or in the world in which one lives, to avoid any political disappointments or gratifications. So political symbols have lost their effectiveness as motives for action and as justifications for institutions.} Mills characterizes various forms of political indifference among the white collar people; some of the people whose lives make no difference escape an awareness of this fact by means of {\em animal thrill, sensation, and fun.} However, political indifference may also {\em be a reasoned cynicism, which distrusts and debunks all available political loyalties and hopes as lack of sophistication. Or it may be the product of an extra‑rational consideration of the opportunities available to men, who, with Max Weber, assert that they can live without belief in a political world gone meaningless, but in which detached intellectual work is still possible}.\footnote{{\em Ibid}., p. 327.} This analysis of political indifference is riot based on statistical studies of white collar people. It is based on personal experience. Mills and Ruth Harper write that that {\em Our knowledge of this is firmer than any strict proof available to us. It rests, first of all, upon our awareness, as politically conscious men ourselves, of the discrepancy between the meaning and stature of public events and what people seem most interested in}.\footnote{{\em Ibid}., p. 328.} {\em Whenever in this book, I have written ‘we’ I mean my wife, Ruth Harper, and myself} \unknown{}\footnote{{\em Ibid}., p. 355.}) {\em It is a sense of our general condition that lies back of our conviction that political estrangement in America is widespread and decisive}.\footnote{{\em Ibid}., p. 331.} Thought is separated from living experience, and the formerly political intellectual becomes a passive spectator.  {\em Most of us now live as spectators in a world without political interlude: fear of total permanent war stops our kind of morally oriented politics.  Our spectatorship means that personal, active experience often seems politically useless and even unreal}.\footnote{“Liberal Values in the Modern World: The Relevance of 19\high{th} Century Liberalism Today,” {\em Anvil and Student Partisan} (Winter, 1952), in {\em Power, Politics and People}, p. 187.}


Mills does not accept this condition. In a section on {\em The Morale of the Cheerful Robot} he writes, {\em whatever satisfaction alienated men gain from work occurs within the framework of alienation; whatever satisfaction they gain from life occurs outside the boundaries of work; work and life are sharply split}.\footnote{{\em White Collar}, p. 235.} Furthermore, Mills does not apologize for this split as the legitimate attitude of the “objective”, detached scholar. For Mills such a pose is the pose of in idiot and Mills remains a different kind of man: we {\em are now in a situation in which many who are disengaged from prevailing allegiances have not acquired new ones, and so are distracted from and inattentive to political concerns of any kind. They are strangers to politics. They are not radical, not liberal, not conservative, not reactionary; they are inactionary; they are out of it. If we accept the Greek’s definition of the idiot as a privatized man,} {\em then we must conclude that the U.S. citizenry is now largely composed of idiots}.\footnote{{\em Ibid}., p. 328.}


Instead of accepting this mass incapacity, Mills seeks to under stand it, so as to get out of it. He asks why men accept themselves with a smile and a cheer, as dependent robots and helpless idiots whose lives make no difference, and he begins to answer. {\em Between consciousness and existence stand communications, which influence such consciousness as men have of their existence}.\footnote{{\em Ibid}., pp. 332‑333.} And the communications provided by the cultural apparatus of the US, consisting of’ mindless commodity propaganda, obfuscating liberal rhetoric and debilitating entertainment, helps explain why the US citizenry is now largely composed of idiots: {\em The forms and contents of political consciousness, or their absence, cannot be understood without reference to the world created and sustained by these media\unknown{} Contents of the mass media seep into our images of self, becoming that which is taken for granted\unknown{} The world created by the mass media contains very little discussion of political meanings, not to speak of their dramatization, or sharp demands and expectations\unknown{} The prevailing symbols are presented in such a contrived and pompous civics‑book manner}{\em {\bf ,}} {\em or in such a falsely human light, as to preclude lively involvement and deep‑felt loyalties \unknown{} The mass media hold a monopoly of the ideologically dead; they spin records of political emptiness.  .   .  . The attention absorbed by the images on the screen’s rectangle dominates the darkened public\unknown{} The image of success and its individuated psychology are the most lively aspects of popular culture and the greatest diversion from politics\unknown{} The easy identification with private success finds its obverse side, Gunnar Myrdal has observed, in ‘the remarkable lack of a self‑generating, self‑disciplined, organized people’s movement in America.’}\footnote{{\em Ibid}., pp. 334‑337.} {\em The ideals of liberalism have been divorced from any realities of modem social structure that might serve as the means of their realization . .. The detachment of liberalism from the facts of a going society make it an excellent mask for those who do not, cannot, or will not do what would have to be done to realize its ideals}.\footnote{“Liberal Values in the Modern World,” {\em loc}{\em {\bf .}} {\em cit}., p. 189.} In his {\em Diagnosis of Our Moral Uneasiness}, Mills turns to the effects of leisure on the drugged and deluded spectators: \unknown{} {\em leisure} {\em itself has largely become merely it part of consumption, no longer part} {\em of a full life, but a substitute for it. For to this sphere also, the mean of mass production—the machineries of amusement—have been applied Rather than allow and encourage men to develop their sensibilities and unfold their creativities, their leisure merely wears them out}.\footnote{“A Diagnosis of Our Moral Uneasiness,” {\em New York Times Magazine} (November 23, 1952); complete version published for the first time in {\em Power, Politics and People}, pp. 332‑333.}


Yet even though Mills rejects the passivity with which men accept their own fragmentation, he no longer struggles against it. The coherent self‑determined man becomes an exotic creature who lived in a distant past and in extremely different material circumstances. The first part of {\em White Collar} opens with the following quotation from R.H. Tawney:


{\em “Whatever the future may contain, the past has shown no more excellent social order than that in which the mass of the people were the masters of the holdings which they plowed and of the tools with which they worked\unknown{}”}\footnote{R.H. Tawney, quoted in Mills, {\em White Collar}, p. 1.} As for the present, cheerful robots, buyers, floorwalkers and salesgirls, professors and managers, shuffle between the Enormous File and The Great Salesroom, purge what remains of their humanity by running in The Status Panic and shopping in The Biggest Bazaar in the the World, while {\em What goes on domestically may briefly be described in terms of the main drift toward a permanent war economy in a garrison state}.\footnote{“ Liberal Values in the Modern World,” {\em loc. cit}., p. 187.} The main drift is no longer the program of the right which can be opposed by the program of the left; it is now an external spectacle which follows its course like a disease.


The American labor movement capable of moving upstream against the main drift, and the leaderless men who kept right on comin’, are abandoned to the media of mass distraction, and to labor leaders. Mills does not excuse this in terms of the political detachment of the objective scholar; he excuses it in terms of the political default of others, even in terms of the default of the workers themselves: the Savior did not arrive, {\em Whatever the political promises of labor and leftward forces 15 years ago, they have not been fulfilled\unknown{}}\footnote{{\em Ibid.}}


As a result, it is not possible {\em to see oneself as a demanding political force}\footnote{{\em White Collar}, p. 327.} since one has not defined a social context {\em in which men willfully modify and create their institutions}.\footnote{“A Diagnosis of Our Moral Uneasiness,” {\em loc. cit}., p. 337.} The field of strategy has been restricted to the point of impotence, since the powerless intellectual has no strategy and no one to communicate it to. Thus restricted, the impotent professor can no longer remain coherent; the rift between theory and practice, thought and action, widens; political ideals call no longer he translated into practical projects, and projected actions are no longer related to any ideals. Thus the same writer who speaks of men willfully creating their own institutions refers to political action as having {\em real demands to make of those in key positions of power}.\footnote{“Liberal Values in the Modern World,” {\em loc. cit}., p. 187.} Willful self‑determination characterizes angels in a city built with words, whereas political activity in the city of men consists of submission to those in key positions of power. Behemoth is everywhere united. But the man who was once moved by this fact into deeper levels of analysis and strategy, now retreats to a post‑World War II formulation of Max Weber’s salvation from impotence and routine {\em \unknown{} there is in America today no set of Representative Men whose conduct and character are above the taint of the pecuniary morality, and who constitute models for American imitation and aspiration\unknown{} Yet it is the moral man—and especially the set of socially visible or Representative Men—who by demanding moral change can best dramatize issues.}\footnote{“ Diagnosis of Our Moral Uneasiness,” {\em loc. cit}., pp. 336‑337.}


\section{Intellectual Default and Escape into Academic Cynicism
}

In the early 1950s, Mills seems to have been in the right frame of mind for his major project with Professor Gerth. {\em Character and Social Structure} is not a political task, it is not a strategy of action addressed to a democratically responsible public. Its aim is not to make the collective dream of the left manifest to potential forces of rebellion. It is a textbook, an encyclopedic compilation of other people’s thoughts, an administrative classification of fragmentary observations, addressed to the powerless people, the status seeking academic bureaucrats who may use it on students who come “either through the gateway of psychology or of’ sociology” for wisdom which “precedes, accompanies or follows intensive study of the short‑run present in the laboratory, field and clinic.” An ironic result of this rational compartmentalization of fragments is that one compartment’s fragments may affirm what is denied by the fragments classified into another compartment. This rationalized incoherence provides a framework in which most of Mills’ earlier observations coexist with their opposites in politically trivial contexts. The book even contains a devastating critique of the bureaucratic structure it is designed to serve. {\em The demand of the state and of corporations for trained civil servants and qualified experts of all sorts has been decisive for the modern development of universities\unknown{} Lorenz von Stein correctly called the modern university ‘a school for bureaucrats.’}\footnote{Mills and Gerth, {\em Character and Social Structure}, p. 254.}


On the basis of the definitions of reality which emerge from the work, a reader {\em cannot responsibly cope with the whole of live experience}.\footnote{Mills, “The Powerless People,” {\em loc. cit}., p. 299.} Instead of asking why people allow themselves to be dehumanized, to be forced to live out their lives on a stage playing the roles of cheerful robots, the authors simply lean back and abandon themselves to the enjoyment of the grand spectacle for which {\em sociologists have fashioned analytical tools. Long‑used phrases readily come to mind: ‘playing a role’ in the ‘great theater of public life,’ to move ‘in the limelight,’ the ‘theater of War,’ the ‘stage is all set.’}\footnote{Mills and Gerth, {\em Character and Social Structure}, p. 10.} Instead of {\em attempts to get to the root of what is happening and what might be done about it,} this textbook provides cold descriptions of what usually happens, presented in such a way that one cannot imagine what might be done about it. {\em An} {\bf institution} is {\em thus (1) an organization of roles, (2) one or more of which is understood to serve the maintenance of the total set of roles.}\footnote{{\em Ibid}., p. 13.} Here slaves, clerks and wage workers are nothing more than obedient sheep, or {\em roles,} and the degradation and self‑annihilation involved in every act of submission is merely the part assigned to supporting characters by the script. {\em The ‘head role’ of an institution is very important in the psychic life of the other members of the institution. What ‘the head’ thinks of them in their respective roles, or what they conceive him to think, is internalized, that is, taken over, by them.}\footnote{{\em Ibid}.} The fact that the {\em head role} has power only because, and only so long as the others voluntarily separate themselves from their own power, and thus annihilate their own humanity, is also mentioned. {\bf Authority}, {\em or legitimate power, involves voluntary obedience based on some idea which the obedient holds of the powerful or of his position. ‘The strongest,’ wrote Rousseau, is never strong enough to be always master, unless he transforms his strength into right, and obedience into duty.}’\footnote{{\em Ibid}., p. 195.} But Rousseau’s lead is not followed; the voluntary alienation of self‑powers is not analyzed in any politically meaningful context.


The authors mention that the social activities in which people engage are not determined by people’s biology, but are specific voluntary responses to particular situations; they are historical, not “natural.” The routinized activities which account for most people’s daily life may well be “roles” which they voluntarily perform in the face of specific obstacles; it may well be true that, in the past, people voluntarily performed the same roles all life long, and thus alienated their selves. However, even if an actor puts on the mask of Oedipus and remains on stage reciting the same lines for the rest of his life, the actor’s self cannot be confused with his mask. Yet this is precisely what the professors confuse. They point out that {\em man as a person is a social‑historical creation,} and they specify that {\em a} {\bf person} {\em (from the Latin} {\bf persona}, {\em meaning ‘mask) is composed of the specific roles which he enacts,} although the word {\em composed} already introduces an ambiguity. But then they say that {\em In order to understand men’s conduct and experience we must reconstruct the historical social structures in which they play roles and acquire selves.}\footnote{{\em Ibid}., p. 14.} In other words, by playing the role of Oedipus a man acquires a self, whereas in actual fact, by playing the role of Oedipus the man becomes a character in a play, spectacle, a dead thing: he alienates his self, and acquires a mask. By confusing the man with his masks, the professors close the very possibility of analyzing man’s self‑alienation in {\em roles} and masks, But if that case they cannot study social {\em institutions} as historical forms of routinized activity, as masks which people voluntarily put on in specific circumstances. Consequently, their frequent use of the term {\em historical} conveys nothing more than the superficial observation that people perform different activities in different periods of time.


Armed with a conception which reduces man to his particular “behavior” in particular circumstances (which coexists with a fragment from Rousseau which points in the opposite direction from this conception), the professors describe social activity as a grand theatrical performance, a vast spectacle. In this enormous drama, there are not merely roles, but bureaucratically arranged sets of roles, or Institutional Orders. These Orders, or {\em Spheres,} are named in terms of the types of roles played within them; the main Orders are political, economic and military; other Orders contain religious, kinship and educational roles Each Institutional Order has a corresponding script, or {\em symbol sphere.} Standard scenes performed in the political and military orders are described in the following dramatic terms: {\em Once a national community is fully a state, it monopolizes the use of legitimate violence within its domain, defends its domain against other states, and may attempt to expand it}\footnote{{\em Ibid}., p. 203.}  {\em \unknown{} When a nation‑state extends political protection to the trading areas of its businessmen we speak of ‘imperialism.’ The most explicit types of imperialism involve the acquisition of a colonial empire by purchase, or conquest, or both.}\footnote{{\em Ibid}., p. 204.}


{\em The violence of a modern national army is legitimated by the symbols and sentiments of the nation and its cause; the men of this army are disciplined for obedience to a hierarchy of staff and line officers.} The following sentence explains that {\em Discipline rests upon acceptance of the nation’s  cause and is guaranteed by sanctions—including loss of status and career chances and, in the last analysis, capital punishment.}\footnote{{\em Ibid}., p. 229.} This explanation of discipline, not merely in terms of force, but in terms of the {\em nation’s cause,} in terms of {\em right,} obscures tile meaning of the statement from Rousseau which was quoted with approval by the professors. {\em The strongest is never strong enough to be always master, unless he transforms his force into right and obedience into duty, This is the origin of the right of the strongest, a right seemingly accepted in irony, and actually established in principle. But will this word never be explained to us? Force is a physical power; I don’t see what morality can result from its effects. To give in to force is an act of necessity, not of will; it’s at best an act of prudence. In what sense could it he a duty?} Asked Rousseau after the statement quoted by the professors\unknown{} {\em What} {\em kind of a right perishes when force ceases? If one has to obey because of force, one need not obey because of duty; and if one is no longer forced to obey, one is no longer obliged. We can see that this word} right {\em does not add anything to force; here it means nothing at all\unknown{} Obey power! If that means give in to force, the precept is good but superfluous; I answer that it will never be violated\unknown{} Since no man has any natural authority over his equal, and since force produces no right only conventions remain as the basis of all legitimate authority among men\unknown{} To alienate is to give or sell\unknown{} (But) to say that a man gives himself freely is to say something absurd and inconceivable; such an act is illegitimate and void if only because the man who does this is not in his right mind. To say the same thing about a population is to suppose a population of madmen; but madness does not create right\unknown{} To renounce one’s liberty is to renounce one’s quality as a man the rights of humanity as well as the duties\unknown{} In short, it is a vain and contradictory convention which stipulates absolute authority for one and unlimited obedience for the others.}\footnote{Jean‑Jacques Rousseau, {\em Du Contrat Social}, Paris: Union Générale d’Editions, 1963, pp. 54‑56.} If Rousseau’s argument had been re‑thought coherently, and not included bureaucratically as a fragment from a file, the professors would have explained that military discipline rests on a complete renunciation of one quality as a man, that even a modern national army is never strong enough to be always master, and consequently that the continued renunciation of one’s humanity cannot be guaranteed by anything.


If obedience and discipline could be guaranteed, man would have no history. But this is not the point of the last paragraph of {\em Character and Social Structure,} where it is said that man not only has a history, but creates it. {\em Neither his anatomy nor his psyche fix his destiny. He creates his own destiny as he responds to his experienced situation, and both his situation and his experiences of it are the complicated products of the historical epoch which he enacts. That is why he does not create his destiny as an individual but as a member of a society. Only within the limits of his place in an historical epoch can man as an individual shape himself, but we do not yet know, we call never know, the limits to which men collectively might remake themselves.}\footnote{Mills and Gerth, {\em Character and Social Structure}, p. 480.} This conclusion is undermined by most of what precedes it. According to paragraphs which immediately precede the conclusion, it is not men, but nations, namely frozen concentrations of men’s alienated powers, that make modern history. {\em On the one hand, there is the U.S.S.R., the world’s greatest land power\unknown{} On the other hand, there is the U.S.A., the world’s greatest industrial and naval power\unknown{} All countries are now interdependent, but all countries are also now directly or indirectly dependent upon the dollar or the ruble standard, upon what the United States or the Soviet Union does or fails to do.}\footnote{{\em Ibid.,} pp. 472‑473.} It is a spectacle with two superhuman heroes; they act, and men obey. The very possibility of collective projects based on shared perspectives and strategies is dismissed by a reasoned cynicism which distrusts and debunks all political activity. In the professional jargon of these authors, reference to straightforward communication among self‑determined individuals would lack sophistication; instead of community, there are roles, and the verbal exchange between roles is not communication but manipulation; the manipulator has a monopoly on his skill: he is a symbol expert; his manipulated audience consists of men who are not specialists in symbols but in other “disciplines” (i.e. they have even alienated their power to express themselves): {\em Skill groups, such as poets and novelists, specialize in fashioning and developing vocabularies for emotional states and gestures; they specialize in telling us how we feel, as well as how we should or might feel, in various situations.}\footnote{{\em Ibid}., p. 56.} In terms of this type of language, political action is reduced to efficient manipulation, because the world consists of rat‑like masses who move and shift in response to particular symbolic stimuli. {\em In the scholar’s study} or {\em the agitator’s den the symbols which legitimate various kinds of political systems may be rearranged, debunked, or elaborated\unknown{} For changes in the legitimating symbols to be realized, masses of people must shift the, their allegiances}.\footnote{{\em Ibid}., p. 298.} In this political world gone meaningless, in which detached intellectual work is still possible, the detached scholar soars so high above human activity that he can no longer distinguish men from things. The lines get blurred, and what had once been political programs and strategies of action now become commodities on a nineteenth century Smithian market ruled by an invisible hand; what was once called the {\em politics of truth in a democratically responsible society} is now seen as big units competing with small fry on a political market where competition leads to concentration and results in the formation if duopolies, monopolies and cartels: {\em If the rival creed cannot be liquidated and is itself not strong enough to establish another} monopoly {\em in the symbol sphere, a ‘duopoly’ may arise. This is a situation of accommodation to a tolerant though} competitive co‑existence\unknown{} {\em Thus out of} competition {\em there occurs a move toward} concentration. {\em One or several} competitors {\em increasingly wins out, and the smaller units, eager to avail themselves of the prestige of the big winner, will jump on the band wagon. Symbol} cartels will {\em thus be formed\unknown{} Another general} mode of concentration {\em occurs by the alliance of a few big units for the more effective suppression of a number of small fry who are thus gobbled up}.\footnote{{\em Ibid}., pp. 289‑290. Italics added.}


Once the detached debunkers who wrote these lines are off the around, they stop at nothing. Even Mills’ early definition of political strategy is so restated that it can he reduced to the manipulative commodity propaganda of a public relations man. {\em Strategic choice of motive is part of the attempt to motivate the act for the} other {\em persons involved in our conduct,} which is here translated to mean that {\em We control another man by manipulating} the {\em premiums which the other accepts}.\footnote{{\em Ibid}., pp. 117, 118.}


The sale of motives on a strategy market does not, however, explain historical change. To explain that, the professors return to Max Weber, and the fourth part of the book, on {\em Dynamics,} deals with {\em The Sociology of Leadership}. This part repeats and elaborates the cynical comments of Mills’ first article with Professor Gerth. The detached professors, one of whom is said to have benefited from contact with the academic wisdom of the other, are once again passive spectators of a familiar drama, the Nazi “revolution,” which has now become, for them, the archetype of all historical change and a synonym for revolution. It is {\em convenient to grasp the psychological and ideological aspects of revolutionary movements by focusing upon their definition of historical time and reality and upon their conception of freedom \unknown{} A keen sense of a new unheard‑of mission inspires the charismatic leader and his followers\unknown{} Optimism, of a previously unheard‑of surge, lifts up the followers of the charismatic leader. With eyes fixed on the distant yet foreshortened goal, they move ahead with the certainty of the sleepwalker, often immunized against the costs of blood, self‑sacrifice and terror which the deliberate destruction of the old entails\unknown{} These experiences of time and reality dovetail with those of the freedom which is to come through detachment in action. Freedom means liberation, and with the increasing size and power of the charismatic following, freedom is felt to increase. For freedom is seen and felt to be a sharing in the expanding movement of the leader. The enthusiasm of the faithful follower is experienced as essential to freedom.}\footnote{{\em Ibid}., p. 445 and p. 447.} As for the outcome of such a “revolution”: the professors restate their conclusion to the article they wrote a decade earlier, clarifying it for those who had not understood its implications the first time: {\em Revolution involves a turnover in personnel; but such turnover is not by itself a revolution. A circulation of elites is not enough; there must also be a restructuring of a system of domination and authority.}\footnote{{\em Ibid}., p. 442.}


{\em Character and Social Structure} may be seen as an {\em index of} a {\em society coming apart.} Neither a cure nor a diagnosis, it is itself symptom of an age when sensitive minds {\em experience stress and strain. } It is a sign of {\em times of distress,} of a {\em state of normlessness,} written by passive spectators of an erupting volcano who do not know or would rather not know that the eruption they’re watching is not natural but social, and that human motion, including their own, is what creates the and maintains the flames. {\em Then occurs in intellectual circles trial and error, criticism and countercriticism, self‑searching and doubt, skepticism and enlightenment, desperate attempts to revive and to reaffirm what proves in the end to be outlived and hollow. Words and deeds fail to jibe, and boredom overcomes many who feel weary of uninspiring days.}\footnote{{\em Ibid}, p. 430.}


\section{Rejection of Crackpot Realism and Academic Incoherence
}

Mills spent the rest of the {\em mindless years,} from 1953 to 1958, recovering from his desperate attempt to revive and to reaffirm what proved in the end to be outlived and hollow. As if to dissociate himself once again from the normless, detached, cynical Spirit that floats above a world of masses shifting enthusiastically under the wands of charismatic leaders, Mills wrote an introduction to the work of a man who was the very antithesis of Max Weber, a man who would have dismissed {\em The Sociology of Leadership} as a second rate mid‑nineteenth century farce, a man who, according to Mills, is nevertheless {\em the best social scientist America has produced,}\footnote{Mills, {\em Images of Man: The Classic Tradition in Sociological Thinking} (anthology with introduction), New York: George Braziller, Inc., 1960, p. 13.} Thorstein Veblen. {\em He was a masterless, recalcitrant man, and if we must group him somewhere in the American scene, it is with those most recalcitrant Americans, the Wobblies. On the edges of the higher learning, Veblen tried to live like a Wobbly. It was a strange place for such an attempt. The Wobblies were not learned, but they were, like Veblen, masterless men, and the only non‑middle class movement of revolt in twentieth‑century America.}\footnote{Introduction to the Mentor edition of Thorstein Veblen, {\em The Theory of the Leisure Class}, New York: New American Library, 1953, p. ix.}


The trip into academic incoherence, or rather the journey to a Paradise where man could be seen through the eyes of God, was an interruption of Mills’ development, but not the end of the road; the trip left deep scars, but it did not stunt him. Mills was, after all, a masterless, recalcitrant man, at times almost a sort of intellectual Wobbly. He seems to have been two different men, and it is significant that the longest quotation from Veblen’s works which he chose for his introduction says, “{\em The current situation in America is by way of being something of a psychiatrical clinic\unknown{} Perhaps the commonest and plainest evidence of this unbalanced mentality is to be seen in a certain fearsome and feverish credulity with which a large proportion of the Americans are affected.}\footnote{{\em Ibid}., p. viii.}


Credulity is a state of delusion; it represents a rift between thought and action. The behavior of a credulous person lacks coherence: he cannot act in terms of what he thinks, and his thoughts are not related to anything he does. It did not take Mills long to remember that his life goal had not been to become a detached inmate in a psychiatric ward; it did not take him long to begin to break loose. He tried to get to the heart of the {\em absence of mind in politics}, the failure of nerve and {\em conservative mood} which had dropped over people like a drugged sleep {\em The psychological heart of this mood is a feeling of powerlessness—but with the old edge taken off, for it is a mood of acceptance and of a relaxation of the political will. The intellectual core of the groping for conservatism is a giving up of the central goal of the secular impulse in the West: the control through reason of man’s fate.}\footnote{“The Conservative Mood,” {\em Dissent}, Vol. 1, No. 1 (Winter, 1954), in {\em Power, Politics and People}, p. 208.} In what seems like a desperate attempt to revive the early framework which had once served as a starting point, Mills returns, in 1954, to what he had called {\em Dewey’s style of liberalism} in his doctoral dissertation. {\em Men in masses have troubles although they are not always aware of their true meaning and source. Men in publics confront issues, and they are aware of their terms. It is the task of the liberal institution, as of the liberally educated man, continually to translate troubles into issues and issues into the terms of their human meaning for the individual.}\footnote{{\em Mass Society and Liberal Education}, Chicago: Center for the Study of Liberal Education for Adults, 1954, pamphlet republished in {\em Power, Politics and People}, p. 370.} The following year, 1955, he reintroduces into the center of his work the ideals he had tried to translate into projects in 1948. {\em Among these values none has been held higher than the grand role of reason in civilization and in the lives of its civilized members. And none has been more sullied and distorted by men of power in the mindless years we have been enduring. Given the caliber of the American elite, and the immorality of accomplishment in terms of which they are selected, perhaps we should have expected this. But political intellectuals too have been giving up the old ideal of the public relevance of knowledge. Among them a conservative mood—a mood that is quite appropriate for men living in a political vacuum—has come to prevail.}\footnote{“On Knowledge and Power,” {\em Dissent}, Vol. 11, No. 3 (Summer, 1955), in {\em Power, Politics and People}, p. 599.} The same man who two years earlier had not opposed a passive, detached, “realistic” description of the state as it {\em monopolizes the use of legitimate violence within its domain,} now indignantly writes: {\em There is no opposition to public mindlessness in all its forms nor to all those forces and men that would further it. But above all—among the men of knowledge, there is little or no opposition to the divorce of knowledge from power, of sensibilities from men of power, no opposition to the divorce of mind from reality.}\footnote{{\em Ibid}., p. 604.} The reality which these {\em men of knowledge} accept without opposition is described in {\em The Power Elite. America—a conservative country without any conservative ideology—appears now before the world a naked and arbitrary power, as, in the name of realism, its men of decision enforce their often crackpot definitions upon world reality. The second‑rate mind is in command of the ponderously spoken platitude. In the liberal rhetoric, vagueness, and in the conservative mood, irrationality, are raised to principle. Public relations and the official secret, the trivializing, campaign and the terrible fact clumsily accomplished, are replacing the reasoned debate of political ideas in the privately incorporated economy, the military ascendancy, and the political vacuum of modem America.}\footnote{{\em The Power Elite}, New York: Oxford University Press, 1956, pp. 360‑361.}


Rejecting the divorce of mind from reality, Mills is able to distinguish the men from the masks, he can see the human beings who {\bf renounce} their humanity and {\bf alienate} their selves in {\em roles} instead of creating their own lives; he does not call it alienation, but he describes it as a dominant fact about everyday life in American society. {\em Today many people have to trivialize their true interests into ‘hobbies,’ which are socially considered as unserious pastimes rather than the center of their real existence. But only by a craftsmanlike style of life can the split domains of work and leisure become unified; and only by such self‑cultivation can the everyday life become a medium for genuine culture\unknown{} The mere chronological fact of more time on our hands is a necessary condition for the cultivation of individuality, but by no means guarantees it. As people have more time on their hands, most of it is taken away from them by the debilitating quality of their work, by the pace of their everyday routine, and by the ever‑present media of mass distraction.}\footnote{“The Unity of Work and Leisure,” {\em Journal of} {\em the National Association of Deans of} {\em Women}  (January, 1954), in {\em Power, Politics and People}, pp. 348‑349.} Mills continues to look for the vehicles between existence and consciousness, the media which guide men to find the aim of life in that tired frenzy by which we strive for the animated glee we call fun .\footnote{{\em The Sociological Imagination}, p. 348.}


His analysis of the mediators between consciousness and existence now has nothing in common with the {\em skill groups} that {\em specialize in telling us how we feel} or with the {\em symbol cartels} selling motives to shifting masses which he had seen from his vantage point on Mt. Olympus. {\em Public relations displace reasoned argument; manipulation and undebated decisions of power replace democratic authority. More and more, as administration has replaced politics, decisions of importance do not carry even the panoply of reasonable discussion} it? {\em public, but are made by God, by experts, and by men like Mr. Wilson\unknown{} The height of such mindless communications to masses, or what are thought to be masses, is the commercial propaganda for toothpaste and soap and cigarettes and automobiles.}\footnote{“On Knowledge and Power,” {\em loc. cit.}, p. 609.}


And when he looks at the intellectuals, Mills does not find them detached: {\em by the work they do not do they uphold the official definitions of reality, and, by the work they do, even elaborate it}.\footnote{{\em Ibid}., p. 612.} The colleagues to whom he devoted a portion of his life, especially those engaged in Scientific Sociology, do not fit Mills’ definition of masterless men. Many of them are engaged in {\em molecular work,} and {\em molecular work has no illustrious antecedents, but, by virtue of historical accident and the unfortunate facts of research finance, has been developed a great deal from studies of marketing and problems connected with media of mass communications}.\footnote{“Two Styles of Research in Current Social Studies,” {\em Philosophy of} {\em Science}, Vol. 20, No. 4 (October,1953), in {\em Power, Politics and People}, p. 554.} His own chosen “discipline” is split into two schools of {\em equally alienated men in whose hands the social studies become an elaborate method} of {\em insuring that no one learns too much about man and society, the first by formal but empty ingenuity; the second, by formal and cloudy obscurantism.} One group engages in {\em the large‑scale bureaucratic style of research into small‑scale problems},\footnote{“IBM Plus Reality Plus Humanism = Sociology,” {\em Saturday Review of Literature} (May 1, 1954), in {\em Power, Politics and People}, p. 570.} while the other group consists of Grand Theorists busy with a {\em seemingly arbitrary elaboration of distinctions which do not enlarge one’s understanding of recognizably human problems or experience}.\footnote{{\em Ibid}., p. 571.} Professors claiming to be detached adapt to the requirements of the dominant bureaucracies; their private interests just happen to coincide with the interests of men with money and power, so that their research is at once a contribution to Pure Science and the source of a comfortable income. These experts are in fact hired technicians and salesmen of knowledge, middlemen who derive their livelihood and status from transforming and processing the discoveries of science, philosophy and art for their employer and customer, the Power Elite, the warlords, corporate chieftains and political directorate of the United States.


The one‑time program of the right has become an accomplished fact, and the left which was to move upstream against the main drift has disappeared. With a renewed will to move against the main drift, Mills seems to have been left completely alone, scarred, but not mastered. He begins, once again, to locate himself in his social context, and thus also to locate his task. {\em What knowledge does to a man (in clarifying what he is, and setting it free)—that is the personal ideal of knowledge. What knowledge does to a civilization (in revealing its human meaning, and setting it free)—that is the social ideal of knowledge.}\footnote{“On Knowledge and Power,” {\em loc. cit}., p. 606.} Neither a charismatic nor a hereditary member of the Power Elite, and clearly neither a self‑sold nor a lucky new arrival, this recalcitrant man who was at times sort of an intellectual Wobbly, cannot find either personal or social meaning in the Higher Circles: {\em I certainly} {\em am not aware of any desire to be more like the rich in the sense that I} {\bf am} {\em sometimes aware of wanting to be more like some of the crack mechanics I know.}\footnote{{\em C. Wright Mills and The Power Elite}, compiled by G. William Domhoff and Hoyt B. Ballard, Boston: Beacon Press, 1968, p. 239.} He defines himself as {\em a third type of man, one whose work does have a distinct kind of political relevance: his politics, in the first instance, are the politics of truth, for his job is the maintenance of an adequate definition of reality. In so far as he is politically adroit, the main tenet of his politics is to find out as much of the truth as he can, and to tell it to the right people, at the right time, and in the right way\unknown{} The intellectual ought to be the moral conscience of society}\unknown{}\footnote{Mills, “On Knowledge and Power,” {\em loc. cit}., p. 611.} {\em This is the role of mind, of intellect, of reason, of ideas: to define reality adequately and in a publicly relevant way. The role of education\unknown{} is to build and sustain publics that will ‘go for,’ and develop, and live with, and act upon, adequate definitions of reality.}\footnote{{\em Mass Society and Liberal Education}, {\em loc. cit}., p. 373.}


However, the major work of this period, {\em The Power Elite}, was not {\em a politically motivated task;} it was {\em suggested by friends that I ought to round out a trilogy by writing a book on the upper classes\unknown{} And yet that is not ‘really’ how ‘the project’ arose; what really happened is (1) that the idea and the plan came out of my files, for all projects with me begin and end with them, and books are simply organized releases from the continuous work that goes into them\unknown{}}\footnote{{\em The Sociological Imagination}, p. 200.} The definition of reality which emerges from these files locates the enemy with a 500 watt glare. And Nazi is only one of his names. {\em The top of modem American society is increasingly unified, and often seems willfully co‑ordinated: at the top there has emerged an elite of power. The middle levels are a drifting set of stalemated, balancing forces: the middle does not link the bottom with the top. The bottom of this society is politically fragmented, and even as a passive fact increasingly powerless: at the bottom there is emerging a mass society}.\footnote{{\em The Power Elite}, p. 324.} However, this large analysis of possible futures which have turned into harsh realities, does not even cast the beam of a pocket flashlight on the alienation of activity, power and intellect, on the comprehensive renunciation of humanity which accounts for, but does not guarantee, the power of the elite. It does not proceed in terms of the struggle between the Power Elite and the alienated population, a struggle in which the right has temporarily realized its program; it does not create on awareness of the field of strategy still open to the left. In his critical introduction to Veblen, Mills even chastises Veblen for overlooking the “social functions” of upper class leisure, status and prestige, saying that leisure activities {\em are one way of securing a coordination of decision between various sections and elements of the upper class,} that status activities {\em provide a marriage market,} and that {\em prestige buttresses power.}\footnote{Introduction to Thorstein Veblen, {\em The Theory of the Leisure Class,} p. xvi.} Mills repeats this critique in {\em The Power Elite}.\footnote{{\em The Power Elite}, pp. 88‑89.} But he thereby completely obfuscates Veblen’s carefully drawn distinction between “social functions” which serve human life and those which stunt it. Indignation about the stunted development and pathological condition of the American population does not become analysis in Mills’ work. He continues to repeat what he already knew in 1948, namely that the “social functions” of the upper class are not going to be destroyed by labor leaders, that {\em the current crop of labor leaders is pretty well set up as a dependent variable in the main drift,}\footnote{“The Labor Leaders and the Power Elite,” {\em Roots of Industrial Conflict}, edited by Arthur Kornhauser, Robert Durbin and Arthur M. Ross. New York: McGraw Hill Book Company, 1954; in {\em Power, Politics and People}, p. 105.} and that {\em within the present framework of political economy \unknown{} unions are less levers for change of that general framework than they are instruments for more advantageous integration with it}.\footnote{{\em Ibid}., p. 108.} Mills ends an article with the statement that, {\em For the businessman, the politician, and the labor leader—each in curiously different ways—the more apathetic the members of their mass organizations\unknown{} the more operating power the leaders have as members of the national power elite.}\footnote{{\em Ibid}., p. 109.} But Mills does not go into the meaning of that apathy as a profound renunciation of self. He seems, rather, to take the apathy as an original datum, as the starting point for analysis, but not itself subject to analysis. As a result, he confines {\em historical change} to events which take place within the {\em higher circles,} and cannot focus on the potential initiative of the alienated, on historical change which consists of de‑alienation and consequently deals with the pathological condition, the unbalanced mentality with which a large portion of the Americans are affected.


Mills is aware of the gap between {\em the central goal\unknown{}: the control through reason of man’s fate,} and the actual condition of the American population. He no longer accepts that stunted condition as the full human stature of {\em the mass,} as a realization of self in the mask and the {\em role.} He writes that, {\em From almost any angle of vision that we might assume, when we look upon the community of publics, we realize that we have moved a considerable distance along the road to a mass society.}\footnote{{\em Mass Society and Liberal Education}, {\em loc. cit.,} p. 358.}


He realizes that the manipulated man of the “mass” is a human being who has alienated what is “inalienable,” his humanity. However, he seems to assume that the “social functions” which serve the Power Elite can guarantee and even deepen the transformation of {\em publics} into {\em masses},\footnote{{\em The Power Elite}, Chapter 13.} and as a result he does not regard the appropriation of the lost humanity as the road to historical change. He turns, instead, to Dewey’s style of liberalism, to “historical change” initiated at the top and by the top, to {\em men selected and formed by a civil service that is linked with the world of knowledge and sensibility}.\footnote{{\em Ibid}., p. 361.}


\chapter{Three:The Intellectual as Historical Agency of Change 1958‑1962
}

\section{The Showdown between Idiocy and Coherence
}

Mills locates the root of the unbalanced mentality, the cause of the intellectual deficiency of the complacent, in {\em the alienation of personal from political life,}\footnote{“The Complacent Young Men: Reasons for Anger,” {\em Anvil and Student Partisan}, Vol. IX, No. 1 (1958), in Power, Politics and People, p. 389.} in {\em the divorce of political reflection from cultural work}.\footnote{{\em Ibid.,} p. 390.} This separation creates a context in which {\em human development will continue to be trivialized, human sensibilities blunted, and the quality of life distorted and impoverished.}\footnote{“The Man in the Middle: The Designer,” {\em Industrial Design} (November, 1958), in {\em Power, Politics and People}, p. 386.} This trivialized, blunted, distorted and altogether private human being is an idiot {\em and I should not be surprised, although I do not know, if there were not some such idiots even in Germany. This—and I use the word with care—this spiritual condition seems to me the key to many modern troubles of political intellectuals, as well as the key to much political bewilderment in modern Society}.\footnote{“The Structure of Power in American Society,” {\em The British Journal of Sociology}, Vol. IX, No, 1 (March, 1958), in {\em Power, Politics and People}, p. 24.} The idiocy is characterized by {\em mute acceptance—or even unawareness—of moral atrocity; the lack of indignation when confronted with moral horror}.\footnote{{\em The Causes of World War Three}, New York: Simon \& Schuster, 1958, p. 77.} {\em Mills looks for historical} origins of this mental illness, and locates some of them in World War II, when {\em Man had become an object; and insofar as those to whom he was an object felt about the spectacle at all, they felt powerless, in the grip of larger forces, with no part in those affairs that lay beyond their immediate areas of daily demand and gratification. It was a time of moral somnambulance}.\footnote{{\em Ibid}., p. 78.} In {\em The Causes of World War Three}, Mills makes it lucidly clear that the enemy, whose name was Nazi during World War Two, was not defeated in 1945: {\em In the expanded world of mechanically vivified communication the individual becomes the spectator of everything but the human witness of nothing. Having no plain targets of revolt, men feel no moral springs of revolt The cold manner enters their souls and they are made private and blase\unknown{} It is not the number of victims or the degree of cruelty that is distinctive; it is the fact that the acts committed and the acts that nobody protests are split from the consciousness of men in an uncanny, even a schizophrenic, manner. The atrocities of our time are done by men as ‘functions’ of a social machinery—men possessed by an abstracted view that hides from them the human beings who are their victims and, as well their own humanity. They are inhuman acts because they are impersonal. They are not sadistic but merely businesslike; they are not aggressive but merely efficient; they are not emotional at all but technically clean cut.}


{\em This insensibility was made dramatic by the Nazis; but the same lack of human morality prevailed among fighter pilots in Korea, with their petroleum‑jelly broiling of children and women and men. And is not this lack raised to a higher and technically more adequate level among the brisk generals and gentle scientists who are now planning the weapons and the strategy of World War III?}\footnote{{\em Ibid}., pp. 78‑79.}


The schizophrenia of {\em the cheerful robot, of the technological idiot, of the crackpot realist,} all of whom {\em embody a common ethos: rationality without} reason\footnote{“T he Complacent Young Men,” {\em loc. cit}., p. 393.} is contrasted by Mills with {\em the ethos of craftsmanship\unknown{} as the central experience of the unalienated human being and the very root of free human development}.\footnote{“The Man in the Middle,” {\em loc. cit}., p. 386.} Craftsmanship is characterized by a {\em unity of design, production and enjoyment.}\footnote{{\em Ibid}., p. 383.} As soon as this unity is destroyed, as soon as these activities become separate masks which “compose” a person, and separate roles which “compose” a social structure, the individual loses coherence and the society lacks reason. This cleavage or rupture, this split between thought, action and feeling, creates a rift, {\em a great cultural vacuum, and it is this vacuum that the mass distributor, and his artistic and intellectual satrap, have filled up with frenzy and trash and fraud.}\footnote{{\em Ibid}., pp.383‑384.} Just like profiteers and capitalist doctors who manage to extort enormous personal gain from war and illness, the cultural middlemen—professional designers, advertisers and propagandists, hired professors, scientists and artists—have managed to extort enormous personal gain from schizo­phrenia. {\em The world men are going to believe they understand is now, in this cultural apparatus, being defined and built, made into a slogan, a story, a diagram, a release, a dream, a fact, a blue‑print, a tune, a sketch, a formula; and presented to them.  Such part as reason may have in human affairs, this apparatus, this put‑together contraption, fulfills; such role as sensibility may play in the human drama, it enacts; such use as technique may have in history and in biography, it provides\unknown{} In the USA the cultural apparatus is established commercially: it is part of an ascendant capitalist economy. This fact is the major key to understanding both the quality of everyday life and the situation of culture in America today.}\footnote{{\em Ibid}., p. 377.}


Among the new profiteers, the cultural, artistic and scientific entrepreneurs, Mills’ colleagues are not a {\em set of Representative Men whose conduct and character are above the taint of the pecuniary morality, and who constitute models for American imitation and aspiration}. In {\em The Causes of World War Three} Mills observes that {\em Most cultural workmen are fighting a cold war in which they echo and elaborate the confusions of officialdoms\unknown{} They have generally become the Swiss Guard of the power elite—Russian or American}{\em {\bf ,}} {\em as the case happens to be. Unofficial spokesmen of the} {\em military metaphysic, they have not lifted the level of moral sensibility; they} {\em have further depressed it. They have not tried to put responsible content into the political vacuum; they have helped to empty it and to keep it empty.}\footnote{{\em The Causes of World War Three}, p. 85.} {\em \unknown{} many, perhaps in fear of being thought Unpatriotic, become nationalist propagandists; others, perhaps in fear of being thought Unscientific, become nationalist technicians.}\footnote{{\em Ibid}., p. 7.}


The first step away from social schizophrenia is to unite one’s split self, or at least to define the conditions for one’s own coherence. Mills tries to define these conditions by referring to the model of the craftsman, whose mind and body are both his own, whose thought and action are inseparable components of projects which consist of intelligent practical activity. {\em In craftsmanship, plan and performance and are unified, and in both, the craftsman is master of the activity and of himself in the process. The craftsman is free to begin his working according to his own plan, and during the work he is free to modify its} {\em shape and the manner of its shaping. The continual joining of plan and performance brings even more firmly together the consummation of work and its instrumental activities, infusing the latter with the joy of the former. Work is a rational sphere of independent action\unknown{} Since he works freely, the craftsman is able to learn from his work, to develop as well as use his capacities. His work is thus a means of developing himself as a man as well as developing his skill.}\footnote{“The Man in the Middle,” {\em loc. cit}., pp. 384‑385.} In political activity, this type of craftsmanly coherence, this unity of plan and performance, requires a definition of reality which sheds light on available courses of action and on obstacles which prevent or block their realization. {\em The less adequate one’s definitions of reality and the less apt one’s program for changing it, the more complex does the scene of action appear. ‘Complexity’ is not inherent in any phenomena; it is relative to the conceptions with which we approach reality. It is the task of those who want peace to identify causes and to clarify them to the point of action.}\footnote{{\em The Causes of World War Three}, p. 82.} However, even though Mills refers to the model of craftsmanship, he does not suggest that social critique is “constructive thinking” in the sense that it finds solutions to the problems of the ruling class, since {\em then we} {\em are foolishly trapped by the difficulties those now at the top have got us into. They do not want us to identify} {\bf their} {\em difficulties as theirs; they want us to think of} {\bf their} {\em difficulties as if these were everybody’s. That is what they call ‘constructive thinking about public problems.’ To be constructive in their sense is merely to stick our heads further into their sack. So many of us have already stuck our heads in there that our first job is to pull them out and look around again for genuine alternatives. In this sense it must be said: the first job of the intellectuals today is to be consistently and altogether unconstructive. For to be constructive within the going scheme of affairs is to consent to the continuation of precisely what we ought to be against.} \footnote{{\em Ibid}., p. 137.} What Mills prophetically called for was a confrontation between idiocy and coherence, a showdown between the fully developed human being and the cheerful robot, technological idiot, crackpot realist, a destruction of the {\em rationality without reason} which degrades and deranges the modern human being: {\em that is the real, even the ultimate, showdown on ‘socialism’ in our time. For it is a showdown on what kinds of human beings and what kinds of culture are going to become the models of the immediate future, the commanding models of human aspiration \unknown{} To make that showdown clear, as it affects every region of the world and every intimate recess of the self requires a union of political reflection and cultural sensibility of a sort not really} {\em known before.}\footnote{“The Complacent Young Men,” {\em loc. cit}., p. 393.}


Mills lucidly defined a large goal, and shortly after his premature death a new left began to take concrete steps toward its realization in every region of the world; even a new American left began to move upstream against the main drift. However, in order to define the available courses of action and the obstacles on the way, Mills himself had to struggle against the frenzy and trash and fraud which had been stuffed into his mind and file by academic bureaucrats and their hired and scared satraps. In this struggle, he had to spend vast amounts of energy to reach a level of coherence and clarity which he had already reached in 1948.


In {\em The Causes of World War Three,} his analysis of the collective self‑alienation, the daily activity which reproduces the Power Elite, does not go beyond insights into the apathy of the population and the powerlessness of’ intellectuals developed a decade earlier. {\em They are allowed to occupy such positions, and to use them in accordance with crackpot realism, because of the powerlessness, the apathy, the insensibility of publics and masses; they are able to do so, in part, because of the inactionary posture of intellectuals, scientists, and other cultural workmen}.\footnote{{\em The Causes of World War Three}, p. 89.} Mills does not regard the daily self‑renunciation as a practical activity (the sale of one’s productive power for a wage) or even as an intelligent practical activity (the sale of one’s mental skills for a salary or a grant), but as a passive condition (apathy, powerlessness, insensitivity). As a result, he is unable to give meaning to a phrase which he believes to be profoundly true but which he cannot substantiate, namely that {\em men are free to make history}.\footnote{{\em Ibid}., p. 14.} The years devoted to Max Weber and Professor Gerth now drive him to repress Rousseau, the Wobblies, Veblen, Marx, and his own experience, and keep him from asking how and why men make power elites through their daily acts of self‑alienation. Mills compulsively repeats: {\em elites of power make history}.\footnote{{\em Ibid.}}


This definition of reality does not adequately clarify how reality can be changed. If the elites of power make history, then the elites of power change history, and the very possibility of changing the reality dominated by {\em The Power Elite} is excluded by definition. Mills’ attempt to emerge from this paradox created by his training is less pathetic than silly. This mid‑twentieth century radical in his early forties is able to write, {\em What man of God can claim to partake of the Holy Spirit, to know the life of Jesus, to grasp the meaning of that Sunday phrase ‘the brotherhood of man’—and yet sanction the insensibility, the immorality, the spiritual irresponsibility of the Caesars of our time?}\footnote{{\em Ibid}., p. 125.} {\em The same man who raises the goal of unifying plan and performance seeks to implement his plan by appealing to the very men who profiteer from the rift between plan and performance, the culture salesmen, the creators of weapons, the makers of images, the perpetrators of religion, the trivializers of knowledge}.\footnote{{\em Ibid}., Part Four.} It is to these men that Mills says, {\em if we are to act as public intellectuals, we must realize ourselves as an independent and oppositional group. Each of us, in brief, ought to act as if he were a political party.}\footnote{{\em Ibid}., p. 135.} It is to the men who specialize in adapting men to what they have become in the modern United States that Mills writes about {\em a show­down on all the modern expectations about what man can} {\bf want} {\em to become.}\footnote{{\em Ibid.,} p. 172.} Mills appeals to the {\em symbol experts,} the fragmented men who occupy {\em the freest places in which to work} precisely because of their fragmentation, as if they were coherent craftsmen, yet he knows that it {\em is the absence of such a stratum of cultural workmen in close interplay with such a participating public, that is the signal fault of the American cultural scene today.}\footnote{“The Man in the Middle,” {\em loc. cit}., p. 386.} Mills’ dilemma deepens: not only are the cultural workmen who could define it strategy of change absent; there is, in addition, {\em no real public for such programs}.\footnote{{\em The Causes of World War Three}, p. 93.} In the absence of both, Mills calls on scientists, priests and professors to tell the Power Elite what they are doing to the United States. {\em To those with power and awareness of it, we must publicly impute varying measures of responsibility for such consequences as we find by our work to be decisively influenced by their actions and defaults.} \footnote{{\em Ibid}., p. 132.} Mills then questions the point of doing this. {\em Any such public role for the intellectual workman makes sense only on the assumption that the decisions and the defaults of designatable circles are now history‑making; for only then can the inference be drawn that the ideas and the knowledge—and also the morality and the character—of these higher circles are immediately relevant to the human events we are witnessing}.\footnote{{\em Ibid}., p. 133.} But even the Mills influenced by Max Weber is a recalcitrant man, and he calls on inexistent {\em cultural workmen} and on profiteering culture experts to change history by changing the ideas of the Power Elite. {\em I am contending that the ideology and the lack of ideology of the powerful have become quite relevant to history‑making, and that therefore it is politically relevant for intellectuals to examine it, to argue about it, and to propose new terms for the world encounter}.\footnote{{\em Ibid}.} But this position takes the sometime “radical” too far, and he backs up. {\em To} {\bf appeal} {\em to the powerful, on the basis of any knowledge we now have, is utopian in the silly sense of that term.}\footnote{{\em Ibid.}} Yet, finding no other alternative in his file, {\em We must accept what perhaps used to be the utopian way} \unknown{}\footnote{{\em Ibid}., p. 93.}


In 1958, Mills had not achieved a unity of plan and performance. Looking for {\em a properly developing society\unknown{} built around craftsmanship} {\em as the central experience of the unalienated human being and the very root of free human development,}\footnote{“The Man in the Middle,” {\em loc. cit}., p. 386.} he convinced himself that {\em It is now sociologically realistic, morally fair, and politically imperative to make demands upon men of power and to hold them responsible for specific courses of events.}\footnote{{\em The Causes of World War Three}, p. 95.} And C. Wright Mills pulled his head out of the sand in an isolated spot of the U.S. desert, and he shouted guidelines and conditions which included “demands” ranging from {\em a senior civil service firmly linked to the world of knowledge and sensibility} to a complete dismantling of the corporate‑military structure of the United States.\footnote{{\em Ibid}., pp. 118‑121.}


\section{The Whole Man as Promethean History‑Maker
}

In 1959 Mills writes, {\em I do not know the answer to the question of political irresponsibility in our time or to the cultural and political question of The Cheerful Robot}.\footnote{“Culture and Politics: The Fourth Epoch,” {\em The Listener}, Vol. LXI, No. 1563 (March 12, 1959), in {\em Power, Politics and People}, p. 246.} Yet he tries, once again, to locate himself in the midst of impotent spectators, apolitical idiots, expert apologists, sophisticated escapists, detached complacents; he tries, once again, to find an exit from a world of rationality without reason. He finds spectacular symbols which embody precisely the opposite traits from those of his friends, his colleagues, his contemporaries. To the clerk with a title, the fragment of a vast project whose sense he cannot grasp, the incapacitated expert, Mills opposes the fully developed man, the man for whom nothing human is alien. {\em The values involved in the cultural problem of freedom and individuality are conveniently embodied in all that is suggested by the ideal of the Renaissance Man.}\footnote{{\em Ibid}., p. 245.}  To the helpless spectator, the political non‑man who watches human life from a distance, the servant who considers himself free the very moment he’s bought, Mills opposes the man who creates his own environment, the man who steals his self‑powers whenever they’re not at his free disposal, the man who bows neither to Zeus nor any master. {\em The values involved in the political problem of history‑making are embodied in the Promethean ideal of its human making.}\footnote{{\em Ibid}.} For Mills, the fully developed man is not a passive spectator engaged in contemplating all that is human, nor is the creative man a detached intellectual whose {\em spirit} creates freely. Both are aspects of a practical man whose coherence does not reside in the comprehensive rationality of his grand theory, but in the unity between his thought and his action. They are symbols of practical‑critical activity, revolutionary activity; they are the two aspects of craftsmanship, {\em the central experience of the unalienated human being and the very root of free human development.} In the previous year’s article, Mills had written, {\em Craftsmanship cannot prevail without a properly developing society}.\footnote{“The Man in the Middle,” {\em loc. cit}., p. 386.} In the article on Renaissance Man and Prometheus, he adds that a properly developing society is one in which men deliberately develop their lives to a level which corresponds to the available instruments, namely a society in a permanent state of revolution. {\em In a} {\bf Properly Developing Society}, {\em one might suppose that deliberately cultivated styles of life would be central; decisions about standards of living would be made in terms of debated choices among such styles; the industrial equipment of such a society would be maintained as an instrument to increase the range of choice among styles of life.}\footnote{“Culture and Politics: The Fourth Epoch,” {\em loc. cit}., p. 240.}


In his next major work, Mills tries to put these precepts into practice. {\em The Sociological Imagination} is a work about craftsmanship. It is the work of a fully developed twentieth century man attempting to link his practical activity to the history of his time. It is an attempt to join thought and action, to unite power with sensibility, to be coherent and not just to think rationally. Mills brings the problem into focus by turning his attention to those nearest to him who are under the impression that they practice a craft, the sociology professors. He exposes them as professional escapists, obfuscators and bureaucrats. Mills again turns to the two dominant schools of social “scientists.” The first rationally constructs a society where abstractions (“values,” “order”) relate to each other as in a medieval Great Chain of Being. In the grand schema of Talcott Parsons, main representative of this school, {\em the idea of conflict cannot effectively be formulated. Structural antagonisms, large‑scale revolts, revolutions—they cannot be imagined. In fact, it is assumed that ‘the system,’ once established, is not only stable but intrinsically harmonious; disturbances must, in his language, be ‘introduced into the system.’ \unknown{} The magical elimination of conflict, and the wondrous achievement of harmony, remove from this ‘systematic’ and ‘general’ theory the possibilities of dealing with social change. With history. Not only does the ‘collective behavior of terrorized masses and excited mobs, crowds and movements—with which our era is so filled—find no place in the normatively created social structures of grand theorists. But any systematic ideas} of {\em how history itself occurs, of its mechanics and processes, are unavailable to grand theory\unknown{}}\footnote{{\em The Sociological Imagination}, p. 42.} The “scientific” practice of the second school is as old as the scribes and tax collectors of the Pharaoh, the bureaucrats hired to gather data which the monarch needs to administer his empire. {\em In so far as such research efforts are effective in their declared practical aims, they serve to increase the efficiency and the reputation—and to that extent, the prevalence of bureaucratic forms of domination in modem society. But whether or not effective in these explicit aims (the question is open), they do serve to spread the ethos of bureaucracy into other spheres of cultural, moral and intellectual life.} Mills notes that it is precisely the men whose work serves administration and repression who claim to be morally neutral, to make no value judgments in their work. {\em It might seem ironic that precisely the people most urgently concerned to develop morally antiseptic methods are among those most deeply engaged in ‘applied social science’ and ‘human engineering.’}\footnote{{\em Ibid}., p. 101.} The result is that professors become administrative technicians, agents of the ruling bureaucracies. {\em Their positions change—from the academic to the bureaucratic; their publics change—from movements of reformers to circles of decision‑makers; and their problems change—from those of their own choice to those of their new clients. The scholars themselves tend to become less intellectually insurgent and more administratively practical.}


{\em Generally accepting the status quo, they tend to formulate problems out of the troubles and issues that administrators believe they face. They study\unknown{} workers who are restless and without morale, and managers who ‘do not understand’ the art of managing human relations. They also diligently serve the commercial and corporate ends of the communications and advertising industries.}\footnote{{\em Ibid}., p. 96.}


It is into this world of hired clerks and servants of repression that Mills sticks his ideal of the intellectual craftsman, the fully developed human being whose knowledge is the basis for changing the world. The projects of such a man are chosen in terms of their contribution to the quality of life, not in terms of their contribution to his personal career. The quality and content of available styles of life among which he can choose are displayed to him by the daily activities of his contemporaries; his ability to see a possible self in the lives of others, an ability acquired by a child when he becomes aware of himself as a choice‑making individual, is what Mills calls the sociological imagination. {\em The first fruit of this imagination—and the first lesson of the social science that embodies it—is the idea that the individual can understand his own experience and gauge his own fate only by locating himself within his period, that he can know his own chances in life only by becoming aware of those of all individuals in his circumstances.} This understanding leads to the awareness that the constraints to his own development are not rooted in his deficiencies, but in the accepted daily activities of others, and with this awareness he is able to translate personal uneasiness into social troubles and public issues. {\em By such means the personal uneasiness of individuals is focused upon explicit troubles and the indifference of publics is transformed into involvement with public issues}.\footnote{{\em Ibid}., p. 5.} Aware of the connection between personal constraints and social activities, the individual learns that the collective transformation of the structure of social activity is the condition for his own liberation. {\em He understands that what he thinks and feels to be personal troubles are very often also problems shared by others, and more importantly, not capable of solution by any one individual but only by modifications of the structure of the groups in which he lives and sometimes the structure of the entire society. Men in masses have troubles, but they are not usually aware of their true meaning and source; men in publics confront issues, and they usually come to be aware of their public terms}.\footnote{{\em Ibid.,} p. 187.} Mills has said much of this before, but in 1959 he is impatient to exit from the {\em Society full of private people in a state} {\em of public lethargy.} In a speech delivered over the Canadian Broadcasting Company ({\em The Big City: Private Troubles and Public Issues}) he is very clear about the connection between people’s daily activities and the shape of their social environment: {\em We must realize}, {\em in} a {\em word}, that we {\bf need} {\em not drift blindly; that we can} {\em take matters into our own hands};\footnote{‘The Big City: Private Troubles and Public Issues” (Speech over the Canadian Broadcasting Company) in {\em Power, Politics} {\em and People,} p. 399.} he ends the speech with the statement, {\em Let us begin this} {\em here and now.}\footnote{{\em Ibid}., p. 402.}


Yet in spite of the lucidity with which he points to the connection between people’s personal constraints and their daily activities, Mills does not begin here and now by cleaning out his files; he leaves matters in the hands of the {\em Power Elite}.\footnote{{\em The Sociological Imagination}, pp. 182‑183.} Consequently, Mills does not translate private troubles into public issues; he does not link his own activity with the daily activities of the underlying population; he does not formulate strategies which can lead to modifications of the structure of the entire society. According to his files elites make history, and consequently Mills addresses himself to the people characterized by Veblen as “the noble and the priestly classes, together with much of their retinue,”\footnote{Thorstein Veblen, {\em The Theory of the Leisure Class}, New York: New American Library, p. 21.} the {\em intellectuals, artists, ministers, scholars, and scientists.}\footnote{Mills, {\em The Sociological Imagination}, p. 183.} Mills himself had called them the {\em Swiss Guard of the Power Elite}{\bf ,} yet he calls on these fragmentary men whose social positions rest on their service to power to annihilate their own “roles,” their “persons,” by becoming Renaissance Men and Promethean history‑makers; Mills calls on Carpetbaggers to overthrow the slave system of the South. He justifies his choice of these profiteering middlemen as a historical agency of change on the grounds that no {\em other group, just now is as strategically placed for} {\em possible innovation as those whose work joins them to the cultural apparatus; to the means of information and knowledge; to} {\em the means by which realities} are {\em defined, by which programs and politics are elaborated and presented to publics}.\footnote{“The Decline of the Left,” {\em The Listener}, Vol. LXI, No. 1566 (April 2, 1959), in {\em Power, Politics and People}, pp. 231‑232.} Mills further justifies his choice by adding that, {\em I do not believe, for example, that it is only ‘Labor’ or ‘The Working Class’ that can transform American society and change its role in world affairs\unknown{} I, for one, do not believe in abstract social forces—such as The Working Class—as} {\bf the} {\em universal historical agent.}\footnote{{\em Ibid}., p. 232.} In other words, it is profiteers who are chosen as historical agents of change; furthermore, it is not because they are manipulated that the ideological middlemen are to struggle for liberation, but because they manipulate. This appeal to the consciences of fragmentary men who live off the scraps of social power they receive in exchange for faithful service to the ruling class has nothing in common with Mills’ definition of {\em the unalienated human being.}


\section{The Intellectual as Revolutionary
}

In 1960, {\em The Fourth Epoch}\footnote{from the title of “Culture and Politics: The Fourth Epoch,” {\em loc. cit.}} suddenly begins; fully developed human beings take matters into their own hands and start to make history {\em here and now.}


{\em Isn’t all this, isn’t it something of what we are trying to mean by the phrase, ‘The New Left?’ Let the old men ask sourly, ‘Out of Apathy—into what?’ The Age of Complacency is ending. Let the old women complain wisely about ‘the end of ideology.’ We are beginning to move again}.\footnote{“Letter to the New Left,” {\em loc. cit}., p. 259.} Yankee “intellectuals” continue to do what they’ve been doing: {\em They see the good, they see the bad, the yes, the no, the maybe—and they cannot take a stand. So instead they take up a tone. But they are never in it; they are just spectators. And as spectators they are condescending, with such little reason to be\unknown{}}\footnote{{\em Listen, Yankee: The Revolution in Cuba}, New York: McGraw Hill Book Company, 1960, p. 150.} They continue to be “detached” while serving power. But they no longer matter. {\em In the showdown these days such people are just no good—for the hungry world.}\footnote{{\em Ibid}.} While they were busy intimidating the powerless with the enormity of the spectacle, while they accumulated career and status by serving the bureaucracy, {\em A man said No! to a monster\unknown{} And then he began to see it; The only real politics possible for honest men in the old Cuba was the politics of the gun, the politics of the guerrilla. The revolution was the only ‘politics’ for an honest man.}\footnote{{\em Ibid.}, p. 40.} The human stature of this refusal and this struggle is in sharp contrast to {\em the weariness of many NATO intellectuals with what they call ‘ideology,’ and their proclamation of ‘the end of ideology.’ The end‑of‑ideology is in reality the ideology of an ending; the ending of political reflection itself as a public fact. It is a weary know‑it‑all justification—by tone of voice rather than by explicit argument—of the cultural and political default of the NATO intellectuals.}\footnote{“ Letter to the New Left,” {\em loc cit}., pp. 247‑249.} The elaborate verbal schemas of the experts who serve corporate and military bureaucracies are destroyed by practical activity, because {\em The revolution\unknown{} smashes whatever is mere artifice.}\footnote{{\em Listen, Yankee}, p. 133.} Revolutionary practice, practical‑critical activity, is the test of {\em the politics of truth,} the test of the adequacy of one’s definition of reality:


{\em The revolution is a way of defining reality.}


{\em The revolution is a way of changing reality—and so of changing the definitions of it.}


{\em The revolution in Cuba is a great moment of truth.}\footnote{{\em Ibid}., p. 114.}


This is why the activity of the intellectual craftsman who unifies plan and performance, the practice of the {\em Renaissance Man} who is also a Promethean history‑maker, is revolutionary practice. In revolutionary activity, self‑changing and the changing of circumstances are part of the same process, the creation of the fully developed individual and of the property developing society. {\em So it is only, we think, in a revolutionary epoch that intellectuals can do their real work, and it is only by intellectual effort that revolutionaries can be truly successful}.\footnote{{\em Ibid}., p. 133.}


{\em The same year that he wrote about the revolutionary moment of truth which} changed reality {\em and so changed the definitions of it,} Mills published another book, on {\em The Classic Tradition in Sociological Thinking.}\footnote{{\em Images of Man: The Classic Tradition in Sociological Thinking} (anthology with introduction), New York: George Braziller, Inc., 1960.} This book is not Mills’ attempt to {\em begin here and now.} It is a return to his files, where all {\em projects with me begin and end}.\footnote{{\em The Sociological Imagination}, p. 200.} The Cuban revolution did not smash whatever was mere artifice in Mills’ files. Mills’ “classics” are not the men who defined reality in ways that clarified possible strategies for revolutionary change. They are the men who shaped Mills’ definition of reality—or rather his definitions of realities, since the book contains the intellectual ancestry of both men who wrote under the name of C. Wright Mills. Here Karl {\em Marx and Max Weber\unknown{} , stand up above the rest},\footnote{{\em Images of Man}, p. 12; following quotation on page 13.} and Veblen, {\em the best social scientist America has produced, who probably\unknown{} was at heart an anarchist and syndicalist},\footnote{{\em The Marxists}, New York; Dell Publishing Company, 1962, p. 35.} stands awkwardly next to, or slightly behind, the father of {\em The Sociology of Leadership}. Rousseau is conspicuously absent among The Classics. The man who rebelled against the fact that “Man is born free, yet everywhere he’s in chains,”\footnote{Rousseau, {\em Du Contrat Social}, p. 50.} is replaced by a man who takes this fact for granted: {\em In all societies­ from societies that are very meagerly developed and have barely attained the dawnings of civilization, down to the most advanced and powerful societies—two classes of people appear—a class that rules and a class that is ruled,}\footnote{Gaetano Mosca, “The Ruling Class,” in Mills, {\em Images of Man}, p. 192.} and by another man for whom the separation between masses and elites is the basic characteristic of social life: {\em So we get two strata in a population: (1) A lower stratum, the} {\bf non‑elite}, with {\em whose possible influence on government we are not just here concerned; then (2) a higher stratum, the} {\bf elite}, {\em which is divided into two: (a) a governing} {\bf elite} {\em (b) a non‑governing} {\bf elite}.\footnote{Vilfredo Pareto, “Elites, Force and Governments,” in {\em Ibid}., p. 264.} The {\em image of man} defined by revolutionary practice is obscured by {\em images of} man which make it impossible to define revolutionary {\em practice.} Mills the independent revolutionary continues to coexist with Mills the academic cynic, even though he is at pains to find justifications for this peaceful coexistence: {\em Back} {\em in the American thirties, there was quite a craze for Pareto\unknown{} I have never understood why, unless it was some kind of attempted antidote to Marxism which was so fashionable at the time. Pareto’s is one of the tougher, even cynical, styles of thought; he seems to relish this posture for its own sake, although he disguises it,/ imagine, by supposing it to be an essential part of Science. Of course it is nothing of the sort. As a whole, I find his work pretentious, dull and disorderly. Yet if one digs hard, one does find useful reflections.}\footnote{Mills in {\em Ibid}., p. 14.}


This is the year when {\em Mills comes face to face with the most important issue of political reflection—and of political action—in our time: the problem of the historical agency of change, of the social and institutional means of structural change.}\footnote{“Letter to the New Left,” {\em loc. cit}., p. 254.} But instead of dealing with the problem in terms of the living experience of revolutionary practice, he pulls dead arguments out of old files. He repeats earlier observations about the {\em collapse} of {\em our historical agencies of change}\footnote{{\em Ibid}., p. 255.} (by which he means trans‑historical Levers which he never believed in, and which therefore could not {\em collapse} for him), and then he states, unambiguously, {\em It is with this problem of agency in mind that I have been studying, for several years now, the cultural apparatus, the intellectuals—as a possible, immediate, radical agency of change.}\footnote{{\em Ibid}., p. 256.} To document his thesis, he lists the activities of students all over the world,\footnote{in {\em Listen Yankee}, pp. 33‑34, and in “Letter to the New Left,” {\em loc. cit}., pp. 257‑259.} and in his book on the Cuban revolution he underlines the fact that {\em The revolution was incubated at the university}\footnote{{\em Listen Yankee}, p. 39.} and that {\em its leaders have been young intellectuals and students from the University of Havana.}\footnote{{\em Ibid}., p. 46.} However, Mills’ documentation is not a proof of his thesis, but an apology for it. Neither the Cuban revolutionaries nor the revolutionary students around the world have anything in common with the {\em intellectuals, artists, ministers, scholars and scientists\unknown{} fighting a cold war in which they echo and elaborate the confusions of officialdoms.}\footnote{{\em The Sociological Imagination}, p. 183.} The young revolutionaries are clearly not the people who are {\em strategically placed for possible innovation as those whose work joins them to the cultural apparatus; to the means of information and knowledge; to the means by which realities are defined, by which programs and politics are elaborated and presented to publics}\footnote{“The Decline of the Left.” {\em loc. cit}., pp. 231‑232.}; the struggling students are the victims of these people, the ones who are manipulated by them.


Mills’ inability to distinguish the bureaucratic agent of repression from his victim does not prove that the Classic Sociology {\em helps one to understand what is happening in the world,} nor that {\em its relevance to the life‑ways of the individual and to the ways of history‑making in our epoch is obvious and immediate.}\footnote{{\em Images of Man}, pp. 16‑17.} Concerned with documenting the role of intellectuals as a revolutionary agency of change and with applying the Sociology of Leadership, Mills does not apply his own analysis of the social function of the university, nor his own analyses of the Leading Roles of academics, to explain why {\em the university was the cradle of the revolutionary ideas,} nor why {\em the politics made there were the politics of revolt and insurgency, of rebellion—­the politics of the revolution}.\footnote{{\em Listen Yankee}, p. 39.} Mills mentions the fact that the Cuban peasants {\em are the people our learned young men joined up with, and mobilized, to make our revolution. Know that we//: these people are the base, the thrust, the power. It is from them that the rebel soldiers came. They are the revolutionaries.}\footnote{{\em Ibid}., p. 45.} He is also aware that the liberation of one individual requires a collective transformation of {\em the structure of the entire society} because his problems are {\em not capable of solution by any one individual\unknown{}}\footnote{{\em The Sociological Imagination}, p. 187.} Furthermore, Mills already knew twenty years earlier that such an individual is able to formulate a political strategy, namely motives for action which appeal to others.\footnote{“Situated Actions and Vocabularies of Motive” (1940), {\em loc. cit}., p. 443.} Yet he does not, in any of his last works, ask about the relationship between the radical individual and the individuals with whom he communicates. Years of interrupted development have closed this question for Mills; it is replaced by a question given to Mills by his intellectual benefactors; the question is,



\startblockquote
{\em Who the hell’s yer leader anyhow?}


{\bf Who’s yer leader?}



\stopblockquote
Mills poses this question “in spite of himself,” or rather, because of an uncritical acceptance of an {\em image of man} based on a separation of men into leaders and led, elites and masses. But he is not comfortable in this framework; he is incoherent: there is a rift between his theory and his practice; his definition of reality does not guide his activity. His single critique of the New Cuban government is: {\em I do not like such dependence upon one man as exists in Cuba today, nor the virtually absolute power that this one man possesses.}\footnote{{\em Listen, Yankee}, p. 182.} In spite of rigid influences which pulled him in the opposite direction, Mills tried to remain a masterless, recalcitrant man, a sort of intellectual Wobbly. {\em What knowledge does to a man (in clarifying what he is, and setting it free)—that is the personal ideal of knowledge\unknown{}}\footnote{“On Knowledge and Power,” {\em loc. cit}., p. 606.} For Mills it remained a personal {\bf ideal}. What he was is perhaps clarified by the suggestion he puts into the words of the Cuban speaker in {\em Listen Yankee: We Cuban revolutionaries don’t really know just exactly} {\bf how} {\em you could best go about this transforming of your Yankee imperialism. For us, with our problems, it was simple: In Cuba, we had to take to our ‘Rocky Mountains’‑—you couldn’t do that, could you? Not yet, we suppose. (We’re joking—we suppose. But if in ten years, in five years—if things go as we think they might inside your country, if it comes to that, then know this, Yankee: some of us will be with you. God almighty, those are great mountains!)}\footnote{{\em Listen, Yankee}, p. 166.} Mills’ knowledge did not set him free for the struggle; it locked him up in a conceptual framework without exit. Unable to think of himself as a leader precisely because he could not accept the “role” of a follower, his knowledge did not inform him that man—all men, not ‘elites’—can make history. Unable to take on the Yankee imperialism by himself, he looked around for eleven companeros to take up the politics of the guerilla. {\em the only ‘politics’ for an honest man,} but what peered back was the cold stare of the scared employee, the hostile indifference of the only {\em agents of change} he had found among the Yankees, the {\em intellectuals, artists, ministers, scholars and scientists.}


{\em And you Yankees are a vigorous people, or at least once upon a time you were.}


{\em If you’d just forget the money—Mother of God, haven’t YOU already enough?}


{\em If you’d just abandon the fear—aren’t you strong enough to?}


{\em If you’d just stop being so altogether private and become public men and women of the world—you could do great things in the World.}\footnote{{\em Ibid}., p. 167.}


\section{An Ambiguous Retreat and an Incomplete Task
}

Bent by several men, Mills bowed to no man. If he sometimes admired the independence and self‑determination of {\em elites,} he felt nothing but contempt for official keepers of seals, and among the keepers, he singled out NATO intellectuals and Stalinists for his greatest contempt. He kept far away from the Talmudic scholars, the high priests and the grand executioners for whom Marx was a Prophet who wrote numerous testaments of a new Bible. And to keep his distance from them, Mills kept his distance from Marx as well. Consequently, when he turns to Marx in his last book, he does not “use Marx” as an occasion for rethinking questions he has not been able to answer, or at times even to pose. He keeps his distance. As a result, he does not read Marx in the clear light of fresh and living revolutionary experience, but through the obscure veil of stale files and dead arguments. Mills’ last book is not a final struggle for coherence; it is not a confrontation between incompatible, never‑synthesized elements which pulled him in opposite directions. It is a retreat from this confrontation. Mills’ {\em The Marxists,} published two years after {\em Listen Yankee,} does not show that, for Mills, {\em the revolution is a way of}


{\em defining reality,} nor that, for Mills, {\em the revolution in Cuba is a great moment of truth.} The Cuban revolution, and the beginning of student rebellions all over the world, stimulated Mills, not to change his definitions of‑reality, but to append revolutionary experience to {\em The Classic Tradition in} {\em Sociological Thinking}. By storing Marx and the Marxists in the Hall of Classical Fame Mills enlarged his menagerie of {\em images of man}; he did not emerge with a coherent synthesis of his own.


For eighteen years, from his attempt to characterize {\em The Powerless People,} through his analysis of {\em White Collar} to his essay on craftsmanship, Mills tried, at times successfully, to deal with the alienation of the individual’s power over his circumstances as a fact about social life in capitalist society. Yet in his last work he reduces the problem of alienation to {\em The question of the attitude of men toward the work they do\unknown{}} \footnote{{\em The Marxists}, p. 112.} He reduces alienation to {\em psychic exploitation,} and using this definition he adds, {\em alienation does not necessarily, or even usually, result in revolutionary impulses. On the contrary, often it seems more likely to be accompanied by political apathy than by insurgency of either the left or righ}t.\footnote{{\em Ibid}., pp. 112‑113.} This superficial definition is a public‑relations man’s concept of {\em alienation:} it means disaffection with the dominant {\em symbols,} and can be remedied by changing the {\em image} with mass circulation newspapers, television, and expensive advertising; if this campaign does not succeed in transforming disaffection to happy acceptance, it can at least channel it into political apathy and thus avoid insurgency. This is not, however, the way Mills had defined alienation in {\em White Collar. When white‑collar people get jobs, they sell not only their time and energy but their personalities as well.} The white collar man sells his creative power and his gestures no matter what attitude he has toward the work he does. {\em Self‑alienation is thus an accompaniment of his alienated labor.}\footnote{{\em White Collar}, p. xvii.} In other words, the salesgirl at Macy’s sells (alienates, separates herself from) her time, energy and gestures even if she enjoys selling herself and thinks she’s Supergirl or Elizabeth Taylor. Public relations men are hired to change her attitude toward her work, and they sometimes succeed, but she remains alienated, because the alienation is a fact about her social situation and not about her {\em image} of it. Mills must have thrown away his file cards for {\em White Collar,} or perhaps he wrote that work before he had developed his files. In either case, the trivial conception of alienation presented in {\em The Marxists} is unrelated to the ideas developed {\em in White Collar.} It is related to the textbook Mills wrote with Gerth nine years before {\em The Marxists.} It was in that book that alienation was treated as a psychic phenomenon, as a concept which did not refer to man’s daily life but to the {\em symbol sphere,} the {\em image} of life. It was there that Mills agreed to put his name over a description of a public relations world where {\em detachment is a step towards alienation}, a world where, {\em In the scholar’s study or the agitator’s den the symbols which legitimate various kinds of political systems may be, rearranged, debunked, or elaborated\unknown{} For changes in the legitimating symbols to be realized, masses of people must shift their allegiances.}\footnote{Mills and Gerth, {\em Character and Social Structure}, p. 298.}


In his {\em Celebration of Marx},\footnote{Chapter 2 of {\em The Marxists}; the quotation which follows is from page 36.} Mills says that Marx’s {\em structural view of a total society results from a classic sociological technique of thinking. With its aid Marx translated the abstract conceptions of con­temporary political economy into the concrete terms of the social relations of men.} However, rejecting even his own structural view of alienation, or forgetting his own characterization of the alienation of living power, time, and gesture which accompany the sale of one’s labor, Mills cannot emerge with a structural view of the total society even in this last work where he directly confronts Marx’s structural view. Having reduced alienation to {\em an attitude,} Mills is unable to relate the state or the corporation or the military to people’s daily activities, he cannot see these “forces” as concentrations of the alienated self‑powers of a population. He need not have taken his clue from {\em Marx}; he could have taken it from Rousseau. In his twenty year long struggle to find the roots of the powerlessness, the private idiocy, Mills might have traced the process through which the voluntarily alienated powers of people become transformed into economic, political and military “institutions.” But Mills retreats from such an analysis once again; he again backs into the textbook he wrote with Gerth. Instead of reducing the “institutions” to the daily activities of people, the daily routines through which they alienate their powers, Mills retreats to the {\em institutional orders} which stand, {\em sui generis,} as structures separate from the activities of daily life which create and reproduce them. Each {\em institutional order} contains decision‑making elites and passive masses; the {\em higher circles} of these {\em orders} are the ones who {\em make history}; since a population’s alienation of energy, mind and time is not seen by Mills as voluntary activity, but as a state of mind, an attitude, these {\em masses} do nothing voluntarily, they do not make history, they simply shift under the symbols dangled before them by the {\em intellectuals} who serve the {\em higher circles.} On the basis of this definition of reality, Mills states that {\em it follows that our conception of the higher circles in capitalist society must be seen as more complex than the rather simple ‘ruling class’ of Marx, and especially later marxists.}\footnote{{\em The Marxists}, p. 118.} And especially {\em later marxists.} Mills seems to have forgotten that the {\em later marxists} who apparently “interpreted” Marx for him in the 1930s, and against whom he reacted for the rest of his life, turned up among the noisiest {\em NATO} {\em intellectuals} of the 1950s. Yet Mills continues to respond to the stale {\em Marx} of the {\em later marxists} with the stale arguments in his files, and finally he evades the problem of alienation altogether by stating flatly that the problem is to define {\em the state, with Max Weber, simply as an organization that ‘monopolizes legitimate violence over a given territory.}’\footnote{{\em Ibid}., p. 119.} With this statement, Mills chooses to keep a bureaucratic conception of reality: society consists of three separate hierarchies, which are not themselves explained in terms of people’s activities; they are {\em defined,} and as definitions they are the starting point for analysis: people’s activities are explained in terms of the hierarchies.


Mills insists on {\em the principle of historical specificity},\footnote{{\em Ibid}., p. 38.} although it has little meaning in his conception. He cannot study the historical forms of concentration of people’s alienated powers, the historical forms of social activity. His framework reduces him to the study of historical successions of institutional orders; historical problems are reduced to questions about the {\em supremacy} of one or another {\em order,} and his {\em more complex} analysis consists of nothing more than the observation that the {\em economic order} is not always {\em supreme}.\footnote{{\em Ibid}., pp. 116‑126.}


With this definition of reality, Mills cannot come {\em face to face with the most important issue of political reflection—and of political action—of our time: the problem of the historical agency of change, of the social and institutional means of structural change.} If Mills does not see that people create their {\em institutions} through their daily activities, then he cannot see how they can change the social system by changing their daily activities, and a historical agency of change must be {\em introduced into the system.} In other words, Mills’ {\em historical agency is} an abstraction which is separate from people’s daily activities; it is some kind of mechanical lever generated by a social machine, and at some point in history the lever automatically destroys the machine. Since such a lever has not overthrown the West European or American capitalist machine, Mills concludes that the agency collapsed. {\em The trends supposed to facilitate the development and the role of the agency have not generally come off—and when they have occurred, episodically and in part, they have not led to the results expected.}\footnote{{\em Ibid}., p. 128.} In other words, Mills defines an entity which cannot exist, projects an event which cannot take place, and then concludes that the entity collapsed because the event did not take place.


In order to prove that the agency which {\em collapsed} was {\em Marx’s agency,} Mills has to prove that Marx had such a conception of an agency. To prove this, Mills has to disprove much of what he learned from Marx. Mills’ often‑repeated proposition that men make their own history within given though transformable material circumstances, comes from Marx. \footnote{{\em Ibid}., p. 122. Mills gives a fuller statement of this view in the last paragraph of {\em Character and Social Structure}, and also in the first seven chapters of {\em The Causes of World War Three}.} However, in order to attribute the theory of the mechanical lever to Marx, Mills has to show that, for Marx, men do not make their own history; that history is inevitable. But in all of the vast tomes of Marx’s writings, stretching over half a century of creative activity, Mills could not find a conclusive statement to that effect by Marx. Mills is too intellectually honest to yank out of context a statement which proves Marx said something which is denied by what precedes and follows it. Consequently, in order to prove that history is inevitable according to {\em Marx,} Mills quotes a statement about the inevitability of history written by {\bf Engels}.\footnote{{\em The Marxists}, p. 91.} But this method of proof is not so honest either, since Engels is Marx and Marx, Engels only for the “marxologists” of the Marx­Engels Institute in Moscow, and not even for all of them. The fact is that men do not make history according to the theory Mills derived from Max Weber; it is because of the influence of Weber that {\em we must construct another model in which events may be understood in closer and in more conscious relation to the decisions and lack of decisions of powerful elites, political and military as well as economic.}\footnote{{\em Ibid}., p. 122.} It is this theory which keeps Mills from seeing that the decisions and lack of decisions of underlying populations create the power of the elites, and consequently that the decisions of these people can also abolish the power of the elites and thus change history. (Mills even suggests that it is the {\em higher circles} of the Soviet bureaucracy who might institute {\em socialism}.\footnote{{\em Ibid}., p. 474.}) Since Mills does not regard the alienation of people’s self‑powers as a daily activity but as a {\em psychic} condition, he cannot regard the de‑alienation of these powers as revolutionary activity but merely as another {\em psychic} condition. In other words, people are doomed to eternal alienation. All that can change is the institutional form of alienation, the type of bureaucratic {\em orders} within which people perform their {\em roles.} And such change can take place either through the intervention of a mechanical lever, which collapsed, or through the morally inspired initiation of the very elites whose power is the inverse reflection of the powerlessness Mills struggled against for over two decades.


The last year of his life, Mills refers nostalgically to the {\em seemingly insignificant groups of scholarly insurgents in the nineteenth‑century capitals of Europe—a kind of man we do not know so well today\unknown{}}\footnote{{\em Ibid}., p. 27.} This is the kind of man Mills did not become when he chose the new and fascinating career chances which often involved opportunities to practice his skill rather freely.\footnote{{\em The New Men of Power}, p. 281.} And in the intervening years, Mills developed a definition of reality which failed to define what such a scholarly insurgent could possibly do: neither a mechanical agency of change, nor a member of the power elite, such an {\em insurgent} is reduced to the impotence of a passive spectator critically observing the moves of elites and the shifts of masses from the fringes of society. Mills’ nostalgia is not related to his theory; he has not achieved the unity of thought, action and feeling which characterizes his ideal of’ all intellectual craftsman. In terms of his theory one cannot imagine what these {\em scholarly insurgents} accomplished in the nineteenth‑century capitals of Europe. Perhaps because of’ the influence of his intellectual benefactors, or perhaps in order to justify his chosen career as professor, Mills has removed the very possibility of politically relevant action from such scholarly insurgents. Throughout his writings, a different Mills had crept into the margins, and at times to the very center of his work: a masterless man, a Promethean history‑maker. But in his last book, a book about revolution, the Promethean history‑maker is conspicuously absent; all that remains is the nostalgia. And even the nostalgia is no longer propped up with theoretical support: whatever might have supported it is beaten and removed from sight. Mills is at pains to remove the very possibility that the practical activity of an insurgent can lead to a transformation of his circumstances. In the process he has to deny much that he once knew. In {\em The Marxists} he returns to a problem which he had treated throughout his works, but which he never developed further than he had taken it in 1942 (in his dissertation), the problem of {\em strategy,} of {\em motives of action which appeal to others.} He uses different words in 1962. {\em This connection of ideal or goal with agency is at once amoral and an intellectual strategy.} But he immediately restates this proposition using the terms of the theory constructed during the intervening period: {\em This connection between built‑in agency and socialist ideal is the political pivot around which turn the decisive features of his} [Marx’s] {\em model of society and many specific theories of historical trend going on within it.}\footnote{{\em The Marxists}, p. 81.} And on the basis of this formulation, Mills reduces the motives for action into the {\em marxian doctrine of later marxists,} and he transforms the {\em others} into a {\em built‑in agency,} the mechanical lever which collapsed. In this context, an analysis of society which defines the conditions for social change, namely the required material instruments and the required knowledge, is not a statement about necessary conditions, but a prediction about the future. And in this context, an individual’s practical attempt to create some of the conditions, namely to provide the required knowledge, to define reality, to formulate a strategy and communicate it to others, is not practical activity at all; it is speculation about what is going to happen in the future, automatically, all by itself. In terms of these cynical, detached and apolitical criteria, Marx was not a committed scholarly insurgent trying to create those conditions for social change which were within his reach; he was a nineteenth century metaphysician who devoted fifty years to speculations, expectations and predictions about the {\em inevitable} future. {\em This being so, it must immediately be said that Marx’s major political expectation about advanced capitalist societies has collapsed., the central agency which he designates has not developed as expected; the role he expected that agency to enact has not been enacted.}\footnote{{\em Ibid}., p. 128.} In other words, if I state that in order to write an article I need certain materials and certain knowledge, I am not stating conditions but making predictions about my future; if I add that my goal is to write an essay on Mills, then this is not a commitment to a project but an expectation that in the face of the books, pen and paper, my mind and hand will mechanically write the essay. If for one of various reasons I fail to write it, then my expectations about myself have collapsed; the central agency which I designated for the task of writing the essay (my hand) has not developed as expected; the role which I expected my mind and hand to enact has not been enacted.


Perhaps because he stood alone for too long, perhaps because lie was recovering from his first heart attack, Mills the detached academic can now only imagine intellectual activity as detached academic activity. Gone is the intellectual craftsman as Promethean history‑maker. Gone is the intellectual architect who wrote, {\em We must realize, in a word, that we} {\bf need} {\em not drift blindly; that we can take matters into our own hands.}\footnote{“The Big City: Private Troubles and Public issues,” {\em loc. cit}., p. 399.} What is left is a detached academic who can merely interpret matters from a distance; who cannot define the conditions required for changing reality, but can only guess about the future; who cannot commit himself to political tasks, but can only have speculative expectations about what others are going to do. Mills’ anthology of Marxist writings contains a short selection which he has either never read or which he has forgotten; in any case, he makes no reference to its presence in the book despite the fact that it is a selection about intellectual craftsmanship, about Promethean history‑making, about the relationship between defining reality, self‑making and history‑making. Mills makes no reference to this selection despite the fact that it takes up questions he regarded as central during more than two decades, and despite the fact that it explicitly denies the main theses he tries to uphold in {\em The Marxists\unknown{} he does not understand human activity itself as} {\bf objective} {\em activity\unknown{} He therefore does not comprehend the significance of ‘revolutionary,’ or practical‑critical’ activity. The question whether objective truth is an attribute of human thought—is not a theoretical but a} {\bf practical} {\em question. Man must prove the truth, i.e. the reality and power, the ‘this‑sidedness’ of his thinking in practice. The dispute over the reality or non‑reality of thinking that is isolated from practice is a purely} {\bf scholastic} {\em question\unknown{} The coincidence of the changing of circumstances and of human activity or self‑changing can only be comprehended and rationally understood as} {\bf revolutionary practice}\unknown{} {\em All social life is essentially} {\bf practical}. All {\em the mysteries which urge theory into mysticism find their rational solution in human practice and in the comprehension of this practice\unknown{} The philosophers have only} {\bf interpreted} {\em the world differently, the point is, to} {\bf change} {\em it.}\footnote{Karl Marx, “Theses on Feuerbach,” in Mills, {\em The Marxists}, pp. 70‑71.} It was in the spirit of these statements that Mills had written, twenty years earlier, that Franz Neumann’s book on Nazi Germany will {\em move all of us into deeper levels of analysis and strategy. It had better. Behemoth is everywhere united.}



\startblockquote
{\bf Kalamazoo}


{\em April, 1969}



\stopblockquote







\page[yes]

%%%% backcover

\startmode[a4imposed,a4imposedbc,letterimposed,letterimposedbc,a5imposed,%
  a5imposedbc,halfletterimposed,halfletterimposedbc,quickimpose]
\alibraryflushpages
\stopmode

\page[blank]

\startalignment[middle]
{\tfa The Anarchist Library
\blank[small]
Anti-Copyright}
\blank[small]
\currentdate
\stopalignment

\blank[big]
\framed[frame=off,location=middle,width=\textwidth]
       {\externalfigure[logo][width=0.25\textwidth]}



\vfill
\setupindenting[no]
\setsmallbodyfont

\startalignment[middle,nothyphenated,nothanging,stretch]

\blank[line]
% \framed[frame=off,location=middle,width=\textwidth]
%       {\externalfigure[logo][width=0.25\textwidth]}


Fredy Perlman



The Incoherence of the Intellectual



C. Wright Mills’ Struggle to Unite Knowledge and Action




April 1969


\stopalignment
\blank[line]

\startalignment[hyphenated,middle]




Retrieved on December 22\high{nd}, 2013 from http://www.autodidactproject.org/other/perlman0.html


\stopalignment

\stoptext


