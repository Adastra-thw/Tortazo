% -*- mode: tex -*-
%%%%%%%%%%%%%%%%%%%%%%%%%%%%%%%%%%%%%%%%%%%%%%%%%%%%%%%%%%%%%%%%%%%%%%%%%%%%%%%%
%                                STANDARD                                      %
%%%%%%%%%%%%%%%%%%%%%%%%%%%%%%%%%%%%%%%%%%%%%%%%%%%%%%%%%%%%%%%%%%%%%%%%%%%%%%%%
\enabletrackers[fonts.missing]
\definefontfeature[default][default]
                  [protrusion=quality,
                    expansion=quality,
                    script=latn]
\setupalign[hz,hanging]
\setuptolerance[tolerant]
\setbreakpoints[compound]
\setupindenting[yes,1em]
\setupfootnotes[way=bychapter,align={hz,hanging}]
\setupbodyfont[modern] % this is a stinky workaround to load lmodern
\setupbodyfont[libertine,11pt]

\setuppagenumbering[alternative=singlesided,location={footer,middle}]
\setupcaptions[width=fit,align={hz,hanging},number=no]

\startmode[a4imposed,a4imposedbc,letterimposed,letterimposedbc,a5imposed,%
  a5imposedbc,halfletterimposed,halfletterimposedbc]
  \setuppagenumbering[alternative=doublesided]
\stopmode

\setupbodyfontenvironment[default][em=italic]


\setupheads[%
  sectionnumber=no,number=no,
  align=flushleft,
  align={flushleft,nothyphenated,verytolerant,stretch},
  indentnext=yes,
  tolerance=verytolerant]

\definehead[awikipart][chapter]

\setuphead[awikipart]
          [%
            number=no,
            footer=empty,
            style=\bfd,
            before={\blank[force,2*big]},
            align={middle,nothyphenated,verytolerant,stretch},
            after={\page[yes]}
          ]

% h3
\setuphead[chapter]
          [style=\bfc]

\setuphead[title]
          [style=\bfc]


% h4
\setuphead[section]
          [style=\bfb]

% h5
\setuphead[subsection]
          [style=\bfa]

% h6
\setuphead[subsubsection]
          [style=bold]


\setuplist[awikipart]
          [alternative=b,
            interaction=all,
            width=0mm,
            distance=0mm,
            before={\blank[medium]},
            after={\blank[small]},
            style=\bfa,
            criterium=all]
\setuplist[chapter]
          [alternative=c,
            interaction=all,
            width=1mm,
            before={\blank[small]},
            style=bold,
            criterium=all]
\setuplist[section]
          [alternative=c,
            interaction=all,
            width=1mm,
            style=\tf,
            criterium=all]
\setuplist[subsection]
          [alternative=c,
            interaction=all,
            width=8mm,
            distance=0mm,
            style=\tf,
            criterium=all]
\setuplist[subsubsection]
          [alternative=c,
            interaction=all,
            width=15mm,
            style=\tf,
            criterium=all]


% center

\definestartstop
  [awikicenter]
  [before={\blank[line]\startalignment[middle]},
   after={\stopalignment\blank[line]}]

% right

\definestartstop
  [awikiright]
  [before={\blank[line]\startalignment[flushright]},
   after={\stopalignment\blank[line]}]


% blockquote

\definestartstop
  [blockquote]
  [before={\blank[big]
    \setupnarrower[middle=1em]
    \startnarrower
    \setupindenting[no]
    \setupwhitespace[medium]},
  after={\stopnarrower
    \blank[big]}]

% verse

\definestartstop
  [awikiverse]
  [before={\blank[big]
      \setupnarrower[middle=2em]
      \startnarrower
      \startlines},
    after={\stoplines
      \stopnarrower
      \blank[big]}]

\definestartstop
  [awikibiblio]
  [before={%
      \blank[big]
      \setupnarrower[left=1em]
      \startnarrower[left]
        \setupindenting[yes,-1em,first]},
    after={\stopnarrower
      \blank[big]}]
                
% same as above, but with no spacing around
\definestartstop
  [awikiplay]
  [before={%
      \setupnarrower[left=1em]
      \startnarrower[left]
        \setupindenting[yes,-1em,first]},
    after={\stopnarrower}]



% interaction
% we start the interaction only if it's not an imposed format.
\startnotmode[a4imposed,a4imposedbc,letterimposed,letterimposedbc,a5imposed,%
  a5imposedbc,halfletterimposed,halfletterimposedbc]
  \setupinteraction[state=start,color=black,contrastcolor=black,style=bold]
  \placebookmarks[awikipart,chapter,section,subsection,subsubsection][force=yes]
  \setupinteractionscreen[option=bookmark]
\stopnotmode



\setupexternalfigures[%
  maxwidth=\textwidth,
  maxheight=\textheight,
  factor=fit]

\setupitemgroup[itemize][each][packed][indenting=no]

\definemakeup[titlepage][pagestate=start,doublesided=no]

%%%%%%%%%%%%%%%%%%%%%%%%%%%%%%%%%%%%%%%%%%%%%%%%%%%%%%%%%%%%%%%%%%%%%%%%%%%%%%%%
%                                IMPOSER                                       %
%%%%%%%%%%%%%%%%%%%%%%%%%%%%%%%%%%%%%%%%%%%%%%%%%%%%%%%%%%%%%%%%%%%%%%%%%%%%%%%%

\startusercode

function optimize_signature(pages,min,max)
   local minsignature = min or 40
   local maxsignature = max or 80
   local originalpages = pages

   -- here we want to be sure that the max and min are actual *4
   if (minsignature%4) ~= 0 then
      global.texio.write_nl('term and log', "The minsig you provided is not a multiple of 4, rounding up")
      minsignature = minsignature + (4 - (minsignature % 4))
   end
   if (maxsignature%4) ~= 0 then
      global.texio.write_nl('term and log', "The maxsig you provided is not a multiple of 4, rounding up")
      maxsignature = maxsignature + (4 - (maxsignature % 4))
   end
   global.assert((minsignature % 4) == 0, "I suppose something is wrong, not a n*4")
   global.assert((maxsignature % 4) == 0, "I suppose something is wrong, not a n*4")

   --set needed pages to and and signature to 0
   local neededpages, signature = 0,0

   -- this means that we have to work with n*4, if not, add them to
   -- needed pages 
   local modulo = pages % 4
   if modulo==0 then
      signature=pages
   else
      neededpages = 4 - modulo
   end

   -- add the needed pages to pages
   pages = pages + neededpages
   
   if ((minsignature == 0) or (maxsignature == 0)) then 
      signature = pages -- the whole text
   else
      -- give a try with the signature
      signature = find_signature(pages, maxsignature)
      
      -- if the pages, are more than the max signature, find the right one
      if pages>maxsignature then
	 while signature<minsignature do
	    pages = pages + 4
	    neededpages = 4 + neededpages
	    signature = find_signature(pages, maxsignature)
	    --         global.texio.write_nl('term and log', "Trying signature of " .. signature)
	 end
      end
      global.texio.write_nl('term and log', "Parameters:: maxsignature=" .. maxsignature ..
		   " minsignature=" .. minsignature)

   end
   global.texio.write_nl('term and log', "ImposerMessage:: Original pages: " .. originalpages .. "; " .. 
	 "Signature is " .. signature .. ", " ..
	 neededpages .. " pages are needed, " .. 
	 pages ..  " of output")
   -- let's do it
   tex.print("\\dorecurse{" .. neededpages .. "}{\\page[empty]}")

end

function find_signature(number, maxsignature)
   global.assert(number>3, "I can't find the signature for" .. number .. "pages")
   global.assert((number % 4) == 0, "I suppose something is wrong, not a n*4")
   local i = maxsignature
   while i>0 do
      -- global.texio.write_nl('term and log', "Trying " .. i  .. "for max of " .. maxsignature)
      if (number % i) == 0 then
	 return i
      end
      i = i - 4
   end
end

\stopusercode

\define[1]\fillthesignature{
  \usercode{optimize_signature(#1, 40, 80)}}


\define\alibraryflushpages{
  \page[yes] % reset the page
  \fillthesignature{\the\realpageno}
}


% various papers 
\definepapersize[halfletter][width=5.5in,height=8.5in]
\definepapersize[halfafour][width=148.5mm,height=210mm]
\definepapersize[quarterletter][width=4.25in,height=5.5in]
\definepapersize[halfafive][width=105mm,height=148mm]
\definepapersize[generic][width=210mm,height=279.4mm]

%% this is the default ``paper'' which should work with both letter and a4

\setuppapersize[generic][generic]
\setuplayout[%
  backspace=42mm,
  topspace=31mm,% 176 / 15
  height=195mm,%130mm,
  footer=9mm, %
  header=0pt, % no header
  width=126mm] % 10.5 x 11

\startmode[libertine]
  \usetypescript[libertine]
  \setupbodyfont[libertine,11pt]
\stopmode

\startmode[pagella]
  \setupbodyfont[pagella,11pt]
\stopmode

\startmode[antykwa]
  \setupbodyfont[antykwa-poltawskiego,11pt]
\stopmode

\startmode[iwona]
  \setupbodyfont[iwona-medium,11pt]
\stopmode

\startmode[helvetica]
  \setupbodyfont[heros,11pt]
\stopmode

\startmode[century]
  \setupbodyfont[schola,11pt]
\stopmode

\startmode[modern]
  \setupbodyfont[modern,11pt]
\stopmode

\startmode[charis]
  \setupbodyfont[charis,11pt]
\stopmode        

\startmode[mini]
  \setuppapersize[S33][S33] % 176 × 176 mm
  \setuplayout[%
    backspace=20pt,
    topspace=15pt,% 176 / 15
    height=280pt,%130mm,
    footer=20pt, %
    header=0pt, % no header
    width=260pt] % 10.5 x 11
\stopmode

% for the plain A4 and letter, we use the classic LaTeX dimensions
% from the article class
\startmode[a4]
  \setuppapersize[A4][A4]
  \setuplayout[%
    backspace=42mm,
    topspace=45mm,
    height=218mm,
    footer=10mm,
    header=0pt, % no header
    width=126mm]
\stopmode

\startmode[letter]
  \setuppapersize[letter][letter]
  \setuplayout[%
    backspace=44mm,
    topspace=46mm,
    height=199mm,
    footer=10mm,
    header=0pt, % no header
    width=126mm]
\stopmode


% A4 imposed (A5), with no bc

\startmode[a4imposed]
% DIV=15 148 × 210: these are meant not to have binding correction,
  % but just to play safe, let's say 1mm => 147x210
  \setuppapersize[halfafour][halfafour]
  \setuplayout[%
    backspace=10.8mm, % 146/15 = 9.8 + 1
    topspace=14mm, % 210/15 =  14
    height=182mm, % 14 x 12 + 14 of the footer
    footer=14mm, %
    header=0pt, % no header
    width=117.6mm] % 9.8 x 12
\stopmode

% A4 imposed (A5), with bc
\startmode[a4imposedbc]
  \setuppapersize[halfafour][halfafour]
  \setuplayout[% 14 mm was a bit too near to the spine, using the glue binding
    backspace=17.3mm,  % 140/15 + 8 =
    topspace=14mm, % 210/15 =  14
    height=182mm, % 14 x 12 + 14 of the footer
    footer=14mm, %
    header=0pt, % no header
    width=112mm] % 9.333 x 12
\stopmode


\startmode[letterimposedbc] % 139.7mm x 215.9 mm
  \setuppapersize[halfletter][halfletter]
  % DIV=15 8mm binding corr, => 132 x 216
  \setuplayout[%
    backspace=16.8mm, % 8.8 + 8
    topspace=14.4mm, % 216/15 =  14.4
    height=187.2mm, % 15.4 x 11 + 15 of the footer
    footer=14.4mm, %
    header=0pt, % no header
    width=105.6mm] % 8.8 x 12
\stopmode

\startmode[letterimposed] % 139.7mm x 215.9 mm
  \setuppapersize[halfletter][halfletter]
  % DIV=15, 1mm binding correction. => 138.7x215.9
  \setuplayout[%
    backspace=10.3mm, % 9.24 + 1
    topspace=14.4mm, % 216/15 =  14.4
    height=187.2mm, % 15.4 x 11 + 15 of the footer
    footer=14.4mm, %
    header=0pt, % no header
    width=111mm] % 9.24 x 12
\stopmode

%%% new formats for mini books
%%% \definepapersize[halfafive][width=105mm,height=148mm]

\startmode[a5imposed]
% DIV=12 105x148 : these are meant not to have binding correction,
  % but just to play safe, let's say 1mm => 104x148
  \setuppapersize[halfafive][halfafive]
  \setuplayout[%
    backspace=9.6mm,
    topspace=12.3mm,
    height=123.5mm, % 14 x 12 + 14 of the footer
    footer=12.3mm, %
    header=0pt, % no header
    width=78.8mm] % 9.8 x 12
\stopmode

% A5 imposed (A6), with bc
\startmode[a5imposedbc]
% DIV=12 105x148 : with binding correction,
  % let's say 8mm => 96x148
  \setuppapersize[halfafive][halfafive]
  \setuplayout[%
    backspace=16mm,
    topspace=12.3mm,
    height=123.5mm, % 14 x 12 + 14 of the footer
    footer=12.3mm, %
    header=0pt, % no header
    width=72mm] % 9.8 x 12
\stopmode

%%% \definepapersize[quarterletter][width=4.25in,height=5.5in]

% DIV=12 width=4.25in (108mm),height=5.5in (140mm) 
\startmode[halfletterimposed] % 107x140
  \setuppapersize[quarterletter][quarterletter]
  \setuplayout[%
    backspace=10mm,
    topspace=11.6mm,
    height=116mm,
    footer=11.6mm,
    header=0pt, % no header
    width=80mm] % 9.24 x 12
\stopmode

\startmode[halfletterimposedbc]
  \setuppapersize[quarterletter][quarterletter]
  \setuplayout[%
    backspace=15.4mm,
    topspace=11.6mm,
    height=116mm,
    footer=11.6mm,
    header=0pt, % no header
    width=76mm] % 9.24 x 12
\stopmode

\startmode[quickimpose]
  \setuppapersize[A5][A4,landscape]
  \setuparranging[2UP]
  \setuppagenumbering[alternative=doublesided]
  \setuplayout[% 14 mm was a bit too near to the spine, using the glue binding
    backspace=17.3mm,  % 140/15 + 8 =
    topspace=14mm, % 210/15 =  14
    height=182mm, % 14 x 12 + 14 of the footer
    footer=14mm, %
    header=0pt, % no header
    width=112mm] % 9.333 x 12
\stopmode

\startmode[tenpt]
  \setupbodyfont[10pt]
\stopmode

\startmode[twelvept]
  \setupbodyfont[12pt]
\stopmode

%%%%%%%%%%%%%%%%%%%%%%%%%%%%%%%%%%%%%%%%%%%%%%%%%%%%%%%%%%%%%%%%%%%%%%%%%%%%%%%%
%                            DOCUMENT BEGINS                                   %
%%%%%%%%%%%%%%%%%%%%%%%%%%%%%%%%%%%%%%%%%%%%%%%%%%%%%%%%%%%%%%%%%%%%%%%%%%%%%%%%


\mainlanguage[en]


\starttext

\starttitlepagemakeup
  \startalignment[middle,nothanging,nothyphenated,stretch]


  \switchtobodyfont[18pt] % author
  {\bf \em

Wayne Price  \par}
  \blank[2*big]
  \switchtobodyfont[24pt] % title
  {\bf

The “Negro March on Washington” movement in the World War II period

\par}
  \blank[big]
  \switchtobodyfont[20pt] % subtitle
  {\bf 

A radical contribution to African-American History Month

\par}
  \vfill
  \stopalignment
  \startalignment[middle,bottom,nothyphenated,stretch,nothanging]
  \switchtobodyfont[global]

January 2013

  \stopalignment
\stoptitlepagemakeup



\title{Contents}

\placelist[awikipart,chapter,section,subsection]



\page[yes,right]


\startblockquote
{\em For Black History Month: In 1941, the African-American labor leader, A. Philip Randolph made a call for a national demonstration by African-Americans in Washington, D.C. The demonstration never occurred, because of sabotage by White and Black liberals. But the organizing for it, and the movement behind it, continuing during World War II, had a big impact on U.S. Black people. It was an important influence preparing for the Civil Rights and Black Liberation struggles of the 1950s and ‘60s.}



\stopblockquote
---


In 1941, the African-American labor leader, A. Philip Randolph made a call for a national demonstration by African-Americans in Washington, D.C. The demonstration never occurred, because of sabotage by White and Black liberals. But the organizing for it, and the movement behind it, continuing during World War II, had a big impact on U.S. Black people. It was an important influence preparing for the Civil Rights and Black Liberation struggles of the 1950s and ‘60s.


At the time, the country was just beginning to come out of the Great Depression, due mainly to government spending to prepare for World War II. Manufacturing of armaments, aircraft, ships, and so on was booming and jobs were expanding. But the good jobs were not for African-Americans, who were at first not hired and later only hired for the most dead-end and menial jobs. Meanwhile relief and public assistance money was being cut back, on the grounds that unemployment was going down (as it was—for Whites).


The military services channeled Blacks into segregated army units, where they were mainly used as laborers. African-Americans could only join the navy as waiters and servants. Black units had White officers and a very few Black officers. No Black officer could command European-American troops. These policies continued all through the “War for Democracy and Freedom.” (Racial segregation in the military did not end until the Korean War.)


This was in the context of a government whose liberal leader, President Franklin D. Roosevelt, would not endorse an anti-lynching bill or an anti-poll tax bill in Congress. He valued his alliance with the segregationists, who then-dominated the Southern Democratic Party, above support for African-American rights—and lives.


U.S. Black people were aware that the Allies whom the U.S. was supporting (as “fighting for their freedom” against the racist Nazis) were the British empire, which ruled vast numbers of people of color in Africa and Asia, as well as the French empire, the Dutch empire, etc.


Not surprisingly, there was a lot of dissatisfaction among African-Americans with the U.S. government and limited support for the developing war. In 1940, one historian warned a Senate committee, “The morale of Negro citizens regarding ‘national defense’ is probably at the lowest ebb in the history of the country” (quoted in Barber; p. 112; Negro” was then used for African-Americans). This popular dissatisfaction continued after Pearl Harbor. A year later, a public opinion poll for an African-American newspaper asked Blacks, “Have you been convinced that the statements which our national leaders have made about freedom and equality for all people include the American Negro?” 82 percent answered “No” (Sugrue; p. 92). An example can be found in Malcolm X’s Autobiography. He describes how, when he was a hustler in Harlem during the war, he got out of being drafted by acting “crazy” (pp. 108—110).


Yet almost all Black organizations and prominent individuals fell in line behind the war. Many argued that Blacks should work in the war industries where they could and should serve in the military however they were assigned, without complaint or protest. This would supposedly prove that they were loyal, competent, and good American citizens. However, these same arguments had been used during World War I, and the aftermath of the war showed no improvement. Returning Black soldiers had been lynched for publically wearing “their country’s uniform.”


It was in response to antiwar feeling that some Black newspapers, for a while, raised the slogan, “Double V for Victory.” That is, victory against fascism abroad and victory against racism at home.


The founding president of the Brotherhood of Sleeping Car Porters, essentially a union of Black workers, was A. Philip Randolph, age 51. Once a militant member of the Socialist Party, he was the president of the National Negro Congress from 1936 to 1940. With other African-American leaders, Randolph, who had been a strong advocate of military aid to Britain, had tried to influence President Roosevelt to make changes. After all, the president had the legal power to outlaw racial discrimination in war industries and in the military with an executive order, at any time. But all they got were vague promises and assurances of goodwill.


\section{The Call for a “Negro March on Washington”
}

Under the pressure of the masses and his own conscience, Randolph decided that some action had to be taken. He formed a coalition with leaders of the NAACP and the National Urban League, and with other prominent African-Americans. They formed the Negroes’ Committee to March on Washington for Equal Participation in National Defense. On May 1, they issued the “Call to Negro America to March on Washington for Jobs and Equal Participation in National Defense on July 1, 1941.” Randolph wrote a press release, “I suggest that ten thousand Negroes march on Washington, D.C., the capital of the Nation, with the slogan, WE LOYAL NEGRO AMERICAN CITIZENS DEMAND OUR RIGHT TO WORK AND FIGHT FOR OUR COUNTRY” (Barber, p. 110; most of my background information comes from James et al., and Barber; also from the essays in Kruse \& Tuck).


Randolph called for mass action, “On to Washington, ten thousand black Americans! Let them swarm from every hamlet, village, and town; from the highways and byways, out of the churches, lodges, homes, schools, mills, mines, factories, and fields. Let them come in automobiles, buses, trains, trucks, and on foot. Let them come though the winds blow and rains beat against them, when the date is set” (quoted in James et al.; pp. 101--102).


The demonstration was to be all-Black. This was not an endorsement of “Black nationalism” as opposed to “integration.” Clearly, Randolph was advocating that African-Americans get their full rights as members of the U.S.A. But he felt that African-Americans should show that they could act on their own, without European-American leadership. Randolph wrote, “We shall not call upon our white friends to march with us. There are some things Negroes must do alone. This is our fight and we must see it through. If it costs money to finance a march on Washington, let Negroes pay for it.\unknown{}Let the masses speak!” (same, p. 102).


Supported by most of the major African-American organizations, the March on Washington Committee began with a network of organizers and potential marchers. Randolph’s union members, in the Brotherhood of Sleeping Car Porters, carried the word of the March to African-American communities across the country. Other mostly-Black unions and locals endorsed it. The NAACP changed the date of its annual convention, so members could come to the March. Black newspapers—of which there were nearly 150—spread information. (White-owned newspapers shut it out of the news.) In at least 19 cities, there developed local committees, including, Los Angeles, Chicago, Trenton, Milwaukee, Washington, Cleveland, Richmond, St. Louis, Atlanta, Savannah, St. Paul, and Jacksonville. By June, March organizers were predicting an estimated 100,000 marchers.


\section{The Pressure to Call It Off
}

This call for a Negro March on Washington was very upsetting to the ruling class. Of course, Southern segregationists denounced it. That was to be expected. But also the liberal Roosevelt administration put great pressure on the organizers to call it off. Roosevelt did not want the U.S. government to be exposed as a racist, oppressive, regime just as it was gearing up for entering a world war supposedly for “freedom and democracy.”


The capitalist class and its politicians, conservative and liberal alike, are not able to get rid of racism. They use the oppression of African-Americans and other people of color to divide the working class, to persuade White workers that they are superior to Black workers and should not cooperate with them in fighting the bosses. Also, racism creates a pool of workers who can be superexploited, paid less than the average for the rest of the working class, and worked harder. White supremacy raises the profits of the capitalist class. Even now, when legal segregation has been ended and there is a Black president, most African-Americans (and Latinos) remain at the bottom of society, the most oppressed and exploited parts of the working class.


By June 7, Roosevelt said that he was “much upset” about the March. He told his aides that they needed to “get it stopped” (quoted in Barber; p. 126). He held a meeting with Randolph and White (president of the NAACP), in which he declared his “firm and positive and definite oposition to the march” (in same; p. 130). He claimed to worry that the March could degenerate into chaos and violence.


At the meeting, the Secretary of State, Knox, asked Randolph, “Do you take the position that Negro and White sailors should be compelled to live together on ships?” (Of course, segregation “compelled” Black and White sailors to live separately.) When Randolph answered yes, Knox said, “In time of national defense, experiments of this kind cannot be carried out” (quoted in James et al.; p. 115).


Afterwards, prominent “friends of the Negro” were mobilized to pressure the leadership to cancel the March. Eleanor Roosevelt wrote to them, “I feel very strongly that your group is making a great mistake\unknown{}to allow this march to take place. I am afraid it will set back the progress which is being made\unknown{}.It may engender so much bitterness\unknown{}” (quoted in James et al.; p. 112).


Under the continuing threat of the mass march, Roosevelt issued Executive Order 8802. It denounced “discrimination in the employment of workers in defense industries or government because of race, creed, color, or national origin.” It obligated all new defense contracts to contain clauses “not to discriminate.” It established a “Fair Employment Practices Committee” to investigate complaints of discrimination and (vaguely) to “take appropriate steps to redress grievances” (quoted in James, et al.; p. 116).


Obviously, this order did nothing against discrimination in the military. Also, obviously, it had no enforcement mechanism. When discrimination in industry would be found by the FEPC, it could not send violators to jail nor could it cancel contracts.


On the surface, this toothless order appeared to acknowledge the government’s commitment to combat racial discrimination in war industries. Really the whole point was to prevent the March on Washington, and, in the future, to channel protest into


investigations and hearings. But quite a number of liberal Black leaders and newspapers praised the Executive Order as a great “victory.”


A few months later, the first chairman of the Committee, Mark Ethridge, a Southern White liberal, wrote a note, “Clearly we have accomplished what the president wanted: we paralyzed any idea of a march on Washington” (quoted in Barber, p. 134). In 1942, Etheridge, made a public statement, “Executive Order 8802 is a war order and not a social document\unknown{}.There is no power in the world—not even in all the mechanized armies of the earth, Allied or Axis—which could force the Southern white people to the abandonment of the principle of social segregation” (James et al.; pp. 191–2). This man had been appointed as chairman of the FEPC by Roosevelt.


In return for this very limited “victory,” barely better than nothing, Randolph cancelled the Negro March on Washington. He did this by himself, speaking only to a few other “leaders,” without any popular conference or discussion. He did try to cover his betrayal by advocating that local MOWM Committees stay intact, but this did not conteract his abandonment of immediate action.


Why did he do this? Randolph cared deeply about the oppression of African-Americans. Also, he knew that his position and influence was entirely based on the support he got from African-American workers and others. Yet he was essentially a liberal (or a very moderate social democrat). He did not want to go against the liberal New Deal administration of Roosevelt, let alone challenge the war effort. He was too tied to the system. Given the excuse of this mild Executive Order, he caved in to the pressure.


\section{The Struggle Continued
}

However, popular discontent continued. A meeting was called by the National Youth Committee of the March. It voted 44 to 1 to repudiate Randolph’s decision and to demand that the march go on. Since discrimination continued throughout industry and in the military, Black anger also continued. So did support for the idea of a national march on Washington.


In 1942, a Black newspaper polled ten thousand African-Americans, “Do you believe that India should contend for her rights and liberty now?” (that is, act against the U.S. ally Britain, even during the war). 88 percent answered yes. (Meanwhile Gandhi and Nehru were in British jails.) This says something about the state of African-American consciousness during the war.


In June 1942, a mass protest rally was held in New York City’s Madison Square Garden, called by the Negro March on Washington Movement. 25 thousand people jammed the site. Similar rallies were held in Chicago (16,000 attendees) and St. Louis (9,000). Meanwhile the FBI “investigated” whether the MOWM had received money from German or Japanese sources!


In the following September, 63 delegates from local March on Washington Committees met and voted to establish a permanent national organization, The language at the conference indicated that they did not place any confidence in Roosevelt’s promises, and that they were for mass action before the war was over. They insisted that the organization be democratic, so that one man could not cancel a march, although they still respected Randolph as a leader. A resolution, titled “Mass Action,” placed the MOWM “on record endorsing mass action, including marches on city halls, city councils, defense plants, public utility works, picketing and sending mass letters and telegrams to the President and congressmen and senators to stress the will and desires of the Negro people for the rights as American citizens” (quoted in James et al., p. 208). It called on Black people to join and respect labor unions and for unions to abandon racist practices.


In 1943, ten thousand Black and White workers held a massive demonstration against racism in Detroit. It was sponsored by the NAACP, the United Auto Workers, African-American fraternal associations, and others. It expressed the discontent which was also behind the MOWM, often by the same organizations which had endorsed it. Issues included police brutality, job discrimination in war plants, especially against Black women workers, refusal to serve African-Americans in Detroit restaurants, discrimination in housing, etc.


That year, the MOWM held a five-day national convention in Chicago. There were about 110 delegates from 36 communities. They adopted a constitution and by-laws. They worked out a program and reaffirmed their goal of a massive African-American March on Washington. The executive board was empowered to set a date. There was a sharp debate over a resolution which gave blanket support to the war. It passed over strong opposition. Randolph also raised the need for “nonviolent, goodwill, direct action.” Apparently some people advocated this as a localized, peaceful, alternative to mass action, while others emphasized more the civil disobedience (law-breaking) aspect, with its militant implications.


In 1945, Congress severely cut back the funding of the FEPC. It ended in 1946. While it had had the limited virtue of shining light on discrimination in war industries, it never made much impact in changing the situation.


Following the end of the war, the March on Washington Movement, dissolved as an organization in late 1947. Randolph and some others kept alive the issues of segregation in the military, until President Truman finally outlawed it. The struggle against racial segregation in industry, and in society in general, continued.


\section{The Role of the Left
}

Tendencies on the Left reacted to the MOWM according to their views on capitalism, the state, and the world war. As I described, liberals and social democrats (reform socialists) were “for” the March, but not so much that they would split with the Roosevelt administration and embarrass it in its war propaganda.


The largest Left organization then was the Communist Party. From 1934 to 1939 it was for a Popular Front (alliance with liberal capitalists). Then the Hitler-Stalin pact was made in 1939. It pledged peace between Germany and Russia, and incidently divided Poland between the two empires, setting off World War II in Europe. Following their new orders, the Communist Party in the U.S. suddenly became very “revolutionary” and pro-“peace” (blaming Britain for the war). This caused a split in the Negro National Congress, with Randolph leading the pro-Allied section and the Stalinists leading the antiwar forces. When the MOWM began, the CP Stalinists criticized it as too moderate.


But in 1941, Germany attacked the Soviet Union. The CP jumped way to the right. It became super-patriotic—not because it suddenly loved the U.S.A. but because the U.S. was Stalinist Russia’s ally. It denounced all strikes and industrial actions by workers, no matter what their grievances. It opposed any movements for national independence in India and the colonies of the Allied empires. It rejected civil liberties, cheering the jailing of leaders of the Trotskyists (which was done with laws which were to be used against the CP after the war).


And, of course, it abandoned support for African-American struggles. It capitulated to White Supremacy. Stalinists denounced the slogan, “Double V for Victory--for Democracy at Home and Abroad.” A writer for the CP’s Daily Worker countered, “Hitler is the main enemy\unknown{}The foes of Negro rights in this country should be considered as secondary” (quoted in James et al., p. 158). In 1945, the national leader of the CP, Earl Browder stated (in the Sunday Worker, March 4), that “it has been the studied policy of American Communists to refrain from public discussion” of racial oppression in the army, because they believed that the army officers were “soundly democratic” and “would move to modify and finally abolish” segregation without any “organized pressure” (in same; p. 344). With this delusional attitude (whether or not they really believed it), the Stalinists opposed the MOWM, at first subtly and then viciously. They ended up denouncing Randolph and other leaders as “traitors.”


The Communists had expected the World War to be followed by a peaceful alliance between a liberal U.S. government and the Soviet Union. They were surprised when, instead, a conservative Cold War atmosphere enveloped U.S. and international politics (a conservatism to which they had contributed by their superpatriotism).


Aside from the Communists, there were the Trotskyists of the Socialist Workers Party. Being opposed to the imperialist war and the capitalist state they were free to support the MOWM. Despite their small size and political limitations (support of the Soviet Union as a “workers’ state,” advocating a “dictatorship of the proletariat,” etc.) they participated in the movement to the best of their ability, in an honorable way.


So far as I can tell, anarchists did not participate in the MOWM. (However, C.L.R. James participated as part of the Trotskyist contingent. He was moving in a direction which took him out of Trotskyism into a politics very close to anarchism.) Anarchist views on the world war varied, including many who were totally against the war, either from revolutionary anti-imperialism or from pacifism. Others (such as Rudolph Rocker) were pro-Allies, in opposition to the horrors of German Nazism. Some took a position not unlike the African-American “Double V” slogan. The anarcho-syndicalist Sam Dolgoff wrote, “\unknown{}It was imperative that the war against fascism be regarded as a two-front war—defeat of fascism abroad by military victory and a relentless campaign” against reactionary forces at home (Dolgoff; p. 114).


It is not my purpose here to discuss World War II and a proper strategy toward it for revolutionary socialists and anarchists. However, there seems to be no reason anarchists would not have supported the MOWM except for their own limited size and resources. Especially, they lacked a specific organization of revolutionary anarchists, able to coordinate activity even among a small number. If they had been able to participate, they would have been able to strengthen the more militant wing of the movement.


\section{The Legacy
}

In his book about the fight to abolish slavery in the British empire, Adam Hochschild concludes: “As with any crusade, there was soon a struggle over just how the anti-slavery movement was to be remembered\unknown{}How, then, was it to be celebrated? As a historic, pioneering mobilization of public opinion, via boycotts, petitions, and great popular campaigns, all powerfully reinforced by the armed slave revolts? Or as a great gift to poor slaves by a group of pious, benevolent men? \unknown{} Not only were\unknown{}radicals\unknown{}and popular protests like the sugar boycott long slighted in British memory, so were the huge slave revolts, especially the final great rising in Jamaica that so clearly hastened the day of freedom” (pp. 349—351). This remains the question: whether to regard progress in freedom as due to benevolent liberal leaders, such as Roosevelt or even Randolph (or Obama, or Lenin), or whether to focus on the mass discontent, self-organization, and struggle.


In his introduction to James et al., Fred Stanton concludes, “The all-Black MOWM never achieved most of its aims, but it forced concessions from the government and had a progressive impact on the labor movement\unknown{}.It was certainly the most important Black movement since the heyday of Garveyism\unknown{}.It was the first large Black organization in which trade unionists played the leading role” (p. 21).


The Negro March on Washington Movement never had its march. It won some limited gains. But it raised the idea of organizing for mass action, of pressure against power rather than quiet pleading. It inspired future efforts, leading to the Civil Rights movement of the 1950s and the Black Liberation movement of the 1960s. These were to shake U.S. society and open the way for further struggles for a truly free and equal society.


- wayne price


{\em References}


Barber, Lucy G. (2002). Marching on Washington; The Forging of an American Political Tradition. Berkeley: University of California Press.


Dolgoff, Sam (1986). Fragments; A Memoir. Cambridge: Refract Publications.


Hochschild, Adam (2005). Bury the Chains; Prophets and Rebels in the Fight to Free an Empire’s Slaves. Boston/NY: Houghton-Mifflin.


James, C.L.R., Breitman, George, Keemer, Edgar, and others (1980). Fighting Racism in World War II. NY: Pathfinder.


Kruse, Kevin M., \& Tuck, Stephen (eds.) (2012). The Fog of War; The Second World War and the Civil Rights Movement. Oxford/NY: Oxford University Press.


Sugrue, Thomas (2012). “Hillburn, Hattiesburg, and Hitler; Wartime Activists Think Globally and Act Locally.” In Kruse \& Tuck, (eds.), pp. 87—102.


X, Malcolm (1999). The Autobiography of Malcolm X; As Told to Alex Haley. NY: Ballantine Books.


{\em *written for www.Anarkismo.net}









\page[yes]

%%%% backcover

\startmode[a4imposed,a4imposedbc,letterimposed,letterimposedbc,a5imposed,%
  a5imposedbc,halfletterimposed,halfletterimposedbc,quickimpose]
\alibraryflushpages
\stopmode

\page[blank]

\startalignment[middle]
{\tfa The Anarchist Library
\blank[small]
Anti-Copyright}
\blank[small]
\currentdate
\stopalignment

\blank[big]
\framed[frame=off,location=middle,width=\textwidth]
       {\externalfigure[logo][width=0.25\textwidth]}



\vfill
\setupindenting[no]
\setsmallbodyfont

\startalignment[middle,nothyphenated,nothanging,stretch]

\blank[line]
% \framed[frame=off,location=middle,width=\textwidth]
%       {\externalfigure[logo][width=0.25\textwidth]}


Wayne Price



The “Negro March on Washington” movement in the World War II period



A radical contribution to African-American History Month




January 2013


\stopalignment
\blank[line]

\startalignment[hyphenated,middle]




Retrieved on July 2, 2014 from http://anarkismo.net/article/24786?search\_text=wayne\%20price\&print\_page=true


\stopalignment

\stoptext


