% -*- mode: tex -*-
%%%%%%%%%%%%%%%%%%%%%%%%%%%%%%%%%%%%%%%%%%%%%%%%%%%%%%%%%%%%%%%%%%%%%%%%%%%%%%%%
%                                STANDARD                                      %
%%%%%%%%%%%%%%%%%%%%%%%%%%%%%%%%%%%%%%%%%%%%%%%%%%%%%%%%%%%%%%%%%%%%%%%%%%%%%%%%
\definefontfeature[default][default]
                  [protrusion=quality,
                    expansion=quality,
                    script=latn]
\setupalign[hz,hanging]
\setuptolerance[tolerant]
\setbreakpoints[compound]
\setupindenting[yes,1em]
\setupfootnotes[way=bychapter,align={hz,hanging}]
\setupbodyfont[modern] % this is a stinky workaround to load lmodern
\setupbodyfont[libertine,11pt]

\setuppagenumbering[alternative=singlesided,location={footer,middle}]
\setupcaptions[width=fit,align={hz,hanging},number=no]

\startmode[a4imposed,a4imposedbc,letterimposed,letterimposedbc,a5imposed,%
  a5imposedbc,halfletterimposed,halfletterimposedbc]
  \setuppagenumbering[alternative=doublesided]
\stopmode

\setupbodyfontenvironment[default][em=italic]


\setupheads[%
  sectionnumber=no,number=no,
  align=flushleft,
  align={flushleft,nothyphenated,verytolerant,stretch},
  indentnext=yes,
  tolerance=verytolerant]

\definehead[awikipart][chapter]

\setuphead[awikipart]
          [%
            number=no,
            footer=empty,
            style=\bfd,
            before={\blank[force,2*big]},
            align={middle,nothyphenated,verytolerant,stretch},
            after={\page[yes]}
          ]

% h3
\setuphead[chapter]
          [style=\bfc]

\setuphead[title]
          [style=\bfc]


% h4
\setuphead[section]
          [style=\bfb]

% h5
\setuphead[subsection]
          [style=\bfa]

% h6
\setuphead[subsubsection]
          [style=bold]


\setuplist[awikipart]
          [alternative=b,
            interaction=all,
            width=0mm,
            distance=0mm,
            before={\blank[medium]},
            after={\blank[small]},
            style=\bfa,
            criterium=all]
\setuplist[chapter]
          [alternative=c,
            interaction=all,
            width=1mm,
            before={\blank[small]},
            style=bold,
            criterium=all]
\setuplist[section]
          [alternative=c,
            interaction=all,
            width=1mm,
            style=\tf,
            criterium=all]
\setuplist[subsection]
          [alternative=c,
            interaction=all,
            width=8mm,
            distance=0mm,
            style=\tf,
            criterium=all]
\setuplist[subsubsection]
          [alternative=c,
            interaction=all,
            width=15mm,
            style=\tf,
            criterium=all]


% center

\definestartstop
  [awikicenter]
  [before={\blank[line]\startalignment[middle]},
   after={\stopalignment\blank[line]}]

% right

\definestartstop
  [awikiright]
  [before={\blank[line]\startalignment[flushright]},
   after={\stopalignment\blank[line]}]


% blockquote

\definestartstop
  [blockquote]
  [before={\blank[big]
    \setupnarrower[middle=1em]
    \startnarrower
    \setupindenting[no]
    \setupwhitespace[medium]},
  after={\stopnarrower
    \blank[big]}]

% verse

\definestartstop
  [awikiverse]
  [before={\blank[big]
      \setupnarrower[middle=2em]
      \startnarrower
      \startlines},
    after={\stoplines
      \stopnarrower
      \blank[big]}]

\definestartstop
  [awikibiblio]
  [before={%
      \blank[big]
      \setupnarrower[left=1em]
      \startnarrower[left]
        \setupindenting[yes,-1em,first]},
    after={\stopnarrower
      \blank[big]}]
                
% same as above, but with no spacing around
\definestartstop
  [awikiplay]
  [before={%
      \setupnarrower[left=1em]
      \startnarrower[left]
        \setupindenting[yes,-1em,first]},
    after={\stopnarrower}]



% interaction
% we start the interaction only if it's not an imposed format.
\startnotmode[a4imposed,a4imposedbc,letterimposed,letterimposedbc,a5imposed,%
  a5imposedbc,halfletterimposed,halfletterimposedbc]
  \setupinteraction[state=start,color=black,contrastcolor=black,style=bold]
  \placebookmarks[awikipart,chapter,section,subsection,subsubsection][force=yes]
  \setupinteractionscreen[option=bookmark]
\stopnotmode



\setupexternalfigures[%
  maxwidth=\textwidth,
  maxheight=\textheight,
  factor=fit]

\setupitemgroup[itemize][each][packed][indenting=no]

\definemakeup[titlepage][pagestate=start,doublesided=no]

%%%%%%%%%%%%%%%%%%%%%%%%%%%%%%%%%%%%%%%%%%%%%%%%%%%%%%%%%%%%%%%%%%%%%%%%%%%%%%%%
%                                IMPOSER                                       %
%%%%%%%%%%%%%%%%%%%%%%%%%%%%%%%%%%%%%%%%%%%%%%%%%%%%%%%%%%%%%%%%%%%%%%%%%%%%%%%%

\startusercode

function optimize_signature(pages,min,max)
   local minsignature = min or 40
   local maxsignature = max or 80
   local originalpages = pages

   -- here we want to be sure that the max and min are actual *4
   if (minsignature%4) ~= 0 then
      global.texio.write_nl('term and log', "The minsig you provided is not a multiple of 4, rounding up")
      minsignature = minsignature + (4 - (minsignature % 4))
   end
   if (maxsignature%4) ~= 0 then
      global.texio.write_nl('term and log', "The maxsig you provided is not a multiple of 4, rounding up")
      maxsignature = maxsignature + (4 - (maxsignature % 4))
   end
   global.assert((minsignature % 4) == 0, "I suppose something is wrong, not a n*4")
   global.assert((maxsignature % 4) == 0, "I suppose something is wrong, not a n*4")

   --set needed pages to and and signature to 0
   local neededpages, signature = 0,0

   -- this means that we have to work with n*4, if not, add them to
   -- needed pages 
   local modulo = pages % 4
   if modulo==0 then
      signature=pages
   else
      neededpages = 4 - modulo
   end

   -- add the needed pages to pages
   pages = pages + neededpages
   
   if ((minsignature == 0) or (maxsignature == 0)) then 
      signature = pages -- the whole text
   else
      -- give a try with the signature
      signature = find_signature(pages, maxsignature)
      
      -- if the pages, are more than the max signature, find the right one
      if pages>maxsignature then
	 while signature<minsignature do
	    pages = pages + 4
	    neededpages = 4 + neededpages
	    signature = find_signature(pages, maxsignature)
	    --         global.texio.write_nl('term and log', "Trying signature of " .. signature)
	 end
      end
      global.texio.write_nl('term and log', "Parameters:: maxsignature=" .. maxsignature ..
		   " minsignature=" .. minsignature)

   end
   global.texio.write_nl('term and log', "ImposerMessage:: Original pages: " .. originalpages .. "; " .. 
	 "Signature is " .. signature .. ", " ..
	 neededpages .. " pages are needed, " .. 
	 pages ..  " of output")
   -- let's do it
   tex.print("\\dorecurse{" .. neededpages .. "}{\\page[empty]}")

end

function find_signature(number, maxsignature)
   global.assert(number>3, "I can't find the signature for" .. number .. "pages")
   global.assert((number % 4) == 0, "I suppose something is wrong, not a n*4")
   local i = maxsignature
   while i>0 do
      -- global.texio.write_nl('term and log', "Trying " .. i  .. "for max of " .. maxsignature)
      if (number % i) == 0 then
	 return i
      end
      i = i - 4
   end
end

\stopusercode

\define[1]\fillthesignature{
  \usercode{optimize_signature(#1, 40, 80)}}


\define\alibraryflushpages{
  \page[yes] % reset the page
  \fillthesignature{\the\realpageno}
}


% various papers 
\definepapersize[halfletter][width=5.5in,height=8.5in]
\definepapersize[halfafour][width=148.5mm,height=210mm]
\definepapersize[quarterletter][width=4.25in,height=5.5in]
\definepapersize[halfafive][width=105mm,height=148mm]
\definepapersize[generic][width=210mm,height=279.4mm]

%% this is the default ``paper'' which should work with both letter and a4

\setuppapersize[generic][generic]
\setuplayout[%
  backspace=42mm,
  topspace=31mm,% 176 / 15
  height=195mm,%130mm,
  footer=9mm, %
  header=0pt, % no header
  width=126mm] % 10.5 x 11

\startmode[libertine]
  \usetypescript[libertine]
  \setupbodyfont[libertine,11pt]
\stopmode

\startmode[pagella]
  \setupbodyfont[pagella,11pt]
\stopmode

\startmode[antykwa]
  \setupbodyfont[antykwa-poltawskiego,11pt]
\stopmode

\startmode[iwona]
  \setupbodyfont[iwona-medium,11pt]
\stopmode

\startmode[helvetica]
  \setupbodyfont[heros,11pt]
\stopmode

\startmode[century]
  \setupbodyfont[schola,11pt]
\stopmode

\startmode[modern]
  \setupbodyfont[modern,11pt]
\stopmode

\startmode[charis]
  \setupbodyfont[charis,11pt]
\stopmode        

\startmode[mini]
  \setuppapersize[S33][S33] % 176 × 176 mm
  \setuplayout[%
    backspace=20pt,
    topspace=15pt,% 176 / 15
    height=280pt,%130mm,
    footer=20pt, %
    header=0pt, % no header
    width=260pt] % 10.5 x 11
\stopmode

% for the plain A4 and letter, we use the classic LaTeX dimensions
% from the article class
\startmode[a4]
  \setuppapersize[A4][A4]
  \setuplayout[%
    backspace=42mm,
    topspace=45mm,
    height=218mm,
    footer=10mm,
    header=0pt, % no header
    width=126mm]
\stopmode

\startmode[letter]
  \setuppapersize[letter][letter]
  \setuplayout[%
    backspace=44mm,
    topspace=46mm,
    height=199mm,
    footer=10mm,
    header=0pt, % no header
    width=126mm]
\stopmode


% A4 imposed (A5), with no bc

\startmode[a4imposed]
% DIV=15 148 × 210: these are meant not to have binding correction,
  % but just to play safe, let's say 1mm => 147x210
  \setuppapersize[halfafour][halfafour]
  \setuplayout[%
    backspace=10.8mm, % 146/15 = 9.8 + 1
    topspace=14mm, % 210/15 =  14
    height=182mm, % 14 x 12 + 14 of the footer
    footer=14mm, %
    header=0pt, % no header
    width=117.6mm] % 9.8 x 12
\stopmode

% A4 imposed (A5), with bc
\startmode[a4imposedbc]
  \setuppapersize[halfafour][halfafour]
  \setuplayout[% 14 mm was a bit too near to the spine, using the glue binding
    backspace=17.3mm,  % 140/15 + 8 =
    topspace=14mm, % 210/15 =  14
    height=182mm, % 14 x 12 + 14 of the footer
    footer=14mm, %
    header=0pt, % no header
    width=112mm] % 9.333 x 12
\stopmode


\startmode[letterimposedbc] % 139.7mm x 215.9 mm
  \setuppapersize[halfletter][halfletter]
  % DIV=15 8mm binding corr, => 132 x 216
  \setuplayout[%
    backspace=16.8mm, % 8.8 + 8
    topspace=14.4mm, % 216/15 =  14.4
    height=187.2mm, % 15.4 x 11 + 15 of the footer
    footer=14.4mm, %
    header=0pt, % no header
    width=105.6mm] % 8.8 x 12
\stopmode

\startmode[letterimposed] % 139.7mm x 215.9 mm
  \setuppapersize[halfletter][halfletter]
  % DIV=15, 1mm binding correction. => 138.7x215.9
  \setuplayout[%
    backspace=10.3mm, % 9.24 + 1
    topspace=14.4mm, % 216/15 =  14.4
    height=187.2mm, % 15.4 x 11 + 15 of the footer
    footer=14.4mm, %
    header=0pt, % no header
    width=111mm] % 9.24 x 12
\stopmode

%%% new formats for mini books
%%% \definepapersize[halfafive][width=105mm,height=148mm]

\startmode[a5imposed]
% DIV=12 105x148 : these are meant not to have binding correction,
  % but just to play safe, let's say 1mm => 104x148
  \setuppapersize[halfafive][halfafive]
  \setuplayout[%
    backspace=9.6mm,
    topspace=12.3mm,
    height=123.5mm, % 14 x 12 + 14 of the footer
    footer=12.3mm, %
    header=0pt, % no header
    width=78.8mm] % 9.8 x 12
\stopmode

% A5 imposed (A6), with bc
\startmode[a5imposedbc]
% DIV=12 105x148 : with binding correction,
  % let's say 8mm => 96x148
  \setuppapersize[halfafive][halfafive]
  \setuplayout[%
    backspace=16mm,
    topspace=12.3mm,
    height=123.5mm, % 14 x 12 + 14 of the footer
    footer=12.3mm, %
    header=0pt, % no header
    width=72mm] % 9.8 x 12
\stopmode

%%% \definepapersize[quarterletter][width=4.25in,height=5.5in]

% DIV=12 width=4.25in (108mm),height=5.5in (140mm) 
\startmode[halfletterimposed] % 107x140
  \setuppapersize[quarterletter][quarterletter]
  \setuplayout[%
    backspace=10mm,
    topspace=11.6mm,
    height=116mm,
    footer=11.6mm,
    header=0pt, % no header
    width=80mm] % 9.24 x 12
\stopmode

\startmode[halfletterimposedbc]
  \setuppapersize[quarterletter][quarterletter]
  \setuplayout[%
    backspace=15.4mm,
    topspace=11.6mm,
    height=116mm,
    footer=11.6mm,
    header=0pt, % no header
    width=76mm] % 9.24 x 12
\stopmode

\startmode[quickimpose]
  \setuppapersize[A5][A4,landscape]
  \setuparranging[2UP]
  \setuppagenumbering[alternative=doublesided]
  \setuplayout[% 14 mm was a bit too near to the spine, using the glue binding
    backspace=17.3mm,  % 140/15 + 8 =
    topspace=14mm, % 210/15 =  14
    height=182mm, % 14 x 12 + 14 of the footer
    footer=14mm, %
    header=0pt, % no header
    width=112mm] % 9.333 x 12
\stopmode

\startmode[tenpt]
  \setupbodyfont[10pt]
\stopmode

\startmode[twelvept]
  \setupbodyfont[12pt]
\stopmode

%%%%%%%%%%%%%%%%%%%%%%%%%%%%%%%%%%%%%%%%%%%%%%%%%%%%%%%%%%%%%%%%%%%%%%%%%%%%%%%%
%                            DOCUMENT BEGINS                                   %
%%%%%%%%%%%%%%%%%%%%%%%%%%%%%%%%%%%%%%%%%%%%%%%%%%%%%%%%%%%%%%%%%%%%%%%%%%%%%%%%


\mainlanguage[en]


\starttext

\starttitlepagemakeup
  \startalignment[middle,nothanging,nothyphenated,stretch]


  \switchtobodyfont[18pt] % author
  {\bf \em

Peter Gelderloos  \par}
  \blank[2*big]
  \switchtobodyfont[24pt] % title
  {\bf

The Rise of Hierarchy

\par}
  \blank[big]
  \switchtobodyfont[20pt] % subtitle
  {\bf 

\par}
  \vfill
  \stopalignment
  \startalignment[middle,bottom,nothyphenated,stretch,nothanging]
  \switchtobodyfont[global]

2005

  \stopalignment
\stoptitlepagemakeup



\title{Contents}

\placelist[awikipart,chapter,section,subsection]



\page[yes,right]

In charting the origin of social hierarchies and control systems, many radical theorists take a materialist stance, and attribute authoritarian behavior to surpluses resulting from agricultural production and other aspects of the civilization process. The fact that some non-agricultural, hunter-gatherer societies developed hierarchical social structures offers a critical contradiction to the materialist view, and presents the key to understanding the origin of hierarchy. Anarchists, whether we wish to abolish all the cultural artifacts of Western civilization as inherently oppressive or to retain certain aspects of civilization, would do well to learn the partial extent to which civilization and hierarchy are concomitant.


Civilization being understood etymologically and culturally as the subjection of human beings to a centralized or common power “to keep them all in awe” in the words of Hobbes, or make citizens out of them, to refer to the Latin, we can turn to hunter-gatherer peoples as a clear example of stateless society. The two major forms of hierarchy evidenced in some hunter-gatherer societies are patriarchy and gerontocracy. Several hunter-gatherer groups are nascent patriarchies. For instance, among the Aché of the Amazonian forests, the sexual division of labor is stark, and men enjoy greater influence in decision-making. The Aranda of central Australia also give greater political influence to men within the group. Additionally, ownership of communal land, which is the source of identity for each band, is traced through the patriline (father to son).


Gerontocracy, age-based hierarchy dominated by elders, is particularly developed among the Aranda, politically, socially, and spiritually. Generally speaking, Aranda children are not active participants in the affairs of the group, whereas elder males are accorded positions of leadership, and the Aranda religion is based on ancestor worship (Lee and Daly, 1999).


The Mbuti of the Ituri forest of central Africa provide an excellent contrast in demonstrating how non-hierarchical a society can be (the Hadza of the Tanzanian grasslands also practice egalitarian social organization, though there is less literature available on them). Though the Mbuti practice some sexual division of labor, the division is not strict, and often manifests as different functions in the same activity, with women and men working together, to care for children or gather food. The Mbuti minimize gender, and except for distinguishing between mothers and fathers use non-gendered familial labels (e.g. sibling, instead of sister) and pronouns. The Mbuti traditionally form exclusive and even lifelong partnerships for raising children, but Mbuti “marriage” does not prohibit extra-marital sex or love.


One of the most important Mbuti rituals might be termed “gender-fuck” by North American anti-oppression activists. It starts as a game of tug-of-war, with the men on one side and the women on the other. But as soon as one side starts to win, a member of the winning side switches teams, and pretends to be a member of the opposite sex, to restore the balance. By the end of the game, everyone has changed their gender multiple times, and they all fall down laughing, having exorcised gender tensions (Turnbull, 1983).


The Mbuti are also an age-equal society. They provide a field of autonomy and a role of importance to each of the five recognized age groups: infants, children, youth, adults, and elderly. Each age group holds a voluntarily recognized power over the others, and it is the healthy symbiosis of the different groups that makes for a well functioning Mbuti band. The youth, for instance, are regarded as defenders of justice, and it is their function to call out problems or conflicts within the group. The adults, though they have substantial influence as the providers of sustenance, are also criticized as being the main sources of {\em akami}, “noise” or conflict, within the group. The role of the elderly is to reconcile conflicts.


Though the embryonic forms of patriarchy and gerontocracy exhibited by some hunter-gatherer groups are harmless compared to hierarchical dynamics in accumulation-based civilizations, the combination of the two systems is a critical milestone in the rise of hierarchical social organization. That historical combination, which almost certainly predates the development of agriculture, marks the first dynamic hierarchies. The permanent division between men and women is bolstered by the aged hierarchy, which bestows privilege over time, in return for cooperation with the hierarchical system. An elite minority, male elders, hold disproportionate influence and the beginnings of political power. Meanwhile, the promise of eventual inclusion into the elite encourages younger males to cooperate with the hierarchy. Females, too, are more likely to cooperate with their own disempowerment; even though they will never ascend to an elite role, they can still win an elevated status as they grow older by participating with the hierarchy.


It seems gerontocracy also makes possible a rudimentary form of policing in stateless society. The age grades that the Mbuti use in a libertarian way become tools for political authority in many West African societies, such as the Ibo (stateless horticulturalists), that subordinate young people to old people. Youth, instead of being autonomous defenders of justice, play a policing function by enforcing the will of the age group above them, thus turning the diffuse sanctions (collectively held enforcement mechanisms) characteristic of anarchy into something closer to the centrally controlled sanctions of the state (Barclay, 1982). This becomes possible in a culture where older people are seen as legitimate leaders and younger people seek to win their favor. Within this context, the concept of lineage becomes increasingly important. The segmentary lineages of many stateless West African tribes appear to open an effective path for the development of government. The “Big Man” leadership evidenced in many simple patriarchies, forager or horticultural, is too unstable to permanently institutionalize political power (an aggressive, strong, or capable man invites competition and resentment, loses these qualities with age, and cannot pass them on to any chosen successor). But segmentary lineages in which each grouping — the family, the sub-clan, the clan — is headed by a leader, the father of the lineage (a concept that requires only patrilineality and gerontocracy), political control over a large population begins to be centralized by a pecking order of leaders, from minor to major; leadership becomes hereditary; and prestigious lineages that have won leadership of the larger structures (clans or the tribe) take on an innate leadership quality: a superiority is believed to run in their blood.


The question remains, why did some human groups develop these forms of hierarchy, while others did not? Patriarchy is often attributed to men winning influence from their role as warriors or providers. But many hunter-gatherer and horticultural groups did not practice warfare, and there is no clear delineation of peaceful political strategies always being practiced by the gender-equal or malineal groups. Neither is there a correlation between men’s role as providers and their role as patriarchs. Patriarchy was as developed or more developed in societies where women provided most of the food, for instance the Aranda, than among groups like the Aché, where men provided roughly 80\% of the diet.


On the contrary, patriarchy seems to be a possible result among any human group (contemporary activists should take note) that does not specifically organize to prevent patriarchy. Gender distinctions are an obvious axis for conflict within human groups, and overcoming conflict must be a constant activity in any society. The development of patriarchy is not inevitable, or natural, it is simply convenient — for those who wish to gain social power, and take the easy way out of dealing with group problems.


Social practices and institutions to prevent or resist the development of patriarchy have been manifold. They range from gender-leveling rituals like those practiced by the Mbuti, to the ritualized collective action, including all-night insult sessions and possible property destruction, practiced by Igbo women against male culprits who have violated a woman’s rights or infringed on the women’s sphere of economic activity (Van Allen, 1972).


Stages of patriarchal development described by Gerda Lerner (1986) include removing women from the divine, most pronounced in the monotheists’ development of a single male God; creating the cultural myth that women are spiritually or mentally incomplete, as in Aristotelian philosophy; and authoring laws or social mores that govern women’s sexuality, as in Hammurabi’s code.


I would add that the first and most important stage of patriarchy is the conceptualization of rigid gender identities. Riane Eisler (1987) and a number of other liberal feminists, in a sincere attempt to liberate an anti-patriarchal history, have resurrected a number of Mediterranean societies dominated by female fertility symbology and marked by less stark class and gender divisions, as evidence of a pre-patriarchal past. Unfortunately, their scholarship still leaves us with an essentialized gender binary in which women’s source of social power is their ability to make babies. In fact, male cooptation of female fertility symbols was a common stage of development in many patriarchal societies. From the Anasazi to the Minoans, male priests recently in charge of religious structures, used, and even wore, yonic symbols as a mark of their power (Donald and Hurcombe, 2000). This occurred in tandem with agriculturalists’ cooptation of the fertility of “Mother Earth.”


One of the earliest known forms of resistance to essentialized notions of gender was artwork, among hunter-gatherers as well as horticulturalists and early agriculturalists. Dating back thousands of years, San rock-art, as well as paintings and figurines from all across the world, frequently contained androgynous figures, encouraging a fluidity to the concept of gender by blurring the distinction or presenting figures that simultaneously exhibited exaggerated female and male characteristics (and often, also the characteristics of other animals). Eisler herself, inhibited by an essentially patriarchal lens, misrepresents her own research, neglecting to mention that the majority of Neolithic figurines in her samples are not female, but androgynous.


Agriculture and civilization did not create hierarchy in human groups, nor did hierarchy lead to the creation of civilization, as evidenced by the existence of egalitarian horticultural and agricultural societies. Rather, hierarchy is a result of a people’s social strategies, but agriculture and other technological progressions allow nascent hierarchies to become much more complex, authoritarian, and violent. Even worse, the military advantages that inhere in agriculture — such as higher population density, disease resistance from living with animals in sedentary communities, and metal tools — allow civilization’s more developed hierarchies to be spread by expanding nations and conquering armies.


To increase our understanding, it would be helpful to know how agriculture developed. It is important to realize that the development of agriculture was not inevitable or universal. Although the vast majority of societies today sustain themselves through some form of agriculture, agriculture’s preeminence is largely a result of population expansion and military dominance by agricultural societies. Perhaps as few as five societies independently developed agriculture in all of human history (in the Middle East, China, sub-Saharan Africa, the Yucatan, and the Andes). This is not to say that agriculture is an unlikely invention; many hunter-gatherer groups demonstrate a knowledge of agriculture but choose not to practice it. Offsetting its military advantages, agriculture was accompanied by a marked decline in human health, which has been sufficiently described elsewhere. Agriculture was often an unpopular invention, spreading through much of Europe less than a mile each year (Diamond, 1992).


In the best-studied example, the Middle East, agriculture developed earliest in the highlands of the Levant, east of the Mediterranean. The process appears to have begun 12,500 years ago, when climatic changes at the end of the Ice Age led to a significant increase of wild-growing cereals and nuts. Natufian hunter-gatherers in the region practiced a simple forager strategy, meaning they gathered and hunted a wide range of plant and animal foods, without specialization, for a diverse diet. After the explosion of cereal and nut populations, the Natufians adopted a complex forager strategy, specializing in the high-energy, easy to gather grains and nuts (Henry, 1989). Accordingly, they went from being nomadic to semi-sedentary, with more permanent dwellings where food could be stored, and seasonal abundances could be exploited. It was a simple matter of economics: they had the opportunity to get by with less effort, so they took it.


However, complex foragers are rare compared to simple foragers, because the complex forager strategy is less adaptive. Complex foragers are more dependent on a small range of foods, and thus vulnerable to the vagaries of climate and other natural changes, and also more sedentary, thus unable to spread out their ecological impact. 10,000 years ago, the climate changed again, and the territory of cereal and nut populations began shrinking. The complex foragers were faced with a choice: adapt to changes in the environment by reverting to a simple forager strategy, or artificially preserve the abundance of their key foods by saving and planting the seeds. Some groups did choose to become simple foragers again, while others developed horticulture and agriculture.


These early farmers were afforded new opportunities. In sedentary communities, they could more easily domesticate animals, develop larger and more complex tools, and create permanent dwellings and property. They could domesticate and manage crop species by storing and replanting seeds with favorable characteristics. They could develop irrigation to grow and harvest beyond the capacities of the local climate. They could store food for times when their staple crops were not in season, cutting out their need to forage. They could use their surpluses to support artisans and others who would not take part in farming. They could raid the stores of neighboring communities in times of scarcity, creating warfare as we know it.


The critical choices of these early agriculturalists, which have affected all of human history since then, would have been profoundly influenced by the social strategies practiced by each particular group. In all likelihood, some of the bands and communities involved in the early development of horticulture and agriculture were egalitarian, like the Mbuti, and others probably practiced patriarchy, or gerontocracy, or both. Patriarchal groups, living in monogamous households, would have been more likely to develop notions of individual property. Gerontocratic groups, by discouraging the role of youth in challenging the status quo, would have been more likely to tolerate and traditionalize social iniquity. Groups with an elite of elder males would have been more likely to develop economic disparities, because the majority in such groups were doing more work and enjoying poorer health than their forager or horticultural ancestors, but those with decision-making authority, the elite, were enjoying the fruits of the surplus.


Though the hierarchies that were in existence before the development of agriculture were insubstantial, and even the groups with dynamic hierarchies, like the Aranda, still exhibit a culture of anti-authoritarianism, these choices took place over centuries, and no one at that time would have known the disastrous consequences of choosing slightly more authoritarian, capitalistic, or warlike strategies. However, over time the massive military advantages that accrued to societies practicing more complex forms of agriculture (having weapons, soldiers, twice the population of your neighbors) meant that just one community pursuing an aggressive strategy could force its neighbors into a sort of arms race, by presenting them with the choice of developing their technologies to stay competitive, fleeing the area, or being overrun, and killed or turned into slaves.


Communities already led by an elite, who would lose the least and benefit the most from warfare and increased production, were certainly more likely to try and out-compete or dominate their neighbors. It was certainly no contradiction for a community to practice horticulture or agriculture and still retain a culture of consensus, communalism, and ecocentrism, but such communities would not have participated in the arms race, and they would have been conquered, allowing for the ascendancy of the culture of domination and accumulation, and the proliferation of the arms race. That is what has been happening ever since.


The meaning of this history for anti-authoritarians today is that domination- and accumulation-based civilizations spread not because of any freely chosen assurances of material improvement, but because of the military advantages, and the imperative to dominate, hardwired into such civilizations. Though it was easy for domination-based civilizations to subjugate surrounding societies, another historical survey could clearly show that these civilizations are quite vulnerable to the internal tension that arises from the antagonism the subjects reasonably develop towards the power structures that dominate them. Recent history shows clearly enough that the military advantages inherent in domination-based civilization do not apply to internal rebellions (provided the rebels have a minimum access to broad support and technologies in the range of firearms and explosives). Whatever occurs after the fall of Authority, a broad cultural remembrance of the dangers of allowing oppressive hierarchies to take root can help prevent a recurrence of the mistakes made by human groups 10,000 years ago, at a time when they could not know the full ramifications of their actions. Oppressive hierarchies are not inherent to any material modes of existence human beings would choose to inhabit (as distinguished from modes that were forcefully implemented from the top, as appears universally to be the case with Western-style industrialism). Rather, oppressive hierarchies allow technologies to become oppressive, and technologies define the range of complexity which those hierarchies can develop. The hierarchies themselves, which foster their own reproduction (in part through the development of technologies that are implicitly oppressive), fall within the range of possible human behavior, but can be prevented when understood as a threat to human freedom and wellbeing. The questions of what to do with this understanding in the present day — which technologies can be kept, which can be reformed, and which must be discarded, as well as the question of how these new material modes (most likely different modes for different bioregions) will interact with our efforts to prevent hierarchy — remain to be explored and answered.


\section{Works Cited
}


\startitemize[1]\relax
\item[] Barclay, Harold, {\em People Without Government: An Anthropology of Anarchy}. London: Kahn and Averill, 1982.




 \item[] Diamond, Jared, {\em Guns, Germs, and Steel: The Fates of Human Societies.} New York: W.W. Norton, 1997.




 \item[] Donald, Moira, and Linda Hurcombe, eds., {\em Representations of Gender from Prehistory to Present}. New York: St. Martin’s Press, 2000




 \item[] Eisler, Riane, {\em The Chalice and the Blade.} San Francisco: Harper Collins, 1995.




 \item[] Henry, Donald O., {\em From Foraging to Agriculture}. Philadelphia: University of Philadelphia Press, 1989.




 \item[] Lee, Richard B., and Richard Daly, ed., {\em The Cambridge Encyclopedia of Hunters and Gatherers.} Cambridge: Cambridge University Press, 1999.




 \item[] Lerner, Gerda, {\em The Creation of Patriarchy}. New York : Oxford University Press, 1986.




 \item[] Turnbull, Colin M., “{\em The Mbuti Pygmies. Change and Adaption}.” Philadelphia: Harcourt Brace College Publishers, 1983.




 \item[] Van Allen, Judith. “Sitting On a Man.” Canadian Journal of African Studies. Vol. ii, 1972. 211–219.




 
\stopitemize







\page[yes]

%%%% backcover

\startmode[a4imposed,a4imposedbc,letterimposed,letterimposedbc,a5imposed,%
  a5imposedbc,halfletterimposed,halfletterimposedbc,quickimpose]
\alibraryflushpages
\stopmode

\page[blank]

\startalignment[middle]
{\tfa The Anarchist Library
\blank[small]
Anti-Copyright}
\blank[small]
\currentdate
\stopalignment

\blank[big]
\framed[frame=off,location=middle,width=\textwidth]
       {\externalfigure[logo][width=0.25\textwidth]}



\vfill
\setupindenting[no]
\setsmallbodyfont

\startalignment[middle,nothyphenated,nothanging,stretch]

\blank[line]
% \framed[frame=off,location=middle,width=\textwidth]
%       {\externalfigure[logo][width=0.25\textwidth]}


Peter Gelderloos



The Rise of Hierarchy






2005


\stopalignment
\blank[line]

\startalignment[hyphenated,middle]


Fall 2005



Personal communication with the author, August 9, 2009


\stopalignment

\stoptext


