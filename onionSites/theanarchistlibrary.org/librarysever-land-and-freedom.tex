% -*- mode: tex -*-
%%%%%%%%%%%%%%%%%%%%%%%%%%%%%%%%%%%%%%%%%%%%%%%%%%%%%%%%%%%%%%%%%%%%%%%%%%%%%%%%
%                                STANDARD                                      %
%%%%%%%%%%%%%%%%%%%%%%%%%%%%%%%%%%%%%%%%%%%%%%%%%%%%%%%%%%%%%%%%%%%%%%%%%%%%%%%%
\enabletrackers[fonts.missing]
\definefontfeature[default][default]
                  [protrusion=quality,
                    expansion=quality,
                    script=latn]
\setupalign[hz,hanging]
\setuptolerance[tolerant]
\setbreakpoints[compound]
\setupindenting[yes,1em]
\setupfootnotes[way=bychapter,align={hz,hanging}]
\setupbodyfont[modern] % this is a stinky workaround to load lmodern
\setupbodyfont[libertine,11pt]

\setuppagenumbering[alternative=singlesided,location={footer,middle}]
\setupcaptions[width=fit,align={hz,hanging},number=no]

\startmode[a4imposed,a4imposedbc,letterimposed,letterimposedbc,a5imposed,%
  a5imposedbc,halfletterimposed,halfletterimposedbc]
  \setuppagenumbering[alternative=doublesided]
\stopmode

\setupbodyfontenvironment[default][em=italic]


\setupheads[%
  sectionnumber=no,number=no,
  align=flushleft,
  align={flushleft,nothyphenated,verytolerant,stretch},
  indentnext=yes,
  tolerance=verytolerant]

\definehead[awikipart][chapter]

\setuphead[awikipart]
          [%
            number=no,
            footer=empty,
            style=\bfd,
            before={\blank[force,2*big]},
            align={middle,nothyphenated,verytolerant,stretch},
            after={\page[yes]}
          ]

% h3
\setuphead[chapter]
          [style=\bfc]

\setuphead[title]
          [style=\bfc]


% h4
\setuphead[section]
          [style=\bfb]

% h5
\setuphead[subsection]
          [style=\bfa]

% h6
\setuphead[subsubsection]
          [style=bold]


\setuplist[awikipart]
          [alternative=b,
            interaction=all,
            width=0mm,
            distance=0mm,
            before={\blank[medium]},
            after={\blank[small]},
            style=\bfa,
            criterium=all]
\setuplist[chapter]
          [alternative=c,
            interaction=all,
            width=1mm,
            before={\blank[small]},
            style=bold,
            criterium=all]
\setuplist[section]
          [alternative=c,
            interaction=all,
            width=1mm,
            style=\tf,
            criterium=all]
\setuplist[subsection]
          [alternative=c,
            interaction=all,
            width=8mm,
            distance=0mm,
            style=\tf,
            criterium=all]
\setuplist[subsubsection]
          [alternative=c,
            interaction=all,
            width=15mm,
            style=\tf,
            criterium=all]


% center

\definestartstop
  [awikicenter]
  [before={\blank[line]\startalignment[middle]},
   after={\stopalignment\blank[line]}]

% right

\definestartstop
  [awikiright]
  [before={\blank[line]\startalignment[flushright]},
   after={\stopalignment\blank[line]}]


% blockquote

\definestartstop
  [blockquote]
  [before={\blank[big]
    \setupnarrower[middle=1em]
    \startnarrower
    \setupindenting[no]
    \setupwhitespace[medium]},
  after={\stopnarrower
    \blank[big]}]

% verse

\definestartstop
  [awikiverse]
  [before={\blank[big]
      \setupnarrower[middle=2em]
      \startnarrower
      \startlines},
    after={\stoplines
      \stopnarrower
      \blank[big]}]

\definestartstop
  [awikibiblio]
  [before={%
      \blank[big]
      \setupnarrower[left=1em]
      \startnarrower[left]
        \setupindenting[yes,-1em,first]},
    after={\stopnarrower
      \blank[big]}]
                
% same as above, but with no spacing around
\definestartstop
  [awikiplay]
  [before={%
      \setupnarrower[left=1em]
      \startnarrower[left]
        \setupindenting[yes,-1em,first]},
    after={\stopnarrower}]



% interaction
% we start the interaction only if it's not an imposed format.
\startnotmode[a4imposed,a4imposedbc,letterimposed,letterimposedbc,a5imposed,%
  a5imposedbc,halfletterimposed,halfletterimposedbc]
  \setupinteraction[state=start,color=black,contrastcolor=black,style=bold]
  \placebookmarks[awikipart,chapter,section,subsection,subsubsection][force=yes]
  \setupinteractionscreen[option=bookmark]
\stopnotmode



\setupexternalfigures[%
  maxwidth=\textwidth,
  maxheight=\textheight,
  factor=fit]

\setupitemgroup[itemize][each][packed][indenting=no]

\definemakeup[titlepage][pagestate=start,doublesided=no]

%%%%%%%%%%%%%%%%%%%%%%%%%%%%%%%%%%%%%%%%%%%%%%%%%%%%%%%%%%%%%%%%%%%%%%%%%%%%%%%%
%                                IMPOSER                                       %
%%%%%%%%%%%%%%%%%%%%%%%%%%%%%%%%%%%%%%%%%%%%%%%%%%%%%%%%%%%%%%%%%%%%%%%%%%%%%%%%

\startusercode

function optimize_signature(pages,min,max)
   local minsignature = min or 40
   local maxsignature = max or 80
   local originalpages = pages

   -- here we want to be sure that the max and min are actual *4
   if (minsignature%4) ~= 0 then
      global.texio.write_nl('term and log', "The minsig you provided is not a multiple of 4, rounding up")
      minsignature = minsignature + (4 - (minsignature % 4))
   end
   if (maxsignature%4) ~= 0 then
      global.texio.write_nl('term and log', "The maxsig you provided is not a multiple of 4, rounding up")
      maxsignature = maxsignature + (4 - (maxsignature % 4))
   end
   global.assert((minsignature % 4) == 0, "I suppose something is wrong, not a n*4")
   global.assert((maxsignature % 4) == 0, "I suppose something is wrong, not a n*4")

   --set needed pages to and and signature to 0
   local neededpages, signature = 0,0

   -- this means that we have to work with n*4, if not, add them to
   -- needed pages 
   local modulo = pages % 4
   if modulo==0 then
      signature=pages
   else
      neededpages = 4 - modulo
   end

   -- add the needed pages to pages
   pages = pages + neededpages
   
   if ((minsignature == 0) or (maxsignature == 0)) then 
      signature = pages -- the whole text
   else
      -- give a try with the signature
      signature = find_signature(pages, maxsignature)
      
      -- if the pages, are more than the max signature, find the right one
      if pages>maxsignature then
	 while signature<minsignature do
	    pages = pages + 4
	    neededpages = 4 + neededpages
	    signature = find_signature(pages, maxsignature)
	    --         global.texio.write_nl('term and log', "Trying signature of " .. signature)
	 end
      end
      global.texio.write_nl('term and log', "Parameters:: maxsignature=" .. maxsignature ..
		   " minsignature=" .. minsignature)

   end
   global.texio.write_nl('term and log', "ImposerMessage:: Original pages: " .. originalpages .. "; " .. 
	 "Signature is " .. signature .. ", " ..
	 neededpages .. " pages are needed, " .. 
	 pages ..  " of output")
   -- let's do it
   tex.print("\\dorecurse{" .. neededpages .. "}{\\page[empty]}")

end

function find_signature(number, maxsignature)
   global.assert(number>3, "I can't find the signature for" .. number .. "pages")
   global.assert((number % 4) == 0, "I suppose something is wrong, not a n*4")
   local i = maxsignature
   while i>0 do
      -- global.texio.write_nl('term and log', "Trying " .. i  .. "for max of " .. maxsignature)
      if (number % i) == 0 then
	 return i
      end
      i = i - 4
   end
end

\stopusercode

\define[1]\fillthesignature{
  \usercode{optimize_signature(#1, 40, 80)}}


\define\alibraryflushpages{
  \page[yes] % reset the page
  \fillthesignature{\the\realpageno}
}


% various papers 
\definepapersize[halfletter][width=5.5in,height=8.5in]
\definepapersize[halfafour][width=148.5mm,height=210mm]
\definepapersize[quarterletter][width=4.25in,height=5.5in]
\definepapersize[halfafive][width=105mm,height=148mm]
\definepapersize[generic][width=210mm,height=279.4mm]

%% this is the default ``paper'' which should work with both letter and a4

\setuppapersize[generic][generic]
\setuplayout[%
  backspace=42mm,
  topspace=31mm,% 176 / 15
  height=195mm,%130mm,
  footer=9mm, %
  header=0pt, % no header
  width=126mm] % 10.5 x 11

\startmode[libertine]
  \usetypescript[libertine]
  \setupbodyfont[libertine,11pt]
\stopmode

\startmode[pagella]
  \setupbodyfont[pagella,11pt]
\stopmode

\startmode[antykwa]
  \setupbodyfont[antykwa-poltawskiego,11pt]
\stopmode

\startmode[iwona]
  \setupbodyfont[iwona-medium,11pt]
\stopmode

\startmode[helvetica]
  \setupbodyfont[heros,11pt]
\stopmode

\startmode[century]
  \setupbodyfont[schola,11pt]
\stopmode

\startmode[modern]
  \setupbodyfont[modern,11pt]
\stopmode

\startmode[charis]
  \setupbodyfont[charis,11pt]
\stopmode        

\startmode[mini]
  \setuppapersize[S33][S33] % 176 × 176 mm
  \setuplayout[%
    backspace=20pt,
    topspace=15pt,% 176 / 15
    height=280pt,%130mm,
    footer=20pt, %
    header=0pt, % no header
    width=260pt] % 10.5 x 11
\stopmode

% for the plain A4 and letter, we use the classic LaTeX dimensions
% from the article class
\startmode[a4]
  \setuppapersize[A4][A4]
  \setuplayout[%
    backspace=42mm,
    topspace=45mm,
    height=218mm,
    footer=10mm,
    header=0pt, % no header
    width=126mm]
\stopmode

\startmode[letter]
  \setuppapersize[letter][letter]
  \setuplayout[%
    backspace=44mm,
    topspace=46mm,
    height=199mm,
    footer=10mm,
    header=0pt, % no header
    width=126mm]
\stopmode


% A4 imposed (A5), with no bc

\startmode[a4imposed]
% DIV=15 148 × 210: these are meant not to have binding correction,
  % but just to play safe, let's say 1mm => 147x210
  \setuppapersize[halfafour][halfafour]
  \setuplayout[%
    backspace=10.8mm, % 146/15 = 9.8 + 1
    topspace=14mm, % 210/15 =  14
    height=182mm, % 14 x 12 + 14 of the footer
    footer=14mm, %
    header=0pt, % no header
    width=117.6mm] % 9.8 x 12
\stopmode

% A4 imposed (A5), with bc
\startmode[a4imposedbc]
  \setuppapersize[halfafour][halfafour]
  \setuplayout[% 14 mm was a bit too near to the spine, using the glue binding
    backspace=17.3mm,  % 140/15 + 8 =
    topspace=14mm, % 210/15 =  14
    height=182mm, % 14 x 12 + 14 of the footer
    footer=14mm, %
    header=0pt, % no header
    width=112mm] % 9.333 x 12
\stopmode


\startmode[letterimposedbc] % 139.7mm x 215.9 mm
  \setuppapersize[halfletter][halfletter]
  % DIV=15 8mm binding corr, => 132 x 216
  \setuplayout[%
    backspace=16.8mm, % 8.8 + 8
    topspace=14.4mm, % 216/15 =  14.4
    height=187.2mm, % 15.4 x 11 + 15 of the footer
    footer=14.4mm, %
    header=0pt, % no header
    width=105.6mm] % 8.8 x 12
\stopmode

\startmode[letterimposed] % 139.7mm x 215.9 mm
  \setuppapersize[halfletter][halfletter]
  % DIV=15, 1mm binding correction. => 138.7x215.9
  \setuplayout[%
    backspace=10.3mm, % 9.24 + 1
    topspace=14.4mm, % 216/15 =  14.4
    height=187.2mm, % 15.4 x 11 + 15 of the footer
    footer=14.4mm, %
    header=0pt, % no header
    width=111mm] % 9.24 x 12
\stopmode

%%% new formats for mini books
%%% \definepapersize[halfafive][width=105mm,height=148mm]

\startmode[a5imposed]
% DIV=12 105x148 : these are meant not to have binding correction,
  % but just to play safe, let's say 1mm => 104x148
  \setuppapersize[halfafive][halfafive]
  \setuplayout[%
    backspace=9.6mm,
    topspace=12.3mm,
    height=123.5mm, % 14 x 12 + 14 of the footer
    footer=12.3mm, %
    header=0pt, % no header
    width=78.8mm] % 9.8 x 12
\stopmode

% A5 imposed (A6), with bc
\startmode[a5imposedbc]
% DIV=12 105x148 : with binding correction,
  % let's say 8mm => 96x148
  \setuppapersize[halfafive][halfafive]
  \setuplayout[%
    backspace=16mm,
    topspace=12.3mm,
    height=123.5mm, % 14 x 12 + 14 of the footer
    footer=12.3mm, %
    header=0pt, % no header
    width=72mm] % 9.8 x 12
\stopmode

%%% \definepapersize[quarterletter][width=4.25in,height=5.5in]

% DIV=12 width=4.25in (108mm),height=5.5in (140mm) 
\startmode[halfletterimposed] % 107x140
  \setuppapersize[quarterletter][quarterletter]
  \setuplayout[%
    backspace=10mm,
    topspace=11.6mm,
    height=116mm,
    footer=11.6mm,
    header=0pt, % no header
    width=80mm] % 9.24 x 12
\stopmode

\startmode[halfletterimposedbc]
  \setuppapersize[quarterletter][quarterletter]
  \setuplayout[%
    backspace=15.4mm,
    topspace=11.6mm,
    height=116mm,
    footer=11.6mm,
    header=0pt, % no header
    width=76mm] % 9.24 x 12
\stopmode

\startmode[quickimpose]
  \setuppapersize[A5][A4,landscape]
  \setuparranging[2UP]
  \setuppagenumbering[alternative=doublesided]
  \setuplayout[% 14 mm was a bit too near to the spine, using the glue binding
    backspace=17.3mm,  % 140/15 + 8 =
    topspace=14mm, % 210/15 =  14
    height=182mm, % 14 x 12 + 14 of the footer
    footer=14mm, %
    header=0pt, % no header
    width=112mm] % 9.333 x 12
\stopmode

\startmode[tenpt]
  \setupbodyfont[10pt]
\stopmode

\startmode[twelvept]
  \setupbodyfont[12pt]
\stopmode

%%%%%%%%%%%%%%%%%%%%%%%%%%%%%%%%%%%%%%%%%%%%%%%%%%%%%%%%%%%%%%%%%%%%%%%%%%%%%%%%
%                            DOCUMENT BEGINS                                   %
%%%%%%%%%%%%%%%%%%%%%%%%%%%%%%%%%%%%%%%%%%%%%%%%%%%%%%%%%%%%%%%%%%%%%%%%%%%%%%%%


\mainlanguage[en]


\starttext

\starttitlepagemakeup
  \startalignment[middle,nothanging,nothyphenated,stretch]


  \switchtobodyfont[18pt] % author
  {\bf \em

Sever  \par}
  \blank[2*big]
  \switchtobodyfont[24pt] % title
  {\bf

Land and Freedom

\par}
  \blank[big]
  \switchtobodyfont[20pt] % subtitle
  {\bf 

an old challenge

\par}
  \vfill
  \stopalignment
  \startalignment[middle,bottom,nothyphenated,stretch,nothanging]
  \switchtobodyfont[global]

April, 2014

  \stopalignment
\stoptitlepagemakeup



\title{Contents}

\placelist[awikipart,chapter,section,subsection]



\page[yes,right]

\section{An old slogan
}

One of the oldest anarchist slogans was “Land and Freedom.” You don’t hear it much anymore these days, but this battle cry was used most fervently in the revolutionary movements in Mexico, Spain, Russia, and Manchuria. In the first case, the movement that used those three words like a weapon and like a compass had an important indigenous background. In the second case, the workers of Spain who spoke of “Tierra y Libertad” were often fresh arrivals to the city who still remembered the feudal existence they had left behind in the countryside. In Russia and Manchuria, the revolutionaries who linked those two concepts, land and freedom, were largely peasants.


It was not the generic working class, formed in the factories and blue collar neighborhoods, for whom this slogan had the most meaning, but those exploited people who had only just begun their tutelage as proletarians.


The reformers of those aforementioned struggles interpreted “Land and Freedom” as two distinct, political demands: land, or some kind of agrarian reform that would dole out to the rural poor commoditized parcels so they could make their living in a monetized market; and freedom, or the opportunity to participate in the bourgeois organs of government.


Land, conceptualized thus, has since become obsolete, and freedom, also in the liberal sense, has been universalized and proven lacking. Yet if anarchists and other radical peasants and workers who rose up alongside them never held to the liberal conception of freedom, shouldn’t we suspect that when they talked about land they were also referring to something different?


Tragically, anarchists became proletarianized and stopped talking about land and freedom. Ever dwindling, they held on to their quaint conception of freedom that did not demand inclusion in government but rather its very destruction. Yet they surrended the idea of land to the liberal paradigm. It was something that existed outside the cities, that existed to produce food, and that would be liberated and rationally organized as soon as workers in the supposed nerve centers of capitalism—the urban hubs—brought down the government and reappropriated the social wealth.


The farthest that anarchists usually come to reject this omission is still within a dichotomy that externalizes land from the centers of capitalist accumulation: these are the anarchists who in one form or another “go back to the land,” leaving the cities, setting up communes, rural cooperatives, or embarking on efforts to rewild. The truth is, the “back to the land” movement and the rural communes of earlier generations, organized according to a wide variety of strategies of resistance, turned up a body of invaluable experience that anarchists collectively have still failed to absorb. Though some such experiments persist today and new versions are constantly being inaugurated, the tendency on the whole has been a failure, and we need to talk more extensively about why.


Non-indigenous anarchists who have decided to learn from indigenous struggles have played an important role in improving solidarity with some of the most important battles against capitalism taking place today, and they have also contributed to a practice of nurturing intimate relationships with the land in a way that supports us in our ongoing struggles. But when they counterpose land to city, I think they fail to get to the root of alienation, and the limited resonance of their practice seems to confirm this.


\section{Land and Freedom unalienated
}

The most radical possible interpretation of the slogan, “Land and Freedom”, does not posit two separate items joined on a list. It presents land and freedom as two interdependent concepts, each of which transforms the meaning of the other. The counter to the rationalist Western notion of land and that civilization’s corrupted notion of freedom is the vision that at least some early anarchists were projecting in their battle cry.


Land linked to freedom means a habitat that we freely interrelate with, to shape and be shaped by, unburdened by any productive or utilitarian impositions and the rationalist ideology they naturalize. Freedom linked to land means the self-organization of our vital activity, activity that we direct to achieve sustenance on our own terms, not as isolated units but as living beings within a web of wider relationships. Land and freedom means being able to feed ourselves without having to bend to any blackmail imposed by government or a privileged caste, having a home without paying for permission, learning from the earth and sharing with all other living beings without quantifying value, holding debts, or seeking profit. This conception of life enters into a battle of total negation with the world of government, money, wage or slave labor, industrial production, Bibles and priests, institutionalized learning, the spectacularization of daily existence, and all other apparatuses of control that flow from Enlightenment thinking and the colonialistic civilization it champions.


Land, in this sense, is not a place external to the city. For one, this is because capitalism does not reside primarily in urban space—it controls the whole map. The military and productive logics that control us and bludgeon the earth in urban space are also at work in rural space. Secondly, the reunited whole of land and freedom must be an ever present possibility no matter where we are. They constitute a social relationship, a way of relating to the world around us and the other beings in it, that is profoundly opposed to the alienated social relationship of capitalism. Alienation and primitive accumulation\footnote{Primitive accumulation, for those unfamiliar with the term, is the process by which the commons are converted into commoditites or means of production; more precisely it is the often brutal process by which capitalist value that can be put to the service of production and accumulation is originally created. A population of rent-paying workers and the factories that employ them already constitute a society organized according to capitalist social relations, in which everything serves the accumulation of ever more capital. On the other hand, things like communal land that directly feeds those who live on it and work with it, or folk knowledge that is shared freely and passed on informally, constitute resources that do not generate capital (that is, alienated, quantifiable value that can be reinvested). To benefit capitalism, such resources need to be enclosed and commoditized, through colonialism, disposession, criminalization, professionalization, taxation, starvation, and other policies. This is primitive accumulation. Marx portrayed this process as one that marks the earliest stage of capitalism but in reality it is an ongoing process active at the margins of capitalism, which crisscross our world with every successive expansion or intensification of the system.} are ceaseless, ongoing processes from one corner of the globe to the other. Those of us who are not indigenous, those of us who are fully colonized and have forgotten where we came from, do not have access to anything pristine. Alienation will follow us out to the farthest forest glade or desert oasis until we can begin to change our relationship to the world around us in a way that is simultaneously material and spiritual.


Equally, anarchy must be a robust concept. It must be an available practice no matter where we find ourselves—in the woods or in the city, in a prison or on the high seas. It requires us to transform our relationship with our surroundings, and therefore to also transform our surroundings, but it cannot be so fragile that it requires us to seek out some pristine place in order to spread anarchy. Will anti-civilization anarchism be a minoritarian sect of those anarchists who go to the woods to live deliberately, because they don’t like the alternative of organizing a union at the local burger joint, or will it be a challenge to the elements of the anarchist tradition that reproduce colonialism, patriarchy, and Enlightenment thinking, a challenge that is relative to all anarchists no matter where they pick their battles?


Land does not exist in opposition to the city. Rather, one concept of land exists in opposition to another. The anarchist or anti-civilization idea against the capitalist, Western idea. It is this latter concept that places land within the isolating dichotomy of city vs. wilderness. This is why “going back to the land” is doomed to fail, even though we may win valuable lessons and experiences in the course of that failure (as anarchists, we’ve rarely won anything else). We don’t need to go back to the land, because it never left us. We simply stopped seeing it and stopped communing with it.


Recreating our relationship with the world can happen wherever we are, in the city or in the countryside. But how does it happen?


\section{History
}

An important step is to recover histories about how we lost our connection with the land and how we got colonized. These can be the histories of our people, defined ethnically, the history of our blood family, the histories of the people who have inhabited the place we call home, the histories of anarchists or queers or nomads or whomever else we consider ourselves to be one of. They must be all of these things, for no one history can tell it all. Not everyone was colonized the same way, and though capitalism has touched everyone on the planet, not everyone is a child of capitalism nor of the civilization that brought it across the globe.


The history of the proletariat as it has been told so far presents colonization (the very process that has silenced those other stories) as a process that was marginal while it was occurring and is now long since completed, when in fact many people still hold on to another way of relating to the land, and the process of colonization that molds us as proletarians or consumers—or whatever capitalism wants us to be in a given moment—is ongoing.


As we recover those histories, we need to root them in the world around us and communalize them, so that they lucidly imbue our surroundings, so that young people grow up learning them, and so they can never be stolen from us again. The printed or glowing page which I am using to share these imperatives with you can never be more than a coffin for our ideas. I seal the beloved corpse within to pass it across the void, but only because I hope that someone on the other side of the emptiness that insulates each one of us will take it out and lay it on firm ground, where it can fertilize tomorrow’s gardens.


\section{Expropriations
}

Armed with this history, but never awaiting it, because limiting ourselves to distinct phases of struggle alienates tasks that must form an organic whole, we must take another step. {\em The embodiment of a communal relationship with the world through increasingly profound expropriations that are simultaneously material and spiritual.}


They are expropriations because they take forms of life out of the realm of property and into a world of communal relations where capitalist value has no meaning.


They are material because they touch the living world and the other bodies who inhabit it, and spiritual because they nourish us and reveal the animating relationship between all things.


Their simultaneity means that they undermine the established categories of economic, political, and cultural. Each of our acts unites elements from all the analytical categories designed to measure alienated life. {\em The transcendence of the categories of alienation is the hallmark of the reunification of what civilization has alienated.}


Do we harvest plants to feed ourselves, as an act of sabotage against a commodifying market, or because our herb-lore and our enjoyment of nature’s bounty tells us who we are in this world? Leave the question for the sociologists: for us it is a no-brainer.


If this quest leads us out of the cities and into the woods, so be it (though many more of us need lessons on how to reclaim communal relationships, how to enact land and freedom in urban space, and fast). But the profound need to overcome alienation and reencounter the world will never take us out of harm’s way. If we go to the woods to find peace—not inner peace but an absence of enemies—we’re doing it wrong. Life lived against the dictates of colonization is a life of illegality and conflict.


Expropriation means we are plucking forms of life out of the jaws of capitalism, or more precisely, ripping them out of its hideous, synthetic body, to help them reattain a life of their own. We do this so that we too can have lives of our own.


This does not mean—and I can’t emphasize this enough—that we measure our struggle in terms of how much damage we do to the State or how much the State defines us as a threat. Although anarchists embody the negation of the State, we are not its opposites. Opposites always obey the same paradigm.


The State has no understanding of the world as community. Capitalists, who lack the strategic and paranoid overview that agents of the State operate in, understand it even less. Some of our expropriations will be open declarations of war, and they will result in some of us dying or going to prison, but other expropriations won’t even be noticed by the forces of law and order, while the capitalist recuperators won’t catch on until our subversion has become a generalized practice.


If we are anarchists, if we are truly enemies of authority, there can be absolutely no symmetry between what capitalism tries to do to us and what we must do to capitalism. {\em Our activity must correspond to our own needs, rather than being inverse reactions to the needs of capitalism.}


\section{Feeding ourselves
}

Little by little, we need to begin feeding ourselves in every sense through these expropriations. And in the unalienated logic of land and freedom, feeding ourselves does not mean producing food, but giving and taking. Nothing eats that is not eaten. The only rule is reciprocity. What capitalism arrogantly sees as exploitation, extracting value, is nothing but a short-sighted staving off of the consequences of the imbalance it creates.


Feeding ourselves, therefore, means rescuing the soil from the prisons of asphalt or monocultures, cleaning it and fertilizing it, so that we may also eat from it. It does not stop there. Feeding ourselves means writing songs and sharing them, and taking hold of the spaces to do so for free. Learning how to heal our bodies and spirits, and making those skills available to others who confront the grim challenge of trying to win access to a healthcare designed for machines. Sabotaging factories that poison our water or the construction equipment that erects buildings that would block our view of the sunset. Helping transform our surroundings into a welcoming habitat for the birds, bugs, trees, and flowers who make our lives a little less lonely. Carrying out raids that demonstrate that all the buildings where merchandise is kept and guarded are simply common storehouses of useful or useless things that we can go in and take whenever we want; that the whole ritual of buying and selling is just a stupid game that we’ve been playing for far too long.


The ways to feed ourselves are innumerable. A body does not live on carbohydrates and protein alone, and anyone who claims that the exploited, the proletariat, the people, or the species have set interests is a priest of domination. Our interests are constructed. If we do not loudly, violently assert our needs, politicians and advertisers will continue to define them.


\section{Finding what’s “ours”
}

In the course of our attempt to nourish ourselves outside of and against capitalism, we will quickly find that there is no liberated ground. No matter where we are, they make us pay rent, one way or another. A necessary and arduous step forward will be to free up space from the grips of domination and liberate a habitat that supports us, a habitat we are willing to protect. In the beginning, this habitat could be nothing more than an acre of farmland, a seasonal festival, a city park, or even just the space occupied by a decrepit building.


There are several important considerations we must explore if we are to find what’s ours. They all have to do with how we cultivate a profound relationship with place. We cannot aim for such a relationship if we are not willing to incur great danger. Making your home on a bit of land, refusing to treat it as a commodity, and rejecting the regulations imposed on it means going to prison or ending your days in an armed standoff unless you can call up fierce solidarity or mobilize an effective and creative resistance. But the more such resistance spreads, the more certain it is that people will die defending the land and their relationship with it.


If you would not die for land or a specific way of moving through it, don’t bother: you’ll never be able to find a home. But how can we build that kind of love when we are only moving on top of the land like oil on water, never becoming a part of it? Everyone yearns to overcome alienation, but very few people still enjoy a connection worth defending.


The fortitude we need takes great conviction, and that conviction can only build over time. Nowadays, perhaps only one out of a thousand of us would give up their lives to defend a habitat they consider themselves part of. The question we need to answer is, how do we foreground that kind of love, how do we spread it, and for those of us who survive and move on, how do we play our part in cultivating an inalienable relationship with place when the misery of defeat and the coldness of exile make it easier to forget?


It is all the more difficult in North America, where society is increasingly transient. Transcience is not a simple question of moving around, as though anarchists should simply stay in their hometown or as though nomads enjoyed a less profound relationship with the earth than sedentary gardeners. But nomads don’t travel just anywhere. They also cultivate an entirely specific relationship with the world around them. Their habitat just has a temporal as well as a spatial dimension.


The problem of transcience in capitalist society is one of not forming any relationship with the place where we live. This is the reason why anarchists who stay anywhere more than a few years drown in misery, and why the anarchists who always move to the new hip spot never stay more than one step ahead of it. It is a key problematic that we need to devote more thought to than we do to the latest French translation or intellectual trend.


In the Americas in particular, there is another great difficulty with finding what’s ours. Our potential relationship to the commodified land (land in the liberal sense that has been imposed by force of arms) is largely codified through a system of race categorization that was developed by colonizers in the 17\high{th} and 18\high{th} centuries. This land was stolen, and it was worked and improved—in the capitalist sense—by people who were stolen from their land. It’s true that the land in Europe was also stolen from those who lived in community with it, and that many of those people were shipped to the Americas and forced to work there. It’s also true that many of them ran off to live with the original inhabitants, or planned insurrections alongside the people kidnapped, enslaved, and taken from various parts of Africa, and that this subversive mingling is what forced the lords and masters to invent race.


It no less true that apart from having money, the surest way to win access to land—albeit commodified land—in the history of the Americas up until the present moment has been by being white. Whatever our feelings or consciousness of the imposed hierarchy of privilege, indigenous people have been robbed of their land and repeatedly prevented from reestablishing a nourishing, communal relationship with it, the descendants of African slaves have been kicked off whatever land they had access to any time it became desirable to whites or any time they had built up a high level of autonomy, while whites, at least sometimes, have been allowed limited access to the land as long as it did not conflict with the immediate interests and projects of the wealthy. The legacy of this dynamic continues today.


The implication of all this is that if white anarchists in the Americas (or Australia, New Zealand, and other settler states) want to form a deep relationship with a specific habitat, claiming land to the extent that it belongs to us and we belong to it, we had better make sure that the only other claims we are infringing on are those of capitalist and government landlords. Are there indigenous people who are struggling to restore their relationship with that same land? Is it land that black communities have been forced out of? How do those people feel about you being there, and what relationship do you have with them? Under what conditions would they like to have you as a neighbor? If white people in struggle continue to assert the first pick on land, this is hardly a departure from colonial relations.


Treating the land like a tabula raza, an empty space awaiting your arrival, is antithetical to cultivating a deep relationship with it. Etched into that land are all the relations with the people who came before you. By trying to become a part of it, will you be reviving their legacy, or destroying it? Find out before you attempt to put down roots.


\section{A longterm proposal
}

The narrative we express in our struggles exerts a huge impact on the outcome of those struggles. Half of domination is symbolic, and by focusing on the quantifiable or the putatively material, rebels have missed out on this other sphere within which battles against power take place.


If we occupy a building as squatters, we signal that our concern is empty buildings and not the land beneath them, nor our relationship with it. If squatters become strong enough that the State is forced to ameliorate and recuperate them, it will take the path of ceding legal spaces and maybe even tweaking the housing laws or creating more public housing. In a revolutionary sense, nothing is won.


If we occupy a building as anarchists who communicate nothing but a desire to destroy all forms of authority, we are safe from recuperation, because we project no way forward for our struggle, no path for the State to reroute. We also make it almost impossible to advance, and we facilitate state repression. With nothing to win, our struggle thrives on desperation, and with nothing to share, no one else will connect to our struggle except the equally nihilistic.


But what if we raised the cry of “Land and Freedom”? What if we projected our struggle as a drive to progressively liberate territory from the logics of state and capitalism? What if we unabashedly spoke about our desire to free ourselves?


While we are weak, we will choose weak targets: vacant lots, abandoned land, an empty building with an absentee landlord. Or a place we already have access to, a home we live in for example. Whether we transform that place into a garden, a social center, a workshop, or a collective house, it must find its way into a specific narrative of liberation. If we justify our use of that space on the grounds that we are poor, that there isn’t enough affordable housing, that the youth need a place to hang out, that people need access to a garden for lack of fresh produce in their diets, or any similar discourse, we are opening the door to recuperation, we are pinning our rebellion to a crisis within capitalism and sabotaging all our work as soon as the economy improves or the government institutes some reform to ease the shortage of housing, produce, youth centers, and so forth.


If we justify our use of that space with a rejection of private property, we have taken an important step forward, but we also construct a battlefield in which our defeat is assured. A rejection of private property is abstract. It leaves a vacuum that must be filled if the capitalist paradigm will be broken. A relationship always exists between the bodies that inhabit the same place. What relationship will we develop to drive out the one of alienated commodities? By refusing to talk about this and put it into practice, we also refuse to destroy private property, no matter how radical a posture we adopt. Nor have we formed and expressed an inalienable relationship with the specific place we are trying to claim. Why that land? Why that building? And it’s true, we want to destroy private property the world over. But you do not form a relationship with the land in the abstract, as a communist might. This is why the spiritual aspect of struggle that the materialists, as priests of Enlightenment thinking, deride and neglect, is important. A communal relationship with the land is always specific.


This means that in every case, we need to assert our legitimacy to claim land over the legitimacy of the legal owners. And while we recognize no claims of legal ownership, we must deny every legal and capitalist claim {\em specifically and generally at the same time.} This means dragging specific owners through the mud as exploiters, colonizers, murderers, gentrifiers, speculators, and so forth, as a part of the process by which we assert our specific claim to that land, but always within a general narrative that refuses to recognize the commodity view of land and the titles, deeds, and jurisdictions that bind it.


While we are weak, it will make more sense to go after owners whose claims to a land-commodity are equally weak—banks that have won property through foreclosure, hated slumlords, governments that are unpopular or in crisis.


Initially, we can win access to land in a variety of ways. Seizing it and effectively defending it, raising the funds to buy it, pressuring the legal owner to cede the title. None of these are satisfactory because all of them leave the structures of capitalist ownership intact. Even in the first case, which clearly seems more radical, the legal owner maintains a claim that they can pursue at a later date, eventually mustering the state support needed to effect an eviction. Ownership has not been undermined, only access.


Once we have access to land, it is crucial to intensify our relationship with it. To share our lives with it and begin to feed ourselves with the relationship we create. To signal that relationship as a reversal to the long history of dispossession, enslavement, exploitation, blackmail, and forced integration that has dogged us for centuries. To announce the place as liberated land, if we are indigenous to the area, and as a maroon\footnote{The maroons were escaped slaves, primarily of African descent but also including European runaways, who inhabited mountains, swamps, and other wild areas in the Americas and Caribbean. They generally mingled with and fought alongside indigenous peoples as they resisted the plantation states being created by European powers.} haven if we are not. In our use of the semi-liberated place, we must communicate to the world that the social contract of capitalism is absolutely unacceptable to us, that our needs are other, and we have no choice but to fulfill them on our own. Simultaneously, we invite all the others who are not fulfilled by capitalism to connect with us.


As we intensify a relationship of land and freedom, our spreading roots will come up against the concrete foundation of property that lies beneath us. The next conflict is to negate the forms by which capitalism binds land (rejecting titles and claims of ownership) and to impugn the right of a government to tax and regulate land that it has stolen.


In the course of this fight, we will lose much of the land we gain access to. Buildings will be evicted, gardens will be paved over, forests will be cut down. This inevitability gives rise to two questions. How to strike a balance between prudence and conflicitivity so that we neither become pacified nor lose our places needlessly? And when we lose, how to do so in a way that is inspiring, that spreads and strengthens our narrative and legitimacy so that next time we will be stronger? The first question will be the harder one. Anarchists have a long history of losing well, but at least since World War II one of our most frequent failings has been the recuperation of our creative projects and the isolation of our destructive projects. Gaining something that they can lose often turns radicals into conservatives. Our semi-liberated places must aid us in our attacks on the State and give solidarity with those who are repressed. Not to do so means losing these places even as they persist in time; they are colonized, they become parodies of themselves and agents of social peace. At the same time, even as they must play a conflictive role, these are the places that nourish us, and we should not risk them needlessly.


Little by little, we will win places where we achieve de facto autonomy, and communal relationships with the land and all other living things can begin to flourish. These places will never be safe or stable. Any moment we are weak, the State may try to take them away from us, with or without a legal pretext. The more widespread support we have, the better justified our narrative and our legitimacy, and the deeper our relationship with a place, the more dangerous it will be for the State to attack us. Additionally, in times of reaction, it will be easier for us to hold on if we have won access to land using a variety of means, from squatting to winning titles. Radical sensibilities will prefer the former, but it should be clear that in both cases the capitalist foundation remains the same. The history of the squatting movements in Europe shows that squatting opens bubbles of autonomy but in and of itself it does not challenge capitalism.


If we have used a variety of means, it will be harder for the State to criminalize us across the board or to construct a legal apparatus capable of evicting us from all of our footholds.


By communicating and building strong networks, these different semi-liberated places can share resources and experiences, broaden their perspectives, and compound their legitimacy. The age-old question of organization is unimportant because such places are heterogeneous. They practice different forms of organization and do not all fit into the same organizational scheme. The present proposal does not envision a movement of urban and rural land projects working towards liberation, as though a thousand people will read this article, understand it in the same way, and all try to put the same thing into practice. The network that will form may well include movements within it, but none will be all-encompassing.


In the Americas, there are already many semi-liberated places in existence that dream of an end to capitalism, and weak networks connect them. Most of these places, or the strongest ones at least, have been created by indigenous struggles. I believe that anarchists who are against civilization can find their place within such networks, defining ourselves in relation to an ongoing attempt to restore a communal relationship with the land, as did the Magonistas in Mexico or many peasant anarchist partisans in the Russian Revolution. Up until now, we mostly define ourselves in relation to an anarchist movement or milieu, or in relation to consumer society. Neither the abstract community of the former nor the posture of rebel and alternative within the latter suit our project of liberation.


In part, this means avoiding sectarian duels with those anarchists who see their battlefield as the workplace or the post-modern city. People who understand themselves as proletarians should struggle as proletarians. I fear that the proletarian worldview is hopelessly poisoned by colonialism and will only reproduce the destruction of nature and the exploitation of all living beings, as proletarian movements have in the past, but using ideology as an indisputable tool for predicting the future just leaves a bad taste in my mouth. It’s better to make criticisms, share them, and back them up with robust struggles that embody a different logic.


If we are to understand ourselves within a network of projects that liberate the land from capitalism and create specific, communal relationships with that land, as newcomers (referring to those of us who are not indigenous) a certain amount of humility is in order. How can we learn from the indigenous struggles that have fought the longest and the hardest for the land without fetishizing them? How can we respect indigenous land claims without essentializing them or legitimizing the state-appointed tribal governments that often manage such claims? I can only offer these as questions, leaving the answers to practice. It is worth signalling, however, that such a practice must build itself on personal relationships of solidarity and friendship rather than abstract notions of unity.


Fortunately, there is a long history for such relationships. In the first centuries of the colonization of the Americas, many people brought over from Africa and Europe and made to work the newly alienated land ran away and fought alongside indigenous people fighting for their freedom and survival. Evidently, there existed a strong basis for solidarity. Today, especially in North America much of that solidarity is absent. Many of the poorest people, regardless of their skin color, are staunch advocates of colonization, Western progress, and capitalism.


Most non-indigenous people in the Americas do not have the practical option of going back to Europe, Africa, or Asia. Yet those of us who are not indigenous, just because we claim solidarity and envision a happy network of communities restoring communal relationships with the land, cannot assume that indigenous people will want us as neighbors. This is a problematic that cannot be resolved with theory or consideration.


Our only option is to struggle for our own needs—this is a prerequisite for any conversation of solidarity, as much as the identity politicians try to avoid it—try to build solidarity with indigenous peoples in struggle, explore the possibilities for a common fight against colonization, and see what answers arise, dealing with the conflicts that inevitably arise with patience and humility.


\section{Communities of the earth
}

As more and more of us begin to wrap our lives into these semi-liberated places, communities will form. Not the alienated pseudo-communities that the very worst of anarchists claim to have today. Communities are built by sharing, and if all we share is a little bit of time in our alienated lives, the bonds will not be strong enough to hold us together, as the failures of “accountability,” resistance to repression, healing, coping with burnout, and intergenerationality in the pseudo-communities amply demonstrate.


When we come together to intensify our relationships with a semi-liberated place, we share so much more. We become part of the web by which the others nourish themselves. At this point, it becomes honest to speak about a community.


As such communities begin to form, certain things will become evident. First of all, while vigorous debate and historical, theoretical clarity are vital in the life of the community, most of the skills and activities necessary for intensifying communal relationships are neither abstract nor discursive. They are practical skills that support the functions of life. Cooking, gardening, childcare, healing, sewing, brewing, dentistry, surgery, massage, gathering, hunting, fishing, trapping, weaving, welding, carpentry, plumbing, masonry, electricity, painting, drawing, carving, animal husbandry, curing, tanning, butchering, apiculture, silvaculture, mycology, storytelling, singing, music-making, conflict resolution, networking, translating, fighting, raiding, and otherwise relating with a hostile outside world (with legal skills, for example).


A community with three web designers, five writers, three gardeners, four musicians, a tanner, a brewer, a painter, and a lawyer will not survive. And not for lack of self-sufficiency. It is not about seceding from capitalism, but about bringing capitalism down with us. Such a community will not survive because they lack the skills necessary to intensify their relationships with one another and with the place they are trying to liberate. With weak relationships, they will not be able to withstand capitalism’s continuous onslaught. They will either be forced to move out or to pacify themselves.


Capitalist deskilling precedes the Fordist economy. Deskilling was present at the beginnings of industrialization, and it was present even earlier in the witch hunts and the attendant creation of universities and scientific professions in Renaissance Europe. Popular knowledge, especially that related to healing, was criminalized and destroyed, whereas a mechanical science of healing suited to nascent capitalism and the modernizing State that was grooming it, was instituted, enclosed, and regulated within the new academies. If we are to create communal relations against capitalism, we must commit ourselves to an intensive, lifelong process of reskilling so that we may nourish ourselves in every sense.


The creation of communities will not only show us the toxic uselessness of liberal education. It will also reveal the inadequacy of that cherished anarchist concept, affinity.


It is time to forget about affinity. Those who currently call themselves anarchists tend to be the warriors and messengers of communities that do not yet exist. Some others are the poets and artists who feed off of the warriors for a while before they go off on their own. We have seen what artists become, surrounded by other artists, and we have seen what warriors do, surrounded by other warriors, and the anarchist struggle has long suffered the consequences. The concept of affinity has done enough damage. It is a thoroughly rationalist notion, based on the idea of sameness as prerequisite for equality, and equality as something desirable.


Members of the much mythologized affinity group do not all experience their affinity in the same way. They do not perceive the group equally, and nearly every group, contrary to its mythology, does in fact have one or two central members. What holds the group together is not affinity, but a collective project. Only amidst a generalized scarcity of trust and sharing does it become possible to confuse these two binding forces.


The community, as a collective project, does not need affinity to hold together. What it needs is sharing, a common narrative, and above all, difference. In every community there should be some anarchists, in the sense given that term today. But a community of anarchists would be intolerable. As long as anarchists remain specialists of propaganda, sabotage, and solidarity—and this is the normative form that is reproduced today—we will scarcely be able to build communities. But as we learn to form connections of complementary difference, the dream of anarchy will become available to people whose temperament is not that of warriors or messengers, and anarchists, for our part, will find our place in a larger social body.


The gamble here is that a great many people are attracted to the dream of anarchy—self-organization, mutual aid, the destruction of all authority—but they are not attracted to the anarchist mode—protests, frequent risk-taking, the constant and scathing analysis of our surroundings; and that this anarchist mode, looped back in on itself, creates a pseudo-community that is toxic and self-defeating, whereas if it found a place within a broader struggle for life lived completely, could defend and spread communities subversive to capitalism.


\section{In Conclusion
}

The challenge presented by a truly anarchist vision of the concepts, land and freedom, center an awareness of colonization as an ongoing force in capitalist society. It is a challenge that requires us to root out the liberal conceptions of land and freedom and all the baggage that accompanies them, including a great many ideations long internalized by anarchists, such as organization through affinity, the pseudo-community and self-referentialization within an abstract milieu, and the externalization of land or the dichotomy city/wilderness.


Above all, it is a challenge that requires a great creative labor. The tasks at hand can take the paths of reskilling, forming a specific relationship with the land, recovering histories that speak of our alienation, expropriating aspects of life, winning access to land, transforming that land, intensifying our relationships with it, and putting our destructive activity at the service of these new relationships.


I want to explore each of these ideas in more depth in future articles. But for now, we have the outlines of a challenge. It is not a new challenge, though I have tried to orient it to the specific problems of our times. Through reflection and action, I hope that once again anarchists can join others in taking up the call for land and freedom, and that when we do, we’ll know what we’re about.


 









\page[yes]

%%%% backcover

\startmode[a4imposed,a4imposedbc,letterimposed,letterimposedbc,a5imposed,%
  a5imposedbc,halfletterimposed,halfletterimposedbc,quickimpose]
\alibraryflushpages
\stopmode

\page[blank]

\startalignment[middle]
{\tfa The Anarchist Library
\blank[small]
Anti-Copyright}
\blank[small]
\currentdate
\stopalignment

\blank[big]
\framed[frame=off,location=middle,width=\textwidth]
       {\externalfigure[logo][width=0.25\textwidth]}



\vfill
\setupindenting[no]
\setsmallbodyfont

\startalignment[middle,nothyphenated,nothanging,stretch]

\blank[line]
% \framed[frame=off,location=middle,width=\textwidth]
%       {\externalfigure[logo][width=0.25\textwidth]}


Sever



Land and Freedom



an old challenge




April, 2014


\stopalignment
\blank[line]

\startalignment[hyphenated,middle]




Black Seed no.1


\stopalignment

\stoptext


